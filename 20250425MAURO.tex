% referencias
\addbibresource{./files/gbTeXbib-20250425MAURO.bib}

% glosarios
\input{./files/gbTeXglo-20250425MAURO.tex}

\ifPDF
% separata de capítulos solo en pdf de pantalla
\newcounter{PrimPag}
\setcounter{PrimPag}{1}
\setcounter{PrimPag}{\theCurrentPage+1}
\newwrite\script
\immediate\openout\script=separatas.sh
\immediate\write\script{\string#!/bin/bash}
\newcommand{\separata}[1]{\immediate\write\script{pdftk ./pdf/\jobname.pdf cat \thePrimPag-\theCurrentPage\space output ./docs/#1-\jobname.pdf}}

% control de inconsistencias de principio y fin de linea
\usepackage[hyphenation,homeoarchy,homeoarchywordcolor=orange, homeoarchycharcolor=orange,draft]{impnattypo}
\usepackage[allcolors=magenta, colorlinks, unicode]{hyperref}
\usepackage{easyReview}
\usepackage{hyperxmp}
\hypersetup{
pdfauthor={Mauro David Dobruskin},
pdfsubject={-},
pdfsubtitle={Aportes al estudio del campo periodístico},
pdfauthortitle={-},
pdfdate={20250428222555},
pdfcopyrightstatus={Public Domain},
pdfrightsinformationurl={https://creativecommons.org/licenses/by-nc/4.0/?ref=chooser-v1},
pdflicenseurl={https://creativecommons.org/licenses/by-nc/4.0/?ref=chooser-v1},
pdfcaptionwriter={Jorge Boromir},
pdfpublisher={Mordoc},
pdftype={Text},
pdfpubtype={book},
pdfvolumenum={1},
pdfissuenum={1},
pdfpagerange={181},
pdfbookedition={1},
pdfisbn={978-950-793-001-0},
pdfpubstatus={P},
pdflang={es-ES},
pdfmetalang={es},
pdfrendition={proof},
pdfsummary={En la Argentina, la intervención de periodistas como autores de libros que discuten sobre el campo periodístico y su legitimidad cuenta con una larga tradición desde los inicios mismos de la Nación. Sin embargo, el período que transcurre entre los años 2009 y 2015 fue especialmente pródigo en la cantidad y diversidad de material que se produjo sobre el tema, con especial énfasis en la crítica a colegas y/o a medios de comunicación.},
pdfkeywords={comunicación; periodistas;},
pdfcopyrightstatustype={Text},
pdftitle={Periodistas en debate. Aportes al estudio del campo periodístico},
pdfcreator={gbTeXpublisher},
pdfproducer={Ecosistema de LaTeX},
unicode=true,
bookmarks=true,
pdfdisplaydoctitle=true,
pdfnewwindow=true
}

	\else
	\ifBNPDF
	\usepackage[width=18truecm,height=25.5truecm,cam,center]{crop}
	\newcommand*\infofont[1]{\sf{\footnotesize #1 (iartola@unrn.edu.ar)}}
	\crop[font=infofont]
	\usepackage[hidelinks, unicode]{hyperref}
	\usepackage{hyperxmp}
	\hypersetup{
pdfauthor={Mauro David Dobruskin},
pdfsubject={-},
pdfsubtitle={Aportes al estudio del campo periodístico},
pdfauthortitle={-},
pdfdate={20250428222555},
pdfcopyrightstatus={Public Domain},
pdfrightsinformationurl={https://creativecommons.org/licenses/by-nc/4.0/?ref=chooser-v1},
pdflicenseurl={https://creativecommons.org/licenses/by-nc/4.0/?ref=chooser-v1},
pdfcaptionwriter={Jorge Boromir},
pdfpublisher={Mordoc},
pdftype={Text},
pdfpubtype={book},
pdfvolumenum={1},
pdfissuenum={1},
pdfpagerange={181},
pdfbookedition={1},
pdfisbn={978-950-793-001-0},
pdfpubstatus={P},
pdflang={es-ES},
pdfmetalang={es},
pdfrendition={proof},
pdfsummary={En la Argentina, la intervención de periodistas como autores de libros que discuten sobre el campo periodístico y su legitimidad cuenta con una larga tradición desde los inicios mismos de la Nación. Sin embargo, el período que transcurre entre los años 2009 y 2015 fue especialmente pródigo en la cantidad y diversidad de material que se produjo sobre el tema, con especial énfasis en la crítica a colegas y/o a medios de comunicación.},
pdfkeywords={comunicación; periodistas;},
pdfcopyrightstatustype={Text},
pdftitle={Periodistas en debate. Aportes al estudio del campo periodístico},
pdfcreator={gbTeXpublisher},
pdfproducer={Ecosistema de LaTeX},
unicode=true,
bookmarks=true,
pdfdisplaydoctitle=true,
pdfnewwindow=true
}

		\else
		\ifPNGEPUB
		\usepackage[hidelinks, unicode]{hyperref}
			\else
			\ifHTMLEPUB
			\usepackage[allcolors=blue,colorlinks,hyperindex=true,unicode]{hyperref}
			\fi
		\fi
	\fi
\fi

\begin{document}
\frontmatter

\textbf{Maestría en Periodismo}

\textbf{Facultad de Ciencias Sociales}

\textbf{Universidad de Buenos Aires}

\textbf{Tesis de Maestría}

\textbf{Periodistas en Debate. Aportes al Estudio del Campo Periodístico}

\textbf{(2009 -- 2015)}

\textbf{Mauro David Dobruskin}

\textbf{Director: Dr. Julio Moyano}

\textbf{Septiembre 2020}

\chapter{Agradecimientos}

A los docentes de la Maestría y en particular a Nadia Koziner, por haberme devuelto la alegría de estudiar

A Julio Moyano que compartió sus conocimientos, experiencia y amistad para que este trabajo llegara a su fin.

A Martín Güelman por su paciencia y dedicación en la corrección de los avances

A Esther Jodice, por haberme acompañado en lo mejor de mi vida

\chapter{Pre texto}

Cuando durante 2015 comencé a dar forma al proyecto de tesis, la bibliografía sobre sociología del periodismo argentino era escasa y/o fragmentaria. De hecho, distintos autores que estudiaban el tema lo presentaban como un déficit a subsanar. Por otra parte ese año, y los precedentes, habían dado lugar a una producción desusada de libros sobre la actividad periodística como expresión del conflicto, inicialmente, entre el gobierno y uno de los grandes grupos de medios de Argentina, pero luego el conflicto se extendió, en apariencia, a todo el periodismo argentino. A la luz de ese conflicto, creí observar la estructuración de un discurso, que daba cuenta del hacer y del sentir del conglomerado de periodistas que intervenían en la serie editorial y que, a mi entender, permitiría descubrir, o al menos reconocer, las formas que la acción del campo periodístico adopta, de manera especial, en una situación de conflicto abierto.

Pierre Bourdieu (2008) plantea que en todo campo hay conflicto, pero que el campo lo gestiona de manera tal que no eclosione. El campo periodístico es experto en la gestión del conflicto. Su capacidad de convertir el conflicto entre actores significativos de la sociedad en acontecimientos de actualidad es, justamente, el sentido de su especificidad. Sin embargo, el conflicto al interior de sus entrañas, parece haberlo descentrado poniendo en riesgo el principio de su existencia, la creencia social colectiva en la identidad entre el discurso periodístico y la realidad, en la verosimilitud de los discursos de los periodistas. Cuando se escriben estas líneas (entre marzo y agosto de 2020), el conflicto ha escalado a niveles difíciles de imaginar en 2015, cuestionando en el escenario público, a periodistas de gran prestigio en Argentina.

Mientras los medios gráficos poseían la exclusividad como medios de información, estos detentaban una unidad discursiva estructural, es decir los periodistas no eran públicamente reconocidos, ya que escasamente firmaban sus notas, por lo que la responsabilidad del periódico residía en los editores y el director. Sin embargo, el advenimiento de los medios audiovisuales promovió el proceso de identificación y autonomización de un número significativo de periodistas, fortaleciendo, a mi entender, el campo periodístico. Sin embargo, el conflicto del que damos cuenta en la tesis parece haber puesto en crisis esa creencia colectiva, haciendo que la existencia misma del campo se halle en riesgo.

\mainmatter
\chapter{Periodistas en Debate. Aportes al estudio del campo periodístico}

\begin{quote}
``-¿Y si se rompe la ilusión? -dije con un hilo de voz.

-Qué se va a romper -me tranquilizó.

-Por si acaso, seré una tumba -le prometí-.

Lo juro por mi adhesión personal, por mi lealtad al equipo, por usted, por Limardo, por Renovales.

-Diga lo que se le dé la gana, nadie le va a creer.''

\emph{Esse est percipi} (Ser es ser percibido)

(Borges y Bioy Casares, 1967)
\end{quote}


\chapter{1. Introducción}

\section{1.1. Planteo del problema de investigación}

En Argentina, la intervención de periodistas como autores de libros que discuten sobre el campo periodístico y su legitimidad cuenta con una larga tradición desde los inicios mismos de la Nación\footnote{Entre otros estudios sobre debates en torno a la construcción y legitimación del campo periodístico desde los orígenes de la prensa argentina, Cfr. Rivera (1968, 1990, 1998), Molina (2012), Alonso (1997, 2004), Moyano (1996, 2008, 2013), Ojeda y Moyano (2015, 2019), Ojeda (2010), De Marco (2006), Díaz, (2012), Auza (1978}. Sin embargo, el período que transcurre entre los años 2009 y 2015 fue especialmente pródigo en la cantidad y diversidad de material que se produjo sobre el tema, con especial énfasis en la crítica a colegas y/o a medios de comunicación.

Desde la reinstalación del sistema democrático, en 1984, es posible identificar el desarrollo y la consolidación de los libros de periodistas como uno de los géneros predominantes del ámbito editorial (De Diego, 2014).Por otra parte, este fenómeno constituye una forma de intervención pública y de jerarquización profesional en el ámbito periodístico (Pereyra, 2013; Vommaro y Baldoni, 2012).

Baldoni y Gómez Rodríguez (2018) dan cuenta del fenómeno mostrando el proceso de incorporación de la producción de libros de periodistas en el mercado editorial, inicialmente desde las pequeñas editoriales. Luego, a partir de la consolidación de este género, las grandes casas editoriales encontraron, como muestra Saferstein (2016), un nicho de mercado propicio para la producción de \emph{best-sellers}.

El período señalado para esta investigación no solo no cambia la tendencia que aprecian los autores mencionados, sino que la profundiza. El conflicto entre el gobierno de la presidenta Cristina Fernández de Kirchner y los intereses de los sectores agrícola-ganaderos durante su segundo mandato presidencial (2011-2015), dará lugar a una producción creciente de libros periodísticos dentro de esta categoría.

Como correlato del conflicto entre el gobierno y el campo periodístico se produjo, simultáneamente, un conflicto de grandes proporciones entre el gobierno y los principales medios de comunicación (Arrueta, 2013; Stefoni, 2013; Gindin, 2014; Ure y Schwarz, 2014). Si bien este fenómeno (el conflicto con los medios de comunicación) no ha sido exclusivo del gobierno de Cristina Fernández de Kirchner, lo novedoso del período es lo prolongado del mismo y que se trasladó al interior del colectivo de los periodistas. Ello produjo disputas discursivas y denuncias recíprocas alrededor de la deontología de la actividad y los principios que la deben regir, personalizando en colegas la crítica sobre el quehacer de la profesión, de manera explícita o implícita, como no se veía desde el debate Sarmiento-Alberdi de 1853 en torno a este tipo de problemas\footnote{El carácter de caso paradigmático de confrontación en torno al rol del periodismo en la construcción de la república moderna ha sido revisitado por los trabajos de Shumway, 2002; De Marco, 2006; Moyano, 2008; Ruiz, 2014 y por Sábato, Myers y Stuven en Altamirano y Myers, 2008, T. 1}.

Fernando Ruiz, en \emph{Guerras Mediáticas} (2014: 4-5), reseña la sucesión de conflictos entre el poder político y los medios de comunicación en la Argentina desde sus orígenes. El autor sostiene que estos conflictos se desarrollan cuando:

\begin{quote}
Un sector importante de la sociedad percibe que su antagonista ideológico ha tomado una dimensión y una actitud amenazantes, se convence de que se acabaron las alternativas y de que llegó la hora del enfrentamiento abierto. En ese contexto se va forjando entre los periodistas un cierto consenso acerca de que el éxito de las ideas del enemigo los encerrará en un país inaceptable. Esa tensión ideológica es la que radicaliza la tensión mediática. En ese momento crece la conciencia en los protagonistas de que el país está en un cruce histórico de caminos, donde la duda, la moderación y el matiz son la trampa de los cobardes y los tibios, de los desorientados o, peor, de los enemigos encubiertos.
\end{quote}

El párrafo precedente expresa con claridad cierta perspectiva dominante en el campo y en buena parte de la intelectualidad, que contempla al periodismo como un espacio de reserva moral o ideológica. Sin embargo, como elemento diferenciador de los conflictos anteriores, nos encontramos ante una situación inédita en la historia de la prensa comercial.

Por un lado, la prensa y el periodismo en general enfrentan una crisis de financiamiento, es decir, la actividad ha devenido poco o nada rentable y en muchos casos deficitaria. Por otro lado, la consolidación de una nueva tecnología que promueve la competencia de \emph{amateur} o de advenedizos en la actividad. Por último, la irrupción en la actividad periodística de grandes conglomerados económicos.

Ramonet afirmaba en 2011, que:

\begin{quote}
El impacto del Meteorito ``Internet'' comparable al que hizo desaparecer a los dinosaurios, está provocando un cambio radical de todo el ecosistema mediático y la extinción masiva de diarios de la prensa escrita (\ldots) En un contexto así el periodismo tradicional literalmente se está desintegrando (p. 11, 23).

Por su parte Françoise Benhamou (2015: 34-35) afirma:

Inútil negarlo: la prensa diaria se hunde en la crisis que no solo pone en jaque gran cantidad de empleos sino también algunos fundamentos de la democracia en la que el diario desempeña un papel esencial (\ldots). Comienza la hecatombe en Estados Unidos. Entre 1991 y 2011 desaparecieron doscientos cuatro diarios y la tirada de la prensa diaria disminuye un 27 \%.
\end{quote}

En síntesis, como ampliaremos luego, el conflicto entre periodistas se suscitó en un contexto de confrontación política, crisis económica y transformaciones tecnológicas de la actividad que, a modo de hipótesis, produjo un estado de inquietud e incertidumbre ante la posible desintegración del campo periodístico. Consideramos que la confluencia de estos tres factores es lo que desencadenó o amplificó la disputa discursiva al interior del campo periodístico.

El análisis de estas producciones, entendemos, permitirá identificar los argumentos que los periodistas desplegaron para sostener y disputar la voz legítima del campo periodístico. Ello permitirá avanzar, estimamos, en el conocimiento de los mecanismos regulatorios del campo, aspectos de su identidad profesional, sus intereses latentes y manifiestos como, así también, aportar al conocimiento del entramado de relaciones entre periodistas.

La propuesta de analizar el contenido de la producción bibliográfica se funda en cuatro razones. En primer lugar, se trata de trabajos en los cuales los autores desarrollaron ideas de manera extensa y reflexiva, ejemplificados con hechos e información propia del campo periodístico que, en general, no utilizan en su actividad dentro de los medios. En segundo lugar, los libros son blasones de prestigio para los periodistas y les brindan la oportunidad de dirigirse a un público que los conoce y los sigue en algunos de los medios, por lo que refuerzan, a través de estas producciones el vínculo con sus audiencias. En tercer lugar, porque el libro los diferencia de sus producciones periodísticas, permitiéndoles tomar algunas licencias y utilizar recursos literarios que, sin poner en riesgo el principio de verosimilitud periodística, lo refuerza. Por último, porque durante el período en cuestión se evidenció, no meramente un aumento considerable en la publicación de títulos sobre la temática, sino el surgimiento de una tipología editorial, como es la de periodismo de periodistas y/o medios, lo que en adelante denominaremos \emph{metaperiodismo}.

Llamativamente, el concepto \emph{metaperiodismo} no es de uso habitual en los trabajos académicos, ya que es difícil rastrear su utilización. Meneses Fernández (2010: 177) lo aplicó al estudio de la prensa durante la denominada \emph{transición democrática española}. La autora llama la atención sobre este hecho: ``Si nos remitimos a la bibliografía, comprobamos que ésta no aporta estudios de la visión endógena publicada por los medios informativos''.

Desde otra perspectiva, Campos-Domínguez y Redondo-García (2015) en \emph{Meta periodismo y transparencia informativa en el periodismo del siglo XXI}, aplican el concepto a modo de instrumento técnico para el perfeccionamiento del periodismo, en un contexto de crisis de financiamiento y consumo de medios periodísticos.

En nuestro trabajo, en consecuencia, aplicaremos el concepto de \emph{metaperiodismo} a la reflexión e interpretación que los periodistas realizan de su actividad, la de sus colegas y de los medios en los libros de periodistas, como núcleo argumental.

Mientras que en los diez años previos al año 2009 se editaron menos de cinco títulos\footnote{En 2007 se publicó \emph{Noticias del poder, buenas y malas artes del periodismo político} de Jorge Halperín. Anteriormente en 2006, Hechos y noticias. Claroscuros de la prensa gráfica en la Argentina, de Eduardo Zukernik, y luego un poco más atrás, en el año 2002, Grandes hermanos. Alianzas y negocios ocultos de los dueños de la información de Eduardo Anguita, para aproximarse a un antecedente del fenómeno.} de esa temática, en el período temporal considerado (entre octubre de 2009 y diciembre de 2015) se publicaron aproximadamente cuarenta, haciendo periodismo sobre periodistas o empresas periodísticas. El primero fue \emph{Diario de guerra. Clarín el gran engaño argentino} de Claudio Díaz (2009), seguido inmediatamente por \emph{678. La creación de otra realidad}, de María Julia Oliván y Pablo Alabarces (2010), los que dan inicio a la saga que, a los efectos del presente trabajo, cerraremos provisoriamente en diciembre de 2015 con \emph{Periodistas depordivos} de Walter Vargas.

\section{1. 2. Objetivos y supuestos de la investigación}

El objetivo de esta investigación es aportar al conocimiento del campo periodístico a través del análisis de los núcleos temáticos de legitimación y reconocimiento, en las producciones bibliográficas de periodistas en la Argentina entre los años 2009 y 2015.

A su vez, se propone indagar en la existencia de estrategias de protección de la legitimidad del campo periodístico que surgen de los libros analizados, e identificar los principios deontológicos de la profesión, en las modalidades argumentativas, implícitas o explícitas, utilizadas por los periodistas en la disputa por el sentido de la actividad periodística.

Esto supone, a modo de petición de principios, la existencia de un campo periodístico, susceptible de ser delimitado y al que se le reconocen determinadas características particulares que, por un lado, lo asimilan al concepto de campo elaborado por Bourdieu y, por otro, lo hacen particular en tanto es posible diferenciarlo del resto de los campos culturales.

Pero, además, supone la existencia de elementos de cohesión identitaria, es decir, ser parte del campo supone ser, existir en tanto periodista; supone adherir a un colectivo social identificado e identificable en el conjunto social.

A su vez, en relación con las estrategias de protección de la legitimidad del campo periodístico sostenemos, abrevando en los desarrollos de Bourdieu sobre esta temática, que el campo periodístico desarrolla o desarrolló durante el período señalado tales estrategias, a partir de una tensión o sentimiento de agresión al interior del campo.


\section{1. 3. Consideraciones preliminares}

El estudio de un campo determinado de la actividad social presenta problemas inherentes al propio campo de estudio, a la importancia y valoración que la sociedad le asigna, su volumen cuantitativo, en relación a la cantidad de sujetos que lo integran, como a su periferia, etc. Bourdieu (2008: 196) plantea con claridad esta cuestión cuando aborda el estudio del campo de la alta costura y considera que este espacio social posee menos protección que otros de mayor interés o de mayor ``legitimidad''. Sostiene en ese trabajo que los ``objetos legítimos se hallan protegidos por su legitimidad contra la mirada científica y contra el trabajo de desacralización que supone el estudio científico de los objetos sagrados'' (Bourdieu 2008: 197).

Ya en 1910, Max Weber expresaba las dificultades que presenta el estudio científico de la prensa y los resguardos que el investigador debía tener frente a la autoconsideración de los periodistas y editores y la potencial sospecha de crítica a su actividad. Sostiene, en su discurso en el Primer Congreso de la Asociación Alemana de Sociología en Frankfurt:

\begin{quote}
(\ldots) un tema que no sólo requerirá unos medios materiales muy importantes para los trabajos preliminares, sino que, de ningún modo, podrá ser tratado objetivamente si los círculos dominantes de la prensa no acogen nuestro proyecto con gran confianza y benevolencia. Es imposible que, si por parte de los representantes de las casas editoras o por parte de los periodistas nos encontramos con la sospecha de que el objetivo de la Asociación es formular críticas moralizantes sobre la situación existente ---es imposible, digo, que alcancemos entonces nuestro propósito; porque es imposible alcanzarlo si no nos podemos proveer, en muy gran medida, de material procedente precisamente de este sector (1924).
\end{quote}

Como observa Weber, la mera sospecha de crítica por parte de los periodistas o las casas editoras, haría imposible alcanzar el objetivo de investigación. Weber tenía conciencia acerca de la impermeabilidad a la crítica por parte del conjunto de los actores de la actividad periodística, empresas editoras y periodistas.

Es un hecho no siempre evidente, que son pocas las actividades profesionales que poseen mayor protección simbólica y jurídica que la actividad periodística, al punto tal que su libertad irrestricta es considerada, en Occidente, condición \emph{sine qua non} de existencia del estado democrático republicano y que, por el contrario, cualquier restricción se interpreta como la demostración de la falta de democracia y libertad.

En síntesis, el campo periodístico se revela como un campo profesional productor de bienes culturales altamente privilegiados y, en buena medida, blindados al estudio científico al punto de que, recientemente, un prestigioso editor de medios gráficos llegó a sostener que ``cuando hay críticas sobre los periodistas o sobre los medios, lo que realmente está poniendo en cuestión la sociedad es el sistema democrático, ya que el periodismo es hijo de ese sistema'' (Fontevecchia, 2018: Diario \emph{Perfil} Lunes 5 Noviembre, 2018).

Por tal motivo, consideramos importante el fenómeno de confrontación entre integrantes del campo, ocurrida entre los años 2009 y 2015, ya que permitió el surgimiento de discursos críticos que resquebrajaron, de alguna manera, el blindaje y pusieron de manifiesto algunos de los mecanismos de legitimación y reconocimiento de sus integrantes, en general sobreentendidos, al tiempo que evidenciaron la existencia de estrategias simbólicamente violentas de protección de su legitimidad.

Como se sostuvo precedentemente, la disputa intelectual al interior del campo periodístico ocurrió en un contexto signado por el conflicto político. Consideramos, sin embargo, que estuvo sobredeterminado por cambios estructurales de la actividad como son la irrupción del proceso de convergencia digital, los nuevos medios de comunicación y la crisis de financiamiento del sector, en simultáneo con los cambios accionarios de sus propietarios.

\subsection{1. 3. 1. Cambio tecnológico y crisis de financiamiento}

A mediados del siglo XIX en Europa y Estados Unidos y a principios del siglo XX en América del Sur, se desarrolla una prensa sostenida económicamente en proporciones variables entre la publicidad y la venta de ejemplares\footnote{En relación con el marco europeo y norteamericano, Cfr. Barbier y Bertho-Lavenir, 1999; Habermas, 1994; Borderia Ortiz, Laguna Platero y Martínez Gallego, 2015. Para el caso argentino Cfr. De Marco, 2006; Ojeda, 2010; Ojeda y Moyano, 2015).}. Durante buena parte del siglo XX, la publicidad cubrió entre el cincuenta y el ochenta por ciento de los ingresos de una publicación, tanto de diarios como de revistas, convirtiendo a la prensa, durante décadas, en un negocio lucrativo.

La incorporación de la publicidad permitió hacer del periódico un medio de comunicación masivo, el primer medio de comunicación de masas, ya que, al poder comercializarse por debajo de su costo de producción, sus editores pudieron ponerlo al alcance de vastas audiencias. El siglo XXI parece haber puesto fin a este modelo.

Wolfgang Donsbach (2014) señala que una de las dificultades que enfrenta la profesión periodística es ``el menguante público destinatario de las noticias''. El autor sostiene que en los países desarrollados la tendencia en la merma afecta principalmente a los periódicos y a los lectores más jóvenes, llegando la reducción al 50\% en los últimos diez años. La reducción en las ventas de periódicos ha venido repercutiendo progresivamente en la reducción de ingresos por publicidad.

El análisis de la inversión publicitaria --su evolución y distribución en los medios de comunicación-- permite evaluar, indirectamente, la percepción que los anunciantes y agencias de publicidad tienen de cada uno de los medios, en relación con la efectividad, penetración, recordación e impacto de sus intervenciones publicitarias. Durante el año 2017, la inversión publicitaria superó en la Argentina los 5.000 millones de dólares, que representaba aproximadamente el uno por ciento del producto bruto interno (PBI).

Los datos del \textbf{Cuadro 1} muestran con claridad el cambio de destino de la inversión publicitaria en los últimos años en Argentina. Hasta el año 2012, se puede apreciar la paridad en la inversión publicitaria entre los medios gráficos y la TV. En el año 2010, la sumatoria de medios gráficos y televisión concentraban el 86 \% de la inversión total.\footnote{A los fines de nuestro análisis conviene considerar los siguientes datos. El conglomerado Área Metropolitana de Buenos Aires (AMBA), incluye la Ciudad de Buenos Aires y los cordones del conurbano bonaerense. Los medios del AMBA suelen tener alcance nacional. Los canales de aire de la región metropolitana tienen una cobertura superior al 80\% de la población nacional, y otro tanto ocurre con los diarios y revistas producidas en el AMBA que, en general, también tienen el mismo alcance.} Esta particular distribución destacaba a la Argentina a nivel regional, a diferencia de lo que ocurre en el resto de los países del Cono Sur donde la preponderancia de la televisión era --y es-- abrumadora, representando, solo este medio, dos tercios de la inversión publicitaria total.

%\texttt{\textbf{Cuadro 1: Evolución de la distribución de la inversión publicitaria por medio entre 2010 y 2018 en Argentina}
%
%\begin{longtable}[]{@{}
%  >{\raggedright\arraybackslash}p{(\columnwidth - 18\tabcolsep) * \real{0.2064}}
%  >{\raggedright\arraybackslash}p{(\columnwidth - 18\tabcolsep) * \real{0.0881}}
%  >{\raggedright\arraybackslash}p{(\columnwidth - 18\tabcolsep) * \real{0.0882}}
%  >{\raggedright\arraybackslash}p{(\columnwidth - 18\tabcolsep) * \real{0.0881}}
%  >{\raggedright\arraybackslash}p{(\columnwidth - 18\tabcolsep) * \real{0.0882}}
%  >{\raggedright\arraybackslash}p{(\columnwidth - 18\tabcolsep) * \real{0.0881}}
%  >{\raggedright\arraybackslash}p{(\columnwidth - 18\tabcolsep) * \real{0.0882}}
%  >{\raggedright\arraybackslash}p{(\columnwidth - 18\tabcolsep) * \real{0.0881}}
%  >{\raggedright\arraybackslash}p{(\columnwidth - 18\tabcolsep) * \real{0.0882}}
%  >{\raggedright\arraybackslash}p{(\columnwidth - 18\tabcolsep) * \real{0.0882}}@{}}
%\toprule\noalign{}
%\multirow{2}{*}{\begin{minipage}[b]{\linewidth}\raggedright
%\textbf{Medio}
%\end{minipage}} & \multicolumn{9}{>{\raggedright\arraybackslash}p{(\columnwidth - 18\tabcolsep) * \real{0.7936} + 16\tabcolsep}@{}}{%
%\begin{minipage}[b]{\linewidth}\raggedright
%\textbf{Inversión publicitaria según medios, en \%}
%\end{minipage}} \\
%& \begin{minipage}[b]{\linewidth}\raggedright
%\textbf{2010}
%\end{minipage} & \begin{minipage}[b]{\linewidth}\raggedright
%\textbf{2011}
%\end{minipage} & \begin{minipage}[b]{\linewidth}\raggedright
%\textbf{2012}
%\end{minipage} & \begin{minipage}[b]{\linewidth}\raggedright
%\textbf{2013}
%\end{minipage} & \begin{minipage}[b]{\linewidth}\raggedright
%\textbf{2014}
%\end{minipage} & \begin{minipage}[b]{\linewidth}\raggedright
%\textbf{2015}
%\end{minipage} & \begin{minipage}[b]{\linewidth}\raggedright
%\textbf{2016}
%\end{minipage} & \begin{minipage}[b]{\linewidth}\raggedright
%\begin{quote}
%\textbf{2017}
%\end{quote}
%\end{minipage} & \begin{minipage}[b]{\linewidth}\raggedright
%\begin{quote}
%\textbf{2018}
%\end{quote}
%\end{minipage} \\
%\midrule\noalign{}
%\endhead
%\bottomrule\noalign{}
%\endlastfoot
%\textbf{Televisión abierta y cable} & \textbf{47,7} & \textbf{44,4} & \textbf{45} & \textbf{46,9} & \textbf{39,5} & \textbf{38,6} & \textbf{39} & \textbf{41} & \textbf{41} \\
%\textbf{Radios en AMBA e interior} & \textbf{3,2} & \textbf{3,2} & \textbf{3,7} & \textbf{4,0} & \textbf{4,6} & \textbf{5} & \textbf{8} & \textbf{8} & \textbf{8} \\
%\textbf{Diarios en AMBA e interior} & \textbf{34} & \textbf{35} & \textbf{34} & \textbf{31} & \textbf{25} & \textbf{22} & \textbf{21} & \textbf{16} & \textbf{15} \\
%\textbf{Revistas} & \textbf{4,7} & \textbf{4,8} & \textbf{4,6} & \textbf{4,2} & \textbf{3,1} & \textbf{3} & \textbf{3} & \textbf{2} & \textbf{2} \\
%\textbf{Vía pública} & \textbf{5,5} & \textbf{5,2} & \textbf{4,3} & \textbf{4,3} & \textbf{4.5} & \textbf{5} & \textbf{5} & \textbf{5} & \textbf{7} \\
%\textbf{Cine} & \textbf{1,3} & \textbf{1,2} & \textbf{1,3} & \textbf{1,3} & \textbf{1,2} & \textbf{1} & \textbf{1} & \textbf{1} & \textbf{1} \\
%\textbf{Internet} & \textbf{4,3} & \textbf{6,1} & \textbf{7,4} & \textbf{8,0} & \textbf{16} & \textbf{20} & \textbf{20} & \textbf{24} & \textbf{25} \\
%\end{longtable}
%
%\ul{Fuente}: Cámara Argentina de Agencias de Medios}

Mientras que en 2010 los diarios y revistas del AMBA recibían el 39\% de la inversión publicitaria y la TV el 48\%\footnote{Fuente: Asociación Argentina de Editores de Revistas.},durante la segunda década esa relación cambia drásticamente, como puede verse en el \textbf{Cuadro 1}. Durante el año 2018, el 41\% de la inversión publicitaria se dirige a la TV y apenas el 17\% a los medios gráficos, en una brecha que tiende a acrecentarse sobre la base de la estabilidad en los medios audiovisuales, el descenso de la gráfica y el vertiginoso crecimiento de la inversión en Internet.

En la actualidad, la venta de diarios en Argentina ronda los 200 millones de ejemplares por año. La evolución de la venta neta total de diarios, en el año 2018, ha registrado un nuevo retroceso (-10\%)\footnote{Fuente: Elaboración propia basada en datos del IVC.}, el cual estuvo acompañado por una merma de los lectores de diarios (-7\%) en 2017 con respecto al año anterior.

\textbf{Cuadro 2: Circulación neta promedio de los principales diarios del AMBA (en miles)}\footnote{Fuente: Elaboración propia según datos del IVC. Se consideró el promedio semanal de lunes a domingo durante el mes de mayo de cada año.}

%\begin{longtable}[]{@{}
%  >{\raggedright\arraybackslash}p{(\columnwidth - 22\tabcolsep) * \real{0.1507}}
%  >{\raggedright\arraybackslash}p{(\columnwidth - 22\tabcolsep) * \real{0.0749}}
%  >{\raggedright\arraybackslash}p{(\columnwidth - 22\tabcolsep) * \real{0.0749}}
%  >{\raggedright\arraybackslash}p{(\columnwidth - 22\tabcolsep) * \real{0.0749}}
%  >{\raggedright\arraybackslash}p{(\columnwidth - 22\tabcolsep) * \real{0.0749}}
%  >{\raggedright\arraybackslash}p{(\columnwidth - 22\tabcolsep) * \real{0.0747}}
%  >{\raggedright\arraybackslash}p{(\columnwidth - 22\tabcolsep) * \real{0.0747}}
%  >{\raggedright\arraybackslash}p{(\columnwidth - 22\tabcolsep) * \real{0.0747}}
%  >{\raggedright\arraybackslash}p{(\columnwidth - 22\tabcolsep) * \real{0.0747}}
%  >{\raggedright\arraybackslash}p{(\columnwidth - 22\tabcolsep) * \real{0.0747}}
%  >{\raggedright\arraybackslash}p{(\columnwidth - 22\tabcolsep) * \real{0.0747}}
%  >{\raggedright\arraybackslash}p{(\columnwidth - 22\tabcolsep) * \real{0.1014}}@{}}
%\toprule\noalign{}
%\begin{minipage}[b]{\linewidth}\raggedright
%\textbf{Circulación diarios pagos}
%\end{minipage} & \begin{minipage}[b]{\linewidth}\raggedright
%\textbf{2010}
%\end{minipage} & \begin{minipage}[b]{\linewidth}\raggedright
%\textbf{2011}
%\end{minipage} & \begin{minipage}[b]{\linewidth}\raggedright
%\textbf{2012}
%\end{minipage} & \begin{minipage}[b]{\linewidth}\raggedright
%\textbf{2013}
%\end{minipage} & \begin{minipage}[b]{\linewidth}\raggedright
%\textbf{2014}
%\end{minipage} & \begin{minipage}[b]{\linewidth}\raggedright
%\textbf{2015}
%\end{minipage} & \begin{minipage}[b]{\linewidth}\raggedright
%\textbf{2016}
%\end{minipage} & \begin{minipage}[b]{\linewidth}\raggedright
%\textbf{2017}
%\end{minipage} & \begin{minipage}[b]{\linewidth}\raggedright
%\textbf{2018}
%\end{minipage} & \begin{minipage}[b]{\linewidth}\raggedright
%\textbf{2019}
%\end{minipage} & \begin{minipage}[b]{\linewidth}\raggedright
%\textbf{Dif.}
%
%\textbf{2010/}
%
%\textbf{2019}
%\end{minipage} \\
%\midrule\noalign{}
%\endhead
%\bottomrule\noalign{}
%\endlastfoot
%Clarín & 301 & 288 & 271 & 279 & 239 & 238 & 207 & 183 & 206 & 213 & \textbf{-33} \\
%La Nación & 150 & 160 & 162 & 168 & 164 & 156 & 130 & 122 & 117 & 100 & \textbf{-37} \\
%Diario Popular & 90 & 91 & 89 & 88 & 82 & 84 & 78 & 69 & 62 & 51 & \textbf{-48} \\
%\multicolumn{12}{@{}>{\raggedright\arraybackslash}p{(\columnwidth - 22\tabcolsep) * \real{1.0000} + 22\tabcolsep}@{}} \\
%\end{longtable}

\textbf{Cuadro 3: Ranking de circulación de diarios. Día domingo, promedio mes de mayo. Primeros 10 títulos (en miles), 2010-2019}\footnote{Elaboración propia con datos del IVC.}

%\begin{longtable}[]{@{}
%  >{\raggedright\arraybackslash}p{(\columnwidth - 24\tabcolsep) * \real{0.0888}}
%  >{\raggedright\arraybackslash}p{(\columnwidth - 24\tabcolsep) * \real{0.1097}}
%  >{\raggedright\arraybackslash}p{(\columnwidth - 24\tabcolsep) * \real{0.0727}}
%  >{\raggedright\arraybackslash}p{(\columnwidth - 24\tabcolsep) * \real{0.0727}}
%  >{\raggedright\arraybackslash}p{(\columnwidth - 24\tabcolsep) * \real{0.0727}}
%  >{\raggedright\arraybackslash}p{(\columnwidth - 24\tabcolsep) * \real{0.0728}}
%  >{\raggedright\arraybackslash}p{(\columnwidth - 24\tabcolsep) * \real{0.0727}}
%  >{\raggedright\arraybackslash}p{(\columnwidth - 24\tabcolsep) * \real{0.0727}}
%  >{\raggedright\arraybackslash}p{(\columnwidth - 24\tabcolsep) * \real{0.0728}}
%  >{\raggedright\arraybackslash}p{(\columnwidth - 24\tabcolsep) * \real{0.0727}}
%  >{\raggedright\arraybackslash}p{(\columnwidth - 24\tabcolsep) * \real{0.0727}}
%  >{\raggedright\arraybackslash}p{(\columnwidth - 24\tabcolsep) * \real{0.0728}}
%  >{\raggedright\arraybackslash}p{(\columnwidth - 24\tabcolsep) * \real{0.0741}}@{}}
%\toprule\noalign{}
%\begin{minipage}[b]{\linewidth}\raggedright
%\textbf{Diario}
%\end{minipage} & \begin{minipage}[b]{\linewidth}\raggedright
%\textbf{Localidad}
%\end{minipage} & \begin{minipage}[b]{\linewidth}\raggedright
%\textbf{2010}
%\end{minipage} & \begin{minipage}[b]{\linewidth}\raggedright
%\textbf{2011}
%\end{minipage} & \begin{minipage}[b]{\linewidth}\raggedright
%\textbf{2012}
%\end{minipage} & \begin{minipage}[b]{\linewidth}\raggedright
%\textbf{2013}
%\end{minipage} & \begin{minipage}[b]{\linewidth}\raggedright
%\textbf{2014}
%\end{minipage} & \begin{minipage}[b]{\linewidth}\raggedright
%\textbf{2015}
%\end{minipage} & \begin{minipage}[b]{\linewidth}\raggedright
%\textbf{2016}
%\end{minipage} & \begin{minipage}[b]{\linewidth}\raggedright
%\textbf{2017}
%\end{minipage} & \begin{minipage}[b]{\linewidth}\raggedright
%\textbf{2018}
%\end{minipage} & \begin{minipage}[b]{\linewidth}\raggedright
%\textbf{2019}
%\end{minipage} & \begin{minipage}[b]{\linewidth}\raggedright
%\textbf{Dif.}
%
%\textbf{2010/2019}
%\end{minipage} \\
%\midrule\noalign{}
%\endhead
%\bottomrule\noalign{}
%\endlastfoot
%Clarín & AMBA & 664 & 591 & 599 & 570 & 513 & 508 & 487 & 446. & 388 & 396 & -40 \\
%La Nación & AMBA & 280 & 332 & 360 & 331 & 323. & 320 & 302 & 288 & 209 & 201 & -28 \\
%Diario Popular & AMBA & 147 & 131 & 135 & 131 & 120 & 118 & 106 & 95 & 85 & 73 & -50 \\
%La Voz del Int. & Córdoba & 99 & 90 & 90 & 90 & 84 & 81 & 73 & 66 & 59 & 53 & -46 \\
%Los Andes & Mendoza & 83 & 78 & 77 & 73 & 68 & 65 & 57 & 52 & 49 & 43 & -48 \\
%La Capital & Rosario & 82 & 80 & 76 & 67 & 62 & 56 & 47 & 41 & 36 & 30 & -63 \\
%La Gaceta & Tucumán & 61 & 62 & 60 & 56 & 56 & 55 & 49 & 45 & 40 & 34 & -44 \\
%Río Negro & Gral. Roca & 45 & 44 & 42 & 36 & 35 & 35 & 32 & 28 & 26 & 21 & -53 \\
%El Día & La Plata & 40 & 41 & 39 & 35 & 35 & 33 & 31 & 26 & 23 & 19 & -52 \\
%Diario Uno & Entre Ríos & 36 & 38 & 22 & 34 & 30 & 28 & 26 & 26 & 25 & 11 & -69 \\
%\end{longtable}

Con respecto a la circulación de diarios pagos, los títulos auditados registraron en general bajas en su circulación. En los últimos 10 años, \emph{Clarín} redujo sus ventas un 33\%, \emph{Diario Popular} el 48\% y \emph{La Nación} el 37\% (ver \textbf{Cuadro 2}).

Como se puso de manifiesto, la reducción en las ventas de ejemplares y la consecuente caída en la recaudación por publicidad colocó a la actividad periodística en una situación de quebranto, es decir, la producción de información de actualidad ya no logra sustentarse por sus propios medios, como lo hizo hasta hace unos años. Sin embargo, el problema del financiamiento de la actividad periodística, ante los cambios tecnológicos y de hábitos de consumo de información, ha sido escasamente tratado en los estudios sobre la prensa y el periodismo.

Recientemente Silvia Cagé (2016: 7, versión digital), en \emph{Salvar los medios de comunicación}, ha descrito de manera minuciosa, este fenómeno en Francia y otros países de larga tradición periodística.

\begin{quote}
Francia perdió dos diarios nacionales en 2012, \emph{France-Soir}y \emph{La Tribune}. En 2014, el grupo Nice-Matin --que cerró el año 2013 con unas pérdidas de explotación de 6 millones de euros-- quedó bajo administración concursal. El diario \emph{Libération} escapó por los pelos a esa misma situación y despidió al llegar 2015 a un tercio de sus trabajadores, mientras \emph{Le Figaro} multiplica los planes de ``bajas voluntarias'' y el diario \emph{Sud Ouest} reduce su plantilla. En Alemania se suprimieron en 2013 más de 1.000 puestos de trabajo y, entre 2008 y 2012, desaparecieron en España cerca de 200 medios de comunicación. Al otro lado del Atlántico, en Estados Unidos, la situación no es más halagüeña. La página web \emph{Newspaper} \emph{Death Watch} lamenta la desaparición de 12 diarios locales desde 2007 y cuenta otros tantos ``en vías de desaparición'', hasta el punto de que numerosos condados ya no tienen ningún diario. Aunque siguen publicándose, el \emph{Chicago Tribune}y el \emph{Los Angeles Times} fueron declarados en quiebra en 2008, un año en el que la pérdida de puestos de trabajo en los diarios norteamericanos superó los 15.000 empleos.
\end{quote}

Como señala Peter Humphreys (2008: 72) el fenómeno no es reciente.

\begin{quote}
La preocupación sobre todo por el aumento de costes, la caída de ingresos y la reducción del número de títulos, en particular en la década de 1970, llevó a una serie de países europeos a introducir subvenciones a la prensa. El apoyo indirecto al sector de prensa, generalmente indiscriminado, ha sido algo habitual. Todos los países europeos occidentales han concedido tipos de IVA preferente; la mayoría ha establecido tarifas postales y de telecomunicaciones preferentes y muchos han ofertado precios especiales para el transporte por ferrocarril. También han sido habituales las desgravaciones fiscales por inversiones. Algunos países han subvencionado la prensa escrita, en reconocimiento a sus altos costes de producción. En otros, el Estado ha subvencionado a las agencias nacionales.
\end{quote}

Cierra su trabajo con una reflexión sobre las nuevas tecnologías

\begin{quote}
Por último, puede afirmarse que la convergencia digital ha producido una apremiante necesidad de replanteamiento de las políticas intervencionistas de manera que en el futuro la financiación pública de la comunicación de ``servicio público'' dependerá cada vez más de la función, en lugar de ser específica de una determinada tecnología. Para la prensa, ya sea impresa o digital, esto podría implicar un mayor apoyo --nunca menos-- para funciones periodísticas concretas de ``servicio público'' amenazadas por la creciente comercialización y competencia entre los distintos medios (Humphreys, 2008: 80).
\end{quote}

A pesar de la frondosa evidencia, parece existir una creencia generalizada entre periodistas e incluso investigadores, acerca de que la prensa escrita sigue siendo un buen negocio. Abordaremos posteriormente este tema ya que es, en sí, uno de los mecanismos de profesionalización y legitimación de la actividad periodística.


\subsection{1. 3. 2. El período dorado}

Durante unos 100 años (entre mediados del siglo XIX y mediados del siglo XX) los diarios y las revistas fueron los principales medios de publicidad. Esta situación convirtió a los emprendimientos editoriales en exitosos económicamente. Sin embargo, tanto la producción de diarios como de revistas periódicas requerían de una condición: ser propietarios de costosos medios de impresión, lo que implicaba disponer de cuantiosos recursos económicos.

En síntesis, los progresos tecnológicos y socioeconómicos dieron lugar al primer medio de comunicación de masas, al producir centenares de miles, e incluso millones de ejemplares por día, pasibles de ser comercializados a un precio módico, gracias al aporte económico de la publicidad. Al decir de Habermas (1994: 212), al acompañar los anuncios con información de actualidad, la prensa pasa de ser ``prensa de opinión, a medio por el que se venden lectores a los anunciantes''.

Pese a ser un buen negocio, la prensa requería un alto nivel de inversión inicial. El proceso de concentración de la prensa fue muy veloz, la exigencia de inversiones significativas estableció lo que se denomina en economía \emph{altas barreras de ingreso}.

Muchos de los títulos históricos de la prensa nacieron asociados al capital financiero o recurrieron a él para enfrentar reconversiones tecnológicas o acosos de otro tipo (Rotemberg, 1999; Muchnic, 2015; Becerra, Hernández y Postolski, 2003; Habermas, 1994).

Habermas (1994) señala que hacia 1850, la mitad de los diarios europeos se habían constituido en sociedades anónimas. Mattelart (1998) refuerza esta idea poniendo de manifiesto el vínculo entre el desarrollo de la prensa, las agencias de noticias y las empresas de publicidad. En el ya clásico trabajo de Siebert y Peterson (1956: 93), escrito en las postrimerías del macartismo, los autores señalan el fenómeno de concentración de la prensa en los Estados Unidos como una de las causas de su empobrecimiento

\begin{quote}
La prensa se convirtió en una entidad omnipotente. También se convirtió en una entidad controlada por propietarios relativamente escasos. El adelanto tecnológico hizo posible que solo unos pocos medios sirvieran a una vasta audiencia. Pero las posibilidades para alcanzar una gran audiencia eran costosas. La propiedad de los medios llegó a concentrarse en comparativamente pocas manos. Los diarios disminuyeron constantemente en número, y de la misma manera las ciudades con periódicos en competencia. Cinco editores gigantes daban razón del gran volumen en la circulación total de revistas y de la suma total gastada en publicidad de revistas.
\end{quote}

Aunque los grandes diarios suelen ocultar el origen de los fondos requeridos para su desarrollo, la bibliografía\footnote{Rotemberg (1999); Muchnik (2012); Anguita (2002); Sivak (2012).} revela que, para lograr hacer de un periódico un medio masivo, se requieren grandes inversiones y que no siempre ha estado garantizado el retorno de capitales en función del riesgo que implicaba. Por otra parte, no es raro encontrar capitales financieros que apoyan los emprendimientos (solo en Argentina, son ejemplos los diarios Clarín, Página 12 y La Opinión entre otros).\footnote{Rotemberg (1999); Muchnik (2012); Anguita (2002); Sivak (2012)..}


\subsection{1. 3. 3. Fin de la época dorada}

El objeto de estudio de este trabajo (el conflicto al interior del campo periodístico, en Argentina en el período 2009 - 2015) se inserta en lo que denominamos \emph{el fin de la época dorada del negocio periodístico}. En los últimos 5 años, los títulos de revistas han venido descendiendo a un ritmo promedio de 15 \% anual.\footnote{Datos de AAER, Asociación Argentina de Editores de Revistas e IVC} Los diarios revelan una mayor fortaleza, aunque la tendencia a compensar la reducción del ingreso por publicidad, con el incremento del precio de tapa\footnote{Los diarios del AMBA han incrementado el precio de tapa en alrededor de un 100\% en los dos últimos años.} probablemente reduzca la cantidad de compradores, alimentando el círculo vicioso de reducción de la inversión publicitaria y el quebranto económico. Solamente en los Estados Unidos, cerraron más de 160 diarios en los últimos nueve años\footnote{~Fuente: Columbia Journalism Review y \href{http://www.wan-ifra.org/}{World Association of Newspapers and News Publishers}~(WAN).}. Actualmente, la producción y circulación de contenidos de actualidad no alcanza a ser sostenida con la venta de ejemplares y de espacios publicitarios, por lo que requiere de recursos externos a la actividad propiamente dicha.

Asistimos a una transformación que distorsiona el lugar de la prensa y del periodismo en los sistemas democráticos, aunque la actividad aún detenta el capital simbólico -la credibilidad- sedimentado en los años precedentes.

La actualidad ya no parece circular por la prensa -digital o de papel- sino que se difumina, se hace evanescente. A pesar de ello, siguen siendo los diarios, las revistas, las agencias de noticias, los grandes productores de contenidos de la misma. Es decir, la producción de contenidos de actualidad sigue estando relativamente concentrada, aunque ahora mal remunerada.

En relación con el fenómeno de Internet y la convergencia, circunscribiremos el análisis, exclusivamente al fenómeno periodístico y de producción de actualidad. El período que transitamos se caracteriza por la hiperconcentración de medios, tanto de distribución como de producción de actualidad, generando la existencia de un excedente de mano de obra periodística, determinado por una menor cantidad de productores de actualidad y por la superexplotación del recurso humano abocado a la tarea periodística, cuya producción es reproducida en multiplicidad de medios y formatos. Al decir de Miguel Wiñazki (2000:266-267):

\begin{quote}
Los nuevos periodistas son ciberproletarios porque se inscriben en una lógica cibermecanicista. Los contenidos que producen se ajustan mecánicamente de acuerdo con la lógica de una línea de producción serial, con una línea de producción neo-manufacturera, que se difunde por la vía virtual y planetaria pero de manera mecánica y fabrilmente ordenada\ldots Son ciberproletarios porque en el mapa móvil de las relaciones de producción de la gran maquinaria de la comunicación global ocupan el sitio de los obreros de antaño: no son los dueños del espectáculo, sino la mano de obra operativa que lo configura.
\end{quote}

A los efectos de comprender el contexto del conflicto, intentaremos recorrer el proceso histórico de desarrollo de medios y, simultáneamente, distinguir en ese proceso la producción y distribución de información de actualidad y el desarrollo de la profesión periodística. A tal fin, proponemos la siguiente periodización, tomando como base la propuesta realizada por Habermas, para los tres primeros momentos.

1º Momento: Pocos medios gráficos producen y distribuyen información de actualidad de manera artesanal y en pequeñas cantidades para audiencias reducidas y sobre temas sociales y de mercado (prensa crematística (Habermas, 1994)).

2º Momento: Incremento de la cantidad de medios gráficos que producen y distribuyen información de actualidad de manera artesanal para audiencias ampliadas aunque aún reducidas sobre temas sociales y políticos (prensa de opinión).

3º Momento: Merma de la cantidad de medios gráficos que producen y distribuyen información de actualidad de manera industrial, aunque aumento significativo de las audiencias y la cobertura territorial. Simultáneamente se asiste al desarrollo de las agencias de noticias, que producen actualidad pero que acceden a las audiencias a través de los medios gráficos (prensa comercial, primer medio masivo de comunicación).

4º Momento: aparición de la radiofonía, incremento de medios de distribución. Inicialmente las emisoras no producen actualidad y reproducen la de los medios gráficos y de las agencias de noticias (segundo medio masivo). Este período coincide con la etapa de sistematización científica y el uso generalizado de la publicidad sistemática, las relaciones públicas (o publcity) y la propaganda de guerra (Habermas, 1994, Bernays, 2008)

5º Momento: aparición de la televisión: nuevo incremento de medios de distribución y leve incremento de productores de actualidad, en desmedro de la diversidad de medios gráficos, que tienden a concentrarse (tercer medio masivo).

6º Momento: estabilización de los tres grandes canales de distribución: gráfica, radiofonía y televisión. Concentración de medios de producción y distribución de actualidad y desarrollo de empresas multimediales y convergentes. Por otra parte, surgimiento de microespacios alternativos de producción y distribución: radios comunitarias y prensa alternativa (medios segmentados).

7º Momento: Internet, que supone una multiplicidad de fenómenos, entre ellos el surgimiento de nuevos actores.

\begin{enumerate}
\def\labelenumi{\alph{enumi}.}
\item
  Proveedores del canal de circulación, conectividad y servidores (Movistar, Fibertel, Arnet, Telefónica, Amazon, Google, etc.) con altísimos recursos económicos y tecnológicos.
\item
  Superplataformas de accesibilidad y redes sociales (Google, Facebook, Twitter, Instagram, Whatsapp.)
\item
  Posibilidad de desarrollo de múltiples productores y emisores de información de actualidad (periodistas \emph{amateur}) que compiten, en muchos casos de manera ventajosa con los periodistas tradicionales.
\item
  Consumidores de la instantaneidad
\end{enumerate}


\section{1. 4. Abordaje Metodológico}

\begin{quote}
Uno de los motivos que dio impulso a la presente tesis fue la observación de la agresividad de los discursos públicos entre periodistas, en el período 2009--2015, fenómeno inusual hasta ese momento o, eventualmente, de muy esporádica ocurrencia. También resultó singular que ese conflicto discursivo se trasladara a las producciones ensayísticas.

En tal sentido, la construcción de los datos empíricos se realizó a través del análisis de contenido en las narraciones realizadas por periodistas en la Argentina entre los años 2009 y 2015, en libros considerados en la categoría editorial de \emph{no ficción}.

Por otra parte, de manera complementaria se estudió el contexto de edición a través del análisis de notas periodísticas que relevan la presentación de las publicaciones seleccionadas y el debate que generó entre los periodistas. Por último, se realizaron entrevistas en profundidad a responsables editoriales de algunas casas editoras.

Durante el trabajo exploratorio, a partir del conocimiento que se posee sobre el campo editorial, se realizó una búsqueda exhaustiva de las publicaciones en formato libro, escritas por periodistas que abordan centralmente la reflexión o el debate sobre periodistas, el periodismo, los medios o sus empresarios. También se incluyeron algunos (pocos) autores, que Bourdieu (1997) denomina \emph{intelectuales periodistas}, y que tienen la particularidad de tener presencia en uno y otro de los campos, siendo reconocidos en ambos.

Por otra parte, no se incluyó en el universo una abundante cantidad de libros que, centrándose en actores o conflictos políticos, abordaban la cuestión periodística de manera tangencial. Tampoco se incluyeron aquellos títulos, cuyos autores tuvieron actividad periodística en el último período pero que no provienen del campo.

Inicialmente, de manera espontánea, se había dado cuenta de una docena de unidades. A partir del estudio sistemático, se desarrolló una búsqueda que incluyó un minucioso rastreo de fuentes bibliográficas, en las principales ciudades del país, como así también rastreo en las bases de datos del ISBN de la Cámara Argentina del Libro (CAL), lo que permitió obtener un \hl{universo compuesto por treinta y siete (37) libros, sin que esto signifique tener la convicción de que se ha agotado el universo ya que durante la elaboración de esta tesis se encontraron seis nuevos títulos, que permitieron llegar a la cifra indicada.}

Dado el carácter de la investigación, no se utilizó una muestra probabilística, sino una muestra intencional, mediante la inclusión en ella de los periodistas cuyos libros sobre el campo mejor representan el universo periodístico. Esta selección no se realizó a priori, por el contrario, adoptó el carácter de \emph{muestreo teórico}, en los términos que lo plantean Glaser y Strauss (1967), es decir, la muestra fue constituyéndose en el proceso de análisis, codificación y evaluación de los contenidos temáticos de cada uno de los libros, no fijándose a priori un número definido de elementos, hasta alcanzar el nivel de saturación de categorías que dé cuenta de la maduración del muestreo desplegado, en los términos de esta teoría.

A los efectos de establecer la condición de \emph{periodistas}, se adoptó el principio de autoadscripción: entendiendo por periodista a todo aquel que se considere a sí mismo como tal y lo haya expresado públicamente en reportajes, libros o notas periodísticas. A su vez, debía contar con no menos de 10 años de actividad en la profesión y reconocimiento interpares, a fin de cumplir con una de las premisas del marco teórico, es decir, ser integrantes del campo periodístico.

Se considera oportuno, en tal sentido, insertar en este espacio la distribución por año y editorial del \emph{corpus} reunido a fin de lograr el dimensionamiento del fenómeno abordado en la tesis.

Gráfico 2: Distribución de las ediciones por años y editoriales
\end{quote}


\chapter{2. Estado de la cuestión y consideraciones teórico-metodológicas}

\section{2. 1. Antecedentes sobre la temática}

La prensa ha sido durante el siglo XX uno de los fenómenos sobre los que las ciencias sociales han reflexionado de manera intensa, considerando su importancia en los sistemas democráticos y consecuentemente por su capacidad de influencia en la opinión pública y las formas que ésta adopta. Entre los temas centrales abordados por los investigadores se halla la relación entre los medios y las audiencias, habiendo quedado, en general, relegada la atención sobre el fenómeno de los periodistas, los que hacen las noticias y/o deciden sobre ellas (\emph{gatekeepers} y \emph{newsmakers}). Wolf (2004: 203) observa lo tardío e importante de dichos estudios frente a las investigaciones centradas en la relación entre \emph{mass media} y opinión pública

\begin{quote}
Los obstáculos que en cambio han limitado este ámbito han sido de distinto orden: por un lado la naturaleza administrativa de buena parte de la \emph{communication research} ha contribuido a atenuar el interés cognoscitivo del tema. En efecto, los estudios sobre los emisores han sido confinados, en general, a los niveles más bajos de las operaciones productivas de los \emph{media}. Los niveles más altos de la planificación económica y de la programación política permanecen prácticamente inexplorados: las cuestiones más amplias e importantes se plantean raras veces y ha habido poquísimos intentos sistemáticos de estudiar al emisor que ocupa una posición crucial en una red social, con la posibilidad de rechazar y de seleccionar la información en consonancia con la gama de presiones que se ejercen en un determinado sistema social.
\end{quote}

Si bien Wolf reconoce las limitaciones que han tenido los estudios sociológicos sobre los productores de actualidad, parece naturalizar las dificultades para llevar adelante estos estudios. En cierta forma, las investigaciones citadas por Wolf parecen repetir un repertorio de voces coincidentes con el discurso del periodismo dominante, cuya orientación es la de dar legitimación a la actividad periodística. (Elliott, (1972); Schlesinger, (1978a); Golding-Elliott (1971); Gans (1979); Altheide (1976).

El desarrollo no ha sido similar en todos los países, mientras que en los Estados Unidos el fenómeno viene siendo abordado de manera intensa desde la década de 1950, en Argentina los trabajos que abordan la sociología y la cultura de los profesionales de la información son relativamente recientes y escasos. Tal vez el resultado de nuestra investigación pueda echar luz sobre las causas de estas diferencias.

Recientemente, algunos autores (Ruiz, 2014; Amado \emph{et al.} 2016) han sostenido que, en general, han sido escasas las investigaciones sobre la sociología de los periodistas en Argentina, al punto de considerar que los periodistas como objeto de estudio sólo fueron incluidos en ensayos y descripciones de circunstancias. Si bien esta afirmación no refleja cabalmente la situación, es cierto que algunos de los trabajos realizados bajo el auspicio de instituciones académicas, fueron abordados desde una perspectiva más de divulgación que académica, muchas veces por estar a cargo, no de investigadores científicos sino de periodistas devenidos académicos.

Amado (2016) cita como antecedentes tres trabajos consecutivos que se propusieron un abordaje sociológico de los profesionales del periodismo: el de Fraga (1997), el de Beliz y Zuleta Puceiro (1998) y el de Majul (1999). Los tres trabajos tienen un denominador común: sus autores son o actuaron en el campo periodístico y además fueron o son consultores de empresas. Fraga, Béliz y Zuleta Puceiro son abogados y consultores de opinión pública; Majul es periodista, pero además empresario de medios.

Rosendo M. Fraga (1997) articula la opinión que los periodistas tienen de su profesión, con el de la opinión pública y la de líderes de opinión, sobre su actividad. El trabajo, basado en encuestas a periodistas, ciudadanos adultos en general, y altos funcionarios gubernamentales, jueces, diputados y senadores, indagó la opinión sobre la política, la justicia y la prensa entre los tres grupos en los que se segmentó la investigación., dando por resultado, en ese momento, una alta imagen de la prensa por sobre los demás factores de poder. Podemos afirmar que Fraga parte desde un marco teórico de \emph{agenda setting}; identifica tres grandes agendas: la política, la de los medios y la de la opinión pública y estudia la relación entre ellas, desde la perspectiva de los actores.

Por su parte, Beliz y Zuleta Puceiro (1998) analizan la cultura profesional del periodismo argentino, a través de una encuesta a periodistas, centrando la atención en la percepción que los periodistas tienen sobre la libertad para expresarse, pero avanzando sobre algunas consideraciones de la cultura del campo. Es de destacar que, entre las principales amenazas a la libertad de expresión, la categoría que concentró la mayor cantidad de respuestas fue ``el proceso de concentración de medios de comunicación'', seguido de la ``dependencia respecto a la publicidad de empresas privadas''. Ambas respuestas fueron señaladas por más del 80 \% de los encuestados.

Por último, en 1999, Majul lleva adelante una encuesta a periodistas junto a entrevistas en profundidad realizadas a algunos de los más reconocidos, intentando validar con datos cualitativos los resultados cuantitativos. El trabajo, de características heterogéneas, no llega a lograr el objetivo de validación, aunque significa un aporte sobre ciertos aspectos de la cultura e identidad del campo periodístico.

Es de destacarse la particularidad del momento en que estos tres trabajos se producen, que encuentra a la prensa y a los periodistas en una situación de unidad frente a las críticas del poder político (gobierno de Carlos Saúl Menem). El proceso de unidad del campo era de tal magnitud que unió en una misma organización, PERIODISTAS, a profesionales que históricamente se encontraban en las antípodas ideológicas y/o políticas\footnote{La organización estuvo integrada por periodistas como James Neilson, Nelson Castro, Horacio Verbitsky, Magdalena Ruiz Guiñazú y Joaquín Morales Solá, entre otros y estuvo dirigida por Mabel Moralejo.}. Otra característica de estos trabajos, auspiciados por universidades privadas y/o la Fundación Konrad Adenauer Stiftung, era el interés por conocer el estado de la libertad de prensa en la Argentina y la percepción que los periodistas, la sociedad y actores significativos tenían de ella.

Esta limitada producción sobre el colectivo periodístico parece estar fundada, entre otros motivos, en el carácter refractario a su indagación que evidencian los periodistas, tal como lo revelan Martini y Luchessi (2004: 13 y 14) en sus registros de campo. Estas autoras explicitan que su trabajo de investigación tuvo por objeto el estudio de las relaciones y prácticas productivas del periodismo desde ``la voz de los periodistas'' a través de una perspectiva etnográfica.

El conflicto que reseña la presente investigación incidió en la promoción de una serie de trabajos académicos contemporáneos, evidentemente conmovidos por la violencia, fortaleza, y lo prolongado del conflicto entre periodistas. \hl{En 2012, Baldoni} percibe la disputa entre el ``periodismo independiente'' y el ``periodismo militante'' y se propone trazar los principales rasgos del conflicto, observando que son las premisas deontológicas las bases argumentativas para la defensa de uno u otro de los tipos de periodismo. De esta manera, los conceptos de \emph{periodismo militante} versus \emph{periodismo independiente} reciben en este trabajo, su bautismo académico.

Unos meses más tarde\hl{, Arrueta (2012)} aborda también este conflicto, poniéndole nombre a las formas de periodismo enfrentadas --\emph{periodismo corporativo} y \emph{periodismo militante}- y rastrea su historia conceptual en el país. Más adelante, continúa con esta búsqueda y aborda los aspectos identitarios del campo periodístico en el proceso de relación con las empresas de medios.

\hl{En 2013, Stefoni} publica un trabajo que pretende vincular el conflicto entre periodistas alrededor de los conceptos de \emph{militantes} y \emph{profesionales}, a partir de los principios deontológicos. A pesar del esfuerzo, el trabajo revela que los periodistas independientes son críticos del concepto de \emph{militante}, pero que los que se oponen a la idea de independencia, en tanto que consideran que todo periodismo es político, también son reacios a catalogar su lugar como el de \emph{periodistas militantes}.

\begin{quote}
Si bien la referencia a un periodismo independiente se presenta como una idea polémica, la voluntad del periodista es un valor que puede ser sostenido en ambas gramáticas. La incomodidad con el término militante que muestran quienes asumen un compromiso político da elementos para pensar en este sentido. A la inversa, quienes rechazan el compromiso político asumen que el periodismo tiene una función social trascendente para el orden democrático, esto es, contraponerse a los gobiernos y ejercer la crítica, para lo que es necesario también ganar en autonomía (Stefoni, 2013:415).
\end{quote}

Por último, \hl{Ure y Schwarz (2014)} proponen un estudio cualitativo de ``las identidades del periodismo argentino y conceptualizan la existencia de dos ``modelos identitarios''. Por un lado, el del periodismo militante y por el otro, de un periodismo profesional, aunque ``débil'', para marcar las diferencias con el periodismo de países que consideran institucionalmente más afianzados.

Los últimos trabajos señalados muestran el interés por descubrir, en el conflicto, la constitución de identidades profesionales, que no parecen consolidarse de manera externa al campo periodístico.

\textbf{2. 2. Enfoque Conceptual}

Como señalaron desde hace tiempo distintos autores que abordaron la problemática, los medios masivos de comunicación se convirtieron en un actor significativo en el campo político y cultural (Borrat, 1989). A pesar de que algunos autores destacan su carácter de meros mediadores --probablemente como reacción a los primeros trabajos encuadrados en lo que Wolf (2004) denominó \emph{teoría hipodérmica}--desde mediados del siglo XX distintos autores dejaron claro el inmenso poder que los medios poseen en la conformación del sentido común, los modos de comprender la realidad y, consecuentemente, en la toma de decisiones de los integrantes de la sociedad en cuyo seno intervienen (Siebert y Peterson, 1967; Borrat, 1994; D`adamo, 2007).

En esta tesis se pretende avanzar en el conocimiento de los procedimientos por medio de los cuales el campo periodístico procura legitimar la información de actualidad como verdadera, real u objetiva. En este sentido, el trabajo se inscribe en la perspectiva de la teoría crítica de la \emph{sociología de los emisores} (Wolf, 2004), no en tanto estudio de los procesos productivos de las noticias (``gatekeepers'' o ``newsmakers''), sino en la comprensión de algunos mecanismos por los cuales el campo periodístico transfiere legitimidad y verosimilitud a las realizaciones de sus integrantes. El conflicto al interior del campo promovió el desarrollo de un metaperiodismo o \emph{periodismo de periodistas,} plasmado en una abundante cantidad de artículos, programas televisivos, mesas de debate y libros. Especialmente en estos últimos se volcaron argumentos que los periodistas no suelen expresar en los medios.

Cuando iniciamos la investigación no éramos conscientes de que abordábamos una tarea poco explorada, como es el fenómeno del metaperiodismo. Descubrimos, a poco de andar, que no abundaban los trabajos que se centraran en el discurso de los periodistas sobre otros periodistas, su quehacer y los medios. Siguiendo a Meneses (2008; 2010), entendemos el metaperiodismo como aquella parte del discurso informativo con la que los periodistas y medios de comunicación se refieren a sí mismos, a los profesionales y circunstancias del campo periodístico, aportando su visión sobre la dinámica del sector y desentrañando las relaciones tejidas entre los periodistas, los medios y los demás actores sociales, y los enfrentamientos entre las facciones internas del periodismo.

\begin{quote}
El metaperiodismo introduce al investigador en la parte del discurso periodístico referida a los medios informativos, al periodismo y a sus profesionales. La prensa española revela la importancia creciente y la evolución que desde la Transición democrática han notado las autorreferencias al sistema informativo. Los periodistas escriben sobre aspectos diversos de su trabajo, creando una imagen propia; también aluden al sector Comunicación, a sus relaciones con los poderes establecidos y con los emergentes, así como a la percepción que la sociedad tiene de los periodistas y los medios. Son unos contenidos que dibujan una faceta de la Prensa y del periodismo extremadamente rica en sus matices, notoriedad y continuidad en el tiempo (Meneses, 2010: 1).
\end{quote}

Gil Bellota (2012) sostiene:

\begin{quote}
El metaperiodismo es lo que hacen los periodistas cuando escriben sobre su propio trabajo. Actualmente es un género de moda. En el metaperiodismo español abundan los lamentos (...). Los metaperiodistas anglosajones no paran de hablar de \emph{paywalls} y cómo \emph{monetizar} contenidos en un mundo plagado de cacharrillos electrónicos y wifis.
\end{quote}

En Argentina, las producciones bibliográficas fueron el modo privilegiado que los integrantes del campo periodístico encontraron para dirimir sus posiciones y legitimar su reconocimiento.

Retrocediendo en la historia, en la primera década del siglo XX, Max Weber (1992) propuso un programa de investigación sociológica de la prensa, a la luz de la creciente importancia y prestigio de la profesión. El autor reveló, así, su original interés por este grupo profesional, que comenzaba a exhibirse públicamente como cohesionado.

A mediados del siglo pasado (1962), Habermas (1994) retomó el tema de los periodistas en una obra que intenta una indagación de la opinión pública y la prensa, revelando el proceso de cambio estructural del lugar de los periodistas desde la prensa de opinión a la prensa comercial. Dado que en ese trabajo su objeto de interés es el proceso de transformación de lo que denomina la \emph{esfera pública}, brinda elementos de análisis para comprender cómo el desarrollo de la prensa desde sus orígenes va logrando un lugar de autonomía de los poderes sociales y políticos en pugna. La prensa, progresivamente, va ocupando el lugar de narrador de la realidad social, como esfera pública de la ``publicidad burguesa''. El autor da cuenta de la \emph{pérdida de la inocencia} por parte de los medios, en el sentido de que toman conciencia de su poder e influencia.

\subsection{2. 2. 1. El campo periodístico}

Bourdieu ha descrito, en múltiples trabajos, la lógica de funcionamiento de los campos de producción simbólica (campo intelectual, artístico, religioso, de la alta costura, científico, etc.). Sin embargo, cuando --tardíamente-- aborda la cuestión del campo periodístico (Bourdieu, 1997), lo hace en un trabajo que no tiene el carácter de una indagación en profundidad, sino de un pequeño ensayo, publicado inicialmente en \emph{Actes de la recherche en sciences sociales}. Más adelante, el autor considera oportuno incorporarlo al libro \emph{Sobre la Televisión} con un ensayo denominado \emph{La influencia del periodismo}. Allí, esboza algunas ideas generales sobre el campo periodístico, entre las que señala dos principios de legitimación de sus integrantes: el de los propios periodistas, que se otorga a aquellos que representan más cabalmente los valores y principio del campo; y el de la audiencia, materializado en el mayor número de lectores, televidentes o entradas a sus notas o sitios. También señala que le preocupa la influencia del periodismo sobre otros campos como el de la cultura, lo que considera pernicioso ya que violenta la autonomía de los mismos.

Como muchos otros autores que abordan aspectos críticos del periodismo, Bourdieu considera a este fenómeno relativamente reciente, sin percibir que la interacción del periodismo con los demás campos simbólicos le es inherente, desde que el periodismo los aborda, ya que por discurrir los mensajes de los demás campos por sus espacios (textuales, radiales, televisivos y redes), el periodismo se convierte en actor significativo de los campos donde focaliza su accionar.

Nos apoyaremos en Bourdieu para el estudio de los campos, pero desde una perspectiva crítica en el caso del campo periodístico y aún de los demás, ya que en su análisis de los campos, entendemos que fue víctima de su propia ``creencia colectiva'' y no incorporó el vector de la ``opinión pública'' en el polígono de fuerzas de cada uno de los campos. En síntesis, Bourdieu descubre, tal vez demasiado tarde, el factor del periodismo operando en campos específicos del quehacer cultural. A los efectos de ordenar el análisis del campo periodístico, comenzaremos desde la teoría general que propone el autor.

En \emph{Algunas propiedades de los campos}, Bourdieu (2008: 112) sostiene:

\begin{quote}
Los campos se presentan a la aprehensión sincrónica como espacios estructurados de posiciones (o de puestos) cuyas propiedades dependen de su posición en estos espacios y que pueden ser analizadas independientemente de las características de sus ocupantes (que en parte están determinadas por las posiciones).
\end{quote}

Para luego afirmar: ``hay leyes generales de los campos'' (113).

\begin{quote}
Cada vez que se estudia un nuevo campo, ya sea el de la filología del siglo XIX, el de la moda de nuestros días o el de la religión en la Edad Media, se descubren propiedades específicas, propias de un campo en particular, al tiempo que se hace progresar el conocimiento de los mecanismos universales de los campos que se especifican en función de variables secundarias. (114)
\end{quote}

En tal sentido, creemos que el estudio del campo periodístico, al ser transversal al resto de los campos, brinda posibilidades extraordinarias de descubrir fenómenos que coadyuven al conocimiento de los otros campos y de la sociedad toda.

\begin{quote}
Un campo -podría tratarse del campo científico- se define, entre otras cosas, definiendo aquello que está en juego y los intereses específicos que son irreductibles a los objetos en juego, y a los intereses propios de otros campos o a sus intereses propios (no será posible atraer a un filósofo con lo que es motivo de disputa entre geógrafos) y que no son percibidos por nadie que no haya sido construido para entrar en el campo (cada categoría de intereses implica indiferencia hacia otros intereses), otras inversiones que serán percibidos como absurdos, irracionales, o sublimes y desinteresados (Bourdieu, 2008: 135).
\end{quote}


\subsection{2. 2. 2. La credibilidad}

La primera cuestión que abordaremos es: ¿qué está en juego? O bien: ¿qué intereses específicos están en juego?

Así como Max Weber caracteriza a la Iglesia como aquella institución que detenta el monopolio de la manipulación de los bienes de salvación, podemos afirmar que el campo periodístico detenta el monopolio de la producción de actualidad. En tal sentido, sostenemos, de manera hipotética, que el campo periodístico disputa por el capital de la credibilidad, sostenida sobre tres ejes: independencia, libertad de expresión y establecimiento de lo que debe ser considerado significativo (la agenda de actualidad).

\textbf{\hl{La credibilidad}}

\textbf{\hl{Independencia Libertad de expresión Agenda de actualidad}}

\hl{Manuel Castells (1998: 48) apunta:}

\begin{quote}
(\ldots) sin credibilidad, las noticias carecen de valor, ya sea en términos de dinero o de poder\ldots La credibilidad requiere una distancia relativa frente a las opciones políticas. Esta autonomía de los medios, arraigada en sus intereses comerciales, también encaja bien con la ideología de la profesión y con la legitimidad y la dignidad de los periodistas. Ellos informan, no toman partido (...). El distanciamiento es la regla.
\end{quote}

Es probable que el campo periodístico haya construido históricamente, de manera no interesada, su principio de credibilidad. El fenómeno se desarrolló de manera compleja, contradictoria, pero una vez establecido el principio colectivo de creencia, por parte de la opinión pública, este se convirtió en el principal capital simbólico a ser protegido y disputado.

Habermas (1994: 211) muestra la forma en que la prensa convierte su interés por la libertad de expresión en los intereses colectivos de sus audiencias

\begin{quote}
Mientras la existencia misma de una prensa políticamente raciocinante es precaria, se ve ésta forzada a la autotematización continua: hasta la legalización permanente de la publicidad políticamente activa, la aparición y el mantenimiento de un periódico político equivalía al compromiso activo con la lucha por conseguir un ámbito de libertad para la opinión pública, con la lucha por la publicidad como principio.
\end{quote}

Habermas (1994) sostiene que, con la consolidación del Estado burgués y la legalización de una publicidad políticamente activa, la prensa se desprende de la carga de opinión (prensa política) y puede atender sus intereses económicos como cualquier empresa. Sin embargo, el autor no da cuenta de que la lucha por esa legalidad puso a la prensa y a sus actores (periodistas, editores y medios) como protagonistas de esa consolidación y consecuentemente de su pervivencia, constituyéndose en un actor estructural del Estado burgués. Es decir, la prensa hace bastante más que atender a sus intereses económicos, se inserta dentro de la estructura social y de relaciones de poder, de manera orgánica, como la voz de lo que ocurre, estableciendo su primacía en el discurso colectivo de lo público, porque proviene de él y porque el resto de los actores sociales dependen de ella para hacer circular sus mensajes.

Otro tanto ocurre con el colectivo de periodistas y redactores. El proceso de consolidación del campo proviene justamente de ``la fase en que la publicidad se impone como publicidad políticamente activa''. En ella, las redacciones de las empresas periodísticas conservan ``el tipo de libertad que caracteriza a la comunicación de las personas privadas reunidas en calidad de público'' (Habermas 1962: 210).

Más allá de cierta idealización que el autor desarrolla sobre el período, es necesario señalar que es justamente este proceso el que va generando el prestigio y valor de la tarea periodística. Esta se estructura como tal desprendiéndose de manera progresiva de su función política y consolidando su autonomía frente a otros campos culturales y de poder.\footnote{Recientemente un actor significativo de la prensa argentina, Jorge Fontevecchia, en una disertación organizada por el Sistema Federal de Medios y la Embajada Británica, sostuvo: "Quienes critican a los medios están criticando al sistema democrático" (05/11/2018). Acceder desde:

  https://www.perfil.com/noticias/medios/fontevecchia-quienes-critican-a-los-medios-estan-criticando-al-sitema-democratico.phtml}

\hl{Definiremos la \emph{credibilidad} para la prensa, como la capacidad de lograr que lo que se informa en un mensaje periodístico llegue a ser significativo, es decir, la tarea de los medios de información no solo es convertir un hecho en verdadero, sino de crear un mundo verosímil en donde lo que no se ha publicado/informado es no significativo. Un mundo verosímil es aquel donde sea significativo lo que se publique.}

Cuando un medio o un periodista alcanzan altos niveles de credibilidad logran que lo verosímil se convierta en verdadero. Es decir, accedieron a la condición de creadores de realidad, creadores de actualidad.

Construir y mantener la credibilidad para un periodista o un medio requiere de condiciones no solo léxicas, culturales y de empatía, sino de condiciones morales y éticas que impidan poner en duda sus intereses, que a los fines de su audiencia solo pueden ser los intereses de esta última. Por todo lo señalado, consideramos que la condición primera de credibilidad es la \emph{\textbf{independencia}}.

En los textos de nuestro \emph{corpus}, se hace evidente la lucha por la deslegitimación del adversario a través de una serie de mecanismos que tienen la pretensión de revelar intereses externos al campo.

\begin{enumerate}
\def\labelenumi{\alph{enumi})}
\item
  El más espurio de esos intereses es la mercantilización de su actividad, es decir revelar que su información/opinión está subordinada a su propio interés económico.
\item
  En segundo lugar, se encuentra la existencia de algún interés político, es decir lo que en el medio suelen denominarse realizar ``operaciones'', que consiste en desacreditar a determinado actor político o, por el contrario, destacar la actividad del mismo positivamente, en función de intereses políticos del periodista o del medio.
\item
  Una última forma de desprestigio consiste en presentar al oponente como carente de los valores del campo y acciones violatorias del \emph{habitus}, o deontología profesional, fundamentalmente la falta de convicción en la defensa de los intereses públicos del campo --la independencia y la libertad de expresión-- lo que revelaría la existencia de intereses desconocidos, pero claramente fuera de los de su profesión y, consecuentemente, de su audiencia.
\end{enumerate}

Bourdieu (1990: 140) establece entre las propiedades de los campos de producciones simbólicas el desinterés material de sus participantes.

\begin{quote}
Debo insistir una vez más en el hecho de que el principio de las estrategias filosóficas (o literarias, etc.) no es el cálculo cínico, la búsqueda consciente de la maximización del beneficio específico, sino una relación inconsciente entre un habitus y un campo.
\end{quote}

Reflexionando sobre el concepto de \emph{habitus}, sostiene que, justamente, el concepto tiene la pretensión de dar fundamento a una ``ciencia de las prácticas''.

\begin{quote}
Cuando las personas no tienen más que dejar actuar a su habitus para obedecer la necesidad inmanente del campo y satisfacer las exigencias en él inscritas (lo que constituye en todo campo la definición misma de la excelencia), no tienen, en absoluto, conciencia de sacrificarse a un deber y mucho menos de buscar la maximización del beneficio (específico). Disfrutan así del beneficio suplementario de verse y ser vistos como perfectamente desinteresados (Bourdieu, 1990: 118).
\end{quote}

La segunda condición es presentarse, siempre y en todo momento, como defensores de la \emph{libertad de expresión}. No es necesario que la prensa o la libertad de expresión estén en riesgo. La prédica de la libertad se halla presente aun cuando en términos comparativos la situación de tal derecho se halle en mejores condiciones que en momentos pasados. Es el campo el que define cuándo y por qué la libertad de expresión está en riesgo y es la audiencia la que le otorga ese privilegio.

Por último, la \emph{\textbf{agenda}} determina el perfil, la cantidad y calidad de la audiencia. La agenda, a diferencia de la credibilidad (dada por los periodistas y redactores), no está signada por los sujetos periodistas sino fundamentalmente por el medio. Definimos a la agenda de actualidad como el recorte de acontecimientos temporalmente próximos o sincrónicos que el medio establece como prioritarios para sus audiencias.

Sostiene Verón (1987: III) en \emph{Construir el acontecimiento} que:

\begin{quote}
(\ldots) ese objeto cultural que llamamos actualidad -- tal como se nos presenta, por ejemplo, el noticiero de un canal de televisión en un día cualquiera- tiene el mismo estatus que un automóvil: es un producto, un objeto fabricado que sale de esa fábrica que es un medio informativo\ldots{} Los distintos modelos de actualidad están construidos para distintas audiencias. Como los distintos modelos de automóviles están concebidos para distintas clientelas.
\end{quote}

Sin embargo, en tanto Verón se ubica desde la perspectiva del emisor; no logra percibir que el resultado de la producción de actualidad no es asimilable estrictamente al de un automóvil.

Según Verón (1987: IV)

\begin{quote}
La actualidad como realidad social en devenir existe en y por los medios informativos. Esto quiere decir que los hechos que componen esta realidad social no existen, en tanto tales (en tanto hechos sociales) antes de que los medios los construyan.
\end{quote}

Si, como afirma Verón, los hechos sociales solo existen porque los medios los han construido, su jerarquización también es una construcción. La importancia de un acontecimiento no está dada por éste ontológicamente, sino por la jerarquización que el medio le asigne.

Borrat (1989) sostiene que la inclusión, exclusión y jerarquización de los relatos informativos se corresponden con las estrategias del periódico. Esa estrategia está definida por el editor, que suele ser un periodista de larga tradición y reputación interpares y, simultáneamente, el ejecutor de las políticas del medio.

\begin{quote}
Puesto que el periódico produce su actualidad periodística según su propia cadena de decisiones y acciones, con sus recursos y en función de sus señas de identidad, bien puede afirmarse que cada periódico produce una actualidad periodística que le es propia, característica, autónoma e irrepetible (Borrat, 1989: 111).
\end{quote}

Incluir, excluir y jerarquizar son los instrumentos de producción de una agenda de actualidad determinada y diferenciada por medios o cadena de medios.

La agenda es la materialización del objeto disputado, que en general ha sido presentado como el logro de la ``primicia'', la ``exclusiva'', etc.

La capacidad del establecimiento de la agenda pivotea sobre el volumen de audiencia (aumento o disminución de ventas, \emph{rating}, etc.), sobre el impacto entre los pares (repetición y agenda de otros medios) y su repercusión en la agenda del subcampo correspondiente (política, social, económica, artística, del espectáculo, etc.). Si bien el establecimiento de la agenda no es potestad del periodista, tampoco lo es de manera exclusiva del editor o del medio. Existen entre ambos --periodista y editor/medio-- procesos de mediación determinados fundamentalmente, por un lado, por la credibilidad y sagacidad del periodista y, por otro, por la capacidad de penetración del medio y las características de su audiencia.

Refiriéndose a los campos de la filosofía o la sociología, Bourdieu sostiene que un problema es legítimo, cuando los filósofos y/o sociólogos lo reconocen como tal y que por la autoridad que se les confiere tienen la posibilidad de reconocerlo como problema legítimo del campo. En tal sentido entendemos que, en el caso de los periodistas, ese problema legítimo es la \emph{\textbf{agenda}}, es el tema o evento, dentro de la infinita cantidad de acontecimientos que ocurren cotidianamente, que el medio ha seleccionado para incluir y jerarquizar de acuerdo a un plan estratégico de producción de actualidad, encubierto en el saber y habilidad de los periodistas. Sin embargo, a diferencia de los campos mencionados donde los problemas de auscultación duran décadas o siglos, con el periodismo --como es evidente-- los problemas devenidos en agenda son volátiles, cambian periódicamente. Ahora, en tiempos de Internet, lo hacen de manera permanente y sincrónica.

A su vez, vemos que la legitimidad y prestigio entre pares está (o estuvo) sesgada por el tiempo de pertenencia de los periodistas al campo y por su capacidad de vinculación, es decir por la cantidad y calidad de relaciones sociales establecidas dentro del campo. En términos generales, se observa que los periodistas de mayor edad, que se encuentran activos en la profesión suelen portar el carácter de \emph{dominantes}, mientras que los jóvenes o recién llegados ostentan (u ostentaban) la calidad de \emph{pretendientes}, como en casi todos los campos. Sin embargo, entendemos que el campo periodístico presenta algunas características particulares:

\begin{enumerate}
\def\labelenumi{\alph{enumi})}
\item
  La existencia, entre los dominantes, de un discurso crítico del presente. Estos detentan una mirada romántica de la profesión en la que todo tiempo pasado fue mejor, lo que fortalece su crédito y reduce el de los recién ingresados.
\end{enumerate}

Ignacio Ramonet, en \emph{La explosión del periodismo} (2011: 17), citando a dos periodistas afirma:

\begin{quote}
Philippe Cohen y Elizabeth Levi (2008) nos recuerdan que, no hace mucho, ``los periodistas gozaban del privilegio, pero también de la responsabilidad, de formar parte de aquellos que tienen voz. Lo que les fascinaba era ser el centro de las miradas. La mayoría de ellos, afortunadamente, ya no tratan de guiar a las masas. Muchos solo aspiran a formar parte de ese mundo del ``famoseo'' que encandila al pueblo.
\end{quote}

Silvia Mercado (2013: 10), en un trabajo reciente, recuerda:

\begin{quote}
El peronismo volvió al gobierno en 1989 y el periodismo vivió su época de oro, en cantidad y calidad de nuevos productos, respaldados en el salto tecnológico de la década y en su capacidad de incidir en el debate público. Fueron los años en los que Clarín se convirtió en el Grupo Clarín, capaz de hacer temblar a cualquier poder, y cada vez más actores empezaron a tallar notablemente en la formación de opinión, como la empresa Telefónica, los empresarios Eduardo Eurnekian, Daniel Hadad y Jorge Lanata ---por nombrar algunos---, que se sumaron a la creciente influencia de La Nación SA, Editorial Perfil, Editorial Atlántida y Torneos y Competencias (TyC).
\end{quote}

Existe una creencia generalizada entre los periodistas de que el periodismo en el presente ha perdido calidad, credibilidad, etc. Lo llamativo, es que ese presente se corre en el tiempo a lo largo de su historia.

Al respecto Bourdieu (1990: 137) afirma

\begin{quote}
Los que, en un estado determinado de las relaciones de fuerza, monopolizan (más o menos completamente) el capital específico, fundamento del poder o de la autoridad específica característica de un campo, se inclinan por las estrategias de conservación --las que, en los campos de producción de bienes culturales, tienden a la defensa de la ortodoxia-, mientras que los menos provistos de capital (que son también frecuentemente los recién llegados y, por tanto, generalmente los más jóvenes) se inclinan por las estrategias de subversión -- las de la herejía).
\end{quote}

Sin embargo, el discurso que adoptan los dominantes del campo periodístico actual es la reivindicación del pasado-cuando ellos eran jóvenes- y de repudio del periodismo presente, en el que los jóvenes disputan la legitimidad con nuevas herramientas tecnológicas.

\begin{enumerate}
\def\labelenumi{\alph{enumi})}
\setcounter{enumi}{1}
\item
  El prestigio y lugar dentro del campo está determinado por la importancia del medio en que se ejerce y la cartera de relaciones que se posee dentro del campo. Es posible aquí hacer distinciones de prestigio endo y exogrupo. Los medios dominantes (masivos), ya sea gráficos, radiales o televisivos, por su capacidad de establecer agendas otorgan a los periodistas alto nivel de prestigio y visibilidad. Por otro lado, algunos medios de nicho, por el cuerpo de los periodistas que reúne, por la antigüedad de su presencia como medio informativo o por el prestigio del editor suelen también otorgar capital simbólico a sus profesionales.
\item
  Por los campos de actualidad en que se insertan. Según el medio de que se trate, en general, en la cúspide se encuentran los periodistas que cubren la actividad política, nacional e internacional, en segundo término, la actividad económica, etc. Lógicamente, en el caso de diarios financieros la situación se invierte y otro tanto ocurre con la prensa temática, generándose situaciones de solapamientos donde es posible ser dominante en un subcampo (el del turf, por ejemplo), pero pretendiente dentro del campo del periodismo en general. Existe una larga lista de periodistas que, en la actualidad, tienen centralidad en el campo y comenzaron sus carreras en periodismo deportivo.\footnote{Víctor Hugo Morales, Nelson Castro, Diego Brancatelli, Mauro Viale, entre otros.}
\item
  El acceso al campo suele estar allanado por periodistas dominantes, que adquieren con ello la posibilidad de ampliar la red de relaciones dentro del campo y eventualmente de nuevas fuentes, capital fundamental de todo periodista. En general, los periodistas suelen poner de manifiesto sus vínculos y sus patrocinadores. A fin de ejemplificar, utilizaré los agradecimientos de Diego Brancatelli, en el libro \emph{Todos contra Branca contra todos} (2015: 10)
\end{enumerate}

\begin{quote}
Gracias a todos aquellos que me dieron una oportunidad en mi vida, que creyeron en mí: Héctor Agüero (el primero que me dio aire); Ezequiel Echeverría (gran maestro gran); Silvia Gottero (María Eva es su segundo nombre); Miguel Friedlander (en el momento justo); Raúl Biaggioni (siempre confió en mí); Hugo Ferrer (¿un genio o un loco?, él sabe por qué); Santiago Del Moro (en todo, un paso adelante); Daniel Hadad (me dio pocas pero sabias palabras y toda la confianza); Diego Gvirtz (por elegirme entre tantos); Daniel Vila y Liliana Parodi (por dejarme ser, con total libertad); Pablo Paladino y Gerardo Foia (me ayudaron a cumplir más de un sueño); Soledad Borinioli, Lucía Ferrari, Santiago Gambaro, Patricio Cristino (compañero de mil batallas), Santiago Carreras (por bancarme siempre), José Luis Pagano y Martín Rubio.
\end{quote}

\begin{enumerate}
\def\labelenumi{\alph{enumi})}
\setcounter{enumi}{4}
\item
  Por último, a pesar del retroceso de los medios gráficos, el origen de los periodistas de estos medios continúa otorgando relevancia al capital simbólico dentro del campo, aunque pueda pasar desapercibido ante las audiencias.
\end{enumerate}

Otra de las propiedades que señala Bourdieu es la de la implicancia de todos los integrantes de un campo con una serie de intereses fundamentales, es decir, a todo lo que ``va unido a la existencia misma del campo: de aquí deriva una complicidad objetiva que subyace a todos los antagonismos.'' (2008: 114)

Según Bourdieu (2008: 115),

\begin{quote}
Los que participan en la lucha contribuyen a la reproducción del juego contribuyendo, más o menos completamente según los campos, a producir la creencia en el valor de los objetos en juego (\emph{enjeux}). Y, de hecho, las revoluciones parciales que tienen lugar continuamente en los campos no ponen en cuestión los fundamentos mismos del juego, su axiomática fundamental, el basamento de creencias últimas en que reposa todo el juego.
\end{quote}

En tal sentido, la disputa del gobierno de Cristina Fernández de Kirchner, con los medios dominantes, luego convertida en disputa entre los actores del campo más o menos \emph{dominantes} versus los más o menos \emph{pretendientes}, puede ser entendida como una de esas revoluciones de las que habla Bourdieu, aunque de una violencia discursiva pocas veces vista en la historia del periodismo argentino. Es probable que esa modalidad pueda adjudicarse a la percepción, por parte del periodismo dominante, de que las formas que adoptaba el conflicto ponían ``en cuestión los fundamentos mismos del juego'', en un contexto de crisis general de la actividad.


\subsection{2. 2. 3. La legitimidad del campo periodístico}

Afirmamos que los periodistas además de obtener información y convertirla en texto narrativo a través de descripciones, comentarios y opiniones, llevan adelante una acción tanto o más importante, como es la de actuar públicamente a los fines de validar y hacer legítima su producción, que es en definitiva lo que da valor a sus informaciones. Sostenemos también, siguiendo a Bourdieu (1997;2008), que esta actuación pública no es individual; los periodistas actúan dentro de un campo profesional que ofrece altos niveles de protección y cobertura.

A diferencia de las producciones de los demás campos culturales cuyos productos poseen, en general, un carácter perenne, las producciones periodísticas fenecen al día siguiente, a la semana o al mes. Esto genera una tensión al interior del campo, donde la cima o el centro nunca es conquistado definitivamente. Por el contrario, se trata de un trabajo cotidiano, y al mismo tiempo, requiere de mecanismos de protección sistémicos. Cualquier ataque o la mera puesta en duda de su valor, objetividad o independencia es reprimido duramente por el campo.

Según Bourdieu, una de las fuentes de la legitimidad de los campos culturales es, justamente, la manifestación del más absoluto desinterés por todo lo que no sea aquello por lo que el campo disputa. En ningún caso, esa disputa puede ser el lucro económico, ya que ese objeto (el dinero o los bienes materiales) corresponde a los campos de la vida profana, de la vida del burgués, del comerciante. Llamativamente, a pesar de que suele considerarse a la prensa como cultura de masas, frente a las producciones de elite, esta aspira a mantener simbólicamente la nobleza de su pasado.

Por ello, si el arte, la ciencia o la actualidad fueran objeto del interés de lucro, carecerían del \emph{maná} (Mauss, 2009), del halo de sacralidad del que están investidos los bienes culturales, entre otras cosas porque sus producciones se regirían por las leyes del mercado y no del campo. Y si carecieran del valor que el campo les aporta, probablemente su precio tendería a cero. Este es uno de los motivos por lo que, en general, las producciones simbólicas se encuentran mediatizadas entre el realizador y el mercado por un facilitador (\emph{marchand}, galerista, editor, fundación, productor, empresa periodística o el Estado). De esta manera el creador, no ve ``perturbado'' su trabajo esencial, por las necesidades de subsistencia, aunque la subsistencia nunca esté garantizada y, en muchos casos, en cualquiera de estos campos, por el contrario, siempre esté en riesgo.

En el caso de la actividad periodística, el bien que provee es tan importante, tan valioso, que continuamente se ve amenazado por el mercado, para incidir, por medio del dinero o favores en su producción. Es por este motivo que una de las formas de desacreditación de la actividad periodística o de un periodista es que tiene intereses monetarios.


\subsection{2. 2. 4. Autonomización del campo periodístico}

En diversos trabajos, Bourdieu afirma que para la constitución de un campo se requiere previamente un proceso de autonomización de los integrantes, es decir, la ruptura de las cadenas que los amarraban a los mecenas y benefactores. Esto es, la existencia de un mercado de bienes culturales donde los integrantes del campo pueden ofrecer sus producciones y vivir de la realización de ellas.

En el caso del campo del periodismo, a diferencia de los de la pintura, la literatura o las ciencias, el campo se presenta en apariencia compuesto por periodistas autónomos, ``independientes''--como gustan denominarse--, pero en realidad esto es una pura ilusión. A diferencia de los artistas, los periodistas no son autónomos en el proceso de producción y circulación de información de actualidad, sino profundamente dependientes del medio donde los periodistas ``brindan su pluma'' o su información a cambio de un salario o retribución por nota escrita, por lo menos hasta hace unos pocos años.

Si bien Internet está produciendo modificaciones significativas en este aspecto, y existen algunos ejemplos de autonomía periodística, todavía no se ha consolidado la posibilidad de que periodistas autónomos sean propietarios de sus propios medios y con ello de los recursos económicos que permitan la reproducción ampliada de sus mensajes. Entre otras cuestiones porque un periodista difícilmente cumpla la expectativa de proveer un conjunto de informaciones que abarquen diversos temas.\footnote{Un caso que merece su seguimiento es el del periódico dominical \emph{El cohete a la luna}, dirigido por Horacio Verbitsky.}

Para abordar la problemática de la autonomización y comprender la actual situación en su contexto histórico utilizaremos la clasificación de etapas que Habermas (1991) propone para la prensa en \emph{Historia y crítica de la opinión pública}. Una de las virtudes de la taxonomía que propone, llamativamente, es que destaca el vínculo entre las características de la prensa y las formas de financiamiento.

Habermas establece un primer momento de la prensa que denomina \emph{crematístico} cuando la producción de hojas y/o pequeños boletines informativos tenían por objeto vender información de los mercados, del comercio y los precios de las mercancías, sobre todo en las ciudades puertos o mercados, en una sociedad que transitaba el paso del mercantilismo hacia el capitalismo pleno. El autor sostiene que estos emprendimientos se desarrollaban al estilo de las pequeñas actividades artesanales, con una rentabilidad reducida, pero con el objetivo de convertirse en una actividad lucrativa. Si bien no fija fechas --y el desarrollo de la prensa ha sido desigual en los distintos centros urbanos de Occidente--, podemos establecer que este período se inicia entre finales del siglo XVI y principio del XVII.

El autor señala un segundo momento de la prensa que denominará \emph{prensa de opinión o periodismo de escritores}. Este se desarrolla progresivamente bajo el manto de aquella prensa de información, signado fundamentalmente por los procesos de autonomización de la burguesía en los Estados monárquicos y los regímenes cuasi feudales tanto en Europa como en América. Habermas, citando a Bücher, señala acerca de este período:

\begin{quote}
Los periódicos pasaron de ser meros lugares de publicación de noticias a ser también portadores y guías de la opinión pública, medios de lucha partidista. Es revelador su énfasis en el desinterés por la rentabilidad, pues pone de manifiesto un cambio cualitativo en las formas de mantener la cohesión y promover la captura de voluntades a través de la prensa: ``Ahora pasa a un segundo plano la finalidad crematística de tales empresas; infringen, en efecto, todas las reglas de la rentabilidad y a menudo son negocios ruinosos desde el comienzo (Bücher, 1917, citado en Habermas, 1991: 210).
\end{quote}

Por último, Habermas (1991: 212) observa una tercera y última etapa:

\begin{quote}
Con la consolidación del Estado burgués de derecho y con la legalización de una publicidad políticamente activa se desprende la prensa raciocinante de la carga de la opinión; está ahora en condiciones de remover su posición polémica y atender a las expectativas de beneficio de una empresa comercial corriente.
\end{quote}

Para este período, arriesga una fecha, y afirma que la conversión en \emph{prensa negocio} se produce simultáneamente en Inglaterra, Francia y Estados Unidos, alrededor de la década del ´30 del siglo XIX. Citando nuevamente a Bücher, Habermas afirma que el periódico adopta el carácter de una empresa productora de espacios para anuncios publicitarios y que esto le devolvió al periódico el carácter de una empresa lucrativa. Más adelante, aclara que esta empresa en nada se parece a la actividad artesanal y que una serie de factores tecnológicos y sociales producirán un cambio significativo de esta segunda gran industria cultural, es decir, la producción de publicaciones periódicas, diarios, revistas y literatura popular por entregas.

Sin embargo, desde el punto de vista técnico, faltan aún algunas décadas para que la actividad periodística deje de ser artesanal y de tiradas reducidas. Recién a mediados del siglo XIX, \emph{The Times}, logra introducir la producción de periódicos en una rotativa\footnote{Existen discrepancias en la fecha de introducción de la rotativa. Según las distintas fuentes consultadas ésta entra en producción entre 1820 y 1840.} (Barbier y Bertho-Lavenir, 1999). En 1884, un nuevo hito en la historia de la impresión transformará definitivamente el negocio periodístico: el invento de la linotipia por el relojero alemán Ottmar Mergenthaler. Basado en la composición automatizada de los textos,\footnote{Una vez finalizada la composición de una línea, se fundía el molde de impresión en negativo, con plomo líquido, obteniéndose un sello de plomo para la impresión.} revolucionó tanto la capacidad de producción como los requerimientos económicos para acceder a esta industria. Por primera vez en la historia de la humanidad, después de casi trescientos años desde la invención de la imprenta de tipos móviles y su aplicación a la difusión de información, el acceso a la opinión pública requiere ingente cantidad de capital.

La prensa comercial, tal como la conocemos hoy, con la incorporación de publicidad como forma de financiar sus costos y abaratar el producto, promovió las condiciones técnicas y de desarrollo comercial para que haga su irrupción la prensa de masas, con altas tiradas y bajos precios\footnote{Ya en 1833, \emph{The New York Sun} se establece como uno de los primeros periódicos populares. ``En concreto, \emph{The} \emph{Sun} se presentó como un diario barato --de un centavo el precio del ejemplar- que para contrapesar las perdidas relativas que suponía costar menos que los competidores, buscaba una tirada grande que se convirtiera en objeto deseado por los anunciantes'' (Barrera, 2004: 89).}. Fue posible entonces satisfacer a una audiencia lectora, ávida de información, junto al desarrollo de una sociedad de consumo de productos masivos provistos por la manufactura industrial, dando lugar al nacimiento de la publicidad profesional. Si bien en su fase inicial la prensa con publicidad y de bajo costo estuvo dirigida a los sectores plebeyos de la sociedad, muy rápidamente los diarios cultos acompañaron la tendencia de incorporar publicidad y vender los ejemplares por debajo del costo de producción.

Es pertinente destacar que el inicio y el fin de las etapas o fases del desarrollo de la prensa operaron de manera asincrónica. Cada una de ellas tuvo solapamientos, hecho que determinó el carácter mixto de la prensa comercial, es decir la articulación de información comercial, social y de interés general con información política y de opinión, beneficiándose cada etapa de la construcción y preparación de las audiencias de las etapas precedentes.

\textbf{Gráfico 1:} Representación gráfica de los momentos de la prensa según la taxonomía de Jürgen Habermas (1994).

%\begin{longtable}[]{@{}
%  >{\raggedright\arraybackslash}p{(\columnwidth - 12\tabcolsep) * \real{0.1741}}
%  >{\raggedright\arraybackslash}p{(\columnwidth - 12\tabcolsep) * \real{0.1327}}
%  >{\raggedright\arraybackslash}p{(\columnwidth - 12\tabcolsep) * \real{0.1382}}
%  >{\raggedright\arraybackslash}p{(\columnwidth - 12\tabcolsep) * \real{0.1555}}
%  >{\raggedright\arraybackslash}p{(\columnwidth - 12\tabcolsep) * \real{0.1382}}
%  >{\raggedright\arraybackslash}p{(\columnwidth - 12\tabcolsep) * \real{0.1307}}
%  >{\raggedright\arraybackslash}p{(\columnwidth - 12\tabcolsep) * \real{0.1307}}@{}}
%\toprule\noalign{}
%\begin{minipage}[b]{\linewidth}\raggedright
%\end{minipage} & \begin{minipage}[b]{\linewidth}\raggedright
%Siglo XVI
%\end{minipage} & \begin{minipage}[b]{\linewidth}\raggedright
%Siglo XVII
%\end{minipage} & \begin{minipage}[b]{\linewidth}\raggedright
%Siglo XVIII
%\end{minipage} & \begin{minipage}[b]{\linewidth}\raggedright
%Siglo XIX
%\end{minipage} & \begin{minipage}[b]{\linewidth}\raggedright
%Siglo XX
%\end{minipage} & \begin{minipage}[b]{\linewidth}\raggedright
%Siglo XXI
%\end{minipage} \\
%\midrule\noalign{}
%\endhead
%\bottomrule\noalign{}
%\endlastfoot
%Crematística & & & & & & \\
%De opinión & & & & & & \\
%Comercial & & & & & & \\
%\end{longtable}

Durante el primer período, los protoperiodistas dependían de los imprenteros y casas editoriales. Con el Estatuto de la~Reina Ana de Inglaterra, promulgado el 10 de abril de 1710, comienzan los escritores a autonomizarse. Sin embargo, ello no aplicaba para los ``periodistas'' dado que, para entonces, los periódicos ya habían pasado de ser ``lugares de publicación de noticias frescas, a portadores y guías de la opinión pública, medios de lucha de la política partidista'' (Bücher, 1917, citado en Habermas, 1994: 210). Habermas (1994: 210) observa que esta transformación produjo modificaciones en la organización interna de la empresa periodística:

\begin{quote}
La inserción de una nueva instancia entre la colección de noticias y su publicación; la redacción. Pero para el editor esto significaba que pasaba de ser un vendedor de noticias frescas a un comerciante de opinión pública.La mutación propiamente dicha no se inicia evidentemente con la constitución y autonomización de una redacción; comenzó con los periódicos cultos en el continente y con los semanarios y revistas políticas en Inglaterra, en cuanto los escritores fueron sirviéndose del nuevo instrumento de la prensa periodística para dotar a su raciocinio, intencionadamente pedagógico, de eficacia publicística. Se ha llamado a esta segunda fase la fase de un ``periodismo de escritores''\ldots Frecuentemente, en Inglaterra fueron los periódicos y revistas de este estilo ``la ocupación predilecta de la aristocracia del dinero.
\end{quote}

Habermas analiza de manera minuciosa este proceso, para comprender fundamentalmente el cambio de la esfera pública y el rol del editor, la redacción y el medio en su constitución. Afirma que la relación entre los redactores y el editor no era una relación de jefe-empleado, que en muchos casos llegaba a participar de los beneficios, aunque el riesgo empresario corría por cuenta de estos jóvenes burgueses u oligarcas del dinero. Sin embargo, allí donde las empresas periodísticas se consolidaron las redacciones tendieron a ser profesionales y autónomas. ``(\ldots) lo poco que el negocio se imponía a la opinión, lo muestra claramente el mismo ejemplo de Cotta, cuya influyente Allgemeine Zeitung fue durante décadas un negocio ruinoso'' (Habermas, 1994: 211).

En un largo párrafo, Habermas desarrolla el proceso por el cual los periodistas, redactores e incluso el editor van convirtiéndose de pretendientes de la conciencia de sus audiencias, en expresión de esa conciencia.

\begin{quote}
En la fase en la que la publicidad se impone como publicidad políticamente activa, conservan también las redacciones de las empresas periodísticas editorialmente consolidadas el tipo de libertad que caracterizaba a la comunicación de las personas privadas reunidas en calidad de público. Los editores aseguraban la base comercial de la prensa, sin no obstante llegar a comercializarla como tal. La prensa, salida del raciocinio del público y constituida como mera prolongación de la discusión del mismo, sigue siendo por completo una institución de ese público; a modo de mediador y vigorizador, no ya como mero órgano de transporte de información, ni instrumento aún de la cultura de los consumidores (Habermas, 1994: 211).
\end{quote}

Habermas describe el proceso de sustitución de los salones de tertulias por el debate en la prensa política o prensa de escritores. Estos escritores, o mejor aún escribas del empresario políticamente activo, se correspondían en general a sectores plebeyos letrados de la incipiente ciudad moderna, que hallaban en las redacciones periodísticas un mecanismo de ascenso social, recursos económicos y, en muchos casos, acceso a una imprenta donde acercar sus trabajos literarios. A nuestro entender este proceso demandó un tiempo prolongado, entre principio del siglo XVIII y finales del XIX dando lugar, por un lado, al desarrollo de la autoconciencia de la autonomía periodística y su lugar estructural en el sistema republicano y, por otro, su contraparte: el desarrollo de audiencias masivas crédulas en la veracidad de las informaciones provistas por los medios.\footnote{Este fenómeno habilitó otro problema fundamental, -según Habermas- a lo largo del siglo XX, capaz de poner en crisis la composición misma de la esfera pública como espacio de convergencia y construcción de opinión pública resultante de un largo proceso de consenso. La aparición de una \emph{publicity} sistemática reconfigura la noción misma de destinatario de la comunicación periodística. Si el reclamo publicitario se dirigía ``a las personas privadas en tanto susceptibles de convertirse en consumidores, las \emph{public relations} se dirigen a la ``opinión pública'', a las personas privadas como público y no como ``consumidores''. La \emph{publicity} (relaciones públicas) se transforma en una actividad tan permanente y sistemática que la esfera pública entra en riesgo de dejar de ser un espacio de construcción de consensos (publicidad burguesa) para convertirse en un espacio de lucha de intereses, causando, en los términos metafóricos propuestos por Habermas, una suerte de ``refeudalización de la esfera pública''. (Habermas, 1994: 209 y ss.).}

\chapter{3. Descripción del campo periodístico en la Argentina, 2009 - 2015}
\section{3. 1. Los periodistas y el fenómeno editorial}

El motivo por el que un libro es editado supone una serie de consideraciones diversas, entre las que se incluyen decisiones económicas, de venta y rentabilidad de la acción editorial, aunque no solamente, ya que en muchos casos existen intereses extraeconómicos: culturales, de catálogo, de prestigio personal e incluso transeconómicos. El rédito económico puede no estar en la venta de ejemplares, es decir que la edición de un título y su presentación pueden tener objetivos que no se agotan en la venta del producto, sino que el mismo es un medio para otras realizaciones. Además, el motivo puede conjugar dos o más de estas variables. Si bien no nos detendremos en esta cuestión es importante desmitificar la acción editorial y de escritura de un libro que, como todo bien cultural, está rodeado de un halo simbólico de alta densidad.

En el caso del período analizado, la publicación de un libro por parte de uno u otro de los integrantes de los sectores en pugna, e incluso de aquellos que pretendían poner paños fríos en la disputa del campo, significaba también un hecho de relevancia noticiosa. Los libros eran, además de un objeto cultural, un instrumento para poner en la agenda de actualidad las posiciones de los contendientes.

Por otra parte, a nuestro entender, desde la aparición de \emph{678. La creación de otra realidad} (Oliván y Alabarces, 2010) los autores eran conscientes de que la presentación de sus libros constituían acontecimientos noticiables y que no solo dialogaban en sus textos con sus lectores, sino con el conjunto de producciones periodísticas precedentes e incluso futuras.

La serie que se inicia en 2009 tiene como fenómeno de contexto el conflicto entre el gobierno y sectores de la alta burguesía agraria y financiera. En tal sentido, muchas de las producciones editoriales del período considerado están determinadas por acciones donde los actores del conflicto y sus aliados intervinieron con la pretensión de incidir en la opinión pública.

La saga se inicia con un libro extremadamente provocativo: \emph{Diario de Guerra. Clarín, el gran engaño argentino}\footnote{Según la Editorial Gárgola, el libro tuvo en dos ediciones una venta apróximada a los 4.000 ejemplares}de Claudio Díaz\footnote{Claudio Díaz falleció el 6 de agosto de 2011, a los 52 años.}, publicado en septiembre de 2009. El libro es editado por una pequeña editorial que, como puede apreciarse en el gráfico 2, son las que inicialmente abordan la tarea de publicar material de crítica al campo periodístico.

Con excepción de \emph{678. La creación de otra realidad}, editado por \href{http://www.lecturalia.com/editoriales/79/paidos}{Paidós}/Planeta durante 2010 y apoyado en un éxito televisivo, serán las editoriales independientes las que tomarán a su cargo los primeros textos en el inicio del conflicto. Entre las editoriales independientes agrupamos, además de las nacionales, algunas pequeñas nacidas para la ocasión. Las editoriales más importantes, en general, se alinearon con los periodistas de los medios o multimedios de mayores audiencias.


\section{3.2. El contexto político del conflicto en la mirada de la prensa}

Durante el período establecido para la investigación, el campo periodístico mayoritario se percibió agredido por el gobierno. Probablemente la frase: ``¿Qué te pasa Clarín? ¿Estás nervioso?'',~que expresó Néstor Kirchner unos días antes de las elecciones legislativas del año 2009, reveló la intención del gobierno de confrontar con el grupo de medios más importante del país. Si bien el enfrentamiento contra este medio ya existía desde el año anterior, cuando ambos rompieron relaciones luego de que el gobierno de Cristina Kirchner presentara~la resolución 125 --que pretendía aumentar los impuestos a la actividad agropecuaria-- este conflicto no había sido verbalizado tan crudamente.

En el mes de marzo de 2009 salió al aire \emph{6 7 8} por la Televisión Pública; un programa que se convertiría en uno de los principales instrumentos de comunicación del periodismo contrahegemónico, minoritario o periférico. Este programa estuvo estructurado sobre un conjunto de periodistas fijos y algunos invitados que debatían sobre temas de actualidad que la producción ponía a discusión, apoyado con imágenes televisivas, audios radiales y/o imágenes de tapas o notas de los principales diarios. A diferencia de otros programas que lo precedieron, en términos de hacer periodismo de periodistas, no utilizaba la crítica de manera humorística\footnote{Al estilo de La Noticia Rebelde, TVR, Videomatch, etc.}.Por el contrario, los panelistas discutían con la información que se analizaba, haciendo énfasis en la carga de subjetividad y artificialidad de la construcción de actualidad. Este programa televisivo concitó una importante audiencia y se constituyó en uno de los blancos paradigmáticos de crítica del periodismo dominante y fue considerado por este como un producto periodístico no legítimo, como expresión del antiperiodismo.

En un libro, publicado en 2010, que pretendió el análisis del fenómeno, estructurado a modo de un diálogo o debate entre un académico y una periodista, quien fuera la primera conductora del programa, María Julia Oliván, afirma:

\begin{quote}
Creo que lo que 6 7 8 incorporó antes que nadie (y no digo que sea un rasgo positivo), fue el choque frontal con otros periodistas. Antes de 6 7 8, nadie se había metido con un periodista desde un programa de televisión. No existía la denuncia contra el medio de comunicación (Oliván y Alabarces, 2010: 72).
\end{quote}

La frase de María Julia Oliván pone de manifiesto el punto de fractura entre los periodistas. Si bien el programa televisivo fue paradigmático en la confrontación entre discursos periodísticos, no fue el único producto de medios que lo incorporó. Muy rápidamente otros canales reprodujeron el modelo, pero de manera reactiva, es decir, respondiendo a los réprobos y revelando mayor capacidad de penetración en la audiencia.

Martín Sivak, en \emph{Clarín, la era Magnetto} (2015: 7) afirma, en el prólogo, que el libro

\begin{quote}
(\ldots) empieza y termina con dos guerras muy distintas. Abre con la de Malvinas de 1982, vista desde las páginas del diario de la señora de Noble; \textbf{cierra con la imaginaria y prolongada que ha librado el Grupo con los Kirchner desde el conflicto agropecuario de 2008.}\footnote{El énfasis es propio.}
\end{quote}

Unos meses después, Julio Blanck, director editorial de la sección política de Clarín, en una entrevista realizada para \emph{laizquierdadiario.com}\footnote{\url{https://www.laizquierdadiario.com/Julio-Blanck-En-Clarin-hicimos-un-periodismo-de-guerra}Domingo 17 de julio de 2016~\textbar{} Edición del día}, ante la pregunta del periodista Fernando Rosso, expresa:

\begin{quote}
Tengo la obligación de hacer una pregunta sobre~Clarín. Martín Sivak, que escribió el libro (Clarín, la era Magnetto), dice que ``por primera vez durante los años Kirchneristas~Clarín~cambió su forma de hacer política e hizo un~periodismo de guerra''. Escribió en 2014 que más allá de cómo termine el kirchnerismo, iba a ser difícil que~Clarín vuelva a tener la gravitación que tuvo en la sociedad argentina. Si es así, ¿cómo es el Clarín~de posguerra, cómo es el~Clarín~poskirchnerista o cuál es tu lectura?
\end{quote}

Responde Julio Blanck

\begin{quote}
A ver: ¿hicimos periodismo de guerra? Sí. Eso es mal periodismo. Fuimos buenos haciendo guerra, estamos vivos, llegamos vivos al final, al último día. Periodismo eso no es como yo lo entiendo, no es el que me gusta hacer. Y yo lo hice, no le echo la culpa a nadie, yo lo hice. Eran las circunstancias e hice cosas que en circunstancias normales por ahí no hubiese hecho, en términos de qué posición tomar o de cierta cosa terminante. \textbf{Hubo un momento en que la guerra contra~Clarín~pasó a ser la guerra contra los periodistas} (\ldots).\footnote{El énfasis es propio.}
\end{quote}

En el citado libro de Sivak (2015: 8, versión digital), más adelante el CEO del medio, Héctor Magnetto afirma:

\begin{quote}
¿ Por qué (los Kirchner) se metieron con nosotros? Por nuestra llegada a un sector importante de la sociedad argentina. Éramos un obstáculo para un poder hegemónico y autoritario que pretendía eternizarse. En la Argentina se producen vacíos de poder que lo llena rápidamente quien controla el Estado. Los medios del Grupo Clarín, con (Carlos) Menem y con Kirchner, actuaron como un límite a los circunstanciales ocupantes del Estado. Es lo que sucede con la prensa que cumple con su tarea.
\end{quote}

Es este un magnífico discurso legitimante de la actividad de los medios. Pero no era solo Clarín o sus periodistas los que se sentían agredidos o en medio de una guerra. Hacia el fin del período seleccionado, Novaro y Birmajer (2015: 8, 11) sostienen, en \emph{Grandes y pequeñas mentiras que nos contaron}:

\begin{quote}
Fue esta una guerra contra la prensa libre, pero también al mismo tiempo una guerra cultural, por imponer una cierta idea de lo bueno y lo malo para el país por sobre todas las demás (\ldots) Aquí contaremos la historia de cómo y por qué el kirchnerismo hizo todo esto y de forma sistemática, haciéndole la guerra al periodismo en estos años. No solo a una empresa o a periodistas de una orientación particular, sino al oficio periodístico, al ejercicio libre del trabajo de informar y opinar en general.
\end{quote}

Esta última frase tal vez valga la pena destacarla porque sintetiza de manera prístina la definición que el campo se asigna: \textbf{el oficio de periodismo como el ejercicio libre del trabajo de informar y opinar en general}. Ya volveremos sobre esta frase, que representa el sentir de buena parte de los periodistas.

Edi Zunino (2013), en un libro de características autobiográficas, intenta reflexionar sobre el conflicto en \emph{Periodistas en el Barro}, reseñando hechos que ponían en tensión el campo editorial. Zunino propone ubicarse en un lugar intermedio, equidistante, casi como un observador independiente del conflicto, afirma que pudo haber sido kirchnerista y luego advierte al lector que perfectamente podría haber sido anti-kirchnerista: ``tengo parientes, amigos y conocidos en la otra vereda''. (11)

Al igual que el resto de sus colegas advierte la conflictividad y la envergadura de la misma. Sin embargo, como veremos más adelante, a diferencia de muchos de sus colegas, no excluye del campo periodístico a ninguno de los actores y es capaz de discernir que, como pocas veces en la historia de la prensa, el conflicto político ``embarró'' el campo periodístico.

\begin{quote}
Así, andando mitos, contramitos, locuras, pataletas, denuncias, tuiteos, mentirosas desmentidas, zancadillas, traiciones e insuperables rencores se fue armando, caso a caso, el siguiente compendio de historias, historietas, subtextos y micro relatos protagonizados por periodistas contra periodistas que se vieron forzados, incluso por sí mismos, a revolcarse en esta guerra de otros que bien podía narrarse desde el clásico balcón de la incordura. O no. (Zunino, 2013: 31).
\end{quote}


\section{3. 3. Acerca del campo periodístico}

A fin de definir algunos conceptos significativos, utilizados en nuestro trabajo, como \emph{periodistas centrales}, \emph{hegemónicos} o \emph{dominantes}, para referirnos a quienes poseen mayor capital simbólico dentro del campo y sus audiencias, así como a sus opuestos, es decir \emph{periodistas periféricos}, \emph{secundarios} o \emph{aspirantes} del campo, consideramos necesario establecer los criterios que utilizamos para indicar la ubicación en unos u otros en los polos del \emph{continuum} que establecimos. ¿Cómo discernir qué periodista se encuentra en la centralidad del campo y cuál en la periferia? Bourdieu utiliza el principio del \emph{valor de la firma}, en el caso de la alta costura. Es decir, que en el caso de un/a modisto/a, el valor de sus prendas determina el lugar que ocupa en el campo. De alguna manera, el sistema es relativamente sencillo, ya que los diseñadores de moda venden a sus consumidores el producto de su arte, por lo que la información está, de alguna manera, disponible en el mercado de la alta costura. Esto no quiere decir que la información sea abierta y explícita, pero suele tener trascendencia, si no es por el valor de las prendas, lo es por la ostentación económica de quienes las producen.

Por otra parte, en distintos trabajos\footnote{En Campo Intelectual, campo de poder (2002) y en Una invitación a la Sociología reflexiva (2008).}, para referirse a los conflictos dentro del campo, Bourdieu (2002) utiliza la metáfora de ``campo de fuerzas'', extraída probablemente de la física. Esta metáfora permite hacerse a la idea de un espacio en movimiento, es decir que, en el proceso de producción y reproducción de un campo determinado, los integrantes y/o sus producciones ocupan, en un momento determinado del tiempo, lugares que pueden mutar según las distintas correlaciones de fuerza. Como en la metáfora bourdieana, el cambio de lugar de uno de los integrantes significa el cambio de posición de todos los demás.\footnote{Mientras se escribe la presente tesis, ocurre en el campo periodístico una verdadera transformación y modificación del polígono de fuerzas. La primicia del medio digital \emph{El cohete a la luna}, que dirige Horacio Verbitsky sobre el \emph{affaire} Marcelo D´Alessio y la presencia de uno de los más connotados periodistas del grupo Clarín, Daniel Santoro, salpicado por este, es un magnífico ejemplo de tensión y movimiento del campo, que incidirá, probablemente en el conflicto, hoy soterrado.}

En el caso del campo periodístico puede replicarse el método de valorizar la firma. Sin embargo, hay que hacer una salvedad. El valor dinerario de la firma no es de información pública. Es más, se considera altamente reservada porque quién paga los servicios de un periodista, no es el consumidor, sino un productor, una empresa o dueños de medios que son parte del campo, por lo que la información suele permanecer oculta.

Por otra parte, salvo muy raras excepciones, los periodistas no hacen ostentación de una vida fastuosa o de lujos porque, entre otras cosas, esto no agrega valor a su firma. Por ello, contar con esta información para hacer comparaciones entre ellos es sumamente difícil, cuando no imposible.

Sin embargo, el conflicto dio oportunidad de acceder de manera indirecta a alguna información cuantitativa de los ingresos de algunos periodistas. Durante el período, se publicaron biografías de algunos de ellos. Ello hace evidente que, por lo menos tres de ellos, se encuentran en el centro del campo: Jorge Lanata, Víctor Hugo Morales y Horacio Verbitsky. Por supuesto, buena parte de los autores de los libros del período también son integrantes reconocidos del campo, ya que de manera indirecta, ser publicados por editoriales comerciales pone de manifiesto que disponen de audiencias que los reconocen y valoran: María Julia Oliván, \href{http://www.tematika.com/buscar.do?seccionDeBusqueda=En+Libros\&seccion=1\&claveDeBusqueda=porAutor\&txtencoded=Graciela+Mochkofsky\&idAutor=35432\&criterioDeOrden=2\&idSeccion=1\&texto=Graciela+Mochkofsky\&optSeleccionada=Autor\&idSeccionPropia=1}{\href{http://www.tematika.com/buscar.do?seccionDeBusqueda=En+Libros\&seccion=1\&claveDeBusqueda=porAutor\&txtencoded=Graciela+Mochkofsky\&idAutor=35432\&criterioDeOrden=2\&idSeccion=1\&texto=Graciela+Mochkofsky\&optSeleccionada=Autor\&idSeccionPropia=1}{Graciela Mochkofsky}}, Pablo Sirvén, Edi Zunino, Luis Majul, \href{http://www.edhasa.com.ar/busqueda-avanzada.php?autor=Daniel+Muchnik\&avanzada=1}{Daniel Muchnik}, Eduardo Blaustein, Gabriel \href{http://www.edicionesb-argentina.com/autor/levinas-daniel/}{Levinas}, Darío \href{http://www.cuspide.com/resultados.aspx?c=VILLARRUEL+DARIO\&por=AutorEstricto\&aut=291343\&orden=fecha}{Villarruel}, ~Jorge \href{http://www.cuspide.com/resultados.aspx?c=FONTEVECCHIA+JORGE\&por=AutorEstricto\&aut=264707\&orden=fecha}{Fontevecchia}, entre otros. Afirmamos también lo son, los periodistas que firman columnas de opinión de los diarios y revistas y los que ofician de conductores de programas de actualidad.

En términos generales, lo que ocurre es un \emph{continuum} entre la mayor centralidad y la periferia. Es por ello que no resulta sencillo distinguir a su contraparte, es decir, a los periodistas periféricos. No obstante, al establecer algunos parámetros para identificar a los periodistas centrales, podemos utilizar esos mismos criterios para reconocer a los que carecen de esas propiedades. A continuación, graficaremos los criterios utilizados para establecer niveles de dominancia/centrales y de periferia/aspirantes.


\subsection{3. 3. 1. Propuesta de clasificación de periodistas según su lugar en el campo}

A los efectos de elaborar una clasificación proponemos, de manera esquemática, cuatro posiciones polares posibles en el campo:

\begin{enumerate}
\def\labelenumi{\arabic{enumi})}
\item
  Hegemónico en el campo, en un medio dominante por audiencia
\item
  Hegemónico en el campo, en un medio alternativo por audiencia
\item
  Periférico en el campo, en un medio dominante por audiencia
\item
  Periférico en el campo, en un medio alternativo por audiencia
\end{enumerate}

Sin embargo, no debemos perder de vista que lo que existe en el campo es una gradualidad entre el centro y la periferia, que está determinada por la posición de los actores y los cambios que operan en los desarrollos profesionales de cada uno de los periodistas. Más clara es la posición de los medios de comunicación que, en general, definen su posición de dominante/alternativo, por la distancia a los factores de poder económicos nacionales e internacionales y por el volumen y características de la audiencia.

A modo de ejemplo, a los efectos de representar esquemáticamente estas nociones, y por el solo hecho de la centralidad que posee, el periodista Jorge Lanata fue hegemónico en el campo en un medio periférico, mientras dirigía el diario \emph{Página/12} y luego, en otro momento de su historia profesional, fue hegemónico en un medio dominante, cuando fue/es el periodista estrella de Clarín, Canal 13 y radio Mitre.

\textbf{Cuadro 5 Condiciones mínimas que reúnen los integrantes del campo de acuerdo a sus posiciones en el polígono de fuerzas}

%\begin{longtable}[]{@{}
%  >{\raggedright\arraybackslash}p{(\columnwidth - 16\tabcolsep) * \real{0.1344}}
%  >{\raggedright\arraybackslash}p{(\columnwidth - 16\tabcolsep) * \real{0.1194}}
%  >{\raggedright\arraybackslash}p{(\columnwidth - 16\tabcolsep) * \real{0.1044}}
%  >{\raggedright\arraybackslash}p{(\columnwidth - 16\tabcolsep) * \real{0.1044}}
%  >{\raggedright\arraybackslash}p{(\columnwidth - 16\tabcolsep) * \real{0.1194}}
%  >{\raggedright\arraybackslash}p{(\columnwidth - 16\tabcolsep) * \real{0.1044}}
%  >{\raggedright\arraybackslash}p{(\columnwidth - 16\tabcolsep) * \real{0.0896}}
%  >{\raggedright\arraybackslash}p{(\columnwidth - 16\tabcolsep) * \real{0.1194}}
%  >{\raggedright\arraybackslash}p{(\columnwidth - 16\tabcolsep) * \real{0.1044}}@{}}
%\toprule\noalign{}
%\begin{minipage}[b]{\linewidth}\raggedright
%\end{minipage} & \begin{minipage}[b]{\linewidth}\raggedright
%\textbf{Medios masivos y dominan-tes}
%\end{minipage} & \begin{minipage}[b]{\linewidth}\raggedright
%\textbf{Medios alterna-}
%
%\textbf{tivos}
%\end{minipage} & \begin{minipage}[b]{\linewidth}\raggedright
%\textbf{Firman sus notas}
%\end{minipage} & \begin{minipage}[b]{\linewidth}\raggedright
%\textbf{Altos ingresos económi-}
%
%\textbf{cos}
%\end{minipage} & \begin{minipage}[b]{\linewidth}\raggedright
%\textbf{Grandes audien-}
%
%\textbf{cias}
%\end{minipage} & \begin{minipage}[b]{\linewidth}\raggedright
%\textbf{Autor de libros}
%\end{minipage} & \begin{minipage}[b]{\linewidth}\raggedright
%\textbf{Presencia}
%
%\textbf{en medios audiovi-}
%
%\textbf{suales}
%\end{minipage} & \begin{minipage}[b]{\linewidth}\raggedright
%\textbf{Objeto}
%
%\textbf{de}
%
%\textbf{Biogra-}
%
%\textbf{fías}
%\end{minipage} \\
%\midrule\noalign{}
%\endhead
%\bottomrule\noalign{}
%\endlastfoot
%\textbf{Hegemónicos Dominantes} & \textbf{Sí} & \textbf{No} & \textbf{Sí} & \textbf{Sí} & \textbf{Sí} & \textbf{Sí} & \textbf{Sí} & \textbf{Sí} \\
%\textbf{Hegemónicos}
%
%\textbf{alternativos} & \textbf{No} & \textbf{Sí} & \textbf{Sí} & \textbf{Sí/No} & \textbf{Sí/No} & \textbf{Sí} & \textbf{Sí/No} & \textbf{Sí} \\
%\textbf{Periféricos dominantes} & \textbf{Sí} & \textbf{No} & \textbf{No} & \textbf{No} & \textbf{Sí} & \textbf{No} & \textbf{No} & \textbf{No} \\
%\textbf{Periféricos}
%
%\textbf{alternativos} & \textbf{No} & \textbf{Sí} & \textbf{No} & \textbf{No} & \textbf{Sí/No} & \textbf{No} & \textbf{No} & \textbf{No} \\
%\end{longtable}

Cuando iniciamos el análisis del \emph{corpus} bibliográfico, presuponíamos que existía cierta coherencia y unanimidad de criterios sobre los valores y principio de legitimidad del campo periodístico, sin importar si eran hegemónicos o periféricos y si trabajaban en un medio dominante o alternativo. Sin embargo, el análisis del \emph{corpus} nos permitió discernir tres categorías de periodistas de acuerdo a los valores y principios que rigen su pertenencia al campo, y dos subcategorías que, sin ser propiamente de periodistas, usufructúan su condición y actúan como tales en situaciones determinadas. A los fines de identificarlas proponemos la siguiente denominación, que intenta sintetizar sus principales características:

\ul{Categorías de periodistas:}

\begin{itemize}
\item
  Defensores
\item
  Profesionalistas
\item
  Predicadores
\end{itemize}

\ul{Pseudoperiodistas}

\begin{itemize}
\item
  Emigrados
\item
  Intelectuales periodistas.
\end{itemize}

Consideramos que estas categorías son preexistentes al conflicto que se suscitó entre el campo periodístico y el gobierno de Cristina Fernández de Kirchner. Sin embargo, fue el conflicto el que permitió apreciar en todas sus dimensiones las diferencias que los separan.


\subsection{3. 3. 2. Categorías de periodistas}

\textbf{Los defensores}

Incluimos en esta categoría a los periodistas que representan más cabalmente al núcleo mayoritario. Adoptamos este término porque los integrantes de este grupo se presentan como agredidos y, en tal sentido, adoptan una estrategia discursiva defensiva a favor de la legitimidad del periodismo independiente. Son jóvenes o mayores, centrales o periféricos, que encarnan la línea de defensa principal del campo dominante desde la perspectiva de las empresas económicamente más sólidas de los medios (dominantes). Expresan las convicciones e intereses de las principales empresas de medios, empresas para las que se desempeñan y a las que consideran parte integrante del campo, en el sentido amplio que desarrollamos.

En el caso de periodistas con firmas de prestigio, construidas gracias a esos medios, muchas veces se han convertido ellos mismos en productores de sí mismos. Editan sus libros, producen sus programas de televisión y venden su publicidad. En no pocas oportunidades, son escribas de intereses que se encuentran fuera del campo, pero cuya pertenencia les permite presentar los contenidos como productos legítimos del periodismo. Pablo Sirvén (2011: 5), en uno de los textos analizados, afirma de manera representativa el sentir de este grupo: ``el periodismo es el único interprete autorizado de la verdad''.

En los términos de Baldoni (2012) este es el grupo de periodistas que representan a los autodesignados ``independientes'', categoría que los define por una de las cualidades que el campo destaca como elemento identitario de su condición. Otra denominación que le asignan los autores de \emph{Periodismo de Infantería} es la de ``jetones''.

\begin{quote}
Para aquellos que no conocen la jerga del diario (Clarín) les comento que los jetones son aquellos periodistas que por experiencia o por prestigio, o por obsecuencia y reconocimiento de parte de la empresa ganan altísimos sueldos, en esta categoría estaban en aquel momento: Van derKooy, Blanck, Cardozo, Canedo, Ona, Sánchez, Bonelli, Seoane, etc. (Márquez y Ces, 2011: 137).
\end{quote}

Expresión de esta tipología de periodismo en el \emph{corpus} son, además de los \emph{jetones} mencionados en la cita: Pablo Sirvén, Luis Majul, Gabriel \href{http://www.edicionesb-argentina.com/autor/levinas-daniel/}{Levinas}. Pero no solo ellos, reconocidos mayoritariamente, sino una gran cantidad de periodistas ignotos que aceptan de manera sumisa, como forma naturalizada de subsistir en la profesión, escribir lo que el editor les pida.

Barone (2011: 4), uno de los expatriados, cita en su libro, \emph{K Letra bárbara. Periodismo sucio y público sublevado,} una leyenda que circula en el medio y que refleja de alguna manera a este tipo de periodista que, en general, suele destacarse en los medios dominantes:

\begin{quote}
El director examina a un nuevo aspirante a redactor:

---Escriba acerca de Dios.

---¿A favor o en contra?

---El puesto es suyo.
\end{quote}

\textbf{Los profesionalistas}

En esta categoría ubicamos a periodistas de larga tradición que, en general, han recorrido distintas editoriales y medios. Provienen, mayoritariamente, de la prensa escrita. Muchos de ellos han sido refractarios a la gestión kirchnerista, por sentir agraviado el campo por parte de los funcionarios y de colegas, como así también por haberse sentido excluidos de los medios privilegiados por el gobierno en la distribución de la pauta periodística. Percibieron que el conflicto con las grandes empresas de medios y la forma en que estas reaccionaron ponían en riesgo el capital acumulado por la profesión, tal cual la conocieron. En el período, sus textos revelan tanto su esfuerzo por defender la profesión periodística como a los medios donde trabajan, sin dejar de percibir el conflicto y las dificultades que enfrentaba el gobierno en su comunicación y el proceso de monopolización de la actividad periodística.

En general, son mayores y perciben la crisis del ``negocio'' periodístico, por lo que la dependencia de la pauta ``oficial'', los tornó enojosos con la situación. Entre ellos señalamos a Edi Zunino, \href{http://www.tematika.com/buscar.do?seccionDeBusqueda=En+Libros\&seccion=1\&claveDeBusqueda=porAutor\&txtencoded=Graciela+Mochkofsky\&idAutor=35432\&criterioDeOrden=2\&idSeccion=1\&texto=Graciela+Mochkofsky\&optSeleccionada=Autor\&idSeccionPropia=1}{Graciela Mochkofsky}, Martín Sivak, Eduardo Blaustein, Walter Vargas, entre otros. Incluso el propio Víctor Hugo Morales puede considerarse dentro de esta categoría,

\textbf{Los predicadores}

Definiremos, al último grupo como los \emph{predicadores}, aquellos periodistas con trayectoria, en general mayores, altamente reconocidos por sus posiciones políticas defensoras de diversas causas (derechos humanos, ecología, feminismo, etc.) no específicamente propias del campo periodístico. Creen fervientemente en la profesión periodística y la necesariedad del campo periodístico, pero descreen de la independencia política o ideológica, ya que imaginan un periodismo comprometido, al servicio de los intereses de su audiencia. Se trata de una audiencia no abstracta, producida por ellos mismos, educada y orientada por ellos.

Estos periodistas pueden desempeñarse o no en empresas periodísticas centrales del campo, pero no suelen estar subordinados a las empresas en las que trabajan, ya que su presencia prestigia a las mismas y no a la inversa. Este grupo es coincidente, aunque no en todos sus términos, con lo que Baldoni denomina periodistas \emph{militantes}. A modo de ejemplo, ubicamos en esta categoría a Horacio Verbitsky, Roberto Caballero y Eduardo Aliverti.


\subsection{3. 3. 3. Pseudoperiodistas}

\textbf{Los emigrados}

Los emigrados no son estrictamente periodistas ya que, por diversos motivos, se autoexcluyeron del campo periodístico. La perspectiva que poseen del periodismo, quienes integran esta categoría, es profundamente escéptica, descreen de los principios que el campo sustenta o defiende. En general, tienen un recorrido prolongado dentro del campo en medios masivos dominantes, no han llegado a lugares centrales ni han tenido las firmas más cotizadas, pero poseen el reconocimiento de audiencias significativas. Ejemplos claros de esta tipología son Claudio Díaz, Orlando Barone y Jorge Asís. Este último se retiró del campo dando un portazo con su libro ``El diario de la Argentina'', aunque luego volvió en calidad de \emph{outsider}. Los emigrados tienden a transmitir a sus lectores/audiencias, el descubrimiento o la transformación sufrida al ``ver la realidad'' de la falacia periodística, como si hubieran tenido una revelación mística y sintieran la necesidad de transmitirla al resto de sus congéneres.

\textbf{Los intelectuales periodistas}

Como ya se señaló, esta es una categoría que introduce Bourdieu, a la que le presta especial atención. El autor sostiene que, quienes la integran, no son completamente periodistas ni son completamente intelectuales.

Si bien esta categoría de \emph{periodista intelectual}, como el resto de las categorías de la clasificación, es preexistentes al conflicto 2009-2015, este le dio una particular relevancia. Quienes la integran se ubicaron, mayoritariamente, en el polo antigubernamental. Entre los más significativos podemos mencionar a Marcos Novaro, Marcelo Birmajer, Santiago Kovadloff, como descalificadores del gobierno; y a Dante Palma, Horacio González y Cynthia Ottaviano, entre los progubernamentales (solo por mencionar algunos ejemplos de medios nacionales, ya que el fenómeno se reproducía, aunque con menor intensidad, en las provincias).

\chapter{4. Análisis de Contenido del \emph{Corpus} Bibliográfico}
\section{4.1. Corpus de Análisis}

El análisis de la producción ensayística ordenado en el tiempo permitirá apreciar el encadenamiento del debate al interior del campo periodístico y, simultáneamente, la existencia de relaciones causales entre los actores del conflicto, en algunos casos motivados por sus casas periodísticas y, en otros, por el fragor de las batallas políticas.

\textbf{Cuadro 4: Universo de los libros considerados}

%\begin{longtable}[]{@{}
%  >{\raggedright\arraybackslash}p{(\columnwidth - 6\tabcolsep) * \real{0.3651}}
%  >{\raggedright\arraybackslash}p{(\columnwidth - 6\tabcolsep) * \real{0.2539}}
%  >{\raggedright\arraybackslash}p{(\columnwidth - 6\tabcolsep) * \real{0.0793}}
%  >{\raggedright\arraybackslash}p{(\columnwidth - 6\tabcolsep) * \real{0.3016}}@{}}
%\toprule\noalign{}
%\begin{minipage}[b]{\linewidth}\raggedright
%\textbf{Título y año de edición}
%\end{minipage} & \begin{minipage}[b]{\linewidth}\raggedright
%\textbf{Editorial}
%\end{minipage} & \begin{minipage}[b]{\linewidth}\raggedright
%\textbf{Nº}
%\end{minipage} & \begin{minipage}[b]{\linewidth}\raggedright
%\textbf{Autor}
%\end{minipage} \\
%\midrule\noalign{}
%\endhead
%\bottomrule\noalign{}
%\endlastfoot
%\textbf{2009} & & & \\
%Diario de guerra. Clarín, el gran engaño argentino & Gárgola & 1 & Claudio Díaz \\
%\textbf{2010} & & & \\
%678. La creación de otra realidad & \href{http://www.lecturalia.com/editoriales/79/paidos}{Paidós}/Planeta & 2 & María Julia Oliván y Pablo Alabarces \\
%Silencio por sangre & Miradas al sur & 3 & Daniel Cecchiniy Jorge Mancinelli \\
%\textbf{2011} & & & \\
%Pecado original & \href{http://www.tematika.com/buscar.do?seccionDeBusqueda=En+Libros\&seccion=1\&claveDeBusqueda=porEditorial\&txtencoded=Planeta\&idAutor=99\&criterioDeOrden=2\&idSeccion=1\&texto=Planeta\&optSeleccionada=Editorial\&idSeccionPropia=1}{Planeta} & 4 & \href{http://www.tematika.com/buscar.do?seccionDeBusqueda=En+Libros\&seccion=1\&claveDeBusqueda=porAutor\&txtencoded=Graciela+Mochkofsky\&idAutor=35432\&criterioDeOrden=2\&idSeccion=1\&texto=Graciela+Mochkofsky\&optSeleccionada=Autor\&idSeccionPropia=1}{Graciela Mochkofsky} \\
%Perón y los medios de comunicación & Sudamericana
%
%\href{https://es.wikipedia.org/wiki/Penguin_Random_House_Grupo_Editorial}{PenguinRandom House} & 5 & Pablo Sirvén \\
%K Letra bárbara & Sudamericana
%
%\href{https://es.wikipedia.org/wiki/Penguin_Random_House_Grupo_Editorial}{PenguinRandom House} & 6 & Orlando Barone \\
%El kirchnerismo póstumo & Ediciones B & 7 & Jorge Asís \\
%Patria o medios & Sudamericana
%
%\href{https://es.wikipedia.org/wiki/Penguin_Random_House_Grupo_Editorial}{PenguinRandom House} & 8 & Edi Zunino \\
%\href{http://www.casassaylorenzo.com/9789872676001/PERIODISMO+DE+INFANTERIA/}{Periodismo de infantería} & \href{http://www.casassaylorenzo.com/resultados.aspx?c=CLARINETE\&ed=7347\&por=editorial\&orden=fecha}{Clarinete} & 9 & Virginia Márquez y Aníbal \href{http://www.libreriapaidos.com/resultados.aspx?c=Ces\%2c+Anibal\&por=AutorEstricto\&aut=59097\&orden=fecha}{Ces} \\
%\textbf{2012} & & & \\
%Lanata & Margen Izquierdo & 10 & Luis \href{http://www.cuspide.com/resultados.aspx?c=MAJUL+LUIS\&por=AutorEstricto\&aut=275032\&orden=fecha}{Majul} \\
%\textbf{2013} & & & \\
%\href{https://www.boutiquedellibro.com.ar/9789876281928/Aquel+Periodismo/}{Aquel periodismo}. Medios y periodistas en la Argentina 1965-2012 & \href{http://www.libreriapaidos.com/resultados.aspx?c=Edhasa\&ed=409\&por=editorial\&orden=fecha}{Edhasa} & 11 & \href{http://www.edhasa.com.ar/busqueda-avanzada.php?autor=Daniel+Muchnik\&avanzada=1}{Daniel Muchnik} \\
%Víctor Hugo, una historia de coherencia y convicción & Ediciones al arco & 12 & Julián Capasso \\
%El inventor del peronismo & Planeta & 13 & Silvia Diana Mercado \\
%Máquinas de captura. Los medios concentrados en tiempos del kirchnerismo & Colihue & 14 & Daniel Rosso \\
%\href{http://www.cuspide.com/9789504932871/1++Clarin/}{Clarín}, el gran diario argentino: una historia & Planeta & 15 & Martín Sivak \\
%Años de rabia & \href{http://www.edicionesb-argentina.com/}{Ediciones} B & 16 & Eduardo Blaustein \\
%Converso & Margen izquierdo & 17 & Pablo Sirvén \\
%El pequeño Timerman & Ediciones B & 18 & Gabriel \href{http://www.edicionesb-argentina.com/autor/levinas-daniel/}{Levinas} \\
%\textbf{2014} & & & \\
%Nuevos desafíos del periodismo & Ariel /Paidós
%
%Planeta- ADEPA & 19 & Daniel Dessein y Gastón Roitberg (Compiladores) \\
%Periodistas en el barro. Peleas, aprietes, traiciones y negocios. Miserias y razones de la guerra mediática en la Argentina reciente & Sudamericana
%
%\href{https://es.wikipedia.org/wiki/Penguin_Random_House_Grupo_Editorial}{PenguinRandom House} & 20 & \href{http://buscapdf.es/busca/?a=edi+zunino\&m=ae}{Edi Zunino} \\
%Tiempos turbulentos & Ariel /Paidós
%
%Planeta- ADEPA & 21 & \href{http://www.planetadelibros.com.ar/carlos-jornet-autor-000063783.html}{Carlos Jornet} y \href{http://www.planetadelibros.com.ar/daniel-dessein-autor-000063287.html}{Daniel Dessein} \\
%Télam, el hecho maldito del periodismo argentino & Ediciones TXT & 22 & \href{http://www.libreriahernandez.com/busquedaMultiple?perPage=18\&authorIds=171462\&sortBy=stockAndTitle\&reverseSort=\&displayMode=\&groupMode=\&page=1}{Ariel Bargach~}y \href{http://www.libreriahernandez.com/busquedaMultiple?perPage=18\&authorIds=118944\&sortBy=stockAndTitle\&reverseSort=\&displayMode=\&groupMode=\&page=1}{Mariano Suárez\emph{~}} \\
%(In)justicia mediática & Sudamericana
%
%\href{https://es.wikipedia.org/wiki/Penguin_Random_House_Grupo_Editorial}{PenguinRandom House} & 23 & Darío \href{http://www.cuspide.com/resultados.aspx?c=VILLARRUEL+DARIO\&por=AutorEstricto\&aut=291343\&orden=fecha}{Villarruel} \\
%Audiencia con el diablo & Aguilar & 24 & ~Víctor Hugo \href{http://www.cuspide.com/resultados.aspx?c=MORALES+VICTOR+HUGO\&por=AutorEstricto\&aut=278005\&orden=fecha}{Morales} \\
%Las locuras del Rey Jorge & Ediciones B & 25 & Eduardo \href{http://www.edicionesb-argentina.com/autor/blaustein-eduardo/}{Blaustein} \\
%Guerras mediáticas & Sudamericana
%
%\href{https://es.wikipedia.org/wiki/Penguin_Random_House_Grupo_Editorial}{PenguinRandom House} & 26 & Fernando Ruiz \\
%\textbf{2015} & & & \\
%\href{https://www.boutiquedellibro.com.ar/9789504946175/Quienes+Fuimos+En+La+Era+K/}{Quiénes fuimos en la era k} & Planeta & 27 & Jorge \href{http://www.cuspide.com/resultados.aspx?c=FONTEVECCHIA+JORGE\&por=AutorEstricto\&aut=264707\&orden=fecha}{Fontevecchia} \\
%El Perro & Ediciones B & 28 & Hernán \href{http://www.edicionesb-argentina.com/autor/lopez-echague-hernan/}{López Echague} \\
%\end{longtable}

%\begin{longtable}[]{@{}
%  >{\raggedright\arraybackslash}p{(\columnwidth - 6\tabcolsep) * \real{0.3651}}
%  >{\raggedright\arraybackslash}p{(\columnwidth - 6\tabcolsep) * \real{0.2539}}
%  >{\raggedright\arraybackslash}p{(\columnwidth - 6\tabcolsep) * \real{0.0793}}
%  >{\raggedright\arraybackslash}p{(\columnwidth - 6\tabcolsep) * \real{0.3016}}@{}}
%\toprule\noalign{}
%\begin{minipage}[b]{\linewidth}\raggedright
%La inconclusa ley de medios: la historia menos contada
%\end{minipage} & \begin{minipage}[b]{\linewidth}\raggedright
%Continente
%\end{minipage} & \begin{minipage}[b]{\linewidth}\raggedright
%29
%\end{minipage} & \begin{minipage}[b]{\linewidth}\raggedright
%Néstor Piccone
%\end{minipage} \\
%\midrule\noalign{}
%\endhead
%\bottomrule\noalign{}
%\endlastfoot
%Grandes y pequeñas mentiras que nos contaron & Planeta & 30 & \href{http://www.planetadelibros.com.ar/marcos-novaro-autor-000058154.html}{Marcos Novaro} y \href{http://www.planetadelibros.com.ar/marcelo-birmajer-autor-000021721.html}{Marcelo Birmajer} \\
%Doble agente. La biografía inesperada de Horacio Verbitsky & Sudamericana
%
%\href{https://es.wikipedia.org/wiki/Penguin_Random_House_Grupo_Editorial}{PenguinRandom House} & 31 & \href{http://www.megustaleer.com.ar/autor/gabriel-levinas/0000030467}{Gabriel Levinas} \\
%Mentime que me gusta & Aguilar & 32 & Víctor Hugo \href{http://www.cuspide.com/resultados.aspx?c=MORALES+VICTOR+HUGO\&por=AutorEstricto\&aut=278005\&orden=fecha}{Morales} \\
%\href{http://www.cuspide.com/9789876842686/El+Rebenque+Del+Diablo/}{El rebenque del diablo} & Colihue & 33 & Víctor Hugo \href{http://www.cuspide.com/resultados.aspx?c=MORALES+VICTOR+HUGO\&por=AutorEstricto\&aut=278005\&orden=fecha}{Morales} \\
%\href{http://www.cuspide.com/9789876275637/Todos+Contra+Branca+Contra+Todos/}{Todos contra Branca contra todos} & Ediciones B & 34 & Diego Brancatelli \\
%Diez ironías sobre la libertad de expresión & Colectivo de Trabajadores De Prensa (CTP) & 35 & Autores Varios \\
%\href{http://www.cuspide.com/9789504947066/Clarin++La+Era+Magnetto/}{Clarín, la era Magnetto} & Planeta & 36 & \href{http://www.planetadelibros.com.ar/sivak-martin-ernesto-autor-000057233.html}{Martin Ernesto}Sivak \\
%Periodistas depordivos & Ediciones al arco & 37 & Walter Vargas \\
%\end{longtable}


\section{4. 2. Periodización}

Consideramos que es posible identificar, dentro del período considerado, dos momentos diferenciables.

Hay una primera etapa, entre los años 2009 y 2011, en la que comienza el proceso metaperiodístico a través de los libros, donde se destacan los trabajos críticos, tanto sobre los grandes medios de comunicación como de las políticas comunicacionales del gobierno, pero en general no expresan una confrontación interperiodística. Las referencias están dirigidas a los medios o políticas en general, pero no hacia los periodistas en particular. Denominamos a este período, etapa preparatoria, o de ofensiva de los sectores críticos del \emph{mainstream}.

Durante el año 2012, primer año del 2° gobierno de Cristina Fernández de Kirchner, Luis Majul edita la Biografía de Lanata, destinada al relanzamiento de unos de los periodistas más prestigiosos de la Argentina y único libro de la serie analizada.

Por el contrario a partir del año 2013 y hasta el 2015, es posible observar lo que consideramos el inicio de la ruptura del campo a partir de la disputa abierta entre periodistas. De los 37 libros detectados, 27 lo son en este breve período. Utilizaremos para definir este período, el concepto del periodista de Clarín, Julio Blanck, periodismo de guerra.

\subsection{4. 2. 1. Etapa Preparatoria}

\textbf{Año 2009}

Como señalamos anteriormente, en 2009, Claudio Díaz, publica \emph{Diario de guerra}. \emph{Clarín, el gran engaño argentino}. Si bien la estructura argumentativa de todo el libro es que el periodismo (¿campo periodístico?) ha sido profundamente antiperonista y que constituyó una ``oligarquía de la comunicación (\ldots), dueños de la riqueza pero también de las palabras'' (pp. 15 y 17), Díaz no puede abstraerse de los valores del campo e intentará mostrar que los prejuicios políticos impidieron la libertad de expresión. Es decir, que la violación del principio de libertad de expresión no estaba en el poder político, sino en el poder empresario o de los grandes grupos económicos.

\begin{quote}
La libre circulación de ``lo que pasa" resulta imposible y hasta inimaginable. La imposición del pensamiento único por los países ricos a través de sus altavoces mediáticos determina la concepción de un único mundo posible, con un único sistema económico viable y con un unificado concepto de modernidad, desarrollo y progreso. En definitiva, la información, en contra de una genuina libertad de expresión, genera dogmas que se resumen en el simple \emph{`lo que no está en los medios, tal y como los medios lo publican y lo} interpretan, \emph{no está en el mundoʹ} (Díaz, 2009: 30).
\end{quote}

Más adelante ejemplifica nuevamente:

\begin{quote}
Hablando de la libertad de prensa y de la manera en que este grupo económico monopólico la práctica, dos episodios ocurridos recientemente nos demostrarán que una cosa son los discursos y otra las realidades. Entre abril y junio de 2008, cuando se desató el apriete destituyente de la alianza de la agroligarquía con la mediocracia, una cuadrilla de empleados del diario Clarín salía todas las madrugadas a tapar las pintadas contrarias al Grupo que por entonces realizaban militantes juveniles peronistas en paredones como los de la Avenida Udaondo, frente a la cancha de River, o en los puentes ubicados sobre la General Paz. La preocupación del Estado Mayor Conjunto que comanda el general Magnetto, estaba dada en los efectos negativos que podía provocar a la imagen de la corporación aquella campaña de esclarecimiento (\ldots). Curiosamente, para la misma fecha, y a raíz de la aparición de nuevas pintadas, la estrategia cambió radicalmente. Esta vez, los graffitis estampados sobre receptorías u oficinas vinculadas al Grupo permanecieron expuestos ante la ``opinión pública'' durante varias horas. Además se fotografiaban y se reproducían en todos los medios --diarios o sitios de Internet-- que pertenecen a la gran familia \emph{clarinesca}. El objetivo parecía estar muy claro: el gran diario argentino es la pobre víctima de una gran persecución, rol que los poderes mundiales han sabido explotar convenientemente a lo largo de la historia del siglo XX (Díaz, 2009: 222).
\end{quote}

La cita anterior permite apreciar no sólo la tensión que soportaban los medios líderes, sino también lo que Díaz llama \emph{el cambio radical de estrategia}. Es decir, muy rápidamente, los medios logran instalarse como víctimas, sujetos de la agresión por parte de ``militantes juveniles''/gobierno. A pesar del esfuerzo de Díaz por diferenciarlos, para las audiencias era muy difícil distinguirlos, sobre todo cuando la dirigencia del gobierno había colocado en el centro de su discurso la crítica a los medios de mayor audiencia.

\textbf{Año 2010}

En 2010, la editorial Planeta, al amparo del éxito del programa homónimo, publicó \emph{6,7,8. La creación de otra realidad.} El libro, que impresiona por lo tosco y mal editado, fue un éxito editorial. Si bien no produjo, como algunos investigadores esperaban (Bilyk, 2012), una saga de investigaciones sobre el programa, no es menos cierto que su existencia influyó, en buena medida, en la importante producción subsecuente que, en general, no dejaban de tener como telón de fondo este producto editorial y/o al programa mismo.

María Julia Oliván, primera conductora del ciclo, en el diálogo con Alabarces, responde en relación a la problemática de la libertad de expresión:

\begin{quote}
Esa es la parte más contradictoria del programa. Teníamos la libertad absoluta de opinar lo que quisiéramos, pero debíamos utilizar esa libertad en diez segundos y siempre en referencia a un tape que era tan radical que permitía ubicarse solo en uno de los extremos. Algunas veces me daba la impresión de que Orlando Barone tenía un poco más de información sobre los \emph{tapes,} porque muchas veces \emph{traía} sus opiniones en un borrador. Pero no sé si esto era así (Oliván y Alabarces, 2010: 83).
\end{quote}

Sin embargo, a nuestro entender, lo sustancial del texto radica en un conjunto de reflexiones que realiza María Julia Oliván y que revelan que el programa televisivo fracturó, o comenzó a hacerlo, uno de los valores del campo. No se cuestionaba la credibilidad de un periodista, de un medio en particular, un día determinado, sino que se ponía en crisis el vínculo entre la audiencia y el sistema de medios, ponía en duda el sistema de creencias de vastas audiencias.

\begin{quote}
El más crítico hacia los periodistas es Orlando Barone, que ha pasado mucho tiempo de su vida en una redacción y en la radio; o Sandra Russo, que tiene una larga trayectoria en gráfica. Pero, en todo caso, esa crítica sobre la televisión, enfocada sobre el periodismo televisivo y sobre el rol de los medios, está hecha con una distancia enorme de los propios medios, no por semiólogos sino por periodistas que se corren y quedan afuera de lo que critican, porque de otro modo no podrían criticarlo. Asumen una posición diferente, no corporativa. Tal vez, eso es lo que les resulta más irritante a los colegas, \textbf{pero nunca antes había sucedido que un periodista asumiera el compromiso de criticar a otro periodista}.\footnote{Destacado por los autores.}Ese lugar incómodo solo lo adoptó 6 7 8, y esa fue una de las cosas que me alejó del ciclo. No me parece posible tomar esa distancia siendo periodistas. Después podemos analizar a quién es funcional ese lugar incómodo o para qué es incómodo y para qué es cómodo (\ldots) La diferencia es que este Gobierno es el que más en serio se tomó el trabajo de contradecir el discurso de los grandes medios, porque cree saber cómo le llegaba ese discurso a la opinión pública. Y lo hizo construyendo un sistema de medios públicos que no solo aportan una mirada oficialista, sino que aportan una mirada multicultural federal, desde un lugar jerarquizado. Eso es algo que nunca había sucedido en los medios públicos (\ldots) Por más incómoda que muchas veces me haya sentido, sé que la discusión que planteó 6,7,8, nos va a servir a todos los periodistas y a la sociedad en general. Porque con formas que no siempre compartí y muchas veces consideré desatinadas, miradores del mundo nos miramos un poco a nosotros mismos. Por primera vez, un programa de televisión analizó el mensaje, al mensajero y al dueño de la mensajería. Y yo estuve ahí (Oliván y Alabarces, 2010: 44, 49, 195).
\end{quote}

María Julia Oliván, toma conciencia del riesgo que corría dentro del campo y se aleja del ciclo meses antes de la publicación del libro. Los años que siguieron al gobierno de Cristina Fernández de Kirchner, parecen confirmar que su alejamiento le permitió ser una de las pocas panelistas y conductoras del ciclo que sobrevivió como periodista en medios dominantes. La irritación de los colegas que menciona María Julia Oliván, se expresó de maneras diversas, una de ellas fue considerar a la crítica de medios como restricción a la libertad de expresión.

Este libro dio lugar a una larga lista de notas periodísticas, en general de tono crítico, por parte de periodistas con larga tradición. Si bien el libro no aborda la problemática de la libertad de expresión o el derecho al disenso, fue la oportunidad de que los periodistas de los medios masivos pudieran realizar la crítica al programa, fundamentalmente, desde programas de televisión o desde algunos blogs. Pablo Sirvén, periodista del Diario La Nación señala:

\begin{quote}
Paradójicamente, 6,7,8 agudizó ciertas aristas del discurso autoritario también en los medios tradicionales. No lo hago cargo de esto a 6,7,8, ya que como producto es un emergente. En general, la prensa ``grande'' nunca se refería a la otra prensa, es decir, la otra prensa podía decir lo que quisiera, hacer lo que quisiera, que no se le daba bolilla. Lo cual, en muchos casos, era tomado como una postura soberbia. Y en verdad puede ser soberbio como en el caso nuestro, de La Nación, pero así es el liberalismo. Es decir, nosotros decimos esto, y Página 12 que diga lo que quiera, todo bien, hay público para todos y, quizás, uno se arma una mejor idea leyendo un poco de todo (octubre de 2010).\footnote{\url{http://adriancorbella.blogspot.com.ar/2010/10/678-la-creacion-de-otra-realidad.html}}
\end{quote}

Sirvén logra percibir el formato de contraataque que adquirió 6, 7, 8. En una nota sin firma, el periódico Perfil levanta la presentación del libro como nota periodística:

\begin{quote}
A tono de la polémica, el libro de Oliván y Alabarces es un extenso diálogo de más de 200 páginas sobre el ciclo ideado y controlado hasta el más mínimo detalle por Diego Gvirtz. En una entrevista con Perfil.com los autores hablaron sobre el programa y si es posible que 6,7,8 sobreviva al kirchnerismo en caso de que en las próximas elecciones el Gobierno pierda.\footnote{\url{https://www.perfil.com/noticias/politica/maria-julia-olivan-y-pablo-alabarces-678-no-puede-vivir-sin-el-kirchnerismo-20101008-0028.phtml}. Octubre 2009.}
\end{quote}

El libro fue objeto de reflexión académica y muchos creyeron que ese sería el inicio de una saga de trabajos que profundizarían la investigación en un tema huidizo como es el periodismo de debate televisivo.

Pablo Andrés Bilyk (2012: 11) analiza el lanzamiento del libro desde esa perspectiva:

\begin{quote}
En su carácter de escrito iniciador de las reflexiones sobre \emph{678}, el libro tiene algunas virtudes que vale la pena mencionar: el esfuerzo por comenzar a definir el género al cual pertenecería el programa, el hecho de reconstruir los antecedentes televisivos, y la posibilidad de pensar las posibles audiencias y sus modos de participación, así como las continuidades/rupturas narrativas de la puesta en escena.
\end{quote}

El gobierno de Cristina Fernández de Kirchner había tomado conciencia de que la prensa mayoritaria le era adversa. Consideraba, además, que utilizaba instrumentos no legítimos del periodismo, por lo que tras la experiencia del programa 678, el gobierno comenzó una larga serie de acciones tendientes a ocupar el espacio de medios.

Durante este año desde una editorial con apoyo oficial se publicó \emph{Silencio por sangre} (2010) de Daniel Cecchini y Jorge Mancinelli, que abordó, de manera crítica, el traspaso de las acciones mayoritarias de Papel Prensa a tres grupos de medios gráficos, durante la dictadura (1976 - 1983). Los autores toman, para sí, el concepto de \emph{libertad de expresión}, ahora en riesgo por la monopolización de la producción del insumo básico de la prensa gráfica: el papel.

En el prólogo del libro, Eduardo Anguita (2010: 7) afirma:

\begin{quote}
El 10 de noviembre de 1976, después de haber arrancado las acciones a Lidia Papeleo de Graiver, los directivos de La Razón, La Nación y Clarín dieron una ``conferencia de prensa" en la sede de la Asociación de Entidades Periodísticas Argentinas (ADEPA), \textbf{esa nefasta asociación de lobby que sobrevive para vergüenza de la libertad de expresión}.\footnote{El énfasis es propio.}
\end{quote}

Más adelante los autores desgranan un discurso que articula la problemática de los derechos humanos y su violación durante la última dictadura, con la libertad de expresión violentada por las grandes empresas.

\begin{quote}
Esa maniobra de apropiación formó parte de la alianza estratégica entre la dictadura iniciada el 24 de marzo de 1976 y los representantes de los grupos económico-mediáticos más grandes del país. Los grupos económicos concentrados necesitaban a los militares para eliminar a la disidencia política y social que se oponía a sus intereses. Los dictadores, por su parte, exigían no sólo una prensa silenciada mediante la censura, sino medios cómplices de sus políticas y de sus acciones. Esa misión la cumplieron Clarín, La Nación y La Razón, y, a cambio de ello, recibieron el monopolio del papel de diario, \textbf{una suerte de dictadura contra la libertad de expresión} (Cecchini y Mancinelli, 2010: 17).\footnote{El énfasis es propio.}
\end{quote}

Más adelante, repiten conceptos propios del campo periodístico, pero resignificando su sentido:

\begin{quote}
El tercer grupo --integrado por diarios privados, maquillados de independientes-- jugó un papel decisivo en la instalación mediática de, primero, la necesidad de un golpe de Estado contra el gobierno constitucional de María Estela Martínez de Perón y de la ``aniquilación de la subversión'' y, después, de las bondades de la dictadura militar y de la justificación de sus métodos represivos. En esa línea se situaron --con matices-- ante la opinión pública Clarín, La Nación y también La Razón, cuya histórica adhesión al Ejército permanecía en las sombras para la mayoría de sus lectores. \textbf{A cambio de ello, recibieron el monopolio del papel de diario, lo que les permitió ejercer} --\textbf{aun después de retornada la democracia}-- \textbf{una suerte de dictadura contra la libertad de expresión} (Cecchini y Mancinelli, 2010: 80).\footnote{El énfasis es propio.}
\end{quote}

Por último, los autores, utilizando las propias palabras de los directivos de Papel Prensa en la denuncia que realizan para mantener el control accionario de la misma, afirman:

\begin{quote}
La última denuncia, firmada por Bartolomé Luis Mitre y Héctor Horacio Magnetto, sostiene lo siguiente: ``Recordemos que la firma Papel Prensa produce el papel de diario que constituye el insumo principal de la prensa escrita. De tal suerte, que la firma Papel Prensa quede en manos del Estado implica para éste el control absoluto sobre la libertad de expresi6n y por ende la desaparición de aquellos medios que critiquen la política oficial. Con este objetivo, el Poder Ejecutivo Nacional intenta -por diferentes vías- desde el año pasado quedarse con la empresa. Ante esa afirmación de los firmantes, cabe el aforismo que dice ``a confesión de partes, relevo de pruebas''. Es que han reconocido, y ante la Justicia, que ellos ejercen desde hace más de 30 años el control sobre la libertad de expresión al tener el poder de decidir quién recibe y quién no, y a qué precio, papel de diario (Cecchini y Mancinelli, 2010: 110).
\end{quote}

\textbf{Año 2011}

El año 2011, año electoral, en el que Cristina Fernández ganó las elecciones presidenciales, luego del fallecimiento de su esposo, comenzará a ser pródigo en libros publicados por las grandes editoriales. La editorial Planeta editó el libro de Graciela Mochkofsky (\emph{Pecado original. Clarín, los Kirchner y la lucha por el poder}), libro que retoma el debate por los hijos adoptivos de la Directora y accionista de Clarín, Ernestina Herrera de Noble. Tanto este último tema como el de la transferencia accionaria de Papel Prensa fueron dos temas de agenda que el gobierno logró instalar de manera consistente durante un largo período de tiempo. En tal sentido, Graciela Mochkofsky (2011: 80-81)lo analiza detenidamente desde una perspectiva estrictamente periodística.

\begin{quote}
El acuerdo por Papel Prensa levantó una ola de indignación entre los editores de los diarios que quedaron excluidos especialmente los del interior, que debían seguir pagando por papel importado, sujetos a las condiciones de vulnerabilidad de las que Clarín, La Nación y La Razón lograban escapar, o someterse a los términos de Papel Prensa, que manejaría en adelante los precios y condiciones (\ldots). En octubre de 1978, el informe anual de la SIP(Sociedad Interamericana de Prensa), que agrupaba a los dueños de diarios del continente, reflejó este clima (\ldots). La misión reportaba, además, que los editores de los diarios (se habían reunido con 60 de ellos) aplicaban masivamente autocensura y se negaban a tomar riesgos y a denunciar el terrorismo de Estado.
\end{quote}

A lo largo del libro la autora desgrana el estilo y el tipo de relaciones recíprocas entre los grandes medios y los gobiernos.

\begin{quote}
Menem estaba furioso; qué ingratitud la de Magnetto y Ernestina. Reprochó a su hermano y a Bauzá por haberlo aconsejado mal -- nunca debió haberles dado tan temprano lo que pedían. El 8 de julio de 1992, tercer aniversario de su asunción, durante una entrevista con un periodista de La Nación que le preguntó cuáles habían sido sus errores, respondió sin dudar: -Sólo uno: haber derogado el artículo 45 de la ley de radiodifusión. No medí las consecuencias. Lo hice para afianzar la libertad de prensa, pero esa anulación permitió la existencia de empresas que tienen un canal de televisión, radio, Papel Prensa y un diario y una agencia informativa. Yo no hablo de coartar la libertad de prensa, pero tampoco hay competencia en lo que hace a la información. Se monopolizó la prensa. No esperaba que algunas empresas se convirtieran en propietarias de diarios, canales de televisión, radios y hasta una cuota de Papel Prensa. Es un error que tendremos que subsanar (Mochkofsky, 2011: 110-111).
\end{quote}

¿Los Kirchner habrían aprendido de la experiencia? Eso parece concluir la periodista.

A inicios del año 2011, la editorial Sudamericana publica tres libros sobre la temática: \emph{K Letra bárbara} de Orlando Barone --a esa altura un expatriado del campo periodístico--, \emph{Patria o medios} de Edi Zunino y por último, reedita un viejo trabajo de Pablo Sirvén, \emph{Perón y los medios de comunicación}, ahora ampliado con el subtítulo \emph{La conflictiva relación de los gobiernos justicialistas con la prensa 1943-2011}. Los dos últimos libros van a dar comienzo, a nuestro entender, a la disputa formal por el monopolio de la legitimidad para informar, a partir de la determinación del contenido del principio de libertad de prensa.

Edi Zunino, desde una perspectiva histórica más cercana, analiza la complejidad de la libertad de expresión en la prensa, en el marco de un gobierno al que caracteriza con rasgos hegemónicos y la pretensión de perdurar por tiempo indefinido (``por lo menos cuatro mandatos''). Por otra parte, Zunino percibe el riesgo de implosión que la confrontación provoca al interior del campo periodístico. Sin intentar mediar, describe fenómenos recientes que ponen de manifiesto que la libertad de expresión --entendida en el sentido del rol de los periodistas y la prensa-- está en riesgo, no porque el gobierno impusiera algún tipo de censura, sino porque, a su entender, el gobierno empujó a los medios a dejar atrás una forma consensuada de periodismo. Si bien la opinión de Zunino es discutible, a nuestro juicio es uno de los pocos periodistas que logra percibir y explicitar el conflicto y sus consecuencias.

\begin{quote}
Desde el mismo 25 de mayo de 2003, Néstor y Cristina dedicaron esfuerzos, cuadros políticos, militantes sociales, amistades empresariales, recursos financieros y estructuras estatales a la sagrada tarea de imponer su versión de la historia y del presente a través de los medios de comunicación. Hay que decirlo: será difícil que el periodismo genuinamente independiente salga indemne y sin manchas de la desigual guerra desatada por el poder político para conquistar títulos elogiosos en las primeras planas y los noticieros. La comodidad de ciertos medios y comunicadores en el papel de opositores endilgado por el propio gobierno \textbf{sirvió para desnaturalizar la utopía de un periodismo crítico y sólo comprometido con el derecho de la población} a mantenerse informada, \textbf{generalizando la sospecha de que, en realidad, la bandera de la libertad de prensa es apenas una coartada para ocultar inconfesables mezquindades personales o corporativas} (Zunino, 2011: 7, 9)\footnote{El énfasis es propio.}.
\end{quote}

Más adelante, Zunino (2011: 10) cita a Verbitsky para sostener su posición:

\begin{quote}
En 2007, al reeditar su libro \emph{Un mundo sin periodistas. Las tortuosas relaciones de Menem con la prensa, la ley y la verdad}, Horacio Verbitsky anotó en el prólogo un diagnóstico brillante: ``Las próximas acechanzas contra la libertad de prensa serían menos burdas que las del menemismo, pero no menos eficaces''.
\end{quote}

Y añade un párrafo que permite deslindarlo del grueso de los periodistas que definimos como \emph{defensores} y caracterizarlo como \emph{profesionalista.}

\begin{quote}
Nadie podría acusar a Verbitsky de opositor. Verdadero titán del periodismo independiente durante los 90 y entendido en guerrillas setentistas, el ``Perro'' pasó a ser un adalid intelectual del autodefinido ``kirchnerismo racional'' (Zunino, 2011: 10).
\end{quote}

A modo de justificación de la obra, Zunino pretende encarnar, desde el campo periodístico, a un iluminador de las complejísimas relaciones entre los periodistas, las empresas de medios y el poder político. En buena medida lo logra, dejando un manual de enseñanzas para jóvenes periodistas, preocupado por salvar al periodismo, al que ve en una seria encrucijada.

\begin{quote}
En demasiadas ocasiones los medios se mostraron bastante cómodos ocupando un lugar de oposición que no se corresponde con su esencia: la lealtad al lector, al oyente, al televidente. También de este lado del mostrador queda mucho por aprender, revisar y corregir. \textbf{No se trata de salvar a tal o cual empresa. Hay que salvar al periodismo}. El bicentenario está a la vuelta de la esquina. Sonará estúpidamente obvio, pero el pueblo sigue queriendo saber de qué se trata, mientras los gobernantes insisten en vendérsela cambiada (Zunino, 2011: 280).\footnote{El énfasis es propio.}
\end{quote}

Uno de los aspectos relevantes del libro, a los efectos de desentrañar su concepción del periodismo frente a la libertad de expresión, es el abordaje y descripción minuciosa del conflicto que produjo la crisis terminal de la Asociación civil PERIODISTAS. Si bien estos hechos ocurrieron fuera del período considerado, al ser reanalizado en el contexto del conflicto del campo periodístico ilumina de alguna manera sus orígenes.

El 23 de octubre de 2004, Julio Nudler había denunciado que los directivos de \emph{Página 12} acababan de censurar su ya clásico panorama semanal, supuestamente por orden del gobierno de Néstor Kirchner. La nota cuestionaba el nombramiento de Claudio Moroni al frente de la Sindicatura General de la Nación, órgano destinado --supuestamente-- al control de la corrupción. Nudler concluía la nota afirmando:

\begin{quote}
¿Qué suponen acerca de la inteligencia de los argentinos? ¿Creen que este pueblo sigue aceptando el roban pero hacen? No: aunque hagan, si roban deben ir presos, hoy, mañana, cuando se los pueda condenar. ¿El títere controlará al titiritero? La Argentina sigue siendo un cambalache.\footnote{Ver nota completa en: Página/12.~Domingo, 14 de noviembre de 2004. Recuperado de: https://www.pagina12.com.ar/diario/elpais/subnotas/43613-14896-2004-11-14.html}
\end{quote}

Según Zunino (2011: 39), Ernesto Tiffenberg, sucesor de Jorge Lanata en la dirección del diario, ``intentó escapar por arriba del laberinto que recién empezaba a rodearlo publicando una columna que terminaría empeorando las cosas y generando lo que fuera de micrófonos suele describirse como flor de puterío''. El miércoles 27 de octubre, Tiffenberg publica su nota de respuesta iniciada con la siguiente frase:\footnote{Ver nota completa en: Página/12.Miércoles 27 de octubre de 2004. Recuperado de https://www.pagina12.com.ar/diario/contratapa/13-42854-2004-10-27.html}

\begin{quote}
Este diario llegó al mundo con muchas buenas intenciones y un solo lema: ``Nacido para molestar''. De las intenciones concretó algunas y se esfuerza por alcanzar o recrear otras. Pero el lema lo cumplió con creces y, claro, también pagó por ello. Desde sus primeros días \emph{Página/12} soportó distintas campañas de desprestigio destinadas a socavar su único capital: la credibilidad. (Zunino, 2011: 39)
\end{quote}

En esta petición de principio hay dos frases que interesa resaltar a los fines de la demostración de nuestros objetivos. En primer lugar, el lema ``nacidos para molestar'', si bien no establece un referente, queda implícito que se trata de molestar al poder político y a los poderes fácticos. Por último, al denunciar la existencia de campañas de desprestigio destinadas a socavar la credibilidad del diario, pone de manifiesto que la denuncia de la violación de la libertad de expresión de Julio Nudler haría que sus lectores desconfíen de su independencia y pongan en duda su credibilidad.

Más adelante refuerza la idea.

\begin{quote}
\emph{Página/12} pelea todos los días por seguir siendo el diario que más primicias publica, pero también está dispuesto a perder una primicia que pueda transformarse en un fiasco. \emph{Página/12} fue muchas veces censurado. \emph{Página/12} no censura (Tiffenberg en Zunino, 2011: 39).
\end{quote}

Por último, intentando demostrar que la actitud de Nudler era inadecuada afirma:

\begin{quote}
El propio director del diario se comunicó entonces con Nudler para ponerlo al corriente de la situación e invitarlo a conversar sobre la manera de avanzar en los temas en cuestión. Si el procedimiento del diario fue el de siempre, la respuesta de Nudler fue inusual, especialmente en una Redacción en la que las diferencias siempre se zanjaron conversando y cotejando datos. Al día siguiente, sin mediar otra charla, Nudler dio a conocer una declaración en la que acusa a este medio de censurarlo. Esa carta se transformó en el mascarón de proa de una intensa campaña de mails que llegó a reflejarse en algunas radios y, especialmente, en el diario de negocios \emph{Ámbito Financiero} (Zunino, 2011: 39).
\end{quote}

Todo el hecho fue sumamente inusual en el campo periodístico. La asamblea de periodistas de \emph{Página/12} defendió a Nudler contra la posición de la Dirección. Como veremos inmediatamente, el hecho repercutió en la crisis de PERIODISTAS, organización nacida al calor de las luchas contra el gobierno de Carlos Saúl Menem, en el marco de las amenazas judiciales que sufría la prensa. El conflicto alrededor de Nudler presagiaba la lucha intestina dentro del campo periodístico y la fractura que se produciría, en el contexto de cambios profundos en el sistema de medios de comunicación.

Zunino denomina \emph{brulote} a la columna de Tiffenberg. Si bien puede parecer un tanto exagerada la denominación, es cierto que la respuesta ponía fuertemente en duda los valores morales de Nudler y le adjudicaba una operación para desprestigiar a \emph{Página/12}.

Una asamblea de periodistas del diario resuelve por unanimidad:

\begin{quote}
``Contra lo que se pretende insinuar en el editorial, decimos que la rigurosidad con que Julio Nudler trata la información es la única manera de respetar y sostener la credibilidad del medio en el que desarrollamos nuestra tarea. Ante una situación de tamaña gravedad, exigimos un inmediato pronunciamiento de las organizaciones que dicen defender el derecho a la información y a la libertad de prensa'' (Zunino, 2011: 41).
\end{quote}

El sujeto de la petición era sin lugar a dudas la Asociación PERIODISTAS, que reunía a muchos de los más prestigiosos profesionales de prensa del país y contaba entre sus integrantes a reputados periodistas de \emph{Página/12}. Desde su fundación en 1995, surgió como una ``Asociación para la Defensa del Periodismo Independiente ante las crecientes amenazas a la prensa y al periodismo independiente. Es una Organización No Gubernamental, independiente de las cámaras de propietarios de medios y de las gremiales de trabajadores''.\footnote{Fuente: \emph{Diario sobre diarios}. Recuperado de: http://www.diariosobrediarios.com.ar/dsd/notas/4/209-se-disolvio-la-asociacion-periodistas.php\#.XDIcXNThDMo}

Zunino reproduce en su trabajo el intercambio de mails entre reconocidos periodistas, que permiten al lector desentrañar las distintas perspectivas que sostuvieron durante el período de incubación del conflicto. Si bien los periodistas no autorizaron indicar la autoría, Zunino los reproduce íntegramente e indica quienes participaron del intercambio. A los fines de aligerar la lectura no se reproducirán los mails, pero en virtud de los conceptos desarrollados y el prestigio de los periodistas que participaron en él, consideramos de interés recomendar algunos tramos significativos que hacen referencia directa a la problemática de la censura y su contraparte: la libertad de expresión.\footnote{Páginas 37 a 52 del Libro Patria o medios, Eduardo Zunino.}

De esta manera concluyó la experiencia de PERIODISTAS. Su análisis permitirá apreciar con detalle las posiciones de lo que podríamos considerar el núcleo central del campo periodístico y que, como diría Bourdieu, puso de manifiesto el conflicto inherente a todo campo. Por otra parte, una de las características distintivas de este, es su alta movilidad o circulación entre los sujetos propiamente dichos y las casas editoriales o medios de comunicación. A diferencia de otras actividades culturales, los periodistas se diferencian entre sí, entre otras cuestiones por su capacidad de mayor autonomía o independencia en relación a sus ``productores''.

Una primera cuestión que es necesario señalar que se desprende del intercambio de correos electrónicos es el reconocimiento entre los integrantes del colectivo de la existencia de diferentes niveles de pertenencia y reconocimiento, tanto de periodistas como de medios. Uno de los integrantes del intercambio señala que:

\begin{quote}
---Preocupante lo de Julio (Nudler). Coincido en que es mucho más generalizado. Por ejemplo, cuando hace poco más de un año \emph{Clarín} despidió a una Comisión Interna íntegra y recién elegida (asunto que PERIODISTAS trató a puertas cerradas), ningún medio del grupo y ningún otro, con una única excepción, dio siquiera la noticia. No traigo a colación el tema para reflotarlo, sino sólo como ejemplo. PERIODISTAS debe analizar la manera de crear un mecanismo que permita denunciar una situación de censura y proteger al periodista que se atreve a hacerla saber. Somos una Asociación para la Defensa del Periodismo Independiente y, por lo tanto, cualquier situación de censura nos compete (Zunino, 2011: 42).
\end{quote}

La descripción revela que la Asociación dio un tratamiento diferente al despido de 20 periodistas de \emph{Clarín}, que al conflicto entre Nudler y \emph{Página/12}. Es decir, Nudler era considerado un par entre los dominantes del campo y la violación de su independencia, a la hora de publicar, tenía un mayor peso específico que el de los 20 periodistas de la Comisión Interna de \emph{Clarín}.

Otro fenómeno no menor es la consideración del medio \emph{Página/12}, en la constelación de medios, como uno periférico al que era pertinente castigar. Es posible pensar que el despido de la Comisión Interna de Clarín habrá disparado igual nivel de debate entre los integrantes de PERIODISTAS. Sin embargo, no suscitó su fractura; ni siquiera una declaración, ni una renuncia.

En términos comparativos y de acuerdo a los nombres que la integraban, podemos afirmar que la organización contaba con periodistas con autonomía. En lo que respecta a lo económico eran altamente reconocidos por sus audiencias y sus pares, lo que les permitía disponer de ofrecimientos laborales en distintos medios. Por otra parte, el intercambio epistolar deja la percepción de que los periodistas de los grandes grupos de medios o los que ya no pertenecían a \emph{Página/12}evidenciaban cierto interés punitivo sobre el diario.

Ahora bien, sobre la problemática específica de lo que estaba en discusión se aprecia un principio general de acuerdo: existe censura cuando el poder político, a través de distintos mecanismos, la aplica para impedir o limitar la difusión de información u opiniones críticas o que perjudiquen sus acciones de gobierno. No existe censura cuando quien limita o modifica una nota es un editor o eventualmente autoridad del medio.

Tomás Eloy Martínez en su carta de renuncia a PERIODISTAS afirma:

\begin{quote}
Entiendo que se apresuró al denunciar la supuesta censura a su artículo. Y coincido, también, que los diarios tienen absoluto derecho a no publicar lo que consideran inconveniente para sus intereses profesionales.\footnote{Fuente: \emph{Diario sobre diarios}. Recuperado de: \url{http://www.diariosobrediarios.com.ar/dsd/notas/4/207-el-episodio-nudler-sacude-al-periodismo-argentino.php\#.XDIdRtThDMp} con fecha 12/11/2004.}
\end{quote}

Sin embargo, unos párrafos más abajo señala:

\begin{quote}
Me parece inadmisible el artículo editorial del miércoles 27 en Página 12, que arroja sombras sobre la buena reputación periodística de Nudler sin que éste tenga derecho a réplica. Si la Asociación tiene por fin sólo defender a los periodistas del ataque de los poderes públicos y no de los abusos de otros poderes, entonces nada tengo que hacer allí.
\end{quote}

Es decir, Tomás Eloy Martínez marca una distinción entre empresa editorial y sus derechos (no publicar algo inconveniente para la empresa) y el campo de los periodistas que debe afirmar una posición de principios de defensa de la libertad de expresión. Objeta, en evidente contradicción, que solo deba defenderse a los periodistas del ataque del poder público y no de otros poderes, en este caso la empresa editora.

La acusación velada a \emph{Página/12} era que estaba siendo premiada por el gobierno con financiamiento publicitario de manera diferenciada. Lo dice, según Zunino, casi a los gritos, Magdalena Ruiz Guiñazú: ``lo que habría que discutir es el vergonzoso manejo de la pauta publicitaria oficial, con que pretenden comprarlo todo''. O el periodista Mariano Grondona de manera más sutil: ``Aquí el tema de fondo no es el Caso Nudler, sino el caso Alberto Fernández, que quiere decidir todo lo que se publica o no se publica''.(129)

El análisis de estos dichos en la actuación privada de los periodistas en su propio campo es inmensamente rico. Ruiz Guiñazú sale de su lugar de periodista para defender los intereses económicos de la empresa donde trabaja en competencia por la pauta oficial. Mariano Grondona, algo más recatado y cuidadoso, señala la existencia de censura explícita por parte del Jefe de Gabinete, Alberto Fernández, sin indicar que estaría utilizando recursos ``ilegítimos''. Tiffenberg y Granovsky, Director y Editor del diario respectivamente, rechazaron ``discutir los criterios editoriales del diario, porque si no habría que debatir los de todos y sería interminable''.

Los criterios editoriales no se discuten con los periodistas, es potestad de la empresa editora y esto es, en general, reconocido por todos. Sin embargo, todos los presentes en la asamblea y en el debate por correo electrónico, eran conscientes de que los medios realizan actividades privadas, que no pueden o no deben ventilarse. Uno de ellos afirma:

\begin{quote}
\textbf{Todos conocemos las reglas del juego}, pero eso no quiere decir que se las deba aceptar como correctas en un Estado de derecho. Aprovechemos lo de Julio {[}Nudler{]} para ver cómo trabajamos estos temas y trasmitir la preocupación a las empresas\footnote{El énfasis es propio.}. (2011: 43)
\end{quote}

Está confirmación de que el campo tiene reglas que no están escritas, pero que son conocidas por los participantes y que esas reglas operan en la Argentina, pone de manifiesto un elemento central. Además de las formas del gobierno, lo que en verdad se discutía, era la discrecionalidad en la distribución de la pauta publicitaria.

Las renuncias a la organización PERODISTAS se desencadenaron en dominó.

La periodista Silvia Naishtat, de \emph{Clarín}, argumentó en su renuncia: ``El comunicado que yo había firmado no fue el que finalmente salió. El que yo firmé tenía un párrafo en donde se le pedía a \emph{Página/12} que reflexionara sobre lo ocurrido, pero finalmente desapareció'' (Zunino, 2011: 49). Parece un débil argumento para tamaña decisión. Algo mucho más importante estaba en gestación.

Otro periodista, Jorge Lanata, presenta en su nota de renuncia la siguiente frase: ``Condeno por igual a la censura pública como a la privada, y lamento que esta asociación se haya transformado en una especie de politburó elefantiásico, lento y acomodaticio'' (Zunino, 2011: 49).

A lo que Verbitsky le responde con un ejemplo de igual tenor al de Nudler, cuando Lanata era el Director editorial del diario.

\begin{quote}
Jorge, cuando me pediste que levantara una nota sobre (Fernando de) Santibañes porque acababas de pedirle ayuda para conseguir un crédito, ¿fue censura o relación entre editor y columnista? (Zunino, 2011: 50)
\end{quote}

Jorge Lanata desviando el eje de la discusión, le responde, poniendo de manifiesto que su renuncia no estaba vinculada al caso Nudler, pero había fijado posición y no estaba dispuesto a moverse, sobre todo aprovechando su propia situación en el entramado de relaciones empresarias de medios.

\begin{quote}
No, Horacio, fue similar al comienzo de Página {[}12{]}, cuando vos le pedías plata al mismo (Enrique) Gorriarán Merlo, en el que después te cagaste. O en aquellos momentos en que hiciste campaña pro Menem o con la venta de Página a Clarín con la que, por tu propia voluntad, claro, te callaste la boca (Zunino, 2011: 50).
\end{quote}

Edi Zunino, es el director de la revista Noticias y en tal sentido uno de los periodistas significativos de la Editorial Perfil, que produce el diario dominical \emph{Perfil}. Este medio fue el único que inició un juicio contra el Estado por discriminación en la pauta publicitaria, aunque probablemente no haya sido el único medio discriminado. De hecho, en su libro, Zunino menciona los diferentes niveles de aporte publicitario a los distintos medios. Por algún motivo, el secretario de Medios, Enrique ``Pepe'' Albistur negaba aportes publicitarios a la Revista \emph{Noticias} y al diario \emph{Perfil}, que, a diferencia de los demás medios gráficos, carecían de otro tipo de aporte económico por parte del gobierno, lo que los llevó a un fuerte enfrentamiento que se manifestó en notas muy agresivas contra este.

Como manifestamos al inicio de esta tesis, la publicidad privada, en la actualidad, no llega a sostener la producción de un medio gráfico. Por ello, la fuente de financiamiento gubernamental es fundamental para la supervivencia económica de un medio, a menos que este se sostenga con otras fuentes de financiamiento u otros emprendimientos económicos para los que el medio opera como estructura de relaciones públicas. Esto último no parece ocurrir con la editorial Perfil, histórica editorial de revistas femeninas y del espectáculo, además de la revista \emph{Noticias}, de actualidad política.

El resto del libro describe de manera detallada los esfuerzos del gobierno por intervenir en la orientación de los programas de información, con una serie de medidas que van desde la ayuda a empresas ``amigas'' para la compra de medios, como la creación de diarios con periodistas cercanos, las consabidas llamadas telefónicas a los periodistas díscolos, etc. Es en este sentido que el libro de Zunino hace énfasis en que la política gubernamental de medios utilizó la ``pauta publicitaria'' como herramienta de disciplinamiento de los medios y que esta era la principal forma de violación de la libertad de prensa.

El libro tampoco oculta operaciones en igual sentido llevadas a cabo por Francisco De Narváez dentro de su canal de televisión. Es decir que, más allá de la centralidad que adopta el gobierno dentro de la narración de Zunino, su valor radica en que revela un esfuerzo por desmitificar la supuesta independencia. En este sentido, su postura es original o por lo menos diferente a la de los periodistas dominantes del campo, que cuando hablan de libertad de expresión ``no se ensucian las manos con dinero''.

Pablo Sirvén (2011: 4, versión digital), en \emph{Perón y los medios de comunicación}, afirma desde el prólogo:

\begin{quote}
La apurada sanción de la nueva ley de medios, el hostigamiento constante al Grupo Clarín y la historieta en torno de Papel Prensa me convencieron de que había llegado la hora de reeditar Perón y los medios de comunicación. Es que, a pesar de su antigüedad por narrar episodios sucedidos seis décadas atrás e investigados por mí hace ya más de un cuarto de siglo, recobraban inesperada actualidad.
\end{quote}

Sirvén realiza una hipérbole histórica, a partir de una implícita petición de principios sobre la veracidad de su libro de 1984. Intenta demostrar que la violación de la libertad de expresión se encuentra, de alguna manera, en la naturaleza misma del peronismo. Si recordamos la perspectiva de Claudio Díaz, todo el libro de Pablo Sirvén parece darle la razón.

\begin{quote}
La relación de los gobiernos peronistas con la prensa y, particularmente, con los medios audiovisuales ha sido, por lo general, tempestuosa (\ldots) El kirchnerismo enlaza esta corriente novedosa combinado con un \emph{déjàvu} de lo realizado por los padres fundadores del justicialismo a mediados del siglo pasado. Con un plus a su favor: lo que entonces parecían \textbf{recortes arbitrarios contra la libertad de expresión} hoy se presentan como necesarias denuncias hacia los ``excesos de las corporaciones mediáticas'', que tienen en sus propios seguidores un núcleo duro de fieles creyentes, y un núcleo blando, mucho más amplio, que sin creer a rajatabla todo lo que dice el Gobierno, ha comenzado a descreer del rol del periodismo tal cual lo conocimos, \textbf{es decir como único intérprete autorizado de la verdad} (Sirvén, 2011: 5, 20, versión digital).\footnote{El énfasis es propio.}
\end{quote}

Hay momentos en que el discurso de los integrantes de un campo, cualquiera sea, revela la naturaleza profunda de sus convicciones y deja expuesta sus reales creencias. La última frase de Sirvén en el párrafo anterior no puede pasar desapercibida. Para Sirvén el periodismo es el ``único interprete autorizado de la verdad'' o por lo menos así fue el periodismo, ``tal cual lo conocimos'' y es esta la creencia profunda que tiene el núcleo del campo periodístico, y que perciben socavada por el kirchnerismo, tanto aquellos que denominamos \emph{defensores} como los \emph{profesionalistas.}

Sirvén, al igual que Saguier, seis años antes, en el Seminario de \emph{Clarín} duda acerca de cómo deslegitimar el discurso del Poder Ejecutivo, sin violentar el derecho a la libre expresión. El gobierno emite mensajes, proyecta agendas, disputa el espacio de comunicación y de acceso a la opinión pública. La prensa se encuentra, según el autor, ``con un ida y vuelta insolente que la cuestiona y le mueve el piso. Si esto empeorará o mejorará la libertad de expresión solo el paso del tiempo lo dirá'' (Sirvén, 2011: 6, versión digital).

Sin embargo, su fe en el credo periodístico lo lleva a considerar que el kirchnerismo convirtió a personas honorables en serviles inmorales.

\begin{quote}
El peor legado que dejará la era kirchnerista en materia de libertad de expresión no será el diario hostigamiento verbal a medios y a periodistas desde lo más alto del poder, inédito en su persistencia desde el fin de la última dictadura militar, sino \textbf{la captación mediante estímulos económicos} o la prédica pertinaz, casi a manera de adoctrinamiento cerril, de personas y personajes que en otros contextos supieron ser honorables y lúcidos (Sirvén, 2011: 38, versión digital).\footnote{El énfasis es propio.}
\end{quote}

En el párrafo anterior, Sirvén se refiere a colegas, que supieron ser honorables y lúcidos, pero que ya no lo son. En otras palabras, comienza a expulsarlos del ``paraíso periodístico'' y lo explicita más adelante:

\begin{quote}
Más llamativo resulta todavía que ese dócil grupo en aumento de periodistas (o, más bien, ``prenseros'') oficialistas festeje y avale alborozado cada uno de esos ataques indiscriminados contra la prensa. ¿Es, tal vez, un tácito reconocimiento de que han dejado de ser periodistas para convertirse en meros divulgadores revulsivos o propagandistas entusiastas de la prédica oficial? (Sirvén, 2011: 39, versión digital).
\end{quote}

En síntesis, el libro de Pablo Sirvén no se dedica tanto a la relación del peronismo con la prensa, como a su relación con la libertad de expresión. El derecho a la libertad de expresión se encuentra, según el autor, en constante peligro durante los gobiernos peronistas. Si bien no es tema de esta tesis, Pablo Sirvén desconoce en su libro hechos trascendentes, en defensa de la prensa en el período peronista como son la Ley del Canillita, la creación del Sindicato de Prensa, el Estatuto del periodista, la creación del Instituto de Verificación de la Circulación (IVC), la creación, organización y centralización de la distribución de diarios y revistas (CDR).

Así como Pablo Sirvén expresa de manera cabal al campo periodístico de mayor centralidad, Orlando Barone, que también proviene del mismo círculo, por algún motivo decidió salirse y convertirse en un expatriado. En su libro \emph{K Letra bárbara} reflexiona sobre su propio escape, el día 5 de octubre de 2008, a partir del conflicto entre el gobierno y el sector agrario a raíz de la Resolución 125.

Barone pone énfasis, al igual que Sirvén, en el problema de la libertad de expresión. Sin embargo, desmenuza el discurso oficial del campo periodístico y logra recuperar la profunda relación entre el discurso y la legitimidad de la producción periodística. Dicho desde el interior, desde las entrañas mismas del campo, sus palabras tienen un peso específico mayor. Barone reflexiona a partir del significado del Proyecto de Ley de Servicios de Comunicación Audiovisual afirmando que:

\begin{quote}
El antioficialismo, común hasta hoy en los grandes medios dominantes, es oficiado por periodistas que graciosamente viven quejándose de que no tienen esa libertad, y entonces hay que creerles y condolerse de ellos. Se asumen como vacas sagradas, y chillan, y patalean, a dúo y armonía con sus empresas si éstas se sienten hipotéticamente damnificadas por el más tenue airecillo crítico o si se las expone en sus negocios y manipuleos informativos con el culo sucio a la vista del público (\ldots) Son esa clase de periodistas los que rechazan la chance de que la nueva Ley de Medios se extienda y se haga más variado y más libre el ejercicio del oficio. Entonces se arrebujan, escandalizados de que un nuevo tiempo de libertades venga a salvarlos. ¿De qué?, si no se sienten en peligro sino, por el contrario, protegidos (\ldots) Y entonces urden amenazas, se victimizan, deslizan sospechas de corrupción y de sobornos para enlodar el resultado legitimado en el Congreso. Pero aun cuando se retarden los cambios con sinuosidades leguleyas, algo está cambiando. La sociedad ya no ignora quiénes somos los mensajeros ni el mensaje ni el soporte. Se ha pisado el hormiguero. Ha sido como obligar a los medios, amparados en una escenografía privada, a descorrer las cortinas y hacerse públicos. A mostrarse sin el falso escudo de la libertad de prensa. A mostrar la hilacha, el ovillo, el cargamento (Barone, 2011: 20, 28, 34, versión digital).
\end{quote}

Barone, de manera similar a Claudio Díaz, reacciona percibiéndose asqueado de ver y verse sumido en la maquinaria de prensa y trata de huir de ella con la esperanza de que la nueva Ley de Medios dé lugar o produzca un nuevo periodismo. Barone (2011: 54, versión digital), citando de memoria a Roger Wolfe, recuerda:

\begin{quote}
Realmente no sé/ lo que me pasa/ No es asco/ No es hastío/ No es abulia/ No es cansancio/ No es indiferencia/ Son todas esas cosas/ y no es ninguna/ Es como si el mundo/ se me hubiera/ parado/ encima. En mi caso no es para tanto. A mí lo que se me paró encima es el mal oficio que he ejercido tantos años creyéndolo bueno. Así de simple y candoroso. Es menos pesado que el mundo. También me acuerdo de una frase suya: ``Tienes el derecho a expresar libremente todo aquello que te esté permitido decir''.
\end{quote}

Barone huye del campo del periodismo y lo narra con poco pudor y hermosa prosa. Es interesante observar que su renuncia es temporalmente coincidente con el proceso que vive Díaz. A diferencia de otros periodistas que se quedan dentro del campo a ``dar batalla'' por la legitimidad, éstos, tal vez por estar dentro de medios centrales, no encuentran espacio y en su huida revelan que todo aquello en lo que creían \emph{se desvanece en el aire}.

El mismo año se publica, \emph{Periodismo de infantería}, de Virginia Márquez y Aníbal Ces\footnote{El libro no tiene registro de la fecha de publicación, sin embargo, los autores dejaron registro en el texto del momento en que fue escrito, alrededor de mediados de 2010, por lo que es muy probable que su edición corresponda al año 2011. Lo que es indubitable es que su redacción se inscribe en el período que consideramos de fractura y crisis del campo periodístico.}. El libro intenta dar cuenta, 10 años después, de la ``lucha de los empleados de \emph{Clarín} por llevar adelante el proceso de sindicalización en la empresa''\footnote{El que diera por resultado el despido de los 20 delegados, citado anteriormente.}. El libro, compuesto por una serie de materiales, es un conjunto coral desordenado de relatos, documentos y fotografías que dan cuenta de la derrota de un conflicto gremial frente a una de las empresas más poderosas de la Argentina. Si bien el libro narra hechos ocurridos en el pasado, anteriores al conflicto con el sector agrario, lo rico del trabajo es que el recuerdo de esa ``lucha'', vista por sus protagonistas, opera desde dentro del conflicto del campo periodístico y los compiladores lo reactualizan a la luz de la nueva situación.

Los autores realizan una petición de principios: ``Clarín se ha caracterizado por el rechazo sistemático a la agremiación de sus empleados'' (Márquez y Ces, 2011: 15). A partir de este principio que guiaría el accionar del diario desde el punto de vista empresario, los autores enhebran elementos diversos que demuestran la subordinación de los intereses periodísticos a los empresarios:

\begin{quote}
Clarín negocia con Menem la destitución de Liliana López Foresi del noticiero de Canal 13, a cambio del despido de Patricio Kelly de Canal 7 (\ldots). También durante las elecciones presidenciales entrevisté a los tres candidatos mejor posicionados. Cavallo me recibió en su bunker de la calle Tagle. Duhalde acudió en persona al diario y aprovechó para saludar a quien se le cruzara, asumiendo su rol en la campaña. A De la Rúa se le permitió responder por escrito. Esa diferencia de trato me dio la certeza: el líder de la Alianza {[}De la Rúa{]} era el candidato elegido por el Holding (\ldots). Dicen que las casualidades no existen en política. Tal es así que el mismo 4 de noviembre, día que Clarín procedía al despido de 117 compañeros, se reunía con Magnetto, con el entonces presidente De la Rúa, para ratificar el Decreto que desregulaba la venta de diarios y revistas. Decreto por el cual, el diario, se hacía con la herramienta necesaria para concretar la asfixia económica de los canillitas (\ldots). Luego de lo cual, Magnetto, le habría solicitado a Pirillo, que dejara de publicar notas sobre los casos de apropiación y tráfico de bebés, alegando que ese era un tema que dañaba en especial, a la ``señora''. Él es dueño de La Razón, dice haberle confesado esta conversación a dos expresidentes, producto de lo cual, habría resultado en su particular perjuicio (Márquez y Ces, 2011: 29, 104, 106, 107).
\end{quote}

\subsection{4. 2. 2. Periodismo de Guerra}

\textbf{Año 2012}

Luego de la reelección de Cristina Fernández de Kirchner, con un triunfo electoral significativo, a partir del año 2012 el \emph{Grupo Clarín} comienza acciones periodísticas tendientes a debilitar al gobierno, desarrollando una serie de prácticas hoy denominadas \emph{fakenews.}\footnote{Las~noticias falsas, conocidas también con el~anglicismo~\emph{fakenews}, son un tipo de~información que se presenta como real y veraz, suelen ser difundidas a través de~portales de noticias, prensa escrita,~radio,~televisión~y/o~redes sociales~y tienen por objetivo producir~desinformación (antiguamente denominadas operaciones de prensa).}El grupo empresario se sintió atacado por las medidas de gobierno, fundamentalmente por el retroceso en las concesiones en telecomunicaciones y la promulgación de la Ley de Servicios de Comunicación Audiovisual, que luego del triunfo electoral contaba con altas chances de aplicarse. También, desde el punto de vista del control de la agenda, el gobierno mantenía la presión sobre el juicio por \emph{Papel Prensa} y por los hijos adoptivos de Ernestina Herrera de Noble, cara visible del grupo Clarín.

Una de las medidas más importantes que toma el \emph{Grupo Clarín}, en términos mediáticos, es el lanzamiento del proyecto televisivo \emph{Periodismo para Todos} (PPT), a cargo del conductor y periodista Jorge Lanata, con guion de Marcelo Birmajer. Jorge Lanata ha sido y sigue siendo un exponente prestigioso y reconocido del campo periodístico, tanto por sus colegas como por su audiencia. Si algo hacía falta para completar su halo de prestigio, fue la publicación de su biografía a cargo del periodista Luis Majul. Tanto la biografía como el proyecto televisivo se inscriben en lo que publicitariamente se denomina, \emph{relanzamiento de la marca}.

El 15 de abril del año 2012 comienza el programa y en noviembre del mismo año sale el libro de Majul, precedido por una gran campaña publicitaria, casi en simultáneo con el lanzamiento del proyecto televisivo, a tal punto que casi ningún argentino desconoce algunas de las ``anécdotas'' que cuenta Majul sobre Lanata, como su intento de suicidio o su consumo de cocaína.

\hl{En síntesis, durante el año 2012, encontramos un único producto editorial dentro de nuestro universo de análisis: \emph{Lanata. Secretos, virtudes y pecados del periodista más amado y más odiado de la Argentina,} de Luis Majul. Según pudo estimarse, el libro vendió aproximadamente 100.000 ejemplares, con una primera edición de 40.000 ejemplares y seis más de 10.000 cada una.}\footnote{La información fue extraída de diversas fuentes: Entrevista a un editor de Planeta.

  Revista \emph{Gatopardo}, noviembre 13 de 2013 https://gatopardo.com/reportajes/siempre-periodista/}

\hl{Llamativamente, por tratarse del libro biográfico de un periodista, hay pocas referencias a su trabajo periodístico, a su ``lucha por la libertad de expresión'', la independencia etc. La única referencia es al conflicto suscitado entre Julio Nudler y \emph{Página/12}, al que hicimos referencia cuando analizamos el libro de Zunino. El texto no aporta nueva información, pero es el ejemplo que toma Majul para construir la imagen de un periodista cabal e insobornable.}

Dos ejes recorren con claridad el texto. El primero presenta algunos aspectos espectaculares de la vida de Jorge Lanata, sus intentos de suicidio, sus adicciones y su particular modo de vida en relación a sus parejas, hija y familia en general. El segundo eje recorre elementos de interés para una psicología del periodista, su vanidad y egocentrismo, excentricidades, despilfarro, etc. Casi no hay recuerdo de su paso por \emph{Página/12}, ni de su eventual relación con exintegrantes del Ejército Revolucionario del Pueblo (ERP). El libro fue un producto eminentemente publicitario a fin de re posicionar al personaje e introducirlo de manera no conflictiva en el ``mundo Clarín''.

\textbf{Año 2013}

En el año 2013, desde el punto de vista editorial, los periodistas dominantes del campo, pertenecientes a los medios de mayor audiencia, dieron inicio explícito a la operación de expulsión de quienes habían sido parte de la crítica al periodismo. El fenómeno es relevante porque durante todo el período considerado (2009--2015) se produjeron 7 títulos biográficos de destacados periodistas, dos de Jorge Lanata, dos de Víctor Hugo Morales, dos de Horacio Verbitsky y una de Héctor Timerman. De los cuatro periodistas citados, durante el año 2013, el campo periodístico dará cuenta de dos de ellos: Timerman y Morales.

Comenzaremos por el texto de Pablo Sirvén, \emph{Converso}, referido a Víctor Hugo Morales. Título de fuerte connotación negativa para definir a un colega, un integrante del espacio periodístico. Suele utilizarse de manera ofensiva hacia alguien que reniega de sus creencias, alguien que ha aceptado una ideología o una creencia religiosa que antes no profesaba, por ejemplo ``judío converso'', ``comunista converso'', etc. La introducción escrita por Jorge Lanata, es aún más dura que estas definiciones.

\begin{quote}
El peor enemigo del doble discurso es el tiempo: lo desnuda, dejándolo en evidencia. Este libro habla del tiempo y de este tiempo: la historia del converso es siempre la historia de Mefisto (\ldots), el \emph{Mephisto}, de István Szabó escrito por Klaus Mann, película ganadora del Oscar en 1981, protagonizada por Klaus Maria Brandauer. El film relata una historia\ldots,la de Hendrik Hoefgen, antes autoproclamado como ``actor del pueblo'', bolchevique, que llega a Berlín durante los comienzos del nazismo y, convencido de que ``el arte continúa el estado puro por sobre el resto de las cosas'', pacta con los nazis hasta convertirse en director del Teatro Nacional prusiano, a costa de dejarse usar como objeto propagandístico del régimen (Lanata 2013: 7).
\end{quote}

El esfuerzo de demostración de Lanata, lo lleva a comparar a Víctor Hugo Morales con Hendrik Hoefgen y al gobierno de Cristina Fernández de Kirchner con el nazismo.

Pablo Sirvén, periodista del diario La Nación, dedicó un esfuerzo significativo a \emph{Converso,} subtitulado \emph{Historia íntima de la brutal transformación personal, profesional y económica del relator más polémico de la Argentina}. El libro trabaja sobre tres ejes.

El primero se halla explícito en el título: demostrar que Morales se convirtió al kirchnerismo en el momento en que este comenzó a atacar al \emph{Grupo Clarín} y a los medios dominantes, tras el impulso a la Ley de Servicios de Comunicación Audiovisual (26.522). Este material permite observar el cambio de perspectiva que adopta Morales con respecto al gobierno, en relación con sus consideraciones anteriores al año 2009.

El segundo eje es que Morales ha mentido u ocultado la verdad en reiteradas oportunidades y en consecuencia es un periodista que no debería tener credibilidad.

Por último, el tercer eje señala que Víctor Hugo Morales no ha sido consecuente defensor de la libertad de expresión. Por ser este uno de los puntos centrales de análisis de esta tesis, nos centraremos en este último eje que consideramos constitutivo del campo, y nos ceñiremos a él en el desarrollo de la crítica de Sirvén a Morales.

Para Sirvén, mentir u ocultar la verdad van de la mano con la poca dedicación puesta por Morales en la defensa de libertad de expresión. A pesar de que el libro cuenta con casi 400 páginas no es posible hallar ejemplos importantes, a nuestro entender, sobre la debilidad de Morales en la defensa de la libertad de expresión. Incluso cuando se refiere a su actuación durante la dictadura en Uruguay, uno de los ejemplos que intenta demoler la credibilidad de Víctor Hugo Morales, no es fácil hallar elementos que lo corroboren. La imagen que pretende construir Sirvén de Morales es la de alguien que aprecia las dictaduras y desvaloriza las democracias.

Vayamos de a poco.

\begin{quote}
VHM: España\footnote{Se refiere al equipo de fútbol español.} no me gustó, no me divirtió, no me entretuvo. Para relatarla, fue insoportable (Refiriéndose al Mundial del 2010, en que España se alzó con la copa) (Sirvén, 2013: 81).
\end{quote}

Interpreta Pablo Sirvén (2013: 81):

\begin{quote}
Se sinceró tras regresar a la Argentina. Sudáfrica 2010 fue una pesadilla en la que se encarnaron las peores cosas que a Víctor Hugo Morales, como persona y como relator, no le daban placer. Para colmo, los años no habían pasado en vano y ya no lo esperaban experiencias excitantes como en mundiales anteriores, donde todavía sumaba fama, potencia y juventud. Todo lo contrario a la mala experiencia sudafricana le pasó en el primer mundial que relató en su vida. Si la libertad y el regocijo que se vivía en la Sudáfrica post apartheid lo aburrió, la Argentina de 1978, sumida en una de las dictaduras más feroces que recuerde América latina, lo fascinó.
\end{quote}

Por si hiciera falta aclararlo, Víctor Hugo Morales, se refiere al partido que España jugó contra los Países Bajos, no a la situación política de Sudáfrica. En relación al Mundial de 1978, Sirvén (2011: 82) encuentra entre las frases más criticables de Víctor Hugo Morales, la siguiente:

\begin{quote}
Morales se relamió porque los dirigentes de los clubes, que le ocasionaban tanta alergia, habían sido desplazados por los más expeditivos uniformados que tomaron en sus manos la tarea de organizar el certamen internacional. Desde la soberbia fiesta de inauguración --- escribió en noviembre de 1979 en su autobiografía ---, hasta el último minuto que duró el torneo, estuve asombrado por el despliegue inteligente, unificado, fervoroso, de los argentinos en torno al evento.
\end{quote}

Luego sobre su relación con la dictadura del Uruguay pregunta:

\begin{quote}
¿A qué distancia exacta Víctor Hugo Morales se mantuvo de ser colaboracionista o heroico durante la dictadura militar en Uruguay? Esa es la gran pregunta que todos se hacen, sólo que algunos la formulan en voz más alta que otros. Dos ex presidentes de la República Oriental del Uruguay, consultados por este libro, \textbf{no recuerdan que Víctor Hugo Morales haya sido heroico, ni mucho menos, durante la dictadura sufrida por ese país entre los años 70 y 80.} --- Que haya sido perseguido, no me consta ---dijo el doctor Alberto Lacalle, el dirigente del Partido Blanco que gobernó entre 1990 y 1995. \textbf{Nunca tuvo una actitud política}. Su familia era del Partido Nacional. ---Víctor Hugo Morales es un gran periodista, tuvo un notable éxito tanto en Uruguay como en la Argentina y su capacidad es indiscutible. Luego ocurrió que ya instalado en la Argentina asumió una actitud como de resistente a la dictadura uruguaya que simplemente no es verdad. Recibió de la dictadura algunos actos muy importantes de apoyo. --- ¿Se refiere al espaldarazo que le dieron los militares cuando los dirigentes del fútbol uruguayo le prohibieron ser relator? ---Exacto. Es un hecho objetivo. La realidad es esa. Ni tanto ni tan poco. No puedo decir que Víctor Hugo fuera parte de la dictadura ni tampoco claramente todo lo contrario, pero tuvo buenas relaciones con los militares. Eso es un hecho. --- ¿Qué opina de la posición tan fervorosa que ahora tiene sobre el kirchnerismo? --- Bueno, eso se mira un poco más a la distancia porque Víctor Hugo no tiene una presencia política en el Uruguay, pero a la gente le ha llamado la atención, sobre todo el entusiasmo con que asume el kirchnerismo (Sirvén, 2013: 105).\footnote{El énfasis es propio.}
\end{quote}

No es difícil observar el intento de Sirvén por comparar la dictadura uruguaya con el kirchnerismo. Más abajo avanza Sirvén (2013: 119):

\begin{quote}
Y, al fin, hizo un reconocimiento valioso de su pasado auténtico: ``como se verá no hay heroísmos significativos en esta historia''. Exacto: ``no hay heroísmos significativos'' en la historia de VHM durante la dictadura en su país. Fue una etapa en la que no se privó de trabajar en los mejores lugares y de escalar hasta puestos relevantes. Nada que cuestionar en ese sentido.
\end{quote}

El gobierno dictatorial de Uruguay respalda a Morales contra una suspensión que le impuso la Asociación Uruguaya de Fútbol. Sirvén (2013: 123) lo narra de esta manera:

\begin{quote}
El general Julio César Rapela, jefe del Estado Mayor Conjunto de las Fuerzas Armadas, también se mostró inquieto con que el silencio impuesto a Morales en los micrófonos futboleros pudiese afectar de alguna manera la imagen del país en el exterior si llegaba a convertirse en tema de debate internacional. Consciente de que era necesario parar la bola de nieve que comenzaba a armarse, el 19 de julio de 1978 el presidente de Uruguay, el dictador Aparicio Méndez, dejó, al fin, sin efecto la resolución de la AUF. Víctor Hugo pudo entonces volver triunfante a sus inconfundibles ``ta-ta-ta'' al micrófono. El castigo impuesto en asamblea por la AUF había sido por 45 días fuera del aire y tan sólo llegó a estar silenciado una semana. El vicepresidente de la AUF, Oscar Schiaffarino, confirmó para Relato oculto que ``la decisión no se consultó con el gobierno militar''. En Mundo color, el periodista deportivo agradeció así a las autoridades castrenses que lo reinstalaron frente al micrófono: ---Sentí una cierta vergüenza por haber distraído a nuestros gobernantes en un tema infinitamente menor al que les ocupa día a día. El Gobierno nacional no me ha condecorado, ni respaldado. Fui respaldado en mi prédica. Apenas (pero eso sí, grandemente) fui defendido en los mismos derechos que usted goza\ldots{} ¿Cuáles eran los derechos de los que ``gozaban'' los uruguayos por ese entonces? \textbf{La mayoría de las garantías constitucionales estaban arrasadas y en especial la libertad de expresión.} La versión VHM 2012, menos obsecuente que entonces y más envalentonada, le encontró otra vuelta de tuerca para justificar cómo los propios militares anularon la prohibición de la Asociación Uruguaya de Fútbol al afirmar que ``los milicos también se cagan'' y que no eran ``tan tontos''. Y agregó:---Cuando vieron que pasaban los días envueltos en un quilombo estúpido, traicionaron a los dirigentes, levantaron la prohibición y los pobres tipos, mis enemigos, renunciaron todos; de rabia más que nada. En esos días en los que yo no podía entrar al estadio venían Edgar Perea, de Colombia, con Fernando Niembro, de la Argentina, muchos más, y trasmitían desde mi cabina. Perea gritaba: ``¡¿dónde estás, Víctor Hugo?!'' Entonces dieron vuelta todo y se mandaron una de película. Yo me estaba jugando mi carrera, pensaba. Luché a brazo partido. Los fui a ver. Me victimicé. \textbf{Victimizarse, recuérdese, es una de sus estrategias preferidas desde siempre}. \footnote{El énfasis es propio.}
\end{quote}

Un último ejemplo a los efectos de mostrar el esfuerzo de Sirvén por sacar del campo a Víctor Hugo Morales:

\begin{quote}
Filippini (jugador de futbol uruguayo) es protagonista del pasado uruguayo de VHM porque una nota radial que le hizo en aquel campeonato del 76, tras un inesperado empate con Nacional, fue el detonante de uno de sus encuentros ríspidos con los militares de la dictadura. --- ¿A quién se lo dedica? ---le preguntó el relator sobre el empate 2-2 de visitante contra Nacional en el Centenario. --- Víctor Hugo, quisiera dedicar estos goles a mi hermano y a todos sus compañeros, que están presos en el penal Libertad --- dijo Filippini. ---Muy bien recibidos, vayan sus saludos ---respondió escueto Morales. Casi un reflejo, tratando de salir lo más pronto posible de algo que, en efecto, lo comprometió. Al día siguiente, el locutor fue citado por las autoridades militares y tras esperar cuatro horas, un coronel lo recibió de mala gana y le hizo escuchar la entrevista, que tenía grabada. --- ¿Qué me puede decir de esto? --- lo interpeló agrio, sobre su diálogo al aire con Filippini. --- Mire, lo que pasa es que uno no escucha el reportaje, está atendiendo 17 cosas de la transmisión, lo dedicó, y la verdad es que no tomé nota de lo que estaba diciendo. Además si hubiera tomado nota, ¿qué le podía decir?\ldots{} nada. Al final me salió un formulismo ---se cubrió Víctor Hugo. --- Bueno, usted me va a entender, tiene tarjeta amarilla ---lo apercibió el coronel. Cuando Filippini se enteró de que Víctor Hugo había tenido problemas lo llamó para disculparse. --- No, por favor, no tenés nada que ver, simplemente te están buscando, te conviene hablar con un coronel que está al mando del operativo --- le recomendó. Cuatro días después, el futbolista llamó al militar que VHM le había recomendado. Le explicó que tenía un hermano y amigos presos en el penal de Libertad, y por eso les había dedicado el gol. --- El operativo de las fuerzas conjuntas termina acá ---respondió lacónico el coronel. A pesar de eso, aquella frase firmó su retiro del fútbol y Filippini tuvo que abandonar su carrera por orden de los militares. Ocurrió en la cuarta fecha del campeonato en el que Defensor Sporting hizo historia (Sirvén, 2013: 130, versión digital).
\end{quote}

Por último, veamos el retrato que hace Sirvén del Víctor Hugo Morales previo al cambio de relación con el campo periodístico.

\begin{quote}
Ahora vayamos con el tiempo hacia atrás para recuperar al VHM pre conversión: ---Me gustaría irme con el botín. Con lo que he juntado. Y hacer un retiro más tranquilo. No es descabellado pensar en pegar la vuelta. Víctor Hugo Morales se sinceró como nunca ante la revista deportiva Ovación, que se edita en Montevideo: ``Por segunda vez desde el año pasado tuve la fantasía del regreso; porque yo tengo mucho quilombo, mucha pelea.'' Corría el mes de agosto de 2009. El ``nuevo'' VHM ya había comenzado a efectuar las primeras maniobras del viraje que lo llevarían a experimentar una aguda metamorfosis. Una transformación que daría muchísimo que hablar. Para bien y para mal. Una nueva y crucial etapa de su vida estaba por empezar. Si en ese momento hubiese pegado la vuelta a su país, su recuerdo en la Argentina habría sido casi inmejorable. Se lo hubiese echado de menos por lo mucho y muy bueno que dio con sus vibrantes relatos a la pasión del fútbol y por sus esfuerzos permanentes por popularizar la cultura y los buenos espectáculos. Tal vez sólo los kirchneristas más recalcitrantes, todavía con tirria porque se había puesto del lado de los intereses ruralistas durante el conflicto con el campo, habrían celebrado su alejamiento. Si, en aquel instante, Morales hubiera clausurado su brillante foja de servicios de casi tres décadas en los medios de la Argentina, su valiosa huella podría haberse conservado intacta y añorada durante mucho tiempo. Quizá para siempre. Pero Víctor Hugo Morales estaba por dar un inesperado volantazo que lo convertiría en el José María Muñoz del kirchnerismo (Sirvén, 2013: 153).\footnote{Es pertinente recordar que José María Muñoz fue el relator oficial del Mundial de Futbol de 1978, durante la dictadura militar, por lo que nuevamente Sirvén asocia al kirchnerismo con una dictadura y a Morales como su relator.}
\end{quote}

Visto a la distancia el objetivo fue logrado, la firma de Morales hoy vale menos que hace 10 años. Ha dejado de participar en los relatos de fútbol de los principales canales de televisión y es un exilado de los programas deportivos.

Si bien en el año 2013, Héctor Timerman era canciller del Gobierno de Cristina Fernández de Kirchner y no periodista, el libro de Gabriel Levinas, tenía la pretensión de desacreditarlo como tal. Es así que comienza su libro y dedica más de la mitad del mismo a analizar la condición de periodista de Héctor Timerman.

La estructura argumental que utiliza Levinas en el caso Timerman, es similar a la que luego utilizará con Horacio Verbitsky en \emph{Doble Agente}. A su vez, Sirvén la reproduce con Víctor Hugo Morales.

Levinas (2013), en \emph{El pequeño Timerman}, comienza mostrando la debilidad de convicciones en la democracia del joven Héctor.

\begin{quote}
La definición y el oportunismo editorial eran inconfundibles; La Tarde contribuyó al clima y apoyó sin rubores el golpe militar que se produjo a los ocho días de su lanzamiento. El título de tapa de la primera edición del diario de Héctor Timerman fue: ``Isabel, ¿sigue gobernando?''. Otro rezaba: ``Costo de vida: nadie lo para''. Abajo, un perseguido por las bandas lopezreguistas contaba sus desventuras bajo el título: ``La AAA quiso asesinarme''\ldots.Y en la del 23, día previo al golpe, adelantaba: ``Se afirma en medios responsables que asumiría el poder un gobierno militar; las FF.AA. se disponen a ocupar el vacío de poder para terminar con el caos''. La Tarde del 24 de marzo, a horas de consumado el golpe, elogiaba la ``absoluta normalidad, serenidad y paz pública'' de la jornada. El título principal era: ``Prestó juramento la Junta Militar, para reorganizar a la Nación''. El diario elogiaba a los tres miembros de la Junta y, en especial, ``la prolongada y brillante foja de servicios'' del almirante Eduardo Emilio Massera (Levinas 2013: 41) .
\end{quote}

Más adelante, señala:

\begin{quote}
El 29 de marzo, día en que Videla juraba como presidente, el diario de Héctor hizo una enternecedora descripción del jefe del Ejército, que anunció desde la tapa. ``Exclusivo: la infancia del Teniente General Videla''. El reportaje incluía, en la doble central, el testimonio de la maestra de cuarto grado del nuevo presidente. La maestra, fotografiada leyendo La Tarde, decía, entre otras cosas: ``Yo les decía a mis alumnos, `si Sarmiento, siendo de origen tan humilde, llegó a presidente, por qué no puede llegar a serlo uno de ustedes'. Se me dio con Jorge (Videla) y le aseguro que me alegro'' (Levinas 2013: 43).
\end{quote}

Inicialmente, Héctor Timerman, según Levinas, apoyó al golpe militar y consecuentemente a la dictadura. Luego, habría demostrado falta de solidaridad con un colega en situación similar a la de su padre, al no informar a la familia de Juan Ramón Nazar, la suerte corrida por el director de un diario homónimo de Trenque Lauquen:

\begin{quote}
Pero, a diferencia de Timerman, que alternó períodos de desaparición y blanqueo y cuyo caso resonó en la prensa mundial, Nazar no estuvo en ninguna lista de detenidos a disposición del Ejecutivo. Durante trece meses y dos días fue oficialmente un desaparecido. Y durante esos trece meses ningún miembro de la familia Timerman se comunicó con la familia Nazar para hacerle saber que Jacobo había visto a Juan Ramón, su colega y director de un diario también llamado La Opinión, en el centro de detención clandestino Puesto Vasco (Levinas, 2013: 68).
\end{quote}

Nuevamente en la página 73:

\begin{quote}
El 24 de agosto, anteúltimo día de la misión de la SIP en Buenos Aires, fue liberado Juan Ramón Nazar, el exdirector del diario de Trenque Lauquen que había compartido diez días de prisión con Jacobo Timerman en un centro de detención clandestino. Nazar dejó de ser un desaparecido al cabo de trece meses durante los cuales su familia no recibió ninguna llamada, ningún aviso de los Timerman para decirles que Jacobo lo había visto en Puesto Vasco.
\end{quote}

Por último, según Levinas, Timerman miente sobre su pasado, al igual que lo hace Víctor Hugo Morales, para Sirvén.

\begin{quote}
``Yo tenía 22 años ---escribió el hoy canciller (en realidad, en 1977 cumplió 24)--- y lo que más ansiaba era una intimidad normal con Anabel''. El espacio íntimo que podían compartir era el departamento de soltero de San Telmo, que, dijo, ``había usado a menudo para ocultar disidentes''. (Levinas 2013: 70)
\end{quote}

Según Levinas, Timerman también miente cuando dice:

\begin{quote}
``Cuando regresé a la Argentina en 1989, trabajé mucho en derechos humanos y me ganaba la vida con la publicidad, si bien de una manera ocasional al principio y luego más regularmente, comencé a escribir notas de opinión. Pero entré de lleno en el periodismo cuando me vino a ver un grupo de periodistas amigos que querían iniciar un proyecto, me interesó y así fundamos la revista \emph{Trespuntos}'' (Levinas, 2013: 96, versión digital).
\end{quote}

Levinas (2013: 96) afirma:

\begin{quote}
La gestación de esa revista fue, en verdad, algo distinta. Atrás, como siempre hasta entonces en la vida de Héctor, estaba Jacobo (\ldots). Cuando, tras varios meses de reuniones, redondearon el proyecto, el viejo Timerman les dijo a Cerruti y Acuña: ``Digan que yo pongo la plata, y les doy mi apellido''. Había que incluir al pequeño Timerman, que tenía el apellido, para conseguir apoyo comercial y atraer a los futuros colaboradores. Por eso ``les hicimos creer a todos que Héctor era el jefe'', dice Acuña, y recuerda que en ese momento el hijo de Jacobo era relacionista de Sprayette, una empresa de ventas por TV.
\end{quote}

Desacreditar a un periodista sobre la veracidad de sus afirmaciones es, tal vez, una de las formas más eficaces de denigrarlo como tal. Es por eso que en estos trabajos el énfasis está puesto en la falta de veracidad de sus afirmaciones, aún en el caso de que tal información sea poco relevante.

Daniel Muchnik puede ser considerado un exponente del grupo que denominamos \emph{periodistas defensores}. Formado en una redacción, reconoce no haber militado ni haber tenido convicciones políticas intensas. Retirado ya del diario \emph{Clarín} introduce, en este año, el libro \emph{Aquel periodismo, medios y periodistas en la Argentina 1965-2012}. Un intento de sociología periodística que articula las distintas situaciones que, según su perspectiva, transitaba el país con las formas y características que adoptaba el periodismo y los periodistas. Una perspectiva clasista en el sentido de una mirada desde la clase media urbana a la que Muchnik pertenecía y parece conocer. A efectos de ejemplificar su mirada y también cierta perspectiva conservadora valga esta frase:

\begin{quote}
Pese a todo, en los años sesenta se levantó la bandera de lo ``anticonvencional''. Villa Gesell, con sus calles de tierra y arena, era un lugar para mochileros que rechazaban la vida en las grandes ciudades, con hábitos, a veces, de libertad exagerada (Muchnik, 2013: 25).
\end{quote}

Sin embargo es uno de los pocos trabajos que permite descubrir o apreciar el vínculo de los periodistas centrales del período de los años ´60 a los ´80 y tal vez algo más allá, con la actuación política en movimientos de izquierda más o menos radicalizados:

\begin{quote}
En Panorama Semanal la Redacción se fue ampliando. Ya estábamos ubicados en las nuevas oficinas, en la esquina de Alem y Paraguay. A Mario Bernaldo de Quirós lo sucedió en la dirección Pedro Larralde, con largos años de trajín en La Razón, bajo la batuta de Félix Laíño. Larralde armó un equipo profesional de muy buen nivel. Como columnistas políticos (de noticias nacionales e internacionales) designó a Marcos Merchensky, un intelectual militante del desarrollismo; a Enrique Alonso, de especial talento y pluma vigorosa, también de las filas del desarrollismo (Muchnik, 2013: 33).
\end{quote}

Luego dedica un apartado a lo que denomina \emph{periodistas de lujo}: \protect\hyperlink{_bookmark8}{Gelman, Urondo y Walsh}:

\begin{quote}
Juan Gelman nació en Villa Crespo en 1930 y comenzó a escribir desde la pubertad. A los 15 años se afilió a la Juventud Comunista, dejó la carrera universitaria que estudiaba y ayudó a crear en 1955 el grupo de poesía El Pan Duro, cuya propuesta se basaba en el compromiso con las necesidades populares (reverenciaban a Raúl González Tuñón). En 1963 lo encarcelaron, víctima del plan de represión Conintes, ideado en su momento por Perón en su segunda presidencia, pero aplicado por primera y última vez por Frondizi. Pocos años después abandonó a los comunistas y a través del grupo Nueva Expresión y de la editorial La Rosa Blindada, donde se vinculó con Juan Carlos Pontantiero y Luis Mangieri, siguió publicando para adherir al peronismo revolucionario, forjado en la Resistencia desde 1955. En 1966 optó por ganarse el sustento como periodista y lo hizo precisamente en Panorama, donde lo nombraron secretario de Redacción a cargo de la sección cultural. En 1971 y hasta 1973 desplegaría su talento en el diario La Opinión, también en la sección cultural\ldots{} (\ldots) Paco Urondo, otro gran poeta, era abierto, canchero y, más afectivo que Gelman, un buen compañero de Redacción, afable y dispuesto a ayudar. Siempre tenía una guiñada cómplice y preguntaba por la vida de los otros, no era ajeno a todo lo que pasaba a su alrededor. De esa clase de tipos que pueblan algunas redacciones para hacer del trabajo compulsivo y de los cierres una tarea grata y creativa. Quizás lo que me unía más a él en esas horas donde convivíamos para sacar la revista a la calle, era su manera de ser, más provinciana y abierta. Los dos nacimos en el Litoral. Él en Santa Fe, en 1930, el mismo año que Gelman. Paco no hablaba de política, de su adscripción a una visión revolucionaria, pero cuando participó en la Redacción de Panorama hacía ya dos o tres años que formaba parte de las FAR, igual que Gelman. Sí, era reconocible su voluntad de un cambio profundo. Quería la revolución, pero no hablaba de cómo llevarla a cabo (\ldots) A Rodolfo Walsh, por entonces, no se lo veía en la Redacción del semanario. Mayor que Gelman y que Urondo, Walsh ya estaba definitivamente consagrado como periodista de investigación, una especie de detective minucioso que mostraba la injusticia y la condenaba (\ldots) Walsh participó sin tregua en el aparato de inteligencia de Montoneros y facilitó varios ataques guerrilleros. El 25 de marzo de 1977, un año y un día después del golpe militar de 1976, escribió Carta Abierta a la Junta Militar, denunciando la matanza de una de sus hijas y sus compañeros. Fue acorralado ese mismo día por una patrulla especial de treinta agentes del Grupo de Tareas de la ESMA, que lo estaba esperando en avenida San Juan, entre Combate de los Pozos y avenida Entre Ríos, y lo acribilló a balazos (Muchnik, 2013: 34, 36, 37).
\end{quote}

No es fácil hallar este tipo de descripciones. A pesar de lo aséptica, pone en evidencia la importancia que, para el periodismo argentino, tuvieron aquellos que vinieron de las filas partidarias, o la militancia política, fundamentalmente peronistas y comunistas, al menos sus cabezas más visibles.

En relación con la problemática de la censura y la libertad de prensa o expresión, Muchnik, nuevamente revela cierta ingenuidad de iniciado, cuando observa el fenómeno de la tensión entre la prensa y el poder político. Refiriéndose a la relación del gobierno de Alfonsín con la prensa, afirma:

\begin{quote}
El radicalismo, en materia de prensa, demostró intolerancia y miedo, diría un psicoterapeuta. No admitía observaciones. No ejercía la censura, no clausuraba medios, no cercaba a los periodistas ni iniciaba juicios por calumnias e injurias contra ellos. \textbf{Eso sí, les tiraba encima una montaña de calificativos y de sospechas. Una actitud de extremo sectarismo e intolerancia}. Desconocía la seducción que ejercería Menem con muchos periodistas en la década de los noventa, invitando a muchos de ellos a sus viajes o a comer pizza y champán, a convertirlos en miembros de la ``tribu del poder'', que incluía partidos de fútbol en la residencia de Olivos, para que todos se complementaran, funcionarios y periodistas (Muchnik, 2013: 147).\footnote{El énfasis es propio.}
\end{quote}

Más adelante:

\begin{quote}
Pero también los que acompañaban a Alfonsín veían en cualquier observación crítica una amenaza destituyente, igualando a los periodistas que no respaldaban el esfuerzo como amigos de los ``carapintadas'' (Muchnik, 2013: 148).
\end{quote}

A lo largo del texto, es posible observar que para Muchnik, la censura es una acción del gobierno y no de los medios de comunicación. Incluso en el caso que el medio ponga límites a la libertad creativa o de expresión, estaría en su derecho de hacerlo:

\begin{quote}
En Clarín se respetaron mis columnas, \textbf{que enviaba, previo acuerdo con la Dirección}, desde mi casa, letra por letra. El diario mantenía contactos institucionales con dos personajes del menemismo: Eduardo Bauzá y Eduardo Menem, los interlocutores de Magnetto y la directora. Y Menem aseguraba y repetía que nunca imperó tanta libertad de prensa como durante su paso por el poder (Muchnik, 2013: 161).\footnote{El énfasis es propio. Recuerda la frase de ``los periodistas tienen la libertad de decir todo aquello que tienen permitido'', que recordó Barone en su libro}
\end{quote}

Para Muchnik, la ofensa a la libertad de expresión se halla en las críticas de los gobiernos a los periodistas dominantes. En relación a Alfonsín afirma:

\begin{quote}
El atril presidencial era usado para burlarse, desde un exagerado uso de poder, de periodistas con nombre y apellido, y mostrando a los fotógrafos las imágenes de las notas en las que, según el gobierno, esos profesionales se habían equivocado. Los que no apoyaban al gobierno ingresaban en la categoría de ``destituyentes'' (Muchnik, 2013: 171).
\end{quote}

Pero sobre el gobierno de los Kirchner la crítica será acérrima:

\begin{quote}
Con el tiempo y las futuras gestiones que se extienden hasta ahora, el enjuiciamiento a algunos medios de comunicación y a muchos periodistas por los gobiernos del matrimonio Kirchner llevó a que periodistas, que mostraban signos de fatiga, fueran víctimas de ensañamiento, ``escraches'' y acorralamientos. Afiches en las calles señalándolos como coimeros o mostrando datos que deberían ser secretos sobre sus ingresos y recursos. O tribunales populares donde se los juzgó en su ausencia, como los organizados por Hebe de Bonafini en Plaza Congreso. O aquel mural panfletario ubicado en Plaza de Mayo en el desfile del 24 de marzo del 2010, \textbf{donde se pegaron fotografías de periodistas y se enseñó a los chicos que llegaban con sus padres a escupir los rostros de los que allí figuraban} (Muchnik, 2013: 15).\footnote{El énfasis es propio.}
\end{quote}

Muchnik lo dice como al pasar, pero en realidad desde el punto de vista jurídico, durante los gobiernos de Néstor Kirchner y Cristina Fernández de Kirchner no solo no hubo juicios contra periodistas impulsados desde el gobierno, sino que además se redujo la penalización de los delitos de calumnias e injurias, reclamado históricamente por el periodismo. La imagen de los niños escupiendo los rostros de los periodistas remite más al intento de construir una representación hitleriana de la pareja presidencial que a cierta realidad. Sin embargo, revela, en un periodista con la relevancia de Muchnik, hasta dónde había calado el sentimiento de acoso que sentían los integrantes dominantes del campo.

Más adelante Muchnik retoma la historia de PERIODISTAS y el conflicto suscitado entre \emph{Página/12} y el periodista Julio Nudler. A partir del conflicto fija su posición. Esta, como vimos, se alineaba con la idea de que es el editor de un diario el que decide qué y cómo se publica

\begin{quote}
Periodistas concluyó su vida a partir de una polémica interna acerca de una supuesta censura. Julio Nudler, valioso periodista de temas económicos del diario Página/12, denunció en Internet que una nota escrita la noche anterior había sido censurada por el editor. Esa imagen dividió en dos aguas muy distantes las opiniones. Hubo en Periodistas quienes avalaron un acto de censura y otros, entre los que me contaba, indicaron que es el editor quien decide qué se publica y qué no se publica. No faltaron quienes atacaron a Página/12 y le transfirieron facturas por viejas rencillas. En ese clima triunfó la propuesta de autodisolución de la entidad. Todo un suicidio con final infeliz (Muchnik, 2013: 163).
\end{quote}

Hacia el final del libro, el periodista vuelve con la pregunta:

\begin{quote}
\textbf{Pero ¿hay o no libertad de prensa? ¿Hay censura? No, no hay censura}. La censura directa, la prohibición a existir de un medio de comunicación, en la que se deben presentar los textos para su aprobación no existe. Sí, en cambio, se conocen condicionamientos, presiones. La utilización discrecional de la publicidad oficial, por ejemplo, a la cual se pliegan ministros y funcionarios varios, con autorización presidencial. La pregunta cándida que cabe a estos imperativos es si el gobierno desea (deseaba y siguió deseando) o no subordinar al periodismo. A mediados de 2012 estaba claro, a través de nuevos grupos periodísticos con publicaciones de escasísimo tiraje, pero con excelente caudal publicitario oficial, más la seguidilla de ``escraches'', que el gobierno quería subordinar. Todo el periodismo tenía un ``pasado'' que podía prestarse a debate. Entre tanto el gobierno no mostraba su propio ``pasado'' (Muchnik, 2013: 172).\footnote{El énfasis es propio.}
\end{quote}

La disputa por la actualidad, o mejor aún por la actualización del pasado, ponía en tensión el discurso periodístico. En el caso de Muchnik, se percibe con claridad la tensión que perciben los periodistas entre la crisis de financiamiento y las nuevas tecnologías:

\begin{quote}
Una encuesta realizada por Unicef delegación argentina a 1.100 adolescentes entre 13 y 17 años que viven en grandes ciudades del país, confirmó que la mayoría se informa por televisión o por Internet. El 51,8\% eligió la televisión; el 41\% Internet; el 3,6\% la radio y sólo el 0,9\% los diarios. La mayoría de los adolescentes no lee diarios, ni en formato de papel (58,3\% del total) ni en su versión digital (69,3\%). Para los expertos, los chicos de hogares socioeconómicos elevados tienen más acceso a los diarios que los que pertenecen a familias de menores ingresos. Así, una conclusión de esa encuesta es que ``el acceso al diario requiere que la familia o la escuela lo facilite, a diferencia de otros medios --como por ejemplo Internet-- a los que el propio joven puede acceder en, por ejemplo, un locutorio''. Eso sí, el 80\% de los adolescentes que entrevistó Unicef aseguró tener teléfono celular, que principalmente usan para enviar mensajes de texto, en segundo lugar para hablar y en tercer término para escuchar música o sacar fotos. Estos hechos pude comprobarlos personalmente como profesor universitario. La mayoría de los estudiantes de Periodismo y de Ciencias de la Comunicación, carreras en las que enseñé durante muchos años, no seguían la información de los diarios (Muchnik, 2013: 178-179).
\end{quote}

Con \emph{Víctor Hugo. Una historia de coherencia y convicción}, Julián Capasso da inicio a las primeras acciones de defensa de los periodistas agredidos por los \emph{defensores}. En el prólogo del libro, escrito por Adrián Paenza, luego de una larga lista de halagos periodísticos a Víctor Hugo Morales, Paenza se pregunta:

\begin{quote}
¿Por qué habría de decir todo esto? ¿Por qué no escribir libremente el prólogo de un libro que habla de Víctor Hugo? Tengo ganas de decir mucho, las ideas se me arremolinan buscando una salida, pero solo doy abasto para escribir de a una por vez. Y tengo miedo de olvidarme de otras que ven cómo su tiempo pareciera no llegar nunca. ¿Por qué no puedo disfrutar de este momento? Es que siento que, en esta coyuntura, pareciera como que tengo que escribir en defensa de Víctor Hugo. Pero, ¿defenderlo de qué ? ¿ De quiénes? (Paenza, 2013: 11).
\end{quote}

Luego afirma:

\begin{quote}
Nadie podrá nunca decir que le pagó a Víctor Hugo para que él dijera lo que no pensaba (\ldots). Lo quisieron comprar de todas las formas posibles. Yo fui testigo de la charla que tuvo con Carlos Ávila en su momento, quien lo quería contratar genuinamente. En definitiva, no arreglaron, pero mientras Ávila lo quería para ponerlo al frente de su enorme batallón de periodistas, quienes estaban detrás de él, detrás de Ávila, querían otra cosa: querían traerlo para ese lado, callarlo, cooptarlo, domarlo. Por eso -\/-creo--- que Ávila no fue responsable. Ávila fue una suerte de emisario, formal o imaginado, pero fue alguien enviado para una misión que finalmente resultó imposible (\ldots). Aunque parezca extraño, quiero recalcar una vez más que no pudieron porque si bien el dinero siempre es un factor en la vida profesional de cualquier persona, ciertamente no fue el único en su caso. Víctor Hugo quería ser independiente (\ldots). El libro servirá como un recorrido imprescindible para entender a uno de los mecenas de este siglo, que elevó la barra de la ética profesional a una altura en donde el oxígeno que allí circula ha sido respirado por pocas personas en la historia del periodismo en la Argentina (Paenza, 2013: 14, 15, 19).
\end{quote}

En este trabajo pueden apreciarse, en las voces de Morales y de Capasso, las diferencias dentro del campo y en relación a la perspectiva de la libertad y la censura:

\begin{quote}
El conductor Marcelo Longobardi lo entrevistaba y escuchaba atentamente la noche del miércoles 30 de septiembre de 2009 en los estudios de C5N, cuando en un momento lo interrumpió y le pidió: ``Pero Víctor Hugo, contame, ¿porqué tu desprecio hacia los monopolios? ¿Qué pasó ? ¿Qué te pasó?''. A lo que el periodista rioplatense respondió: ``Yo he sufrido los monopolios, los he detestado, \textbf{me parece que han hecho un enorme daño, que han conspirado contra la calidad del periodismo que tenemos, contra nuestra libertad de expresión}, que han dañado la imagen del periodismo que más o menos nosotros soñamos desde la condición que tenemos ambos que es estrictamente de periodistas. No somos dueños de medios. En consecuencia, a la ley de medios la leí con actitud positiva. Pero después me di cuenta de que era una muy buena ley'' (Capasso, 2013: 121).\footnote{El énfasis es propio.}
\end{quote}

Más adelante Capasso reseña la respuesta de Morales a la diputada Elisa Carrió. El debate de la ley de Servicios de Comunicación Audiovisual fue refinando las posiciones alrededor de la misma y distanciando a los integrantes del campo, fundamentalmente alrededor de ella:

\begin{quote}
Parece atinado defender la libertad de prensa contra lo que sea, como ella dijo, no obstante, algunos ya le hemos agradecido la intención y decimos que si es por nosotros que no se moleste, porque no está en juego la libertad de los periodistas si no, en todo caso, la libertad que gozan los medios, sus dueños. Una libertad que muchas veces se agota en el dueño de la empresa, y ahí termina, que nada tiene que ver con nosotros, los periodistas. Para entendernos mejor vamos a empezar escuchando a Lilita Carrió: ``No tenemos ningún problema en defender los llamados grupos económicos si es para defender la libertad de prensa''. Parece fantástico lo que dice Lilita, sin embargo, quiero dejar un solo ejemplo, por ahora, para permitir un razonamiento más cómodo y que pulveriza, desde mi punto de vista, el razonamiento que presenta Lilita Carrió. Una nueva ley que limite los monopolios es lo mejor que puede sucedernos a los periodistas (Capasso, 2013: 178).
\end{quote}

Posteriormente, en un largo párrafo, Morales desarrolla aún más su argumentación con ejemplos tomados de la propia realidad periodística del momento:

\begin{quote}
Te invito a pensarlo de esta manera: tomen mentalmente todas las empresas periodísticas del multimedio más mencionado como destinatario de la ley, tomen las radios del grupo, ahora tomen sus canales y después tomen sus diarios, ahora su agencia de noticia y ahora las decenas y decenas de empresas que el multimedio posee. Imaginen a continuación a sus periodistas, a los más destacados que usted quiera, los que son de los más importantes que hemos tenido de muchos años a esta parte. Ahora imagínelos hablando a favor de la ley de radio difusión, o imagínelos nada más, declarando en otros medios que están a favor. Esto es imposible, entonces el solo hecho de no encontrar una sola voz a favor del proyecto en esos medios, le demuestra que quizás y desde nuestro punto de vista, a la señora Carrió y a la oposición que lo que en realidad sucede es que esos periodistas no tienen la libertad de decir lo que quieren. Ésa es la libertad de prensa que dice la oposición que quiere defender. Y yo insisto, francamente, en pedir que no lo hagan, por lo menos no en nombre de todos los periodistas. Nos hace tanto daño que podemos pensar por lo contrario. Si hay periodistas que perteneciendo a los medios implicados están en contra del proyecto por otras cuestiones, pero no lo pueden decir tampoco. Los multimedios son lo suficientemente dañinos para nuestra profesión, para que no sonemos creíbles, tengamos la posición que tengamos frente a un hecho como este, por ejemplo, de la ley de radiodifusión. Libertad son muchos medios. Libertad es que si me echan de este trabajo porque molesta lo que digo pueda encontrar, rápidamente, otro sitio donde expresarme. Los medios concentrados nos quitan el trabajo, nos condicionan, nos limitan, ahí está uno de los reales enemigos de la libertad de prensa, por lo cual yo discrepo, respetuosamente, con lo que ha dicho Lilita Carrió\ldots.'' (Capasso, 2013: 178).
\end{quote}

En otra reflexión sobre la libertad de prensa, que el libro recupera de un reportaje realizado a Morales por el Suplemento \emph{Ni a Palos,} del periódico dominical \emph{Miradas al Sur}:

\begin{quote}
NiaPalos: ---Te escuché decir que la libertad muchas veces se agota en el dueño de la empresa, pero que los periodistas tienen margen de maniobra. ¿Qué alcance tiene eso? ¿Por qué entonces los periodistas son tan consecuentes con los dueños de los medios donde trabajan?

VH: ---Yo creo que tiene que ver con la relación que tienen con el gobierno. Yo no he tenido nunca una actitud de malicia con el gobierno, sino más bien de disgusto. Lo que debe haber es un componente de mucha bronca para que no puedas ver las ventajas de la Ley para los periodistas. Las ventajas son poder decir lo que quieras porque va a generar mucho más laburo. Con cada radio nueva que se forme a partir de la ley, va a haber 20 ó 30 tipos más que van a laburar. Pero además se van a crear situaciones de mercado por las cuales las viejas injusticias de los oligopolios que se quedaron con una porción brutal del mercado van a saltar por los aires y la carrera va a empezar de nuevo. Una de las preguntas que se hacen imbécilmente es por ejemplo cómo van a hacer para sostener la radio, porque conciben un mercado empequeñecido, no conciben un mercado que se amplíe, no conciben que Clarín suelte todo lo que tiene prisionero. Clarín no te permite hacer publicidad en otros medios. (...) No pueden estar tan en contra, yo creo que lo que pierde a la mayoría es el resentimiento contra el gobierno en general. Y si vos funcionás desde esa calentura, no ves nada. No era verdad que entraban las telefónicas, pero lo querían ver así. Entonces pedalearon con lo de las telefónicas, se sacaron las telefónicas y pedalearon con lo de la licencia cada dos años. Ninguna de estas cosas nunca fue real (Miradas al Sur, 11 de octubre de 2009, Suple Ni a Palos) (Capasso, 2013: 180).
\end{quote}

En síntesis, la reflexión de Víctor Hugo Morales, lo ubica en un lugar diferencial de buena parte del periodismo en Argentina. Sin provenir de una militancia política o de una posición ideológica definida, e incluso especializándose en el fútbol, Morales logró por sus condiciones personales y profesionales, ser uno de los primeros entre los pares, llegando a ser incorporado al selecto grupo de periodistas de la Academia Nacional de Periodismo, a la que renunció en el marco del conflicto.

Morales, brega por la libertad de expresión, no en abstracto sino en concreto, es decir, la libertad para decir lo que se piense, sin importar el dueño de la empresa para la que se trabaje. Por otra parte, enarbola causas de la agenda en la que se especializa: lucha contra el monopolio de la televisación del futbol, contra los negociados en la Asociación del Futbol Argentino, etc.

Eduardo Blaustein publica un verdadero tratado de periodismo que salió con el sugestivo título de \emph{Años de rabia. El periodismo, los medios y las batallas del kirchnerismo}. Realiza un serio intento por desmitificar la profesión y vuelca experiencias personales que intentan quitarle al periodismo su halo de heroísmo y ética moralizante:

\begin{quote}
Que las derechas, las empresas de medios o generaciones de periodistas algo livianos no se hayan enterado o se muestren distraídas, y que pretendan hablar de la libertad de imprenta como si viviéramos los virginales años del despegue de la prensa escrita hacia 1780, no es culpa del Kirchnerismo (Blaustein, 2013: 33).
\end{quote}

Sobre la problemática de la libertad de prensa o la censura, Blaustein se atiene a información de fuente reconocida:

\begin{quote}
En tiempos kirchneristas un examen posible del concepto de objetividad puede hacerse en torno de la discusión de si en nuestro país tenemos o no libertad de prensa. Los sectores opositores sostienen desde hace años, como mínimo, que esas libertades están en peligro inminente o que la libertad de expresión recibe un gancho a la mandíbula detrás del otro. Hay una fuente posible para analizar el tema, opinable, pero que tiene su ``prestigio global'': la gente de Reporteros sin Fronteras. Todos los años esa organización difunde su propia clasificación de ``libertad de prensa'', en la que no se precisa demasiado qué actores contribuyen a más o menos grados de libertad o sobre qué pautas se hacen los rankings. Según Reporteros, como al gordito de Alfonsín, a Argentina no le va tan mal. ``Ocupa un lugar más bien envidiable en la última clasificación mundial'', decía el informe 2011. Nuestro país aparecía en ese informe en el puesto 48 (comparte la posición con Estados Unidos, algo llamativo) en el período 2011-2012 entre 179 naciones. Finlandia, Noruega, Estonia y los Países Bajos encabezaban los índices de ``respeto a las libertades fundamentales''. De las naciones latinoamericanas, a Argentina solo la precedían Costa Rica, Uruguay y El Salvador; en ese orden. Chile ocupaba el puesto 80. Brasil, el 99. México, que desde hace años vive un cuadro desgarrador en buena medida derivado de ``la guerra contra las drogas'' (al menos 80 periodistas asesinados desde 2007), aparecía en la ubicación 149. Los subsuelos del horror\textquotesingle, según Reporteros sin Fronteras, eran los que ocupaban, a partir del puesto 175, Siria, Turkmenistán, Corea del Norte y Eritrea (Blaustein, 2013: 49-50).
\end{quote}

Es decir, la situación no era tan grave como los sectores dominantes del campo periodístico pretendían presentarla. En la actualidad, en el mismo ranking del año 2019, Argentina retrocedió al puesto 57 y a fin de mantener las comparaciones, Estados Unidos subió al puesto 48.

Volviendo a Blaustein:

\begin{quote}
Del discurso de Ernestina Herrera de Noble: ``Felicito al gobierno del presidente Menem por haber privatizado los canales al inicio de la gestión y no al terminarla. Esto subraya el respeto por la libertad de las opiniones''. En los 90 se abriría la fase de más vertiginoso crecimiento del Grupo Clarín (\ldots). Una hipótesis en sentido contrario respecto a esos años: fue en los 90 cuando los periodistas, empoderados y halagados, ayudamos a extender un malentendido desde nuestros puestos de combate. Precisamente porque disfrutamos de mejores márgenes de libertad y de protagonismo, creímos que vivíamos las mejores esencias del oficio; que las empresas podían ser nuestras aliadas, negociando aquí, cediendo allá, otras veces imponiéndonos; las cosas habían sido o serían siempre así. Podría haber sido un modesto malentendido de cabotaje, entre nosotros, los periodistas, en privado. Pero le confiamos el malentendido a la sociedad, con la inestimable colaboración de las empresas de medios. En esencia, el malentendido consistía en, relativamente, creer que no solo éramos (o somos) unos ídolos, sino que además éramos libres (Blaustein, 2013: 106, 118).
\end{quote}

El reciente párrafo revela un profundo conocimiento del campo. Cuando el periodismo se siente llamado a jugar el rol que, según considera, la historia le ha asignado, es cuando el campo se realiza como tal, se pone en acto. La posdictadura, el período menemista y en cierto sentido el período que analizamos, se revelan como pródigos en el procesos de construcción de su legitimidad y, por lo tanto, de su capacidad para convertir en verosímil su construcción de la realidad social. Tal vez no sea casual que el metaperiodismo se active en estas situaciones, ya sea a la manera de autoelogio o como crítica de lo que se debería haber hecho y no hizo, o de lo que tendría que haber sido y no fue.

Blaustein se refiere al fenómeno como \emph{malentendido,} es decir, considera que el periodismo no es ese periodismo y la sociedad no tendría que creer que lo es. Sin embargo el periodismo es lo que es porque existen esos períodos, en los que la capacidad simbólica del campo se expresa en su máxima potencia. Podemos afirmar que los campos en general --y el campo periodístico en particular--, se expresan cabalmente cuando se sienten agredidos o cuando condiciones favorables para mostrarse en plenitud le permiten hacerlo.

\begin{quote}
La noción de independencia arrastró estos últimos años una impugnación un tanto panfletaria y cansadora ---sin embargo necesaria--- desde el kirchnerismo, especialmente cuando se contrapuso a ``independencia\textbf{'' la desafortunada figura del periodismo militante}. A la idea de la independencia periodística se la puede cuestionar desde lo ideológico (``desde qué lugar hablás'') tanto como desde el punto de vista de la subordinación y obturación de la independencia dados los intereses y necesidades empresarias. Hay otro enfoque posible para deconstruir la noción de independencia que tiene que ver más con la esfera íntima del periodista (Blaustein, 2013: 122). \footnote{El énfasis es propio.}
\end{quote}

Nuevamente Blaustein revela su capacidad de percepción del \emph{hábitus,} al que se refiere Bourdieu y la manera en que el campo reaccionó ante la expresión \emph{periodismo militante}, realizada en una entrevista por el presidente de Telam, Martín García. Este último no era periodista, y es probable que por este motivo haya verbalizado esta provocación semántica para un colectivo tan sensible y protector de su identidad profesional. Blaustein se refiere a \emph{desafortunada figura} porque percibe el profundo desagrado que provocó entre los periodistas, abroquelando al campo periodístico contra las políticas comunicacionales del gobierno, al que percibían contrario a la profesión periodística. Es más, si García hubiera querido disputar la legitimidad de la realidad social debería haber acusado a sus oponentes de militantes y no autoadjudicarse ese mote.

Sigue Blaustein (2013: 122) reflexionando sobre el significado de independencia según su perspectiva:

\begin{quote}
Ser ``independiente'', según cómo, puede resultar una comprensible maniobra personal para permanecer a salvo, quieto y confortable en el propio lugar a la hora de sincerar la propia opinión (pregunta clásica: por qué los periodistas no dicen a quién votan). En el mejor de los casos, y suponiendo que todo se trate de razones y no de emociones o pasiones, ser independiente es el resultado de poner en ejercicio la propia honestidad intelectual (con los riesgos de esa noción ``vaga'' de honestidad a la que se aludió en un capítulo anterior). Ser independiente puede ser el fruto del intento de conservar la propia pureza ---"yo no me mancho, no me juego por nadie''---, con lo que la pureza puede convertirse en descreída mesura.
\end{quote}

Blaustein sabe que no existe la independencia, pero simultáneamente comprende en plenitud que fuera de ella no hay periodismo. Que ese concepto es el blindaje protector con el que se configura y legitima el vínculo entre medio--periodista--audiencia. A partir del capítulo 4 del libro, Blaustein intenta historizar el conflicto a los fines de encontrar el origen:

\begin{quote}
En 1985, el primer secretario de Cultura, Carlos Gorostiza y algunos otros funcionarios mostraron simpatía por el asunto del NOMIC.\footnote{NOMIC, acrónimo de Nuevo Orden Mundial de la Información y Comunicación, concepto desarrollado en el Informe MacBride, elaborado por la Comisión creada por Unesco para el estudió de los problemas de la comunicación. El Informe propuso cambios para distribuir y equilibrar los flujos de información entre los países, sin embargo, la oposición de las organizaciones privadas de medios y el retiro de los Estados Unidos y el Reino Unido de la UNESCO, desfinanciándola, a cabó relegando el proyecto al olvido.}Los grandes medios de prensa saltaron en masa y en cadena para morderles la yugular. Denunciaron ataques estatistas contra la libertad de expresión, afirmaron que UNESCO se había convertido en tierra sovietizada, emplearon expresiones tales como ``dialéctica del Kremlin''. Recortes de la época:``Dijimos en sucesivas oportunidades que detrás de ese pretendido \textquotesingle nuevo orden' ---denominación que por sí sola despierta sombrías reminiscencias dictatoriales--- se esconde una agresión gravísima y muy concreta contra la libre expresión de las ideas. El `nuevo orden\textquotesingle{} no es otra cosa que el manejo directo o encubierto de la actividad informativa por los organismos del Estado'' (La Nación, 7 de marzo de 1985).``La Asociación de Telerradiodifusoras Argentinas (ATA) ve acrecentado su desconcierto ante la ratificación de la adhesión del gobierno nacional al llamado Nuevo Orden Mundial de la Comunicación e Información'' (Clarín, abril de 1985). El contexto en el que finalmente el radicalismo intentó discutir y poner en debate una nueva ley ya le era adverso: había perdido las elecciones de 1987, la inflación era de dos dígitos, se habían producido las rebeliones carapintadas, existía cansancio político en lugar de lo que había sido, dos o tres años atrás, la vitalidad y potencia de la experiencia alfonsinista. Tan decaído estaba el alfonsinismo que Eduardo Angeloz, futuro candidato radical a la presidencia, a mediados de 1986 decía ``A este gobierno le hace falta un Balbín''. Mientras diputados y senadores radicales proponían sus proyectos de nueva ley de Radio- difusión, Alfonsín había atentado una discusión paralela dentro del Consejo para la Consolidación de la Democracia (COCODE). Con lo que se habían sumado cuatro espacios de discusión: SIP, COMFER, el Congreso y el COCODE (\ldots). Cuando el Gobierno anunció el envío de un proyecto final de ley, se multiplicaron las presiones de las entidades de medios privados. La diferencia de las campañas institucionales respecto de las que conocimos desde 2009 es que se hacían más visibles en la prensa gráfica. Ya se veía por parte de las empresas la suma de todas las destrezas: las jurídicas, las de la propia potencia de comunicación, las del arte publicitario. Una ``puesta en escena'' posterior recordable fue la campaña del CEMCI (la nueva entidad de medios independientes fogoneada por Clarín) (\ldots), Que las cosas definitivamente se habían puesto agrias lo muestra el célebre speech de Raúl Alfonsín contra Clarín que tantas veces reiteró 6,7,8. El diario había publicado a mediados de febrero de 1987 un artículo con datos recortados de los provistos por el INDEC, poniendo el acento en los problemas de desempleo. Alfonsín entendió que el recorte era mal intencionado y en una visita al mercado comunitario de Flores tronó de esta manera:``Yo no les voy a pedir a los medios de difusión que varíen su prédica, soy respetuoso de la libertad de prensa, pero ustedes tienen un ejemplo hoy, en los diarios de hoy. Yo les pido que vean el Clarín, que se especializa en titular de manera decidida, como si realmente quisiera hacerle caer la fe y la esperanza al pueblo argentino. Yo respeto al diario el Clarín, y el Clarín respeta al presidente sin dudas, y no ha de pretender que calle su opinión. Lean ese artículo que está vinculado a los anuncios sobre la desocupación, sabemos que es un opositor acérrimo, y no nos interesa; sabemos que es también este tipo de artículo el que aparece cotidianamente en el diario, pero léanlo, porque en la forma falaz en que está presentada la noticia de una disminución de la desocupación en la Argentina es un ejemplo vivo contra lo que tenemos que luchar los argentinos'' (\ldots). En la primera semana de marzo de 2008 la presidenta Cristina Fernández mencionó un pronunciamiento de la Facultad de Ciencias Sociales de la UBA que cuestionaba duramente el perfil de la cobertura hecha por los medios de comunicación sobre las manifestaciones cruzadas que desató el \emph{lock-out} agrario, particularmente por entender que había de por medio retóricas discriminatorias. La carrera de Comunicación Social anunció que, a partir de ese episodio, articularía su producción académica con la de su Observatorio de la Discriminación en Radio y Televisión creado no ese mismo año de 2008, sino tres años atrás, y que haría lo mismo con el COMFER y el INADI para poner en discusión social la eventual comisión de los actos discriminatorios denunciados. No se había producido entonces el anuncio sobre el envío de la Ley de Servicios Audiovisuales al Congreso, pero la reacción de los medios fue parecida. TN apeló, en la voz de sus conductores, flashes y zócalos a los imaginarios de la mordaza y la censura. La Nación habló del ``temor a una nueva ofensiva contra la libertad de prensa". ADEPA publicó una solicitada alertando sobre la pretensión gubernamental de ``controlar al periodismo para adocenarlo'' (Blaustein, 2013: 166, 170, 171, 180, 181).
\end{quote}

El rastreo histórico de las posiciones de los grandes medios frente a los intentos del Estado por regular de algún modo su autonomía, parece no dejar lugar a dudas de la agresividad con que estos reaccionan frente a cualquier propuesta.

\begin{quote}
Daniel Rosso, un muy buen cuadro técnico de la comunicación gubernamental, escribió en la revista Contraeditorial (diciembre de 2010), a propósito de las políticas del Gobierno anteriores a 2009, que quienes tenían el control privado de la esfera pública ``produjeron una novedad: que el gobierno fuera excluido de la libertad de expresión. Y que, paralelamente, fue presentado como quien la restringía. El gobierno generaba hechos, pero no sus sentidos. En ese vacío de explicación las políticas de transformación quedaban reducidas a conflictos sin razón'' (Blaustein, 2013: 185).
\end{quote}

Esta cita de Daniel Rosso, explicita uno de los mecanismos que el campo implementó para desacreditar las voces del gobierno, pero también otras voces externas al campo o expulsadas de él.

Más adelante Blaustein explora también otro de los argumentos: el referido al financiamiento de la prensa, que el campo explotaba con argumentos plausibles. En síntesis, que el gobierno no usaba los mismos criterios de financiamiento de la prensa y los medios de comunicación que las empresas privadas.

\begin{quote}
Hay más, la Relatoría para la Libertad de Expresión dice que ``la publicidad estatal puede compensar los vastos recursos de la comunicación controlados por intereses empresariales o por los círculos financieros, pues pueden ampliar la voz de periodistas y medios de comunicación locales, de los medios más pequeños y de los que critican a las empresas'' (Blaustein, 2013: 221).
\end{quote}

Hacia el final del libro, el autor concluye desencantando al lector con lo que podrían ser los aspectos más despreciables del campo. El anteúltimo capítulo, \emph{Ética cínica, deontología, populismo mediático}, revela una postura similar al primer libro de nuestra saga, el de Claudio Díaz. Si Díaz consideraba al periodismo de los grandes medios como expresión de antiperonismo, Blaustein lo descubre como expresión de la antipolítica.

\begin{quote}
En Argentina, por supuesto, la antipolítica viene de lejos y con buenas razones, tiene su propio marco reciente: kirchnerismo. El salto a más o más densa antipolítica puede analizarse según las dosis e intensidades de cinismo de los discursos mediáticos. Escribió nuestro amigo del seudónimo, don Enrique Orozco, en la revista \emph{Crisis}: ``La ideología periodística no se agota en lo que sale hacia fuera. Hacia adentro, los medios sistémicos son incubadoras de cinismo, escepticismo y sumisión. El cinismo es el elemento no dicho del catecismo de la libertad de expresión. Desconfiar siempre de las buenas intenciones de los que proponen algún tipo de transformación. Descreer siempre (...)``Nunca nada cambia para mejor''. Escribió también Orozco (y me gusta citarlo porque es un sub 40): ``El periodismo ortodoxo entiende la política únicamente en dos de sus facetas, la del robo, evidente y a gran escala, y la de la rosca. Allí donde la política habla de militancia y ofrenda, épica y amor (haciendo uso de su propia dosis de cinismo), el periodismo repone las nociones de punteros, activistas, mafias y piqueteros''. El periodismo, que perdió su épica, se aplana así en una mirada despolitizada de la política (...) Contribuye a eso la ausencia de roce social ---tan constitutiva como la de cualquier elite--- de la casta que conduce los medios (Blaustein, 2013: 386-387).

\textbf{Año 2014}
\end{quote}

Durante el año 2014, se consolidó en el campo periodístico y en la sociedad la percepción de que el conflicto había cobrado significativa virulencia.

FOPEA (Foro de Periodismo Argentino)\footnote{El Foro de Periodismo Argentino (FOPEA) es una organización fundada en el año 2002, que tiene por objetivo constituirse en un espacio de reflexión, diálogo y promoción de la calidad del periodismo. La integran destacados profesionales de medios de comunicación, principalmente de los grandes medios nacionales.} que reúne al grupo de periodistas vinculados a los grandes medios, produjo dos materiales editados: \emph{Nuevos desafíos del periodismo} y \href{http://www.cuspide.com/9789871496990/Tiempos+Turbulentos/}{\emph{Tiempos turbulentos}}. Edi Zunino dará cuenta del conflicto con \emph{Periodistas en el Barro}\footnote{Si bien el libro tiene fecha de edición en noviembre del año 2013, su distribución fue realizada durante finales de ese año y su impacto mediático se desarrolló durante el año 2014.}. Otro tanto hará Eduardo Blaustein analizando el derrotero de Jorge Lanata en los últimos años. Finalmente, algunos periodistas que comenzaron a percibir el intento de expulsión del campo, iniciaron su autodefensa: \emph{Audiencia con el diablo} de Víctor Hugo Morales e \emph{(In) Justicia Mediática} de Darío Villaruel.

El libro de Edi Zunino(2013), \emph{Periodistas en el barro. Peleas, aprietes, traiciones y negocios. Miserias y razones de la guerra mediática en la Argentina reciente,} desde el título expresa el riesgo en que se percibe el campo y pone de manifiesto la percepción de algunos periodistas sobre el estado que alcanzó el debate metaperiodístico. El libro de Zunino retoma algunas ideas centrales de \emph{Patria o Medios} (2011), pero dándole mayor espesor en el debate dentro del campo. El autor intenta dar cuenta de lo que denomina ``la guerra mediática en la Argentina reciente''. Recorre, a nuestro modo de ver, las distintas posiciones que se expresaban en el campo periodístico en relación a la libertad de prensa. Considera que tanto a Néstor Kirchner como a Cristina Fernández de Kirchner el tema de la prensa los obsesionaba.

La primera referencia a la libertad de prensa la pone en boca de un empresario de medios audiovisuales: Daniel Hadad. En relación a la supuesta negociación que habría tenido con Néstor Kirchner, Hadad afirma:

\begin{quote}
El éxito económico no es un lujo: es un pilar sobre el que descansa la libertad de prensa. En otras palabras, para la empresa informativa ganar dinero es un deber ético (Zunino, 2013: 27).
\end{quote}

Luego en el proceso de venta de la señal C5N al grupo de Cristóbal López, Zunino (2013: 36) narra una situación de supuesta censura en el canal.

\begin{quote}
A las 23.07 del martes 13 de marzo de 2012, la emisión en vivo del programa de Marcelo Longobardi fue levantada del aire de C5N sin aviso previo. El periodista entrevistaba al ex jefe de gabinete Alberto Fernández, verdadero inventor de la confrontación con la prensa en el auge del período K convertido, por obra y gracia del agotamiento de los ciclos políticos y de su propio oportunismo, en el opositor más quisquillosamente racional de la gestión CFK. En el bloque anterior, el ex periodista y escritor Jorge Asís había descuartizado al vicepresidente Amado Boudou por sus estrechos vínculos con las extrañas movidas accionarias en la imprenta Ciccone Calcográfica, principal fabricante privada de billetes del país por concesión de la Casa de Moneda. El ``Boudougate'' estaba en boca de todos. En la oscuridad del estudio esperaba su turno el periodista Alberto Padilla, quien, ante el abrupto corte de la transmisión subió a Twitter su comentario como testigo de un acto de ``represión a la prensa argentina''. \textbf{Nadie podría asegurar que se trató de un mensaje directo de Hadad a la Presidenta de la Nación, destinado a aclararle qué cuernos habría querido decirle aquella dura mañana en Olivos}. Pero el escándalo de ``censura'' y ``autoritarismo'' estalló sin atenuantes en todas las radios, desde bien temprano, la mañana siguiente (\ldots).---Se hicieron muchos comentarios en el piso sobre las razones, las causas, los llamados (...) Le pido que me libere de hablar ---esquivó el ex hombre fuerte del gabinete K, quizás tratando de liberar de cargas a su viejo amigo Hadad. Los supuestos llamados para pedir el corte de la emisión habrían surgido del celular de Julio De Vido. \footnote{El énfasis es propio.}
\end{quote}

Luego, Zunino cita un artículo de Verbitsky, sobre un comentario de Alfredo Leuco, en relación a iniciar un juicio contra el canciller Héctor Timerman.

\begin{quote}
---Es insólito que a un periodista le moleste que el canciller haya divulgado su punto de vista sobre asuntos de gobierno. Lo habitual es que los funcionarios cultiven el secreto y los periodistas les exijan el acceso a la información. La segunda sorpresa fue que Timerman contara un encuentro anterior en el que Leuco, según dijo acompañado por el gerente comercial, le pidió publicidad para hablar bien de Kirchner. Otros periodistas han escrito en forma crítica sobre las condiciones del ejercicio de su profesión en estos tiempos, \textbf{pero ninguno fue tan extremo como Leuco en su curiosa descalificación de este período como el más restrictivo para la libertad de expresión en democracia.} Esta enormidad, que lo llevó incluso a polemizar con el director del diario donde escribe, Jorge Fontevecchia, es incomprensible en un periodista (...) (Zunino, 2013: 53).\footnote{El énfasis es propio.}
\end{quote}

Zunino (2013: 53) va revelando paso a paso, el proceso que devino en guerra de periodistas. Cita a Leuco:

\begin{quote}
Verbitsky dejó de ser uno de los mejores periodistas argentinos, incluso para mí, y se convirtió en el jefe de inteligencia informal del kirchnerismo ---sacudió Alfredo Leuco desde Radio Continental el primer día de agosto de 2010, mientras releía sin parar los ácidos comentarios de ``El Perro'' en Página/12 sobre su resonante entredicho con Héctor Timerman, ex periodista devenido ministro de Relaciones Exteriores y flamante figurón de la red social Twitter.
\end{quote}

Luego cita a Lanata:

\begin{quote}
Yo soy incapaz de matar a nadie y Verbitsky, no. ¿Vos serías capaz de pegarle un tiro en la cabeza a un tipo atado y desarmado? Bueno, así fue el asesinato del general Aramburu. Que quede claro: Verbitsky fue montonero, yo no. Y los montoneros reventaron a Aramburu. Tampoco defendí jamás la lucha armada (Zunino, 2013: 279).
\end{quote}

Más adelante:

\begin{quote}
El que padeció la censura fue Nelson Castro. Y ya en ese momento, Verbitsky se puso del lado de los victimarios y cuestionó la situación de Nelson planteando que todo había sido una cuestión comercial y contractual. De veras exageraba Leuco al considerar al kirchnerismo la etapa más restrictiva de la historia en materia de libertad de expresión, aunque de seguro lo hacía condicionado por la persecutoria dureza con que los K venían castigando su desencanto con un gobierno que en un principio lo había entusiasmado. A eso se refería Leuco al ubicar su profesión ``más allá del corazoncito político'', aunque sin detenerse a pensar que ``el que se enoja, pierde'' (Zunino, 2014: 54).
\end{quote}

En otra parte del libro:

\begin{quote}
En la edición dominical de Página/12 inmediatamente posterior a los Premios Perfil, HV acusó en un recuadro a Noticias de haberlo censurado por no publicar su discurso completo, lo cual era cierto pero por razones de edición. En la nota, Verbitsky había hablado hasta por los codos sin ninguna clase de condicionamiento. Lo tituló ``Menos premios y más respeto''. A mí, ``censura'' me pareció demasiado. Le respondí en dos páginas de Noticias que concluían apelando al sentido común de los lectores: ¿Alguien entregaría un Premio a la Libertad de Expresión para censurar al premiado seis días después? No es aquí donde Verbitsky ni nadie serán censurados. Estamos dispuestos a correr un montón de riesgos en el ejercicio del periodismo, pero no el del suicidio. Incluso el de animarnos a suponer que, cuando toda esta espuma de chifladura baje, mucho más probable que censurar al Perro resultará pedirle que escriba en la revista (Zunino, 2013: 60, versión digital).
\end{quote}

Tanto en este trabajo como en \emph{Patria o Medios}, Zunino permite aproximarse a lo que podríamos denominar una estructura de personalidad en estos periodistas, que conforman el más alto estatus dentro del campo. Una fuerte egolatría, una autoconciencia de su poder de daño sobre el poder político y, consecuentemente, su capacidad de negociación y potencial disponibilidad de recursos simbólicos. Pero por fuera de estas características que parecen compartir como parte de su \emph{habitus} bourdiano, en algunos casos su afinidad ideológica, o su desmesurada necesidad de disponer de recursos económicos, los convierte en actores significativos del campo político tanto como periodístico de donde obtienen su poder.

Morales (2014), con el libro \emph{Audiencia con el diablo}, parece iniciar su defensa con un buen ataque, utilizando una metáfora futbolística. Intenta intervenir sobre el significado de libertad de expresión que utilizan los integrantes del campo dominante:

\begin{quote}
Magnetto pone un límite que todos conocen en la Argentina. Nadie va más allá de esa línea que él ha movido hasta estrangular la libertad de expresión. Magnetto sabe quiénes pueden ser comprables y hasta se desafía en conquistar la conciencia de los que supone como más difíciles (Morales, 2014: 19).
\end{quote}

Víctor Hugo Morales, por su conocimiento del \emph{habitus} del campo, sabe que lo que debe ser discutido, es la libertad de prensa, la libertad de expresión.

\begin{quote}
Magnetto es como el ladrón que con una bolsa de joyas en cada mano deja su huida en suspenso para volver por un brazalete que se le quedó en las vitrinas. Y no le importa que las cámaras de seguridad lo registren una y otra vez al saberse protegido por la máscara de la libertad de expresión (\ldots) La grosería fue más notoria que la hipocresía cuando un editor de Tucumán presentó la jornada del 8 de agosto como un ataque a la libertad de expresión. Los marrulleros de Adepa exhibieron el desenlace de la mediación con Magnetto como una prueba de la persecución periodística que existe en la Argentina. No era Magnetto quien pretendió acallar a un periodista, sino el gobierno que intentaba silenciar a Magnetto (Morales, 2014: 36, 38).
\end{quote}

Por último, afirma:

\begin{quote}
El límite de la libertad de expresión es el propio Magnetto (Morales, 2014: 40).
\end{quote}

El libro reflexiona sobre la forma que adoptó la agresión hacia su persona, por apoyar la Ley de Servicios de Comunicación Audiovisual y por considerar que la compra de la empresa \emph{Papel Prensa}se realizó bajo presión represiva de la dictadura. Víctor Hugo Morales fija el momento de inicio de la campaña contra su persona el 22 de agosto de 2010, luego de una carta que le enviara a Héctor D'Amico\footnote{Hector D´Amico era en ese momento Director ejecutivo de la Revista Noticias.}:

\begin{quote}
Desde aquel 22 de agosto en adelante (fecha de la carta a D'Amico) , la cantidad de apariciones negativas que sumaron a \emph{Clarín} y \emph{Perfil}, a \emph{El País} y \emph{El Observador} de Montevideo, a la CNN, los alineó como emprendedores de una campaña de desprestigio tal que es imposible imaginar algo parecido en los anales del periodismo. La nota que se escribió en la revista \emph{Noticias} de Perfil en septiembre de 2009, contemporánea al inicio de las acciones de \emph{La Nación} y \emph{Clarín} (cuando marcó el punto final al ninguneo de veinte años), contrasta con el ensañamiento posterior, y exime de presentar otras pruebas sobre la razón del \emph{periodicidio} que intentan cada día. En esa nota, la revista usaba expresiones como: \emph{Una pelea a muerte con Clarín. Uno de los periodistas más prestigiosos. Periodista intachable y relator exquisito. Milita activamente para que salga la Ley de Medios}. \emph{Con el rol de cruzado público anti-Clarín pone mucho en riesgo. Su apoyo a la nueva ley contrasta con los intereses de la propia empresa que lo tiene contratado. Su opinión combinada con su trayectoria es un mix explosivo para la imagen del monopolio. Es un periodista independiente al que solo un desvelado podría acusar de ser una voz cooptada por el oficialismo. Tiene el respeto de sus colegas, muchos de los cuales lo toman como un ejemplo de su profesión, y una trayectoria intachable} (\ldots) La Ley de Medios y Papel Prensa, nacidos a la luz de ese tiempo de bisagras de 2009 y 2010, explican por qué los Saguier y Magnetto dan letra y aval a unos quince o veinte torturadores que, para rebajarme hasta su condición moral, me tocan con la picana de sus mentiras (Morales, 2014: 42).
\end{quote}

Morales describe su percepción acerca del intento de devaluar su lugar dentro del campo, por parte de los medios dominantes, promoviendo la idea de que su actuación tiene intereses ideológicos. En tal sentido, señala una serie de datos que fueron utilizados con el objeto de desacreditar su imagen.

\begin{quote}
El día 24 de octubre de 2010 escribí un correo a Claudio Gurmindo, entonces jefe de Redacción del diario \emph{Perfil}. En ese \emph{mail} decía: Le quiero contar a Jorge (Fontevecchia) que el escrache va bien. Al cabo de meses tomado de punto ya sea por lo que dije de los que escriben a Perfil.com, por la nota a un demente que permaneció en pantalla varias semanas con el título Farsante, o por lo que dijo de mí La Cámpora cuando yo tenía una posición favorable al campo, ya contabilizo un par de agresiones con la bandera de ``tiene razón Perfil''. Vengo del cine de Patio Bullrich. Fui con mi mujer y mi hija más chica. Una provocación absurda terminó con una discusión, o, mejor dicho, con un tipo insultándome con el pretexto de una respuesta mía a ``es como dice Perfil, sos un vendido''. El 26 de octubre, Fontevecchia respondió mi correo. Escribió: Víctor Hugo: Lamento mucho lo que te sucedió en el cine agravado por estar con tu hija más chica. El viernes de la semana anterior tuve una situación similar en la Facultad de Filosofía de la UBA donde realizo una maestría. Deploro la violencia aun verbal\ldots{} A fin de 2010, en su edición del 24 de diciembre, y como regalo de Navidad, la revista Noticias me eligió, junto con Orlando Barone, como EL PEOR PERIODISTA DEL AÑO. El 16 de febrero de 2011, Jorge Fontevecchia envió un correo a Gustavo González, con copia a Fabiana Segovia, en el que hablaba de esa encuesta. Decía: Acabo de leer un reportaje a Víctor Hugo en la revista Pronto que me llena de tristeza porque se refiere a mí muy injustamente. Te pido que como director de la revista Noticias le expliques que yo no tengo nada que ver con la elección del mejor y el peor periodista del año que hace esta revista, ni con la elección del jurado, que tampoco nunca tuve nada que ver con todas las selecciones de los años anteriores, y que estuve muy apenado cuando me enteré que él había sido electo por el jurado de este año al punto de solicitarte que lo llamaras para transmitirle nuestras excusas y explicaciones del caso (Morales, 2014: 44, 45).
\end{quote}

Morales desgrana, ejemplo tras ejemplo, las acciones de demolición de su imagen. Lo invitan a una entrevista para la cadena CNN, para hablar sobre libertad de expresión en la Argentina. Acepta. La primera pregunta que le formulan es: ``Señor Morales, dicen que usted es la voz del gobierno...'' (Morales, 2014: 50). Interpreta Morales (2014: 51):

\begin{quote}
``La voz del gobierno'', como me llamó la CNN durante la entrevista, es la construcción diseñada por los medios de Magnetto-Saguier para debilitar mi credibilidad en la discusión de la Ley de Medios.
\end{quote}

Percibe, también, los mecanismos de desacreditación que aplican los medios a través de sus periodistas:

\begin{quote}
Con los siete sicarios que me atacan desde \emph{La Nación} a partir de la batalla de la Ley de Medios suprimieron mi honor frente a mucha gente, como si apretasen una tecla de la computadora. ``Está por borrar el honor de una persona. ¿Está seguro de que quiere hacerlo?''. La máquina da opciones. Continuar o Cancelar. Ellos ordenan continuar. Lo hacen mientras a cara descubierta, con la libertad de expresión en ristre, dejan de pagarle cientos de millones de pesos a la AFIP (Morales, 2014: 58).
\end{quote}

Dentro de los mecanismos que utilizan señala el ``escudo hipócrita de la libertad de expresión'':

\begin{quote}
Utilizar el aspecto emocional más que la reflexión para abrir la puerta al inconsciente e implantar ideas, deseos, miedos y comportamientos manteniendo al lector, al telespectador en la ignorancia y la mediocridad con la que se lo quiere, complaciente, cerrando los caminos de una meditación más profunda y personal. El escudo hipócrita de la libertad de expresión para corromper el entendimiento de la realidad, creando una ficción agobiadora (Morales, 2014: 59).
\end{quote}

Más adelante, el libro reproduce una parte sustancial del fallo de la Corte Suprema de Justicia según el cual la ley de medios no vulnera la libertad de expresión:

\begin{quote}
No se discute en estos autos una cuestión meramente patrimonial, dado que el derecho de propiedad queda a salvo en caso de probarse daños emergentes de actos lícitos del Estado; tampoco se agota la discusión en torno a los derechos de información ni de expresión que, por otra parte, no están lesionados por esta ley. Lo que en el fondo se discute ---apelando a tesis descartadas hace más de un siglo en su país de origen--- es si se deja o no la configuración de nuestra cultura librada a la concentración de medios en el mercado. Jurídicamente, permitirlo sería una omisión inconstitucional, porque lesionaría el derecho a nuestra identidad cultural.{[}\ldots{]}. Permitir la concentración de medios audiovisuales, renunciando a una regulación razonable, que puede discutirse o ser todo lo perfectible que se quiera, pero que en definitiva no se aparta mucho de lo que nos enseña la legislación comparada (a veces más limitativa, como respecto de la prohibición de la propiedad cruzada), en estos tiempos de revolución comunicacional y más aún de nuestras características, sería simple y sencillamente un suicidio cultural (Morales, 2014: 65).
\end{quote}

Según Morales (2014: 68), los medios dominantes recurrieron a la Comisión Interamericana de DDHH.

\begin{quote}
Llegaron a la Comisión Interamericana de Derechos Humanos en Washington, en los Estados Unidos, con el discurso de los medios dominantes. Pero la traza era distinta de la habitual, y chocaron a la salida de la primera curva inesperada. Fue cuando una integrante de la Comisión preguntó con aire ingenuo y malicia de jugador de póquer si lo que estaban denunciando como un ataque a sus personas no era parte de lo que ellos reclamaban, o sea la libertad de expresión. Miles de kilómetros para que ese simple interrogante arruinara una presentación sobre la que la corporación había puesto sus expectativas más dañinas.
\end{quote}

El análisis, en detalle, del proceso de conceptualización del par censura/libertad de expresión, tiene por objeto revelar el proceso de su construcción como instrumento de legitimación.

\begin{quote}
Los dirigentes del fútbol que discutieron el dinero que les pagaba por el fútbol se convertían en títulos del mal manejo de los clubes. Los obedientes, en cambio, eran poco menos que Gardel. El presidente de AFA, Julio Grondona, fue el hombre con mayor protección en la historia de los medios. Les dio la llave del negocio más indecente que Clarín haya tenido porque la estafa era doble. El Grupo, \textbf{con su rebenque de la libertad de expresión} haciendo círculos en el aire, robaba a los clubes despojándolos de su dinero y de su dignidad. Los empobrecía y los obligaba a venir al pie. A pedirles a ellos miserables adelantos que le servían a Magnetto para exigir más, siempre más (\ldots). Hacen sus negocios offshore, tientan a la embajada más poderosa del mundo para que actúe contra un gobierno democrático, distorsionan la plaza económica con sus operadores, atizan los fuegos de las hogueras en las que atan a un palo a los funcionarios y giran en círculos la copa de sus cócteles para que el hielo les asegure el placer del trago. Una buena vida, matizada por alguna citación judicial. ``Andá a ver qué quieren ahora'', le dicen a su abogado e instruyen a sus confidentes de la redacción para que busquen antecedentes del juez que envió la cédula, ``\textbf{y vayan viendo cómo conectar el asunto con un ataque a la libertad de expresión}'' (Morales, 2014: 77, 114).\footnote{El énfasis es propio.}
\end{quote}

Morales tiene perfecta conciencia de la función que cumple el descrédito y el estigma dentro del campo y la desvalorización de sus acciones simbólicas dentro del mismo:

\begin{quote}
¿Sabe, Magnetto, cuándo escribí por vez primera sobre Clarín? En 1987, hace veintiséis años. ¿Sabe desde qué fecha está documentado que hablo contra los multimedios como el suyo y denuncio los perjuicios que provocaría al periodismo, a la sociedad y a las relaciones del poder? Desde 1991. ¿Entiende lo que eso significa de libertad en mi conciencia? La misma que me provoca saber lo que he perdido económicamente en estos años, porque, mientras usted me ensucia, la realidad es que de publicidad he dejado de percibir más del sesenta por ciento de loque está pautado, usted puede preguntarle al actual director de Radio Continental(2014: 15).
\end{quote}

Eduardo Blaustein, periodista reconocido de múltiples medios, interviene en el conflicto con un libro complejo, tras la aparente simplicidad o espectacularidad de su título: \emph{Las locuras del Rey Jorge}. Se trata de un intento de biografía periodística de Jorge Lanata, con un subtítulo provocativo: \emph{1983-2014: Periodismo, política y poder. El ascenso al Trono de Jorge Lanata}. Afirma Blaustein (2014: 11) en la introducción para aclararnos de quién estará hablando:

\begin{quote}
Peso pesado, pesadísimo, Jorge Lanata es el periodista de mayor influencia en \emph{eso} que se llama la opinión pública argentina.
\end{quote}

Blaustein intenta defender y comprender a un Jorge Lanata con quién ha trabajado, y a quien admira y respeta:

\begin{quote}
Si es por la discusión sobre la independencia periodística, hay que revolver bien el archivo para encontrar, por cada cien veces en que Lanata se mostró transparente y corajudo, una frase que sinceró el vínculo entre el periodismo, su viabilidad económica y el poder. ``Cuando un socio viene a poner guita no le mirás la cara'' ``¿Vos pensás que esto es una historia de amor?''\,``Son relaciones donde a vos te usan y vos usás''. Lanata es el representante más excelso del periodismo justiciero (Blaustein, 2014: 12, 13).
\end{quote}

Concluye su introducción:

\begin{quote}
El intento de hacer un libro justo, si es que eso existe siendo que este es un libro hecho desde mi propia subjetividad, obligaba entre otras cosas a retroceder en el tiempo. Un libro justo no podía centrarse en el escarnio que hace el kirchnerismo de Lanata. Eso no quita lo que ya asoma en esta introducción: mis propias lecturas, mis propias irritaciones con el Lanata contemporáneo, el que en los últimos años ocupa un lugar pesado ---muy pesado y muy rico de analizar--- entre las corporaciones y las masas (Blaustein, 2014: 17).
\end{quote}

Consideramos que Blaustein se inscribe entre los periodistas que denominamos \emph{profesionalistas}. Cree profundamente en la profesión de periodistas, se sentía cómodo en el campo periodístico y se encuentra en medio de una batalla donde el grupo dominante pretende expulsar a los herejes.

Sobre la perspectiva de Lanata en relación a la libertad de expresión y la censura, deja una crónica rica en anécdotas periodísticas desde la juventud de Lanata. Repasando la experiencia de Lanata en la revista \emph{El Porteño} (publicación que data de fines de la última dictadura) recuerda:

\begin{quote}
Lo interesante era cómo, veloz, George comenzaba a ocupar espacio (y aquí lo mismo: \textbf{los periodistas matamos por firmar, por figurar}). Inmediatamente después de su nota de cuatro páginas, Lanata firmaba su columna de opinión ---``Sobre la libertad de expresión''---en la que cuestionaba, con buenos argumentos, la actitud del gobierno radical de impugnar opiniones o informaciones periodísticas con el sambenito de la ``desestabilización''. ``Habitualmente ---decía Lanata--- se confunde defensa del sistema con defensa del gobierno. Así, Alfonsín pasa a encarnar a la democracia''. Escribía también Lanata: \textbf{``La prensa, y por su intermedio la opinión pública, son elementos de control gubernamental, no de control social''} (Blaustein, 2014: 35).\footnote{El énfasis es propio.}
\end{quote}

Inmediatamente, sigue Blaustein reflexionando sobre la problemática de la relación entre la prensa, el gobierno. Aborda, sobre todo, el derecho que tienen los funcionarios de discutir o refutar a la prensa:

\begin{quote}
La discusión al respecto, eterna, estalló, se enriqueció, se extendió y se polarizó mal en el ciclo Kirchnerista, particularmente a partir de la ruptura de relaciones entre el Grupo Clarín y el Gobierno. Sobre eso, el rol transparente e inimputable de ``la prensa'' se hablará mejor y más adelante en este libro (Blaustein, 2014: 35).
\end{quote}

En estos párrafos, Blaustein no solo nos transmite la perspectiva de Lanata sobre la libertad de expresión, sino también pequeñas apostillas de la particular personalidad de los periodistas. La frase ``matamos por firmar, por figurar'' --y la habilidad de Lanata para lograrlo-- permite acercarnos al proceso de encadenamiento de la relación entre periodistas y audiencias. Nos permite también ver de qué manera, editor, audiencia y periodistas se relacionan en un proceso de demanda y satisfacción recíproca.

Lanata expresa, y lo hemos visto en otras referencias a su persona, a un periodista que concibe al Estado y al gobierno como el poder censor. No le reconoce al gobierno derecho a réplica y considera a la prensa como expresión de la opinión pública y no a la inversa. Esta perspectiva, que hoy podríamos denominar ingenua, permea muy fuertemente la creencia entre los periodistas.

Blaustein destaca el compromiso de Lanata con las políticas de derechos humanos, memoria, verdad y justicia:

\begin{quote}
De hecho, la primera nota que firmó George, el 2 de junio de 1987, tuvo relación con la herencia de la dictadura. Pero no exactamente con los juicios sino con el debate provocado por un grupo de procesistas que pretendió publicar una solicitada a favor de Jorge Rafael Videla. Para explicar su postura o la del diario respecto a ese dilema sobre libertad de expresión (¿había que publicar esa solicitada?) Lanata se apoyó con elegancia en un editorial del diario \emph{Libération}---como si Europa proveyera una autoridad moral que Argentina no--- que plantea el mismo dilema en torno de las afirmaciones de un historiador negacionista del Holocausto. El artículo de \emph{Libération} concluía más o menos con un que se vaya a cagar su libertad de expresión, aunque en francés: ``En nombre del principio de prohibido prohibir se acaba por tomar libertades insoportables con la libertad, hasta comprometer a esa misma libertad (. . .) Es la ocasión para \emph{Libération} de reafirmar que esta empresa de prensa responde a una cantidad de reglas éticas también en el correo de lectores. Más allá de esas reglas, la publicación no puede ser posible''. George, reitero, solo citaba. Ya se ``jugaría mejor'' en muy poco tiempo (2014: 63).
\end{quote}

Estos ejemplos que Blaustein narra sobre Lanata vienen a recordarnos que tendría principios éticos. Blaustein intenta una defensa de Lanata en tanto lo considera el periodista más influyente de la Argentina. Más adelante agrega:

\begin{quote}
Una buena porción de los textos que publicó en Página {[}12{]} están dirigidos a defender la libertad de expresión, a cuestionar lo que hoy se llaman embates contra la prensa, a responder las agresiones brutales, hoy olvidados, de funcionarios menemistas y del propio Carlos Menen por cada denuncia aparecida en el diario. El menemismo no ahorraba calificativos que casi parecen ganarles en voltaje a los de los tiempos Kirchneristas: ``traidores'', ``judíos'', ``terroristas'', ``delincuentes periodísticos''. Se multiplicaban por entonces las querellas por \textbf{calumnias, injurias o desacato, figuras delictuales que fueron abolidas en tiempos Kirchneristas, para tranquilidad y mejor libertad de los periodistas} (Blaustein, 2014: 69).\footnote{El énfasis es propio.}
\end{quote}

El campo periodístico no tolera una perspectiva no monista de la información. En general, los sujetos sobre los que discurre la información carecen de la posibilidad de réplica; los sujetos sobre los que recae el acontecimiento informativo o periodístico no puede aclarar o contradecir la información periodística y, los actores políticos, han aceptado el rol de operadores de medios como fuentes calificadas que negocian información por favores. Cuando alguien disputa la información, el periodismo lo considera una acción de censura, ``embates contra la prensa'', aunque quienes lo hacen hayan eliminado las leyes que limitaban la libertad de los periodistas.

El campo es una construcción colectiva, intersubjetiva y autónoma de la voluntad de sus integrantes. Eso no quiere decir que la voluntad de cada uno de los periodistas no intervenga en el proceso de su propia incorporación al campo y, una vez dentro, mantener la pertenencia y disputar la centralidad.

\begin{quote}
Tema importante subrayado por Lanata en esa conversación: que el diario gustara mucho a periodistas de otros medios significó mucho en términos de irradiación (o simple copia) en otros espacios. Otro igualmente importante: si la revista El Porteño, en su criterio, alcanzaba como para ``asustar a las abuelitas'', George apostaba a más masividad. Decía también, reivindicando la categoría, que Página era ``un diario alternativo", aun cuando fuera al mismo tiempo un diario comercial viable (Blaustein, 2014: 71).
\end{quote}

El reconocimiento interpares es central a los integrantes del campo. El reconocimiento por parte de los actores centrales brindaba legitimidad, prestigio y protección a un medio pequeño:

\begin{quote}
Aceptaba también la necesidad de apoyarse en determinados actores políticos para sostener la viabilidad comercial del diario (en algún momento fue Carlos Grosso desde la municipalidad porteña, Antonio Cafiero desde la Provincia). Me tocó alguna vez, como editor de la sección Sociedad, tener que evitar algún tema que irritara a la Municipalidad en tiempos de Grosso. Me molestaba un poco entonces; lo entendí mejor después. Imaginen los lectores de otros medios, mucho menos libres que Página/12, y sin embargo mucho más ricos económicamente en cuanto a ingresos por ventas o publicidad. El más poderoso de todos es el Grupo Clarín. Sin embargo calla mucho, como sabe buena parte de los lectores, porque lo primero son los dividendos (Blaustein, 2014: 71).
\end{quote}

Este tipo de declaración de principios de un periodista como Blaustein no impresiona tanto por la sinceridad, sino sobre todo por la creencia que el campo tiene acerca de la capacidad de sus lectores para discernir las condiciones y limitaciones autoimpuestas para generar la agenda, los temas y los encubrimientos. Llamativamente, \emph{Página/12} fue y es un diario con escasa superficie dedicada a la publicidad. Desconocemos el esfuerzo puesto en vender espacio publicitario, pero el reconocimiento de Blaustein permite inferir que el sostén económico venía dado bajo formas menos profesionales, es decir de financiamiento no explícito.

En el párrafo siguiente Blaustein (2014: 83) afirma:

\begin{quote}
Periodismo y política suelen discurrir por lógicas, límites y problemas comunes. Y no necesariamente, como proponen el propio discurso y la ``ideología periodística'', o George en particular, como un plano de elevación de todas las calidades, un horizonte superador de todas las cosas. Alguna vez el ex presidente Eduardo Duhalde debió desdecirse en cuotas para pasar de aquella expresión sobre ``la mejor policía del mundo'' a otra cosa muy distinta. Lanata, lo reitero, entendiblemente, calló durante años la historia del financiamiento inicial de Página. Hasta que tal vez su dolor por verse sin diario, su afán de protagonismo, o sus peleas con quienes quedaron al frente de Página, lo llevaron a boquear el asunto en entrevistas públicas. Demás está decir, a la cúpula directiva esa ruptura del acuerdo de silencio no le cayó nada bien. Si había una separación ya considerablemente tensa entre Lanata y los otros fundadores del diario, ese foso se ensanchó.
\end{quote}

En el párrafo anterior queremos destacar la percepción que el autor tiene acerca de que habría una ideología periodística que considera al periodismo como algo que está por encima de todas las cosas.

La cercanía de Blaustein con el protagonista de su libro, permite recorrer las idas y vueltas de los periodistas que acompañan a Lanata en sus reflexiones sobre el periodismo y la libertad de expresión. Lanata respondiendo a Martín Caparrós sobre una nota crítica del denuncismo afirma:

\begin{quote}
La nota de Martín no solo es altanera y un poco despreciativa, sino que le adjudica a la prensa una importancia que, gracias a Dios, no tiene. Sostener que la difusión de los hechos de corrupción evita el protagonismo de la política es demasiado lineal y excluyente. Ambas cosas bien podrían ser reflejadas por los medios si les interesara hacerlo (\ldots) Bien podría decirse que su brillante y un poco extensa obra sobre los setenta dejó en Caparrós un discutible tono vanguardista (Blaustein, 2014: 129).
\end{quote}

Y continúa:

\begin{quote}
Caparrós define al periodismo como ``el arte de categorizar el mundo''. Me encantó; desde Karl Kraus que no leía una definición tan redondita. No estoy de acuerdo: no somos tan importantes. Creo que es la cultura la que categoriza el mundo, con la colaboración estelar de la tradición, las iglesias, la policía, la escuela, la publicidad y ---no seamos tampoco tan modestos--- también los medios (Blaustein, 2014: 130).
\end{quote}

En el párrafo anterior puede apreciarse, por un lado, la autoconciencia de Caparrós y, por otro, la claridad con que Lanata minimiza la importancia que el campo periodístico le asigna a su labor. De modo que, para Lanata, los periodistas solo son narradores de un mundo que ha sido previamente significado.

Hacia el final del libro, Blaustein sigue el recorrido que llevó a Lanata a posarse en las manos de Clarín. Lanata es perfectamente coherente con el campo periodístico. A pesar de sus críticas a algunos medios --incluso a Clarín--estas críticas siempre fueron desde el campo y nunca disputó discursivamente con otros periodistas:

\begin{quote}
Pero también entiendo que hoy la discusión de Clarín va más allá de Clarín mismo, \textbf{hoy lo que se está discutiendo con Clarín es la libertad de prensa de la Argentina}. Entonces yo sé de qué lado estoy. Esto empezó de una manera totalmente accidental. Cuando Moyano le manda camiones a Clarín le bloquea la salida yo dije que me parecía mal, yo estaba en la tele y dije que Clarín tenía que salir. Al otro día me llamó toda la línea de Clarín para agradecerme. Todos, Rendo, Kirschbaum, Adrián, todos, D'EIía (N. de A: por el gerente de noticias de Canal 13, Carlos D'Elía). Y yo dije, no hay nada que agradecer, yo quiero que los diarios salgan (...) No es un problema de que yo estoy a favor de tal o en contra. Y después otra cosa: yo fui yo antes de Clarín y voy a ser yo después de Clarín. Entonces por eso a mí como que no me preocupa en nada. Hoy las circunstancias objetivas hacen que a Clarín le sirva y a mí me sirva estar ahí (Blaustein, 2014: 294).\footnote{El énfasis es propio.}
\end{quote}

Blaustein comienza a cerrar el libro confrontando su imagen del Lanata que conoció al Lanata del momento en que escribe:

\begin{quote}
Los últimos textos que tomé como \emph{corpus} para este libro antes de darle un final se hacen algo desoladores. Desoladores desde la subjetividad del que escribe, o vistos desde el punto de vista de lo que fue el George de los 80 y los 90 (Blaustein, 2014: 346).
\end{quote}

Luego de ejemplificar con dos o tres temas\footnote{Conflicto con la Policía de Córdoba en conjunción con un conflicto con la Gendarmería Nacional, por aumento de salarios y reconocimiento de incentivos y el Proyecto del nuevo Código Penal.} que considera significativos en los que Lanata lo defrauda en su carácter de periodista, Blaustein enumera una serie de ítems en los que desgrana puntualmente su crítica. En esta crítica es posible hallar elementos significativos sobre ciertos principios que si bien no pueden considerarse deontológicos, parecen recorrer ciertos elementos que corresponden al perfil del buen periodista y que Lanata habría vulnerado. Ordenaremos los argumentos de manera diferente al autor, a fin de que sean funcionales a nuestro interés por elucidar las principales consideraciones sobre la acción periodística.

Es necesario destacar que Blaustein en ningún momento considera que Lanata ha dejado de ser periodista o viola principios de independencia y libertad de conciencia, ni siquiera que su cambio periodístico se deba a intereses económicos, como suele estigmatizarse a aquellos periodistas a los que se pretenden expulsar. Por el contrario, Blaustein observa una deriva ideológica y psicológica en el marco de un conflicto del que Lanata no puede o quiere rehuir.

En el punto cuatro de su argumentación, Blaustein muestra de manera sutil que el poder político disputaba la legitimidad de los principios de verosimilitud y realidad al campo periodístico:

\begin{quote}
4) El George dominante desde sus tiempos de Perfil, Crítica de la Argentina y fundamentalmente desde su trabajo en el Grupo Clarín achicó agenda al hacer un negocio personal de oposicionismo a ultranza contra un gobierno particular. De alguna manera y con sus evidentes límites, ese gobierno, el kirchnerista, amenazaba con quitarle banderas y de ahí en parte la furia de George: él en competencia no podía ser menos corajudo que el kirchnerismo, puro relato, pura impostura. En tiempos anteriores, crónicas, cuentos, notas, denuncias, investigaciones o editoriales tenían destinos más amplios y diversos. Todas las baterías de George, desde el Grupo Clarín, apuntan sí o sí contra un único actor político y sus aliados, algo que él desmiente sin mayor entusiasmo. No hay ya denuncias o críticas contra otros sectores (sí para cuestionar la debilidad general que padeció la oposición por lo menos hasta el 2013) ni contra el poder económico ni crítica estructural. La crítica que planteaba en sus años mozos contra el Poder Judicial hoy se dirige exclusivamente a los presuntos ``jueces k'' o a los ``embates'' del kirchnerismo contra la independencia de la Justicia. Hoy George, en pantalla o radio, no haría papelonear a Carlos Melconian como lo hizo en 2003 hablando, con Melconian en estudios, de su declaración de bienes. La agenda periodística de George es, puesto en sus palabras, de ``uso mutuo''. La agenda de George es la del Grupo Clarín, y la supera. George tiene una capacidad admirable para producir daño (Blaustein, 2014: 355-366).
\end{quote}

En el punto dos, Blaustein realiza una interpretación teórica de la actitud de Lanata en sintonía con esta deriva psicológica:

\begin{quote}
2) Las violencias de George contribuyen a una rabiosa polarización que es tóxica para la sociedad y empobrece todo debate posible. Me apoyo ahora en especialistas en comunicación. No en aquellos que optan por legitimarse dentro mismo de los medios naturalizando las lógicas perversas de la comunicación. Hablo de los críticos. Félix Ortega, por ejemplo, sociólogo y catedrático español, impugna dos categorías: la del ``populismo mediático'' y la del ``caudillismo electrónico'' (\ldots) Lo que entre nosotros se ha dado en llamar ``crispación'' es ante todo un efecto producido por una acción mediática tendiente, prioritariamente, a hacer inaceptable cualquier medida tomada por la ``otra parte''. Caudillo electrónico es George. Caudillo dedicado a la exacerbación del conflicto. 5) En \emph{Años de rabia} escribí con algún exceso de amabilidad o ingenuidad: ``No le cuestiono tanto que trabaje para el sistema Clarín como el pacto eminentemente político (...) que hizo para poner su revólver en alquiler, acaso como último y desesperado efecto del síndrome de adicción a la popularidad''. \textbf{Me desdigo parcialmente (en aquello de trabajar para el Grupo) porque no se puede tener la centralidad de George dentro del Grupo Clarín sin haber cedido}. Me apoyo en la frase utilizada por Reynaldo Sietecase cuando la entrega de los premios Tato: \textbf{``Una cosa es la crítica y otra es la operación política a medida de los sectores empresarios que me contratan''.} Operación de razonamiento simple, entre muchas posibles desde que trabaja para el Grupo, George olvidó continuar (o solo recordó superficial y convenientemente) sus denuncias sobre concentración comunicacional, objetó por inaplicable y un poco pelotuda a la Ley de Servicios de Comunicación Audiovisual ideada para paliar esa concentración. Dejó en el olvido la complicidad de Clarín con la dictadura, el maltrato de las empresas del grupo a sus trabajadores, la contaminación del río Baradero, ``peor que Botnia'' (Blaustein, 2014: 355-356).\footnote{El énfasis es propio.}
\end{quote}

Finalmente, Blaustein introduce dos elementos que deben considerarse a la hora de analizar el conflicto y los procesos de independencia y exclusión: los fenómenos de popularidad y espectacularización del periodismo:

\begin{quote}
The Lanata Syndrome, aquello de la adicción a la popularidad. Es sencillamente peligroso para la salud del buen periodismo, que una figura del peso de Lanata se deje arrebatar por el personaje que lo domina y desborda, aunque así se lo demanden las reglas del mercado, las del periodismo industrial o las audiencias. La valentía consiste en enfrentar esas reglas también. Un periodista no puede hablar desde la ceguera, desde la violencia, desde la soberbia (\ldots) (Blaustein, 2014: 357).
\end{quote}

Blaustein no solo realiza una reseña histórica de un periodista popular como Lanata, sino de su personaje, que representa un momento del periodismo argentino. La espectacularización del periodismo y su consecuente adicción a la popularidad produjo como resultado este tipo de fenómenos dentro del campo, que no son aislados sino que por el contrario, tienden a convertirse en elementos inherentes al mismo. Cuando Bourdieu señala la doble dependencia del campo periodístico, diferenciándolo de los demás campos intelectuales, expresa justamente que la popularidad, es decir la dependencia del nivel de consumo, se torna significativa y tracciona al resto de los campos, incluso al campo de poder político.

Durante este mismo año, 2014 se publica, un trabajo de índole político-gremial: \emph{Telam. El hecho maldito del periodismo argentino, una historia narrada por sus trabajadores}. Con prólogo de Alejandro Bercovich, el libro es un clásico exponente de bibliografía para la acción gremial, en el marco de un contexto de disputa dentro de la agencia de noticias oficial del Estado Argentino. De hecho, el sello editorial es el de la agrupación gremial que cobija o cobijó a la Comisión Interna. Esta publicación no puede escindirse de la disputa por la legitimidad del campo.

El libro expresa a sectores periféricos del campo, no por su reconocimiento interpares, que en muchos casos es elevado, sino por su carácter de periodistas desconocidos para el público. En general, los periodistas de agencias de noticias carecen de reconocimiento entre las audiencias.

Por otra parte, si bien los autores realizan un esfuerzo por reconvertir el significado de \emph{periodista} en \emph{trabajador de prensa}, el esfuerzo se ve frustrado, ya que la idea de trabajadores de prensa, ubica al profesional del periodismo, en una situación de dependencia con respecto a su empleador y esto es lo que se pretende ocultar. Tal vez, esto revele las dificultades de agremiación por un lado y al mismo tiempo los escasos éxitos gremiales obtenidos por el colectivo:

\begin{quote}
Este trabajo de Suárez y Bargach recupera y socializa desde la perspectiva de los laburantes la historia de Télam. Una historia que es la de sus periodistas y administrativos, sus luchadores y traidores, sus constructores y destructores. Y también la de sus fracasos y sus logros, que fueron muchísimos (Bercovich, 2014: 6).
\end{quote}

El libro parece intentar librar varias batallas simultáneamente. Por un lado una disputa con los periodistas dominantes de los grandes medios, a los que recrimina la carencia de independencia y libertad de conciencia. Además, sus autores son conscientes de que la narración del conflicto gremial se inscribe dentro del conflicto del campo periodístico y parecen intentar aprovecharse del mismo, presentándose como la expresión pura del periodismo argentino, enmarcando su lucha como expresión de una defensa general de la profesión como ``trabajadores de prensa''.

Por último intentan aprovechar la situación de relativa debilidad del gobierno frente al campo periodístico, para imponer condiciones en el emprendimiento periodístico (TELAM) con mayor cantidad de empleados sindicalizados.

Dice Bercovich (2014: 6-7) en su prólogo:

\begin{quote}
La libertad para \textbf{escribir rigurosamente y sin condicionamientos} en Télam, como lamentablemente no ocurre en todos los medios públicos, \textbf{es inversamente proporcional al poder del gobierno de turno}. En una democracia inmadura y que aún se debe el debate sobre la división entre Estado y partido gobernante, el afianzamiento del kirchnerismo y la reconstrucción de la autoridad presidencial que habían dinamitado Menem y De la Rúa no podían tener otro corolario que la pérdida de la autodeterminación que habíamos conquistado durante los años de la resistencia al cierre y la liquidación que querían imponer las gestiones de impronta neoliberal. \footnote{El énfasis es propio.}
\end{quote}

La frase de Bercovich asombra por lo que deja entrever como perspectiva idealizada de la tarea del periodista: una persona que ``escriba rigurosamente y sin condicionamientos'' solo existe en el imaginario de un periodista sin experiencia o para discursos de ocasión. Sin embargo, el análisis siguiente es revelador de las condiciones de trabajo de una agencia de noticias estatal. La condición de que a menor poder del gobierno de turno, menor condicionamiento de los periodistas, era reveladora de la tensión y la disputa al interior de la empresa estatal y de la batalla por la autonomía del campo en relación al funcionariado político, que en cada cambio de gobierno, hace sentir su capacidad de incidencia. Más adelante los autores afirman:

\begin{quote}
Luego, con la salida de Carlos Menem del poder retomó ciertos rasgos de independencia informativa, sobre todo durante el Gobierno de Fernando de la Rúa (Bargach y Suárez, 2014: 35)\footnote{La cita se refiere a TELAM}.
\end{quote}

Bercovich, irrumpe con un discurso de periodista militante en el mejor sentido del término; alguien que concibe al periodismo como un oficio violento, una herramienta comprometida con ``el futuro de su clase''.

\begin{quote}
Una contribución invalorable a la necesidad imperiosa de formación de quienes debemos tomar en nuestras manos la defensa de nuestros derechos laborales y la \textbf{simultánea reivindicación del violento oficio de escribir que ejerció Walsh hasta que la dictadura ahogó en sangre sus denuncias. Porque desvincular una tarea de la otra es propio de ganapanes alienados y burócratas sindicales y no de trabajadores conscientes y comprometidos con el futuro de su clase} (Bercovich, 2014: 8).\footnote{El énfasis es propio.}
\end{quote}

El libro realiza un \emph{racconto} histórico de la Agencia Télam, desde la primera Agencia de noticias. En este repaso, los autores van dejando su impronta acerca de la tarea y función de la prensa:

\begin{quote}
A Havas lo ayudó el contexto: plena expansión del capitalismo, Paris como centro cultural del mundo, una prensa en crecimiento, el auge de los Estados-nación, traslados masivos del campo a las ciudades. En esa conjunción de elementos que marcaron la transformación de la sociedad debe contarse también la lenta mudanza de la prensa política a la comercial (\ldots) Fundamental para comprender la Europa ---y no sólo--- de años siguientes, la Revolución Francesa tuvo características particulares: un fuerte consenso social, un corpus de ideas nuevas, la cohesión de esa nueva clase social y la defensa de derechos hasta entonces impracticados, entre ellos la de la libertad de opinión y expresión. Aunque estos derechos debieron atravesar una larga lucha antes de quedar establecidos, fueron en muchas sociedades el horizonte al que aspirar (Bargach y Suárez, 2014: 12).
\end{quote}

La percepción que los periodistas de una agencia de noticias tienen sobre la problemática del acceso a la información es sumamente interesante, ya que están en condiciones de evaluar las formas en que se propaga la misma. Asimismo, el libro revela como Internet se fue posicionando en el mercado de la información, pero no a favor --como suele considerarse-- de la democratización de la información:

\begin{quote}
Muy en contra de lo que suele pensarse, los avances tecnológicos, la multiplicación de plataformas, la explosión y expansión de internet y la diversificación de la información no generaron por sí mismo una democratización de la comunicación. Mucho menos en el universo de las agencias de noticias, que, por estructura y poderío económico, supieron adaptarse muy rápidamente a las nuevas necesidades del mercado, de modo de no perder influencia y retener el lugar de poderosas empresas controladoras, de modo más o menos explícito, de buena parte del mercado de la información (Bargach y Suárez, 2014: 49).
\end{quote}

En este intento de colocar en el contexto histórico la propia historia de Télam, los autores recuerdan el proceso que devino en la Comisión MacBride:

\begin{quote}
De alguna manera, ese clima fue la base para que se genere el debate sobre el rol de los medios. En junio de 1977, el irlandés Sean MacBride planteó ante la UNESCO la necesidad de que la organización analizara el rol de la prensa, el peso de las presiones económicas y financieras que sufrían los medios de comunicación y cómo influían los intereses de las multinacionales, que hasta manipulaban a los gobiernos de turno. Sus palabras tuvieron eco: apenas unos meses después, en la reunión de Nairobi, el entonces director general de la UNESCO, el senegalés Amador MahtaM`Bow, le propuso a MacBride encabezar una comisión internacional que se encargara de estudiar los problemas que existían en el ámbito de la comunicación\ldots{} Aunque el nombre oficial de ese cuerpo fue el de Comisión Internacional de Expertos en Materia de Comunicación, pasó a la historia simplemente como Comisión MacBride.En diciembre mismo de ese año la Comisión inició sus tareas, que durarán cerca de dos años y \textbf{se dieron en medio de un tironeo previsible: el que se generaba entre los países del Tercer Mundo, que protestaban por las informaciones que los medios de países centrales divulgaban, y las naciones desarrolladas, que defendían el principio de libertad informativa} (Bargach y Suárez, 2014: 55).\footnote{El énfasis es propio.}
\end{quote}

Los autores cierran esta parte de la historia comentando sobre el destino de la famosa Comisión MacBride:

\begin{quote}
En poco más de cinco años, ya con Ronald Reagan en el Ejecutivo estadounidense, una catarata de hechos terminó con el informe MacBride en un cajón: EEUU anunció su retiro de la UNESCO después haría lo mismo el Reino Unido-, habló del documento ``Voces múltiples/un solo mundo'' como un ``proyecto sovietizante'' y elaboró un texto propio que iba en dirección exactamente contraria al de Mac- Bride, y el senegalés M´Bow ---fuerte apoyo para el irlandés- fue relevado de la dirección de la UNESCO (\ldots) La llegada del español Federico Mayor Zaragoza, en 1987, al sillón principal de la UNESCO terminó por echar por tierra lo poco que quedaba de las aspiraciones de MacBride y su informe. Mayor Zaragoza, en busca de lograr el regreso de EEUU y el Reino Unido a la UNESCO, hizo público su deseo de enterrar el NOMIC y hasta habló de la Comisión como una iniciativa propuesta ``desde afuera'' de la organización (\ldots). Lo que seguiría sería aún más claro: En la reunión de París de 1989, la UNESCO se pronunció \textbf{por el libre flujo de la información como forma de resolver las dudas en lo referente a la eliminación de los desequilibrios: la supresión de los efectos negativos de ciertos monopolios; la eliminación de barreras internas y externas a la libre circulación de información y la pluralidad de fuentes y canales de información (\ldots).} Fue en esa carrera que todos los países atravesaron situaciones similares, con bandera de largada en la conformación de un sistema de medios estatales fuerte: duros enfrentamientos con medios privados, hasta entonces cómodos decisores de políticas y agendas, denuncias de las organizaciones internacionales de medios, cuestionamientos de fundaciones supuestamente guardianas de la libertad de expresión y denuncias también de opositores internos sobre los fondos que requerían esos medios estatales (\ldots). Télam tuvo el raro privilegio de sobrevivir a 1.000 días de intervención y varios intentos de liquidación de la fuente de trabajo. Tiene ahora la posibilidad de enorgullecer a todo un gremio como emblema de la libertad de expresión y de la lucha democrática por los derechos a la información y a la libre agremiación (\ldots). Lo que no hay que olvidar es que a pesar de todo, de la dictadura, de la debacle de la etapa de Alfonsín, de la precarización de los '90 y de embates flexibilizadores de toda índole, nadie pudo derogar el Estatuto del Periodista y el Estatuto de Empleados Administrativos de Empresas Periodísticas, aunque hoy, visto a la distancia y a la mutación de las empresas, no hay duda que muchos de los dueños de los medios violan estas normas que amparan derechos laborales pero también la libertad de Conciencia, de prensa, de expresión y los valores de la democracia (\ldots). El primer secretario general del SAP fue el periodista peronista Valentín Thiebaut, que se había iniciado en el oficio como crítico teatral y luego devino en especialista económico y editorialista del diario Democracia. Alineado con el gobierno, Thiebaut participó de la reunión anual de la Sociedad Interamericana de Prensa (SIP), donde la delegación argentina se retiró en la Jomada inaugural denunciando maniobras para ``desestabilizar'' al gobierno nacional bajo el enunciado de la ``libertad de prensa". La SIP no cambió (Bargach y Suárez, 2014: 58-59, 65, 85, 135, 148).\footnote{El énfasis es propio.}
\end{quote}

Durante el año 2014, ADEPA\footnote{Asociación de Entidades Periodísticas Argentinas. Agrupa a las empresas editoras de medios gráficos nacionales.} editó dos libros. Estos intentan insertarse en el debate de las nuevas problemáticas del periodismo, sin desatender su disputa con el gobierno y los medios no críticos.

\emph{Nuevos desafíos de la prensa}, trabajo colectivo de periodistas argentinos y extranjeros, reflexiona centralmente sobre el significado que tiene para la prensa el surgimiento de los nuevos medios de comunicación, \emph{Internet}, \emph{smartphones}, redes sociales, etc. En la mayoría de los artículos se rescata la importancia y el valor de la tarea del periodista, pero se minimiza la importancia de los medios tradicionales y se propone una adecuación de la profesión a los nuevos medios.

Hay pocas referencias en el libro sobre la problemática de la libertad o la censura, como si los nuevos medios, de alguna manera, estuvieran exentos de dichos conflictos. Sin embargo, José Ricardo ``Pepe'' Eliaschev (2014: 38) no se priva de mencionar el autoritarismo del gobierno nacional.

\begin{quote}
Los gobiernos de fuerte pulsión autoritaria y dirigista, como el de Cristina Fernández en la Argentina, piensan con obsoletas categorías de hace un cuarto de siglo cuando fantasean con batallas liberadoras para jaquear a ese fantasma vaporoso llamado ``los medios''.
\end{quote}

\emph{Tiempos turbulentos. Medios y libertad de expresión en la Argentina de hoy} el otro libro editado por ADEPA es un libro colectivo escrito por periodistas e intelectuales periodistas que desarrollan de manera meticulosa las diversas formas de violación del principio de libertad de expresión desde que asumiera el gobierno Néstor Kirchner. En sus 146 páginas, ``libertad de expresión'' o ``libertad de prensa'' aparecen en 112 oportunidades y su antónimo ``censura'' en 19 ocasiones, es decir, ambos términos aparecen casi una vez por página. El nivel de recurrencia llama la atención en la lectura más ingenua.

Explícitamente el libro se presenta como un incentivo a la reflexión sobre el problema de la libertad de expresión.

\begin{quote}
Aspiramos a que los escritos incluidos en esta obra alienten la reflexión sobre el valor de las libertades de expresión y de prensa como piedras basales del debate democrático (Jornet y Dessein, 2014: 11).
\end{quote}

Desde el prólogo, los compiladores establecen una línea divisoria que fortalece el principio rector del campo: la libertad de expresión como madre de todas las libertades- Por otra parte, existiría una ideología de la censura destinada fundamentalmente a limitar el control ciudadano sobre los actos de gobierno.

\begin{quote}
Si bien los autores ahondan en la realidad argentina y específicamente en un momento histórico, sus análisis constituyen aportes válidos para comprender la ideología que subyace en quienes, aquí o en otros países, atacan a la prensa independiente como un modo de silenciar la discrepancia, de instaurar un discurso único, de limitar el control ciudadano sobre los actos de gobierno. Por ello, y más allá de los tiempos turbulentos que hoy nos ocupan, este libro debe ser leído, antes que como un retrato de época, como una advertencia aplicable a todos los tiempos y en todo lugar. Porque la libertad de expresión, vale recordarlo, permite comprobar que sean respetadas las demás libertades ciudadanas (Jornet y Dessein, 2014: 11).
\end{quote}

Magdalena Ruiz Guiñazú (2014: 19), expresa en el libro, el sentimiento de los periodistas enfrentados a una voz que disputaba la centralidad y legitimidad de la narrativa periodística.

\begin{quote}
Son tiempos difíciles para el periodismo en la Argentina. Siempre hemos pensado que informar y tener opinión propia eran la base de nuestra profesión. Y que, también, por ese derecho todos los periodistas argentinos viviríamos, en democracia, una realidad en la que el disenso pudiera expresarse sin violencia. Hoy (y en los últimos años) han proliferado los carteles callejeros sin pie de imprenta con nuestros rostros y textos insultantes que funcionan como instancias de ``juicios'' ilegales y plagados de falsedades frente a la Casa de Gobierno, y el uso de la cadena nacional de radio y televisión como medio de acusación. Se ha instalado en nuestro país algo así como ``el que no piensa como nosotros es un enemigo''.
\end{quote}

Analizado con detenimiento, el texto de Ruiz Guiñazú considera que hay violencia en el disenso si este no se expresa en los medios masivos y preferentemente pertenecientes a empresas privadas. Más aún, disentir con los grandes medios sería, en sí, un acto violento.

Más adelante, Ruiz Guiñazú (2014: 19) busca legitimar su posición con una palabra autorizada dentro del periodismo internacional: Ignacio Ramonet. La autora utiliza una entrevista realizada por Jorge Fontevecchia en septiembre del año 2011.\footnote{http://www.attacmadrid.org/?p=5496}

\begin{quote}
Se acuña también ahora el término ``periodismo militante'' (tal como decía Ignacio Ramonet, en una de sus últimas visitas a Buenos Aires, ``periodismo militante no es información''). Creo, citando siempre a Ramonet, que ``el periodismo tiene que aportar cívicamente algo a la sociedad. Si está al servicio de tal interés privado o de tal interés político, no está construyendo ciudadanía'\textquotesingle{} Y me atrevo a añadir que medios de propaganda (y mucho menos pagados por el gobierno de turno) no son periodísticos.
\end{quote}

Sin embargo en esa entrevista a Ramonet, el reconocido periodista, más allá del concepto ``periodismo militante'', postula una hipótesis opuesta a la sostenida por la periodista. En la entrevista mencionada, Jorge Fontevecchia hace un serio esfuerzo por traccionar a Ramonet hacia su propia posición y, finalmente, logra que diga las palabras mágicas ``periodismo militante''. Sin embargo, Ramonet no le asigna esta categoría a los periodistas del programa televisivo \emph{6 7 8}

El capítulo 4 del libro corresponde a Gustavo González (2014: 30, 32), director periodístico de Editorial \emph{Perfil}. El capítulo desarrolla una estructura argumentativa sumamente interesante sobre las formas comunicativas que implementaron en sus mandatos los gobiernos de Néstor Kirchner y Cristina Fernández de Kirchner:

\begin{quote}
En esa construcción, los Kirchner le concedieron a la televisión abierta un papel fundamental en la batalla. Al igual que las anteriores administraciones, usaron a Canal 7 como un medio de propaganda gubernamental, pero con variantes clave. Primero, triplicaron la audiencia con el Fútbol para todos, llevando un punto habitual de rating a casi tres de promedio mensual, con picos de más de treinta puntos durante algunos partidos (casi 6 millones de espectadores en todo el país). Desde allí se emiten alabanzas y avisos de kirchnerismo explícito gol tras gol y jugada tras jugada (\ldots). También se le sumó al oficialismo tradicional de su programación el oficialismo combativo de 6, 7, 8, con emisiones diarias (salvo los sábados) y en horario central. Sus cerca de tres puntos de rating equivalen a alrededor de 300.000 espectadores entre la ciudad y el Gran Buenos Aires, más otros 300.000, aproximados, que se calculan en el resto del país. Total: 600.000 personas sometidas todos los días a los mismos mensajes (\ldots). Al oficialismo militante del canal público se le suma Canal 9. Desde hace tres años, la emisora abrió su burbuja de culebrones latinos al oficialismo crudo de los programas del creador de 6, 7, 8, Diego Gvirtz, quien en 2010 llegó a la emisora con el pan de la publicidad oficial debajo del brazo. Así comenzaron a emitirse \emph{Duro de domar}, de lunes a viernes; y TVR, los sábados. Ambos programas le garantizan al Gobierno unos 660.000 espectadores, promedio, por Duro...de lunes a viernes, más otra cantidad similar por TVR (\ldots). A la programación de Gvirtz en Canal 9, a mediados de 2010 se agregó Bajada de línea, el envío conducido por Víctor Hugo Morales, también alineado con las opiniones del oficialismo. El célebre conductor de radio representó una incorporación clave para los estrategas gubernamentales, por avalar cada medida del Gobierno con su prestigio personal. Aporta una audiencia de entre 400.000 y500.000 espectadores en todo el país.
\end{quote}

El periodista analiza con detenimiento el volumen de audiencia y concluye que:

\begin{quote}
Es el mayor aparato comunicacional puesto en marcha por el Estado desde 1983. Una ``Korpo'' que no se limita a la obsecuencia clásica de medios que quieren quedar bien con el poder de turno, sino que le suman una militancia desembozada y belicosa, por momentos demasiado similar a la que acompañaba a los distintos gobiernos militares (González, 2014: 33).
\end{quote}

En un capítulo posterior Sergio Berensztein (2014: 35), insiste en la hipótesis general del libro:

\begin{quote}
¿Cómo es posible que los medios de comunicación independientes estén siendo sometidos a un ataque tan brutal, desembozado y deplorable por parte de un gobierno que cuenta con un importante apoyo popular y plena legitimidad de origen?
\end{quote}

Cada uno de los capítulos aborda desde distintas perspectivas la siguiente petición de principios: Argentina, vivió uno de los ``períodos más bochornosos para la historia de la libertad de expresión''. Y continúa:

\begin{quote}
Por eso, lo que experimenta la Argentina no es solamente un conflicto grave y lamentable entre un gobierno y los medios de comunicación independientes, cuestión que de por sí constituye un hecho sumamente preocupante y merece una enérgica condena y repudio, pues \textbf{el riesgo real consiste en la censura abierta} o encubierta. Vale decir, una limitación insoportable y venal a la libertad (Berensztein, 2014: 37).\footnote{El énfasis es propio.}
\end{quote}

Marcos Novaro (2014: 43), en un intento por mostrar la estructuralidad del tipo de vínculo que el kirchnerismo mantuvo con la prensa, sostiene:

\begin{quote}
La relación de los Kirchner con el periodismo profesional e independiente estuvo plagada de conflictos y abusos de todo tipo ya \textbf{desde los inicios de su carrera política, en Santa Cruz,} y a todo lo largo de su ejercicio del poder político nacional. Pero no debe olvidarse que fue la crisis del campo, desatada en marzo de 2008, la que radicalizó drástica e irreversiblemente su actitud en la materia: las tensiones se convirtieron entonces en guerra abierta.\footnote{El énfasis es propio.}
\end{quote}

Alejandro Alfie (2014: 63), en su capítulo \emph{Recursos públicos, fines políticos}, hace hincapié en la problemática del financiamiento diferencial de medios de comunicación por parte del Estado:

\begin{quote}
En Abuso de publicidad oficial y otras formas de censura indirecta en América Latina (2008-2010), la Asociación por los Derechos Civiles (ADC) mencionó que el informe ``Diez desafíos claves para la libertad de expresión en la próxima década', en febrero de 2010, advirtió que "el abuso en la distribución de la publicidad del Estado o en el ejercicio de otras facultades estatales para influir en la línea editorial'' es ``uno de los aspectos más preocupantes dentro de los mecanismos ilegítimos de control gubernamental sobre los medios de comunicación''.
\end{quote}

Pablo Secchi y Rosario Pavese (2014: 74), en el artículo \emph{¿De qué hablamos cuando hablamos de publicidad oficial?}, se concentran en la problemática de la pauta publicitaria durante el período de los gobiernos de Nestor y Cristina Kirchner, aunque la cuestión no es privativa de estos gobiernos, ya que no existe normativa que establezca los criterios de distribución de la pauta publicitaria por parte del Estado.

\begin{quote}
Los posibles impactos del manejo discrecional de la publicidad oficial sobre la libertad de expresión son realmente relevantes. Si bien no existe un derecho intrínseco de los medios a recibir recursos del Estado por publicidad, sí existe el derecho de no ser discriminados por sus contenidos editoriales.
\end{quote}

Sin embargo, el artículo no logra encontrar lógicas relevantes de asignación discrecional de fondos, salvo el privilegio de asignar algunas publicidades a pequeñas publicaciones.

Arturo Guardiola (2014: 91-94), en \emph{Pensar, decir, debatir}, sostiene:

\begin{quote}
Un daño, más profundo que el causado por los ataques que sufre la libertad de prensa en nuestro país, es el que generan los agravios y avances del poder sobre la libertad de pensamiento. Se trata de un camino que nos lleva a los tiempos de la Inquisición y nos arrastra seiscientos años hacia el pasado, sembrando en la sociedad argentina un estado de ánimo colectivo de opresión e intolerancia (\ldots). La libertad de palabra es, desde entonces, un pilar del Estado democrático, indispensable para la formación de una opinión pública libre y el debate abierto de las ideas. Es una forma de ejercitar el autogobierno, han llegado a decir los tribunales; por eso, un ataque a la libertad de expresión es simplemente un ataque a la democracia (\ldots). La persecución y la estigmatización de personas o grupos, que se predica desde el poder, se acerca peligrosamente a lo que se conoce como ``hatespeech'' o ``discurso del odio''. Concepto que surge del impulso que las redes sociales le imprimen a la tendencia creciente a ejercer la libertad de expresión en los espacios públicos, donde las ideas se manifiestan muchas veces en forma relajada y violenta, definido como ``aquellas palabras que por su sola pronunciación infligen daño o tienden a incitar un inmediato quebrantamiento de la paz'' y ha dado lugar a que el Consejo de Ministros de Europa recomendara a los Estados miembros que deben impedirse.
\end{quote}

Como puede apreciarse, para este sector del campo periodístico, la crítica sobre la falta de libertades difícil de realizar y solo puede ser formulada de manera hipotética o como afirmaciones con escaso fundamento.

Silvia Mercado (2014: 107-108) en \emph{Años perdidos, batallas inútiles}, compara y asimila la década peronista (1945-55) con la situación que se vive en el momento en que escribe:

\begin{quote}
El peronismo se había transformado en una democracia netamente autoritaria, que regía la vida y la libertad de todas las personas. Mirtha Legrand, que en ese tiempo ya era una jovencísima y exitosa actriz, me dijo en una entrevista que "en esa época, o te ponías de acuerdo con Apold\footnote{Raúl Alejandro Apold~fue un~periodista~y~político~peronista~que ocupó la Subsecretaría de Prensa y Difusión durante la primera y segunda presidencia de~Juan Domingo Perón. Algunos historiadores le asignan haber sido el diseñador de las políticas de comunicación durante los primeros gobiernos del General Perón} o no trabajabas. Así de simple''. Y es imposible no relacionar lo que se vivía en esa época con la actualidad (\ldots). En las dictaduras, por supuesto, todo fue peor. \textbf{A la censura y las persecuciones}\footnote{El énfasis es propio.} se sumaron los secuestros, torturas y desapariciones. Pero estamos hablando de democracias que se vuelven autoritarias, de gobiernos votados por mayorías, que no respetan a las minorías, que quieren controlar la opinión y denuestan la investigación periodística independiente del poder. O peor, que cuando esas minorías se transforman en mayorías, como sucedió en 2009, haciendo un abuso de legalidad, votaron leyes que declaraban el fin de la diversidad.
\end{quote}

Llama la atención el intento de promover la percepción de la existencia de censura y persecución durante los gobiernos peronistas y asimilarlos a la última dictadura.

Alfredo Leuco, en \emph{Libertad de prensa de baja intensidad}, un trabajo escrito en 2006 e incorporado al libro de 2014, comienza con esta frase ``Este es el momento de menor libertad de prensa en la Argentina desde 1983''. Analizar esa afirmación y confrontarla con la realidad llevaría mucho espacio, deberíamos recorrer las experiencias de periodistas amenazados, asesinados, programas de televisión censurados, etc., entre 1983 y 2003. También en este artículo es preciso remarcar el esfuerzoe interés por instalar esta percepción.

El artículo termina diciendo:

\begin{quote}
Lamentablemente, la lista de situaciones que erosionan la libertad de prensa podría continuar. Con el modelo de hiperconcentración de muchos medios en pocas manos. O con el bloqueo que están sufriendo la ley de acceso a la información o la mismísima ley de radiodifusión de la dictadura que todavía nos rige. \textbf{También con el desprecio y maltrato que existe desde el Gobierno hacia las entidades que representan a las empresas o a los periodistas. Hablo de ADEPA, SIP (Sociedad Interamericana de Prensa), Fopea (Foro de Periodismo Argentino)} y tantas otras, que solo han recibido ofensas o negativas ante los pedidos de entrevistarse con el Presidente para dialogar civilizada y democráticamente sobre los temas en común (Leuco, 2014: 114).\footnote{El énfasis es propio.}
\end{quote}

La tensión que provocaba la exclusión de los ámbitos cercanos al poder y la confrontación pública con periodistas y organizaciones del campo periodístico es uno de los argumentos recurrentes para argumentar la limitación a la libertad de expresión.

Í talo Pisani (2014: 127), en \emph{Tolerancia cómplice}, se preguntaba:

\begin{quote}
¿Qué misteriosa razón ha provocado que líderes de opinión e influyentes personajes de la Argentina hayan refrendado a ciegas un modelo de simulación y permitido que se voltearan pilotes democráticos, se osara desafiar la Constitución y se avanzara contra derechos personalísimos como las libertades ciudadanas?
\end{quote}

Le asigna a la pregunta varias respuestas:

\begin{quote}
En el caso de los intelectuales y artistas, simpatizantes o militantes del Kirchnerismo, se impone la seducción ideológica, que en ciertos casos se combina con la seducción del dinero o la del trabajo garantizado y bien remunerado (\ldots). Con los medios de comunicación, ocurre en parte algo parecido a los intelectuales K. Pero solo en parte. (\ldots). Pero el hostigamiento a la prensa independiente tiene múltiples variantes más: la más sonora ha sido la ley de medios, que hasta ahora lo único que ha provocado fue concentrar más el espectro periodístico en manos oficiales y viene fracasando con su verdadero fin: ver derrotado al Grupo Clarín. Otras perversidades son el escrache a periodistas críticos (desde la propia presidenta o mediante afiches), la prohibición de publicar avisos en periódicos (vía dueños de cadenas de supermercados y electrodomésticos), la tentación de intervenir las asambleas de Clarín y la cruzada contra Papel Prensa, con la vocación de expropiar la parte necesaria de las acciones para hacerse del control de la distribución del principal insumo para diarios (Pisani, 2014: 128, 129).
\end{quote}

Fernando González (2014: 136), en \emph{La imprescindible recuperación del periodismo desafiante}, centra su análisis en el proceso de cooptación de medios por parte del gobierno kirchnerista y de qué manera este proceso traccionó a la exageración de la crítica a los medios:

\begin{quote}
Los empresarios cercanos al Kirchnerismo crearon nuevos medios de comunicación, o directamente avanzaron sobre los existentes mediante la compra a buen precio de diarios, emisoras de TV, radios y sitios web. El zar del juego, Cristóbal López, se quedó con el grupo de radios líderes y el canal de noticias C5N, que había construido Daniel Hadad; favorecido por contratos de obra pública generosos, Gerardo Ferreyra, de Electroingeniería, compró radio Del Plata y puso en marcha el canal digital 360TV; los hermanos Olmos, socios del gremio metalúrgico, compraron el diario de economía BAC y el popular e histórico diario Crónica, a los que sumaron una parte de la productora de TV Underground, y la sociedad entre Sergio Spolszky y Matías Garfunkel acumuló el diario Tiempo Argentino, los diarios gratuitos El Argentino y Miradas al Sur, la revista Veintitrés, las radios América y Vorterix, y el canal de noticias CN23. Todo ese conglomerado, sumado a algunos medios privados que cedieron parte de sus contenidos a la letra oficial y a la red de medios estatales, se convirtieron en los grandes propaladores de lo que se llamó el ``relato kirchnerista''.
\end{quote}

Finalmente reconoce:

\begin{quote}
Agobiados por ese escenario desfavorable y hostil que tuvo su pico máximo tras la contundente victoria y la reelección presidencial de Cristina Kirchner en 2011, los periodistas independientes habituados a ejercer la profesión con mirada crítica debimos extremar los esfuerzos para mantener el rigor y el equilibrio editorial. Muchas veces lo logramos y en ciertas ocasiones, es nuestro deber reconocerlo, cometimos el error de exacerbar la crítica en la misma medida que el llamado periodismo militante difundía patéticamente la propaganda oficial y se ensañaba en términos personales con muchos colegas por el simple hecho de opinar distinto.
\end{quote}

Fernando Laborda (2014: 140), en \emph{De la concepción kirchnerista de la prensa}, \emph{al espejo chavista} sostiene:

\begin{quote}
No falta mucho para que la Argentina kirchnerista pueda reflejarse en el espejo de la Venezuela chavista, que poco a poco ha ido sustituyendo a la sociedad por el Estado y que exalta un liderazgo personalista incompatible con instituciones y principios republicanos que despersonalizan el poder. Quizá pocas cosas como la relación entre gobierno y prensa asocien a ambos países. El chavismo ideó un modus operandi contra el periodismo independiente cuyos pasos de manual podrían resumirse así:•Difamar públicamente a propietarios de medios periodísticos críticos del Gobierno para desprestigiarlos y asociar a los periodistas capaces de incomodar al poder político con los supuestos intereses económicos de sus empleadores, apuntando a negarles toda independencia profesional y credibilidad. •Fortalecer a los medios de comunicación afines y debilitar al resto mediante la asignación arbitraria de publicidad oficial como herramienta para premiar a los complacientes y castigar a los desobedientes. •Generar nuevos marcos legales para restringir la libertad de prensa y para debilitar económicamente a determinadas empresas periodísticas. •Estatizar o crear medios bajo la tutela gubernamental a través de empresarios amigos, cuya fidelidad al Gobierno estaría garantizada con jugosas inyecciones de pauta publicitaria oficial. •Abusar de la cadena nacional para inocular en la población el relato sobre las bondades del modelo oficial y para tildar de conspiradores o golpistas a los medios de prensa críticos del Gobierno. Una serie de ataques sufridos por la prensa en la era kirchnerista refleja las similitudes entre la Argentina de los últimos años y el régimen fundado por Hugo Chávez. • En reiteradas ocasiones, la presidenta de la Nación denunció infundadamente a periodistas y medios. El vicepresidente Amado Boudou llegó incluso a comparar a dos periodistas de los diarios Clarín y La Nación con quienes ``limpiaban las cámaras de gas durante el nazismo''. •Distintos periodistas críticos del Gobierno fueron dejados cesantes en medios privados para los cuales se desempeñaban por aparentes presiones oficiales (el caso de Marcelo Longobardi en Radio 10 ha sido uno de los más emblemáticos). •Gremios afines al Gobierno bloquearon plantas impresoras de diarios e impidieron su distribución, en medio de la pasividad de las autoridades. •Se repitieron maniobras intimidatorias, como la presencia de unos doscientos inspectores de la Administración Federal de Ingresos Públicos (AFIP) en las oficinas de Clarín o un embargo contra La Nación pese a la existencia de una medida cautelar dispuesta por la Justicia que lo impedía. •La arbitrariedad y discrecionalidad con que se manejó la asignación de la pauta publicitaria oficial quedó claramente de manifiesto. Medios gráficos cercanos al oficialismo de escasísima circulación han llegado a recibir en un año diez veces más pautade la que reciben del Estado nacional canales de televisión con elevado índice de audiencia...
\end{quote}

Un balance editorial del año 2014, permite concluir, por un lado, que da comienzo al inicio de la resistencia de algunos periodistas que observan el proceso de confrontación. Por otro, las dos producciones de ADEPA, parecen marcar cual será la agenda metaperiodística de los grandes medios para el año que se avecina.

\textbf{Año 2015}

El año 2015 marca el momento que todos los actores del campo consideraron determinante. Hacia fin de año se realizarían las elecciones presidenciales. Los periodistas y medios no dependientes de los grandes grupos mediáticos consideraban que si las elecciones premiaban la continuidad gubernamental sería la oportunidad de implementar la nueva Ley de Medios Audiovisuales. Esta, desde su perspectiva, ampliaría las oportunidades de trabajo, multiplicando los medios de comunicación. Por el contrario, los periodistas y medios dominantes comprendían que las elecciones eran la oportunidad para intentar impedir tal éxito.

Desde la perspectiva editorial, el año resultó en una eclosión de publicaciones metaperiodísticas de las grandes editoriales. Once libros de periodistas sobre periodistas, empresas de medios y acción periodística fueron puestos en el mercado editorial, la mitad de ellos producidos por las grandes editoriales. De los once libros señalados, comenzaremos el análisis del que refleja, tal vez de mejor manera, una de nuestras hipótesis.

Gabriel Levinas, prestigioso periodista y productor con una larga trayectoria, interviene con un libro que se inscribe en la disputa por expulsar del campo a referentes de las opciones alternativas: \emph{Doble agente. La biografía inesperada de Horacio Verbitsky.} Al igual que el libro sobre Víctor Hugo Morales, analizado anteriormente (\emph{Converso} de Pablo Sirvén) se trata de una obra que tiene la pretensión de devaluar, si no de expulsar directamente del campo periodístico, a un periodista con alto grado de reputación y trayectoria. El libro trabaja sobre dos líneas argumentales. La primera, presenta al periodista con un pasado oscuro, por haber trabajado para personajes e incluso para instituciones vinculadas a la última dictadura militar, de manera simultánea a la de su pertenencia a Montoneros. A los ojos de quienes promocionaron el libro, es la publicación de las pruebas que demostrarían que ``el Perro'' fue colaborador de la represión. La gravedad de la acusación es tal que sobre ella se juega toda la credibilidad del libro y la profusa campaña publicitaria que lo sostuvo. La segunda línea, como ya señalamos, es la del escaso compromiso de Verbitsky con la libertad de expresión y la lucha contra la censura. En virtud de que el personaje en cuestión tiene una alta reputación en la defensa de los derechos humanos, de las libertades en general y de la de prensa en particular, los argumentos de Levinas apuntan a demostrar sus contradicciones y, eventualmente, sus inconsistencias ideológicas.

Levinas(2015: 9) comienza planteando que ``pocos días antes de terminar este libro, se suscitó una polémica en torno a la información que adelantamos acerca de los vínculos de Horacio Verbitsky con la Fuerza Aérea durante la última dictadura''. El autor comenta que, a raíz de esa polémica, mantuvo una serie de conversaciones con allegados y que, una de ellas, la opinión del filósofo y ensayista Alejandro Katz, le ayudó a comprender la razón de ser de este libro''. La misma se transcribe a continuación porque si bien Katz no es periodista\footnote{Dentro de nuestra categorización podríamos considerarlo un pseudoperiodista ( intelectual periodista).}, hizo una declaración original acerca de la tarea del periodismo, que Levinas hizo suya:

\begin{quote}
Otra controversia hará a la necesidad de este libro. ¿Por qué escribir, publicar, un libro de este tipo? Hay una primera respuesta: porque no es posible no hacerlo. Cuando un periodista recibe información de interés público, su obligación y su deseo coinciden: esa información debe ser compartida con la audiencia más amplia posible. \textbf{No es el periodista el que debe decidir si una información debe restringirse o circularse,} porque actuar de ese modo sería equivalente a ejercer censura o, cuando menos, a asignarse el derecho de efectuar un juicio que el resto de la sociedad no podría realizar. Por tanto, esta controversia tampoco tiene para el autor mucho interés (\ldots) (Levinas, 2015: 10).\footnote{El énfasis es propio.}
\end{quote}

Nos adentraremos en la cita anterior más adelante. Por ahora diremos que una vez que Levinas se sintió justificado para actuar, actuó.

La síntesis argumental del libro es que existiría documentación probatoria que avalaría el vínculo entre Verbitsky y la dictadura militar iniciada en 1976. A su vez, desde el punto estrictamente periodístico, Verbitsky subordinaría los intereses del campo y consecuentemente de su audiencia, a su ideología y/o eventualmente a sus intereses (de poder).

\begin{quote}
El número 8 de El Periodista, en noviembre de 1984, marca la gran irrupción de Verbitsky en el semanario mediante la publicación de un suplemento de ocho páginas con los nombres de ``Los 1351 represores denunciados en el informe secreto de la Conadep''. La primicia se anunció en tapa: ``Exclusivo: los nombres de la infamia'' (\ldots) Aunque se destacó de entrada en la cobertura del frente militar y del juzgamiento por la represión, Verbitsky tenía libertad para escribir sobre lo que quisiera. Y desde donde quisiera. A principios de diciembre de 1984, por caso, firmó una extensa cobertura desde La Habana, presentada como ``Informe especial: Cuba hoy'' (\ldots). En un recuadro sobre la formación musical de los cubanos, decía que ``la libertad de expresión y de búsqueda formal está consagrada en la Constitución cubana'', y aclaraba que la asistencia a conciertos y festivales no es obligatoria'' (Levinas, 2015: 172-74).
\end{quote}

Por la nula crítica al castrismo cubano, Levinas, lo ubica como alguien no confiable. Sin embargo, el párrafo no revela un solo dato falso; era esperable que el joven izquierdista Verbitsky, se dejara deslumbrar por la Cuba socialista. De hecho Levinas reconoce en el libro:

\begin{quote}
También Verbitsky encontró definitivamente su lugar: gran denunciante de la corrupción, severo impugnador de la impunidad de los crímenes de la dictadura que Menem quiso consagrar con los indultos y el posterior ascenso de algunos represores, riguroso crítico de la política económica y \textbf{orgulloso abanderado de la libertad de expresión}. Fueron las mejores páginas del Perro, de las que resultaron sus libros más exitosos (\ldots) A fines de 1995, además, se había fundado la Asociación Periodistas, nacida como gesto de solidaridad hacia Verbitsky ante el asedio judicial menemista y también como una forma de curarse en salud. En el reducido grupo inicial, el Perro se rodeó de figuras como Joaquín Morales Solá, Roberto Guareschi, Hermenegildo Sábat, Magdalena Ruiz Guiñazú y Andrew Graham Yool, entre otros colegas prestigiosos y de larga trayectoria (\ldots) Periodistas creció en número e influencia cuando \textbf{Verbitsky arrimó el calor de la Fundación Ford, cuyos recursos permitieron, a cambio de ciertos requisitos formales, aumentar la membresía} (en especial, con la incorporación de socias, pues hasta entonces Magdalena era la única mujer) e iniciar un monitoreo a partir del cual la Asociación empezó a publicar sus informes anuales sobre ``La libertad de expresión en la Argentina'' y a denunciar sistemáticamente los ataques a la prensa (Levinas 2015: 227, 229, 240).\footnote{El énfasis es propio.}
\end{quote}

El párrafo anterior resulta confuso, ya que Levinas afirma que la organización PERIODISTAS nace como gesto de solidaridad con Verbitsky, pero luego parece enunciar que es Verbitsky quien busca rodearse de figuras prestigiosas. También llama la atención la forma en que el autor introduce a la Fundación Ford en esta biografía:

\begin{quote}
Los premios y reconocimientos, locales e internacionales, se fueron apilando. En la Feria del Libro de 1991, Verbitsky recibió la distinción ``Al mejor trabajo y al más pedido'', por \emph{Robo para la Corona}, y fue premiado por el Colegio de Abogados de San Isidro; en 1994 recibió el ``Diploma al Mérito'' de la Fundación Konex en las categorías Ensayo y Análisis Políticos; en 1996, la Latin American Studies Association le otorgó su premio a los medios de comunicación; en 1997, en una encuesta de la alemana Fundación Konrad Adenauer y el Centro de Estudios para una Nueva Mayoría, de Rosendo Fraga, sus colegas lo eligieron como el `ideal de periodista'' (\ldots) El cetro de campeón de la libertad de expresión era paradójico. No sólo porque \textbf{Verbitsky había sido miembro de Inteligencia de Montoneros, organización que había matado gente ---entre otras cosas, por sus ideas---, sino también por su condición de ``periodista militante''.} Recordemos que, cuando el diario \emph{El Mundo} fue clausurado por el gobierno de Perón, \emph{Noticias} (donde el Perro era jefe de Política) criticó el hecho por ser una acción ``contra el campo del pueblo`: por entonces, la libertad de prensa era considerada un ``principio burgués''. \textbf{También en democracia Verbitsky tuvo acciones reñidas con la libertad de expresión}. En 1987, después del levantamiento carapintada de Semana Santa, presentó una medida cautelar e impidió la publicación de una solicitada en apoyo de Videla y los jefes militares que habían sido condenados en el Juicio a las Juntas, argumentando que el texto constituía ``apología del delito'' y formaba parte de una escalada golpista (Levinas, 2015: 240, 241).\footnote{El énfasis es propio.}
\end{quote}

El texto es impreciso y tensiona la imagen pública presente del periodista con algunos datos de su pasado. Sin embargo, acusar a Verbitsky de tener acciones reñidas con la libertad de expresión por haber participado del colectivo de intelectuales que presentaron una medida cautelar para que no se publique una solicitada que reivindicaba a la dictadura demuestra las dificultades argumentales que enfrenta el autor. Más adelante, haciendo referencia al conflicto suscitado en la Asociación PERIODISTAS, pone la responsabilidad de la decisión mayoritaria en cabeza de Verbitsky, es decir, la consideración que no se trató de un acto de censura periodística:

\begin{quote}
En el plenario de Periodistas, el debate sobre el ``caso Nudler'' fue candente. Muchos dijeron sin dudar que era un episodio de censura. En la discusión se colaron argumentos sobre el uso discrecional de la publicidad oficial, las presiones de los funcionarios sobre periodistas y medios y otras cuestiones que hacían a la libertad de prensa. Pero al final la mayoría, hábilmente conducida por Verbitsky, se plegó al argumento de que un editor/director tenía derecho a decidir sobre la calidad y solidez de una nota así como sobre su publicación o no. En votación dividida, el veredicto fue: no hubo censura (\ldots) En todo caso, al Perro le quedaba su temible pluma y el espacio en Página/12 para ajustar cuentas con Nudler, sin concesiones a la verdad ni a la moral. El 14 de noviembre, más de tres semanas después de la fecha en que el texto original de Nudler debió haber sido publicado en Página/12 y luego de que hubiera circulado por varios medios escritos y digitales, Verbitsky reprodujo el texto como ``La nota de Nudler'' y ocupó, al lado, el triple de espacio en un intento de desacreditar el contenido y a su autor:Lejos de aclarar, la interpretación sobre una campaña confunde. Más verosímil es el propio Nudler cuando en un reportaje en la revista Noticias relaciona el humo del cigarrillo que tuvo que aspirar durante años en la Redacción con el cáncer de pulmón con metástasis ósea que padece y dice que como esto lo tiene muy sensible y se hartó de las triquiñuelas de los corruptos decidió incurrir en una locura hamletiana y romper los códigos (\ldots) creo que el suyo fue un conmovedor grito de desesperación y despedida, que merece el mayor respeto, y nada tiene que ver con una calculada conjura. Tras esa bajeza disfrazada de piedad, y ya sobre el final, Verbitsky refutó las apreciaciones políticas de su colega: ``Según Nudler, un gobierno de esta calaña necesita una prensa amordazada y organismos de control inutilizados''. Verbitsky no resistió allí la tentación de citarse a sí mismo: Es un esquema familiar, que expuse hace una década en el libro Hacer la Corte. La construcción de un poder absoluto, sin justicia ni control. No veo cómo pueda proyectarse al presente con fundamentos tan etéreos. Ni la prensa está amordazada ni los organismos de control esterilizados. Hubiera sido mejor que Página/12 publicara la nota, pero su escaso sustento no habilita a considerarla censurada (Levinas, 2015: 330, 332, 333).
\end{quote}

Más abajo:

\begin{quote}
Mientras, la mayoría de las notas que vindicaban las denuncias contra Bergoglio desapareció misteriosamente del archivo digital de Página/12. En noviembre de 2014, el tuitero @mis2centavos advirtió la ausencia de las columnas críticas escritas por Verbitsky. Durante el acalorado debate en las redes sociales acerca de si se trataba de una maniobra de censura por parte del matutino, el mismo Horacio salió a aclarar los tantos: ``Fui yo'', fue el título de una breve explicación que dio en el mismo diario (Levinas, 2015: 384).
\end{quote}

El último capítulo del libro, a modo de \emph{post scriptum}, lo denomina \emph{Backstage}. El mismo está dedicado a recopilar opiniones personales de múltiples personajes sobre el periodista objeto del libro:

\begin{quote}
A lo largo de la investigación consultamos a casi sesenta personas, en aproximadamente setenta entrevistas personales, por correo electrónico y por Skype. No todos los entrevistados accedieron a que se conociera su nombre, y de varios de ellos quedaron pasajes que no fueron incluidos en el cuerpo de este libro. Lo que sigue es una selección de algunos de esos fragmentos (Levinas, 2015: 408).
\end{quote}

Reproduce fragmentos de nueve entrevistas, cuatro de ellas realizadas a periodistas. Tres de ellas tienen un comentario crítico, pero solo una, la de Leuco, tiene cierta sustancia negativa:

\begin{quote}
Alfredo Leuco: Me parece un tipo emblemático de la traición del trabajo periodístico, porque más allá de la simpatía política, \textbf{Horacio Verbitsky deja de ejercer el periodismo crítico de investigación para ponerse al servicio de un gobierno} (\ldots). Una vez llevé {[}a Roberto Perdía{]} al programa de tele y ahí me dijo {[}refiriéndose al Perro): ``Tengo algún papel para darte, hay que contar la verdad porque este se lava las manos de todo y fue un traidor'' (\ldots) Yo lo asocio mucho al miedo que tiene la gente del espectáculo a {[}Jorge{]} Rial: tiene capacidad de daño, y Verbitsky también. Creo que la gente piensa que es más de lo que es (Levinas, 2015: 410).\footnote{El énfasis es propio.}

Silvina Walger: Lo conocí cuando estaba en La Opinión. Yo estaba de novia con ``el Vasco'' Mouriño, lo conocí por él (\ldots). Ya en Confirmado, \textbf{Horacio ``olía a servicio''.} Hubo una época en que figuraba como "asesor de la dirección'' (Levinas, 2015: 418).\footnote{El énfasis es propio.}

Jorge Lanata: Un día me entero del tema de su relación con la Fuerza Aérea, de su relación con Güiraldes. En esa época yo salía con Graciela Mochkofsky. Ella estudiaba en El Salvador, y un día le digo: ``¿Por qué no vas a preguntar por Verbitsky?''. Ella lo va a ver Güiraldes y consigue el libro de la Fuerza Aérea. El Perro se entera, me llama y me dice: ``¿Vos me estás investigando?'' Me tomó por sorpresa, y le digo que sí. ``¿Por qué?'', me pregunta. ``Porque quiero saber con quién trabajo'' (Levinas, 2015: 422).
\end{quote}

El periodismo requiere de información para ser. Los periodistas son portadores de información. Los periodistas no son testigos de la información que portan; son, en la mayor parte de los casos, intermediarios de múltiples fuentes, muchas de las cuales son servicios de inteligencia nacionales y/o extranjeros. El campo periodístico no ignora ese hecho, pero el mismo es ocultado meticulosamente, porque de lo contrario revelaría el riesgo que conlleva la información y las opiniones que se vierten. La confidencialidad de las fuentes está en la estructura, en la naturaleza más profunda del campo. Muy de vez en cuando esa estructura se trasluce, en general, en procesos de crisis del campo, de lucha desatada entre los actores.

Levinas, con datos y/o argumentos verdaderos o falsos rompe uno de los códigos del campo a efectos de descalificar y expulsar del mismo a un jugador con gran capacidad de mantenerse en el centro de la disputa, aunque desde medios periféricos. Tal vez esta característica de Verbitsky, de permanecer fuera del \emph{mainstream} periodístico y, al mismo tiempo, por no incumplir con principio periodísticos centrales, le otorgue un lugar destacado a la hora de privilegiarlo como recurso informativo, tanto entre su audiencia como entre sus fuentes. Esta característica lo hizo objeto del interés delos \emph{defensores} del campo, a los fines de desprestigiar su figura y el valor de su cualidad periodística.

Con \emph{Mentime que me gusta}. \emph{De los intereses del periodismo al autoengaño del lector}, Víctor Hugo Morales se introduce nuevamente en el conflicto. Si bien ya había sido biografiado por un periodista que lo defendió, su imagen estaba siendo fuertemente agredida e incluso enfrentaba juicios millonarios iniciados por una de las personas más poderosas del país: Héctor Magnetto. En tal sentido, el libro representa, de alguna manera, una perspectiva nueva, de un actor significativo del campo:

\begin{quote}
En esta obra queremos detenernos en analizar un sector que debería ser pilar fundamental en la formación de un ciudadano libre en democracia pero que lleva años demostrando que ha abandonado esa labor. \textbf{Los medios de comunicación, que no dejan de escudarse en la libertad de prensa y la libertad de expresión para defenderse de cualquier crítica}, están teniendo una responsabilidad fundamental en que las nuevas generaciones no logren comprender nada de lo que suceda, asuman como inevitables guerras de las grandes potencias en su búsqueda de recursos y no perciban alternativas a las bárbaras decisiones de mercados. En las siguientes páginas hemos querido hacer un recorrido ameno por los senderos de esa tragedia (2015: 3, versión digital).\footnote{El énfasis es propio.}?
\end{quote}

Morales percibe que ese debate y su defensa como periodista no puede ser llevado a cabo bajo los viejos paradigmas del periodismo. Recurre, entonces, a la mención de lo que, a nuestro entender, deberían ser llamados nuevos paradigmas de la prensa, en el marco del derecho a la información:

\begin{quote}
La comunicación juega un papel fundamental, tanto en la existencia de la democracia como en el funcionamiento del propio sistema de derechos, oportunidades y libertades, y su democratización tiene relación directa con la construcción de nuevas sociedades más justas y equitativas, participativas: es decir, construir el futuro de nuestros pueblos donde el ciudadano sea el protagonista. \textbf{Uno de los derechos ciudadanos es el acceso a una información independiente, imparcial, veraz y plural, que le suministre a la ciudadanía conocimiento y orientación suficientes sobre las alternativas existentes para la toma de sus propias decisiones políticas y su desarrollo como ciudadano} (2015: 3, versión digital).\footnote{El énfasis es propio.}
\end{quote}

Sin embargo, vuelve sobre sus pasos e intenta demostrar que los principales medios han renunciado a sus propios principios:

\begin{quote}
De las palabras preliminares: Este Manual de estilo resume lo que es el ``periodismo de \emph{Clarín}''. Actualiza y detalla nuestro compromiso editorial con los argentinos. Explicita de qué manera asumimos cotidianamente la ética, el rigor profesional y la calidad periodística. Héctor Horacio Magnetto-Vicepresidente Ejecutivo y Director General\emph{:} \textbf{Clarín es un diario independiente, comprometido con las producciones culturales y el trabajo de los argentinos que marcan nuestra identidad como nación y contribuyen al desarrollo de una sociedad solidaria y justa. Promueve la libertad de expresión, el pluralismo y el fortalecimiento de las instituciones que sustentan el régimen democrático.} Manual de Estilo Clarín, Clarín -- Aguilar 1997 (\ldots). \textbf{Pluralidad de enfoques.} En los temas en los que haya posiciones contrapuestas, La Nación recogerá en sus páginas todas las disidencias, a fin de ofrecer al lector una cobertura completa del asunto. La opinión propia del diario será tratada en la columna de editoriales. Este principio se aplicará también en las crónicas, a fin de que el lector pueda tener un conocimiento completo de lo que arguyen las partes enfrentadas con relación a un suceso. Manual de estilo y ética periodística La Nación, Compañía editora Espasa Calpe Argentina S.A. Grupo Editorial Planeta 1997 (2015: 13, versión digital).\footnote{El énfasis es propio.}
\end{quote}

A tal efecto recorrerá una serie de noticias contenidas en los principales medios durante los últimos meses, y demostrará que estos falsearon los hechos. De esta manera, intentará argumentar que sin diversidad no hay derecho a la información, aunque formalmente la libertad de prensa esté garantizada:

\begin{quote}
El 29 de enero de 2014 la escalada a la cumbre de la imaginación fue una cuestión de fronteras. ``Corrieron la frontera con Bolivia y exigen trasladar a familias salteñas''. Espere, hay más. ``En noviembre pasado se permitió modificar el límite 30 kilómetros al sur\ldots. \emph{La Nación} sabe cuál es el perfil de sus lectores. El razonamiento que acompaña la información es instantáneo. ¿Así que encima de que vienen aquí cuando se les antoja a vivir de lo nuestro los muy taimados nos andan robando pedazos de tierra? La última información que se conoció al respecto sucedió cuando la cancillería dijo lacónicamente y con un bostezo a las tres de la tarde de ese mismo día que la zona se encuentra demarcada desde 1925 y no existen reclamos (Morales, 2015a: 32, versión digital).
\end{quote}

Otra:

\begin{quote}
\emph{La Nación} se hacía eco del reclamo y titulaba: ``Por los saqueos, el gobierno chino pide a la Argentina protección para sus ciudadanos. Desde la cancillería china pidieron que el gobierno `tome medidas tangibles para proteger la seguridad y los intereses''. La nota mencionaba como fuente a la agencia oficial de noticias \emph{Xinhua}y a su competidor y amigo, \emph{Clarín}, quien había recibido de primera mano declaraciones de diplomáticos chinos. Por su parte, el matutino de Magnetto titulaba ``El gobierno chino reclamó por la seguridad de sus ciudadanos'' y agregaba: ``Desde su cancillería exigieron medidas para proteger a comerciantes de ese origen. Y enviaron a un viceministro'' (\ldots) Dos días después, la agencia estatal de noticias Télam daba por tierra con la versión \emph{Clarín-La Nación} al señalar que ``China nunca ha ejercido presiones o presentado protestas formales a la Argentina por este asunto''. Respecto a la visita de Li Wei, el embajador aseguró que se trató de ``una visita acordada entre la parte china y el Ministerio de Seguridad argentino a partir del 15 de octubre de 2013 y realizada entre los días 7 y 10 de diciembre para reforzar la cooperación institucional'' (Morales, 2015a: 34, 35, versión digital).
\end{quote}

Otra:

\begin{quote}
La nota transcurría en TN. El mismo 27 de octubre de 2014, Moira era entrevistada en su calidad de deportista ganadora de seis medallas de oro en el Sudamericano de Kayak. El periodista afirma casi compungido que esas medallas tienen ``un valor distinto\ldots{} imaginamos la historia, tiene un valor distinto y seguramente esto de vender el pelo\ldots{} ¿por qué tomaste esta decisión? Ya sé que porque no te alcanzaba el dinero, pero hay otras opciones. Vos dijiste esto: ``quiero vender mi pelo''. La respuesta de la adolescente dejó a los periodistas helados: ``En realidad, fue una confusión. Lo de vender el pelo fue hace muchos años. Lo que pasó es que el Sudamericano, en realidad, me lo pagó Chubut Deportes'', agregó que estaba muy agradecida por eso, que tenía dos becas y los viajes cubiertos por el Estado. Cuando la periodista le pidió detalles sobre la decisión de cortarse el pelo, Moira dijo que se lo había cortado por comodidad: lo tenía hasta la cintura y no resultaba fácil entrenar con semejante melena. La deportista llamó ``confusión'' al titular que había publicado Clarín y agregó que se veía perjudicada frente a una gobernación que la había apoyado tanto en su dedicación al deporte (2015: 36-37, versión digital).
\end{quote}

Se podría seguir enumerando las mentiras que recuerda. Hay sin embargo algo muy significativo: en el libro no se menciona el nombre de ningún periodista, a menos que se trate como fuente de alguna información, aunque probablemente muchas de las notas mencionadas por Morales lo tuvieran enunciado. El dato no es menor a los efectos de la presente tesis, porque será uno de los elementos que utilizaremos como principio deontológico. Morales utiliza el ``ellos'', ``el medio'', ``el periodista'' o ``los periodistas'', pero nunca hace referencia a periodistas particulares en tono de crítica o como autores de notas que incluyan, según él, falsedades:

\begin{quote}
El 23 de febrero de 2012 \emph{La Nación} tituló ``Un tren no frenó y provocó un desastre''. Desastre rinde más que desgracia. La nota cita ``versiones extraoficiales'', que son las de nadie, una suposición \textbf{hecha por ellos mismos} según las cuál es el maquinista de la formación llegó a decir que ``había advertido en Haedo que el tren no frenaba'' (\ldots) El propio Grupo Clarín envió a \textbf{una periodista} a la puerta de la radio para cazar (el término es irremplazable) a VHM (\ldots). El cameraman le comentó a \textbf{la periodista}: ``Nos cagó, nos dio la nota''. Pasaron dos días y el encuentro no fue divulgado porque quedaba demasiado clara la infamia que habían cometido. A la mañana siguiente, VHM se quejó por radio de la falta de ética, pero tampoco la pusieron al aire durante esa jornada. Al final, \textbf{un periodista} que también esperaba en la puerta de la radio la salida de VHM por otra razón profesional había tomado nota de la nota y la misma empezó a trascender y llegó a contar con cientos de miles de visitas (\ldots). Sin mayor respaldo que sus propias afirmaciones, \textbf{el periodista que firma la nota} asegura que el funcionario (por De Vido) está alterado porque ``dejó sustanciales evidencias de que participó en la redacción del contrato de Cerro Dragón que investiga la SEC'' (\ldots). Repugna, a criterio de quien escribe estas líneas, el proceder de FOPEA, foro creado por \textbf{los laderos más importantes de los diarios dominantes}, sus reflejos para condenar lo que condena \emph{Clarín} y su visión indiferente de los ataques a periodistas de parte de Héctor Magnetto (\ldots). Hablan de las pautas publicitarias como herramientas para acallar periodistas como si \emph{Clarín} o \emph{La Nación} fueran a dejar de hacer lo que hacen por un ingreso mayor o menor de esas pautas. Pero callan, sumisos y cobardes a la hora de denunciar que Magnetto inicia juicios penales contra periodistas o de carácter económico contra el firmante de este libro, por millones de pesos, carcomiendo, si fuera preciso, la libertad de expresión que atropella con sus denuncias intimidatorias y costosísimas, aun si no gana los juicios, asunto improbable en esa justicia que parece un campo minado por los intereses del Grupo (\ldots). La noticia \textbf{fue firmada por una periodista} que por entonces estaba en Punta del Este, y hasta figuraba en la galería de fotos de los personajes divertidos de la ciudad uruguaya. Sin embargo, Timerman, que pasaba por París en escala de un viaje no relacionado con los tristes episodios, participó de la manifestación (\ldots). El 12 de enero, Alberto Nisman, al cabo de una discusión familiar, dejó a su hija en el aeropuerto de Madrid y se volvió a Buenos Aires. Llamó al periodista Morales Solá de \emph{La Nación} y TN y le comunicó su decisión. Había que explicar el retorno inexplicable (\ldots). En el afán de menguar la capacidad de respuesta de los legisladores del gobierno, \emph{Clarín} y sus canales operaron asignando como propia de la diputada Conti una frase \textbf{del periodista Jorge Rial} (\ldots). Para \textbf{algunos periodistas}, no ya para los medios, el asesinato evitaría preguntarse cuánto colaboraron ellos con el trastorno de ese hombre en las últimas horas de su vida (Morales, 2015a: 78, 109, 110, 119, 134, 154, 155, 158, 161, versión digital).\footnote{El énfasis es propio.}
\end{quote}

Concluye Morales con una línea argumental en la que parece asentar algunos principios generales de periodismo:

\begin{quote}
Falsear la realidad para tener razón es no tenerla. Un buen lector acompaña el análisis editorial dando por sentado que los hechos no fueron deformados para poder avanzar. El árbol y la raíz, el mar y su lecho, el periodismo y la verdad. La mentira cuando, por su insistencia, no puede confundirse con el error, rompe el contrato que han sellado el medio y sus seguidores. Hace poco en los EE.UU. un periodista que había adulterado su participación en la guerra de Irak confesó la verdad. El héroe que había arriesgado su vida devino en un vulgar tramposo y el canal de televisión donde tenía su programa decidió suspenderlo, acaso para siempre. Pero el público fue más allá en su castigo y la audiencia se contrajo hasta hoy porque no pudo perdonarse su ingenuidad. (2015: 174-175, versión digital)
\end{quote}

A \emph{Mentime que me gusta}, le sigue el mismo año, un nuevo libro de Víctor Hugo Morales, \emph{El rebenque del Diablo} (2015b). Recoge el juicio que Cablevisión le hizo al autor por el que tiene que pagar casi cuatro millones de pesos. Si bien el libro es sumamente revelador de la perspectiva que el autor tiene sobre los grandes medios, no avanzaremos con él ya que de alguna manera reitera posturas y perspectivas sobre la prensa y los periodistas.

A lo largo del análisis de los textos fue haciéndose más y más revelador el hecho de que el conflicto se agravó de manera determinante a partir de que el gobierno de la presidenta Fernández de Kirchner presentó a debate y finalmente promulgó la ley de Servicios de Comunicación Audiovisual. Entre 2010 y 2015, la ley enfrentó diversas instancias judiciales que impidieron ponerla en vigencia de manera plena. Las principales trabas judiciales tuvieron como eje argumental la problemática de la desapropiación de medios de los principales grupos, dada la concentración en pocas empresas.

Néstor Piccone (2015) en \emph{La inconclusa ley de medios, la historia menos contada}, hecha algo de luz sobre el conflicto y el posicionamiento del campo. El prólogo escrito por Carlos Ulanovsky, es llamativo, ya que este periodista se caracterizó por tener una postura no confrontativa en los medios masivos:

\begin{quote}
De eso habla el libro. De un lado quedaron los interesados en sostener los nuevos paradigmas mediáticos, tan ligados a la defensa de lo monopólico, a la manipulación opositora, a los propósitos destituyentes y, últimamente, a la cada vez más notoria partidización. Señala, sin temores, los nombres de quienes desde lugares de poder ayudaron a postergar la sanción definitiva de la ley y a mirar para otro lado con tal de no enfrentar los intereses hegemónicos. Son los muchos a los que el periodista Víctor Hugo Morales agrupa en lo que llama ``el lado Magnetto de la vida''. ¿Por qué una embestida tan permanente y tan feroz contra el proyecto? Seguramente, por lo que tiene de necesario, oportuno e imprescindible; por audaz e inclusivo; por pensar en la gente como sujeto activo de la comunicación y por tomar a los contenidos audiovisuales como bienes y servicios públicos, por plural, por crear figuras como la Defensoría del Público (2015: 8).
\end{quote}

El texto de Piccone, autoproclamado partícipe inicial de la lucha por la ley, descubre el cambio de paradigma que encubre la misma, al poner en un pie de igualdad el derecho a la información y el derecho a la libertad de prensa. Por otra parte desarrolla, de manera sintética, el proceso de constitución de la Coalición por una Radiodifusión Democrática, visto desde dentro del movimiento. Este movimiento expresó --y tal vez siga expresando-- al sector más periférico del periodismo, con fuerte presencia entre los comunicadores y periodistas del interior del país y, en general, con formación universitaria:

\begin{quote}
Pero la lucha por la hegemonía de una clase sobre otra requiere una lucha semántica previa, tarea fundamental para gobernar y definir un nuevo modelo de sociedad. Esta confrontación semántica también explica la alianza que en Argentina, Brasil, Ecuador, Venezuela, Bolivia se construyó y de donde surgieron los conceptos que dieron forma a los proyectos democratizadores de la palabra que cuestionaron la independencia y la libertad de prensa y avanzaron con el derecho humano a la comunicación, que se sintetiza en la libertad de expresión y el derecho a estar informado (Piccone, 2015: 36-37).
\end{quote}

Por otra parte, el libro tiene la virtud de mostrar el fenómeno social que la concentración produjo. Este fenómeno supuso la expulsión de una importante cantidad de periodistas fuera del campo o a su periferia:

\begin{quote}
Los periodistas y camarógrafos que cubrían fútbol como noticia fueron prohibidos por la AFA. Ya no podían ingresar a la cancha y tampoco transmitir el partido, debido a que eso había quedado en manos exclusivas del canal del Grupo Clarín. Como si eso fuera poco, tenían prohibido realizar notas y grabar escenas del partido, de los vestuarios o del ambiente. No sólo se apropiaban de los partidos y secuestraban los goles hasta la noche del domingo a las 22 horas, sino que también se apoderaban de la información futbolística. Sólo un canal tenía derecho a la información. Esos cientos de periodistas deportivos que tuvieron que reciclarse, nunca perdonaron al Grupo. La pelea por la Ley de Medios les dio la posibilidad de vengarse de aquel atropello (\ldots). El fútbol fue igualmente la herramienta que utilizó el Grupo Clarín para presionar a los cables surgidos y administrados por los empresarios pymes de cada ciudad. Clarín llegaba con el fútbol y abría un canal de cable. Con la exclusividad, ``apretaba'' a los canales. Con el tiempo, y ante la imposibilidad de transmitir los partidos, estos perdían clientela y Clarín se quedaba con el canal local que distribuía las señales. Una vez liquidada la competencia, el Grupo intentaba cerrar la señal local. De esta manera, la ciudad se quedaba sin información propia y los trabajadores que hacían programas periodísticos o de información general se quedaban sin trabajo (Piccone, 2015: 50).
\end{quote}

Pocos textos son tan reveladores como este para poner de manifiesto el apoyo con que contó la ley en los sectores marginales del campo periodístico.

\emph{Grandes y pequeñas mentiras que nos contaron}, es un libro editorialmente extraño, ya que, si bien fue publicado a finales del año 2015, da la impresión de que hubiese estado preparado para adecuarlo al resultado de las elecciones presidenciales. Es probable que los autores o la editorial hayan tenido la intención de publicarlo antes de las elecciones para intervenir como herramienta propagandística contra el oficialismo.

Los autores, Marcos Novaro y Marcelo Birmajer (2015), se corresponden de manera precisa al concepto que Bourdieu asigna al de ``intelectuales periodistas''. En relación con esta categoría Bourdieu (1997:112) señala:

\begin{quote}
Estos ``intelectuales periodistas'', que utilizan su doble pertenencia para sortear las exigencias específicas de ambos universos e introducir en cada uno de ellos unos poderes mejor o peor adquiridos en el otro, están en disposición de ejercer dos efectos importantes: por una parte, introducir formas nuevas de producción cultural, situadas en una zona mal definida entre el esoterismo universitario y el exoterismo periodístico; por otra parte, imponer, en particular a través de sus juicios críticos, unos principios de valoración de las producciones culturales que, al conferir la ratificación de una apariencia de autoridad intelectual a las sanciones del mercado, y al reforzar la propensión espontánea de determinadas categorías de consumidores a la \emph{alodoxia,} tienden a reforzar el efecto de los índices de audiencia o de la \emph{best seller list} sobre la recepción de los productos culturales y también, indirectamente y a mediano plazo, sobre la producción, al orientar las decisiones (las de los editores, por ejemplo) hacia productos menos exigentes y más vendibles.
\end{quote}

Si bien dudamos en incorporar estos autores al \emph{corpus}, ya que no son expresiones puras del campo periodístico, por el contrario, ambos parecen pertenecer al campo intelectual. Marcos Novaro es investigador principal del CONICET y dirige un centro de investigaciones políticas. Marcelo Birmajer, que ha escrito en múltiples publicaciones periodísticas, es mucho más reconocido como escritor de ficción, obras de teatro y guiones televisivos, que como periodista. Sin embargo se decidió su incorporación porque expresaban justamente este carácter de doble pertenencia que, como veremos, les permite --como diría Bourdieu (1997:112)-- ``sortear las exigencias específicas de ambos universos e introducir en cada uno de ellos unos poderes mejor o peor adquiridos en el otro''.

En las 300 páginas del libro de Novaro y Birmajer, los conceptos \emph{libertad de prensa} o \emph{libertad de expresión} son mencionados en 70 oportunidades. Ello produce un fenómeno de reiteración convincente, en el sentido de las múltiples formas de su vulneración.

El libro da cuenta de todas las formas en que el periodismo dominante se sintió agredido por el poder político. En buena medida, aporta también al discurso de ese periodismo los argumentos empíricos y analíticos para confirmar su opinión y, con mayor precisión, su sentimiento.

Su hipótesis, presentada en el prefacio del libro, es la siguiente:

\begin{quote}
El mito de que los medios de comunicación condicionaron espuriamente a los gobiernos democráticos argentinos hasta la llegada de Néstor Kirchner al poder, y que tanto él como Cristina liberaron a la ciudadanía de esa opresión mediática, ejercida por una ``corporación que nadie votó y nadie más que ellos se animó a desafiar'' es una de la más perniciosas entelequias que ha contaminado nuestra vida política. \textbf{Y uno de los argumentos más elaborados de los muchos ensayados a lo largo de la historia argentina por gobiernos deseosos de dañar, acotar y en la medida de lo posible suprimir la libertad de expresión}(...)Su apropiación y repetición por parte de una gran variedad de actores políticos y sociales, desde dirigentes partidarios y sindicales hasta intelectuales y artistas, e incluso muchos periodistas, logró dar cierta legitimidad a una guerra llevada adelante en estos años desde el poder contra la prensa independiente (2015: 7).\footnote{El énfasis es propio.}
\end{quote}

Esta es la primera vez, en el largo tránsito del conflicto al interior del campo, que se deja registrado teóricamente que lo que estaba en juego era la legitimidad del periodismo como efector monopólico del discurso de la realidad social. También señala que ese discurso, que disputa con la prensa dominante, no solo era propio de la conducción política, sino que desbordaba hacia ``una gran variedad de actores políticos y sociales, desde dirigentes partidarios y sindicales hasta intelectuales y artistas, e incluso muchos periodistas'' (2015: 7).

Los autores apuntan que lo que estaba en riesgo, era la existencia misma del campo, entre otras cuestiones, porque la investigación en temas de comunicación se encontraba en manos de sectores críticos de los medios dominantes que ponían de manifiesto la falta de objetividad en el proceso de la comunicación social.

\begin{quote}
Pero por años, la Argentina vivió un tiempo que tuvo poco de normal. Vivió en medio de una guerra, una ofensiva abusiva e injustificada de los gobernantes contra los disidentes. Y de esos disidentes los privilegiados por los ataques gubernamentales fueron los periodistas (\ldots). Hay quienes dicen, sobre todo desde el gobierno, que gozamos en la última década y media de la más plena libertad de expresión. Que se puede decir cualquier cosa y no hay periodistas presos ni diarios cerrados. Así que ¿por qué preocuparse \textbf{si las autoridades impugnaron la tarea periodística de determinada empresa} o \textbf{pretendieron regular sus actividades para limitar sus audiencias}? (\ldots) Todo esto importa y mucho porque \textbf{libertad de expresión no solo es decir lo que uno quiera, es no ser amenazado ni de mil maneras se le haga la vida imposible a las empresas de medios (\ldots).} En ello cumplió un rol fundamental \textbf{un recurso argumental} que acompañó todas las iniciativas oficiales: \textbf{la pretensión de desconocer que existiera algo así como una profesión o un ejercicio independiente del periodismo}, dado que toda noticia u opinión sirve abierta o solapadamente a un interés y proyecto político. Por lo que solo habría periodismo interesado, sea a favor o en contra del gobierno, y no ``prensa libre'' (Novaro y Birmajer, 2015: 10, 11, 13).\footnote{El énfasis es propio.}
\end{quote}

Nuevamente, hacen explícito que el campo político disputaba la verosimilitud de la información provista por los periodistas. Y esto, para los autores, significaba poner en duda el principio de la libertad de expresión:

\begin{quote}
\textbf{La misma existencia del principio de ``libertad de expresión'' fue de este modo puesta en duda por el proyecto oficial}. Él postuló que no existía nada que se pudiera llamar información objetiva, ni independencia para informar y para informarse, porque toda la actividad de informar es parte de una guerra política entre intereses enfrentados en la que no existen periodistas, ni ciudadanía, ni instituciones ni principios que nos contengan a todos. Solo hay facciones y militantes y armas de uno u otro bando (Novaro y Birmajer, 2015: 13).\footnote{El énfasis es propio.}
\end{quote}

Podemos apreciar que el discurso de estos dos profesionales es propio del periodismo y no de investigadores académicos. Sin embargo, como diría Bourdieu, usufructúan su condición de tales como legitimadores del mensaje a favor del campo periodístico.

La idea de que el principio de libertad de expresión fue puesto en duda, en el país, al desconocer o impugnar la tarea de los periodistas, es un argumento muy propio de un campo en disputa, que defiende su derecho a la mayor autonomía. Este fenómeno (actores políticos disputando la legitimidad de la verosimilitud de sus representaciones de la realidad) es relativamente nuevo y está, probablemente, asociado a la difuminación de la realidad social a partir de la masificación de las redes sociales y el periodismo amateur.

Una frase al inicio del libro da cuenta de lo que representó para el campo la disputa de la legitimidad:

\begin{quote}
En verdad fue antes de la aparición del Kirchnerismo, entre el retorno de la democracia y la presidencia de Duhalde, que el país vivió ``la más intensa y prolongada era de expansión de la libertad de expresión de toda su historia. Más allá de los errores y visiones bastante tradicionales que pueden atribuirse a Alfonsín en la materia, más allá de los abusos de poder que cometió el menemismo y de sus complicidades mafiosas, que entre otras cosas terminaron con la vida de José Luis Cabezas, tanto en términos de las normas dictadas como de las prácticas habituales del poder del Estado se respetó bastante más que en el pasado la libertad de opinión y de acceso a la información, gracias a lo cual la prensa independiente floreció (2015: 15).
\end{quote}

Es decir, incluso el asesinato de José Luis Cabezas representó para estos autores un hecho menor, comparado con el ataque a ese mismo derecho de la ``era kirchnerista''. Esta frase no debe tomarse a la ligera, el asesinato de Cabezas representaba el ataque a un periodista por lo que hacía, e incluso su muerte acrecentaba el prestigio y autoestima del campo periodístico, reforzando la imagen heroica que poseen los fotoperiodistas. El componente heroico de la actividad es muy importante, un elemento que debería estudiarse con detenimiento ya que forma parte del capital simbólico del campo, en tanto que estos periodistas arriesgan su vida en aras de que podamos acceder a la información.

\backmatter
\chapter{5. Conclusiones}

\section{5.1. Presentación}

El fenómeno metaperiodístico, -es decir el discurso de los periodistas sobre su propia actividad- tuvo su mayor expresión en Argentina entre los años 2009 y 2015. En ese contexto, las condiciones políticas y sociales ponían en discusión el rol de la prensa y los periodistas. El rol preponderante que adoptó el periodismo en el período que analizamos y la tensión entre los actores del campo, a favor o en contra del gobierno de los Kirchner, abrió el camino para la expansión del discurso metaperiodístico.

Meneses Fernández (2010), quien estudia el fenómeno del metaperiodismo durante el período posfranquista en España, lo describe de manera detallada pero no extrae conclusiones sobre las causas del mismo. Al parecer, el metaperiodismo emerge con virulencia en períodos en que el rol del periodismo es puesto en duda o, por lo menos, cuando su lugar en la sociedad se torna un tema de debate.

Debemos agregar, además, como elementos coadyuvantes a la tensión dentro del campo la persistente crisis de financiamiento de la prensa\footnote{En un reciente trabajo, Mastrini y Becerra (2017: 53, 54), señalan en distintos pasajes una cantidad significativa de datos que dan cuenta de esta crisis: ``El mercado de medios de la Argentina es inestable y tiene problemas de subsistencia. De las cinco mil estaciones de radio que funcionan en el país, menos del 3\% se autofinancia con publicidad. Las empresas de televisión abierta, sector que absorbe el 35\% de la torta publicitaria, sostienen que no son rentables (\ldots). Para comprender la magnitud del problema es preciso reconocer que los medios comerciales no hubieran sobrevivido a sucesivas crisis sin el diligente auxilio prestado por las administraciones a través de la periódica condonación de deudas fiscales y previsionales, la venta de pliegos de televisión y radio a precios irrisorios, el socorro financiero para evitar la convocatoria de acreedores, la desgravación impositiva (\ldots) entre otras medidas''.} y el surgimiento de periodistas \emph{amateurs} que encontraron canales de comunicación en las redes sociales, \emph{blogs} y páginas \emph{web}. Esto último desdibujó el campo periodístico asentado en los medios tradicionales.

Como hemos visto, en el período estudiado, hallamos un extenso \emph{corpus} de libros de periodistas. De estos libros, recuperamos un amplio conjunto de argumentos cuyo objetivo fundamental fue legitimar la actividad periodística y de la prensa en general, frente a quienes consideraban detractores de su actividad y, establecer una frontera simbólica que distinguiera a quienes se hallaban dentro del campo de quienes habían violado principios que ponían en duda su condición, lo suficiente como para ser excluidos del mismo.

La primera cuestión que interesa distinguir es que, en general, la prensa argentina ha sido muy poco o nada cuestionada en su tarea, incluso en el período posdictatorial. En el ya clásico trabajo de Blaustein y Zubieta (1999), \emph{Decíamos ayer. La prensa argentina bajo el Proceso}, los autores se extrañan del escaso debate que se suscitó en la prensa sobre su rol en la dictadura, al regreso de la democracia. Por lo que el fenómeno kirchnerista, al intentar discutir en igualdad de condiciones con los medios, fue considerado por la mayoría del campo periodístico, como una afrenta a la actividad.

La oposición al gobierno que adoptaron los principales medios de comunicación estuvo determinada por un sinnúmero de causas. Sin embargo, la forma y virulencia que asumió el conflicto en el campo periodístico, junto con la cantidad de libros editados, permite sostener que la actuación del mismo fue determinante para proteger su legitimidad al mismo tiempo que fue funcional a los intereses de los medios dominantes.

Dijimos en la introducción, a manera de hipótesis, que el periodismo detenta el monopolio de la producción de actualidad. Ahora bien, luego de nuestro trabajo podemos agregar que, a los efectos de conservar tal privilegio, el periodismo lleva adelante acciones de legitimación de ese monopolio y de disputa por su derecho al mismo.

El campo periodístico acepta que haya distintas actualidades, producidas por periodistas o medios que, como señala Verón (1983), están destinadas a distintas audiencias. Lo que no es aceptable para el campo, es que se ponga en duda o en disputa su construcción. Es decir, el campo reconoce a cada uno de los periodistas y medios el derecho a construir narrativamente su actualidad, que incluye su agenda, sus fuentes y sus interpretaciones. Lo que no acepta, por lo que se rebela de manera violenta, es que un medio, un periodista o un actor externo o interno al campo, dispute o ponga en duda la realidad social o la verdad creada por otro periodista o medio.


\section{5. 2. Composición del campo}

A fin de avanzar en los objetivos propuestos, comprender las estructuras argumentativas y la disputa entre periodistas devenida en metaperiodismo, intentamos responder las siguientes preguntas: ¿De qué está hecho el campo periodístico?; ¿qué elementos lo componen?, ¿qué tipo de relaciones establecen esos elementos entre sí?

Afirmamos que el campo periodístico está constituido por los periodistas, pero también por los medios de comunicación y agencias de noticias. Consideramos necesario plantear con claridad que no es posible escindir a los medios de comunicación del campo propiamente dicho.

Los medios de comunicación, hoy también denominados \emph{plataformas de contenidos}, están constituidos por diarios, revistas, canales de televisión de aire y de cable \emph{online} y \emph{offline}, emisoras de radio AM, FM, radios por Internet y redes sociales por Internet. A esta clasificación podría hacérsela más compleja con las múltiples combinaciones posibles: canales por redes sociales, \emph{streaming}, \emph{blogs}, sitios de Internet de diarios y revistas, etc. Pero a los efectos de la presente tesis, cuyo objetivo no es analizar los integrantes del campo de manera exhaustiva, con estos elementos parece suficiente.

Dentro de los medios de comunicación, hallamos importantes diferencias, entre otras cuestiones, por su situación de dependencia económica de diversas fuentes. Como ya se indicó, la prensa gráfica sufre una fuerte caída de ventas e ingresos publicitarios, lo que la torna dependiente de otras fuentes de financiamiento.

Algunas empresas de medios gráficos iniciaron procesos de conversión, primero en empresas multimediales y luego en empresas globales. Esto permitió que los medios gráficos fueran subsidiados de alguna manera por los otros emprendimientos económicos, generalmente ligados a la comunicación (producción de revistas y libros, canales de aire y de cable, productoras de contenidos, proveedores de televisión por cable, internet, telefonía, etc.), pero no exclusivamente, ya que en muchos casos se articularon con inversionistas globales.

Sin embargo, el proceso de concentración no se replicó en todos o en una buena cantidad de medios. Muchos emprendimientos periodísticos encontraron financiamiento en distintos grupos económicos y/o en la publicidad estatal. En tal sentido, aquellos medios rezagados en la carrera de concentración dependen exclusivamente de la publicidad privada y de la pauta publicitaria estatal, abierta o encubierta. Si como afirma Bourdieu, la constitución de un campo está determinado por el proceso y capacidad de autonomía de quienes potencialmente lo integran, el proceso de reversión de esta autonomía parece producir fenómenos de crisis, fracturas y disputas, hacia el regreso a formas premodernas de la actividad, mecenazgo y dependencia de grupos de poder, políticos o económicos.

En tal sentido consideramos que no pueden distinguirse del campo periodístico a los medios de comunicación. Es a través de ellos que se materializa el periodismo, pero fundamentalmente porque los medios y las empresas que los cobijan o sostienen, son quienes expresan de manera organizada las líneas argumentales de la deontología profesional.

Con respecto a los periodistas, estos sólo adquieren la condición de tales si se encuentran dentro de un medio de comunicación o agencia de noticias. Fuera de un medio, un periodista es un experiodista o un aspirante a periodista. Si bien no es el único campo de bienes simbólicos en el que se expresa esta dependencia entre los actores y sus contratantes, es probablemente el único que hace esfuerzos por ocultarlo, presentándose a sí mismos como autónomos y libres dentro del campo de la producción de actualidad.

Una de las primeras cuestiones significativas en relación con este tema es la condicionalidad que expresa el Estatuto del Periodista. La Ley 12.908, en su artículo 2, establece que sólo se consideran periodistas profesionales a quienes reciban una paga en dinero por su trabajo periodístico. El artículo completo, detalla con precisión esta cuestión.\footnote{Art. 2º - Se consideran periodistas profesionales a los fines de la presente ley, las personas que realicen en forma regular, mediante retribución pecuniaria, las tareas que les son propias en publicaciones diarias, o periódicas, y agencias noticiosas. Tales el director, codirector, subdirector, jefe de redacción, secretario general, secretario de redacción, prosecretario de redacción, jefe de noticias, editorialista, corresponsal, redactor, cronista, reportero, dibujante, traductor, corrector de pruebas, reportero gráfico, archivero y colaborador permanente. Se incluyen como agencias noticiosas las empresas radiotelefónicas que propalen informativos o noticias de carácter periodístico, y únicamente con respecto al personal ocupado en estas tareas.

  Se entiende por colaborador permanente aquel que trabaja a destajo en diarios, periódicos, revistas, semanarios, anuarios y agencias noticiosas, por medio de artículos o notas, con firma o sin ella, retribuidos pecuniariamente por unidad o al centímetro, cuando alcance un mínimo de veinticuatro colaboraciones anuales. Quedan excluidos de esta ley los agentes o corredores de publicidad y los colaboradores accidentales o extraños a la profesión. No se consideran periodistas profesionales los que intervengan en la redacción de diarios, periódicos o revistas con fines de propaganda ideológica, política o gremial, sin percibir sueldos (el énfasis es propio).}

Esta definición, establecida durante el primer gobierno de Perón, parece haber tomado en cuenta cierto estado precedente de la cuestión y con consenso internacional. A modo de demostración es interesante observar que los premios a periodistas (Pulitzer, Cabot, etc.)\footnote{Últimos premios Pulitzer, en la categoría Periodismo nacional: 2010: Matt Richtel y miembros del~\emph{The New York Times;} 2011: Jesse Eisinger y Jake Bernstein de~\emph{\href{https://es.wikipedia.org/wiki/ProPublica}{ProPublica};} 2012: David Wood de~\href{https://es.wikipedia.org/wiki/HuffPost}{\emph{The Huffington Post}}.; 2013: Lisa Song, Elizabeth McGowan y David Hasemyer de~\emph{\href{https://es.wikipedia.org/w/index.php?title=InsideClimate_News\&action=edit\&redlink=1}{InsideClimate News};} 2014: David Philipps de~\href{https://es.wikipedia.org/w/index.php?title=The_Gazette_(diario)\&action=edit\&redlink=1}{\emph{The Gazette}}; 2015: Carol D. Leonnig de~\emph{\href{https://es.wikipedia.org/wiki/The_Washington_Post}{The Washington Post};} 2016: El equipo de~\emph{\href{https://es.wikipedia.org/wiki/The_Washington_Post}{The Washington Post};} 2017: David Fahrenthold de~\href{https://es.wikipedia.org/wiki/The_Washington_Post}{\emph{The Washington Post}}.} suelen indicar, junto al nombre del premiado, el nombre del medio para el que trabaja.

Retomando la idea central, el ser del periodista está dado por su adscripción y su adhesión al medio. Sin el medio, el periodista es un sujeto sin intervención en la construcción social de la realidad. A la inversa, muchos suelen adoptar la identidad de periodista por el mero hecho de haberse incorporado a un medio.

El medio de comunicación valida de manera colectiva a los periodistas que lo integran, pero ese colectivo ha comenzado a transformarse aceleradamente. El cambio sustancial en las relaciones de propiedad de los medios de comunicación y la pérdida de rentabilidad, basada en la publicidad explícita, han dado por resultado la preeminencia de la empresa editorial por sobre el colectivo de periodistas. Cuando los periodistas ven disminuir las fuentes de trabajo, las empresas periodísticas, agencias y medios de comunicación se tornan subordinantes, si se pretende mantener la condición de periodistas.\footnote{Según el Informe elaborado por el Observatorio y alerta laboral de periodistas de FOPEA 2017 -- 2018, se han perdido 1.791 puestos de trabajo en la prensa argentina entre 2017 y 2018. https://www.fopea.org/informe-observatorio-y-alerta-laboral-de-periodistas-de-fopea-2017-2018/}

Por otra parte, como ya vimos, lo que ha cambiado es la relación estructural de estas empresas con el sistema de financiamiento. Si hasta hace dos décadas el financiamiento de los medios se basaba fundamentalmente en la venta de espacios publicitarios (centímetros de columna, segundos radiales o televisivos) produciendo lo que Habermas (1994) consideraba la venta de audiencias a los anunciantes, hoy esa relación con el mercado publicitario ha devenido en la venta de audiencias a empresas globales, no ya para la promoción de la venta de bienes o servicios, sino por el interés de convertirse en difusores globales de la realidad social adecuada a intereses que exceden en mucho el mero incremento de la venta de tal o cual producto. Este cambio estructural, no solo ocurre con las grandes empresas de medios, sino aún con las pequeñas e incluso con las autogestionadas.

Las pequeñas empresas de medios se tornan dependientes de una o muy pocas empresas anunciantes, patrocinantes y/o del propio Estado. En el caso de las autogestionadas, requieren del financiamiento de partidos, sindicatos, ONGs y/o universidades, poniendo en juego, de esta manera, la condición de independiente, según la definición de Borrat (1989). Ocurre, incluso, que periodistas destacados se ven compelidos a buscar sus propias fuentes de financiamiento en calidad de patrocinantes, a los fines de completar sus ingresos económicos.

En síntesis, el cambio estructural ha producido nuevas relaciones en el campo, tanto entre los periodistas y las empresas como entre los periodistas entre sí, convirtiendo a la disputa entre periodistas en un conflicto más encarnizado, no ya por destacar en el universo del campo, sino por la propia sobrevivencia como tal.


\section{5. 3. Estrategias de protección de la legitimidad del campo periodístico}

Nos propusimos, entre los objetivos específicos, indagar la existencia de estrategias de protección de la legitimidad del campo periodístico. Es decir, consideramos, a modo de hipótesis, que el conflicto debió producir acciones discursivas que intentaran proteger la legitimidad de la actividad periodística, cuestionada por la propia naturaleza del conflicto, donde los actores polares ponían en duda el pilar fundamental de la actividad periodística: la credibilidad. En tal sentido, hemos seleccionado, dos ejes que, como hemos visto en el \emph{corpus}, prevalecen en la estructura argumental de los integrantes dominantes del campo.

El primer eje es la libertad de expresión. Este se presenta como el eje preeminente, es decir, como el eje sobre el que se sostiene el discurso de los sectores dominantes del campo periodístico. El principio podría ser formulado de la siguiente forma: los integrantes del campo tienen el deber de manifestarse públicamente a favor de la libertad de expresión y contra la censura, de acuerdo a lo que en cada momento el campo considere que se está violando. Esta última parte de la formulación es sustantiva, ya que las acciones o discursos que violan el principio son cambiantes, según la consideración que el campo realice en cada momento.

El segundo eje, que agrupa la mayor cantidad de argumentaciones, es el de la violación del principio de independencia. Este puede dividirse en dos tipos de independencia: la económica y la ideológico-política.

A continuación, desarrollaremos cada uno de los ejes y las líneas argumentales sobre las que estructura el campo su discurso. Referiremos tanto a los sectores dominantes como a los periféricos.


\subsection{5. 3.1. Protección de la legitimidad del campo. Libertad de expresión}

Afirmamos que el campo periodístico se estructura alrededor del derecho a la libertad de expresión. Su vulneración por parte de cualquier actor político o social lo coloca fuera del entramado social del sistema democrático y, por lo tanto, fuera de toda legitimidad. Como ya vimos, Habermas (1991: 211), en \emph{Historia y crítica de la opinión pública}, refleja con precisión este fenómeno

\begin{quote}
Mientras la existencia misma de una prensa políticamente raciocinante es precaria, se ve ésta forzada a una autotematización continua: hasta la legalización permanente de la publicidad políticamente activa, la aparición y el mantenimiento de un periódico político equivalía al compromiso activo con la lucha por conseguir un ámbito de libertad para la opinión pública, con la lucha por la publicidad como principio.
\end{quote}

La libertad de expresión es un derecho que interesa a los medios. Fuera del campo periodístico, los ciudadanos carecen de ese derecho de manera plena, ya que no acceden a los medios de comunicación.

La burguesía protegió este derecho, por encima del de la propiedad y de la libertad individual. Lo convirtió, así, en un derecho absoluto, de manera tal que la existencia de tal libertad o derecho era demostración y/o garantía de democracia, cualquiera fuera el régimen de selección de autoridades de gobierno, mientras que su vulneración era demostración de lo contrario.

Al mismo tiempo, en virtud de que ese derecho es propio del campo y, de alguna manera, exclusivo, la definición de su violación le es inherente, y es susceptible de ser resignificada de acuerdo a las condiciones y contingencias sociales. Es el campo el que establece cuándo el principio ha sido vulnerado, aunque son los medios quienes representan institucionalmente ese derecho.

En un párrafo a destacarse de \emph{El periódico, actor político}, Hector Borrat (1989) señala

\begin{quote}
El periódico califica al Gobierno como democrático o no según asuma, frente a él o al conjunto de la prensa, una política comunicativa que favorezca o perjudique a los intereses de la gran industria periodística. La libertad de prensa --entendida como epifanía de la libertad de empresa- deviene así, desde la perspectiva del periódico, la medida de todas las cosas.
\end{quote}

Una de las primeras acusaciones realizadas desde el campo periodístico al gobierno de Cristina Fernández de Kirchner, en el marco del conflicto, fue la violación de ese derecho. Consecuentemente, en el conflicto intraperiodístico el eje de la libertad de expresión dividió las aguas, convirtiendo a todo aquel que no fijara una posición crítica sobre la política de prensa del gobierno, en un expatriado de la profesión.

Es importante destacar que el campo periodístico sólo sintió violentada la libertad de expresión a partir del inicio del conflicto. Existieron antecedentes previos como el conflicto entre \emph{Página/12} y el periodista Julio Nudler, que derivó en el estallido de la Asociación PERIODISTAS, pero no se había llegado al grado de que un sector mayoritario del campo pretendiera la expulsión de los réprobos. Durante el desarrollo del conflicto, los periodistas de los medios dominantes fueron mutando su opinión hasta concebir que todos los gobiernos, el de Néstor Kirchner, el de Cristina Fernández de Kirchner y, en general, los gobiernos de signo peronista violentaban la libertad de expresión.

No es sencillo hallar referencias que acrediten un discurso positivo sobre la garantía de la libertad de expresión durante casi ningún gobierno en la República Argentina, ya que el campo lo considera una precondición naturalizada y solo reacciona cuando considera que se ha vulnerado. Durante el \emph{Seminario Internacional} que organizó el diario Clarín para conmemorar sus 60 años de existencia, a mediados de 2005, denominado \emph{Desafíos del periodismo real: Los diarios en la encrucijada del siglo XXI} (2006), Fernán Saguier, en el marco de la Mesa \emph{El desafío de la independencia}, afirmó lo siguiente:

\begin{quote}
En los últimos meses, cada vez son más los colegas del exterior que preguntan si la prensa argentina tiene problemas en su libertad para expresarse. La respuesta que yo les doy es que no existen problemas formales para la libertad de prensa. Este gobierno no ha mandado al parlamento ningún proyecto de Ley que sea lesivo para la libertad de prensa y no ha tomado decisiones económicas y financieras que afecten aún más la situación de las empresas (pág. 160).
\end{quote}

En su misma alocución deja entrever lo que luego se convertiría en el centro de la crítica respecto a la limitación de la libertad de expresión: las reacciones verbales del presidente Néstor Kirchner.

\begin{quote}
No vivimos una persecución institucional hacia la prensa. Pero si se suceden tres tipos de actos desconcertantes: llamados telefónicos en grado inusual, la centralización de la información pública en pocas manos que deciden qué medios y qué periodistas la reciben, y el manejo arbitrario y selectivo de la publicidad. Se podría mencionar los ataques verbales, pero aquí me detengo. Porque un presidente está en su derecho cuando se expresa en desacuerdo con lo que dicen los medios. Más que nada, el poder nos ha sorprendido con su obsesión, su queja constante y su descalificación, cuando la realidad publicada lo contradice (pág.161).
\end{quote}

Si bien a los actores centrales del campo periodístico no les fue sencillo encontrar elementos para demostrar que la libertad de expresión estaba siendo vulnerada, el argumento vino de la mano del proyecto de la ley de Servicios de Comunicación Audiovisual.\footnote{En agosto de 2009, la~presidenta~Cristina Fernández de Kirchner~envió al Congreso de la Nación~el Proyecto de Ley de Medios de Comunicación Audiovisual, para reemplazar la Ley de Radiodifusión de la dictadura,~que venía siendo impulsada por la periferia del campo desde el año 2004. La\textbf{~}\emph{Coalición por una Radiodifusión Democrática}\textbf{~s}e conformó durante el año 2004, convocada por el Foro Argentino de Radios Comunitarias (FARCO). Fue un grupo conformado por sindicatos, universidades y carreras de comunicación, organizaciones sociales, radios comunitarias, pequeñas radios comerciales y organismos de derechos humanos, entre otros. La coalición abogó por 21 puntos básicos para una nueva Ley de Radiodifusión.} No obstante, desde el punto de vista argumental, fue mucho mayor la incidencia de la crítica a los periodistas de los medios dominantes, realizada por funcionarios del gobierno y posteriormente por periodistas, que no pertenecían a los grandes medios de comunicación. Un fenómeno paradigmático fue el programa de crítica de medios \emph{6 7 8}.

La legitimidad de pertenencia al campo y la verosimilitud se articularon definiendo a los contendientes. Por un lado, los que se encontraban en medios masivos y de alta penetración, susceptibles de verse afectados por la ley de Medios. Por el otro, los que se hallaban fuera de esos medios, en la periferia del campo y llamados a ser los beneficiarios potenciales del proyecto de ley.


\subsection{5. 3. 2. Protección de la legitimidad del campo periodístico. La independencia}

Afirmamos anteriormente que en la actividad periodística la independencia no existe, pero que fuera de ella no hay periodismo. Intentando buscar una metáfora que lo represente creemos que la independencia es al periodismo lo que la estética es al arte. Aún los periodistas más comprometidos políticamente con determinado partido o corriente de opinión adscriben a su independencia y alegan que los ``independientes'' no son tales.

Durante mucho tiempo, y aún hoy, suele relacionarse de manera mecánica la falta de independencia con los anunciantes publicitarios. Sin embargo, en general, los grandes medios poseen muchísima autonomía de sus anunciantes explícitos, es decir de aquellos que ponen anuncios en los medios.

En la disputa metaperiodística observamos dos formas de violación al principio de independencia:

a- Independencia económica: los integrantes del campo no deben subordinar su autonomía a ningún interés económico por fuera de su casa editorial, e incluso debieran resistir presiones también de ella.

b- Independencia política o ideológica: los integrantes del campo no deben subordinar su autonomía a intereses políticos o ideológicos, en consecuencia, no son independientes quienes subordinan su objetividad o la verdad, a intereses que no son los de la audiencia, sino los de su ideología o sector político.

La independencia económica fue argumentada enfáticamente por parte de los periodistas \emph{defensores.} Institucionalizados en organizaciones económicamente muy sólidas, presentaban con desconfianza a los periodistas periféricos que accedían a posiciones destacadas, tanto en medios públicos como en nuevos emprendimientos privados, en muchos casos sostenidos con apoyo estatal. Simultáneamente, se hacía evidente la escasa inversión publicitaria del Estado a los medios más críticos y eso produjo en los periodistas de esos medios un alineamiento rígido con la directiva editorial, que incluso puede apreciarse en los periodistas \emph{profesionales}.

La falta de independencia por subordinación económica se expresó, por ejemplo, a través de la ``denuncia'' de los contratos y montos cobrados por periodistas del programa \emph{6 7 8} y de otros medios. Sin embargo, salvo en raras excepciones, no se brindaba información sobre esos montos, es decir, las denuncias eran veladas y hacían hincapié en la mejora de la situación económica de algún periodista.

Es llamativo incluso que, en las tres biografías críticas, contra Timerman, Verbitsky y Víctor Hugo Morales, los autores no ponen el factor económico como un elemento que hubiera traccionado a estos periodistas al lugar en que los ubican. Héctor Timerman era un hombre de buena posición económica; Víctor Hugo Morales también lo era. En el caso de Verbitsky, no pudieron encontrar indicios de posesiones materiales que lo mostraran como a un hombre que se vendiera por dinero.

La segunda causa que, en general, se ha esgrimido como limitante de la independencia fue la ideológica o política. En esto coinciden casi todos los periodistas tanto \emph{defensores} como \emph{profesionalistas} y \emph{predicadores}. Parece probable que la falta de independencia por defender una postura política o un gobierno, sea menos peligrosa para la legitimidad del campo y pueda ser expresada con menos riesgo para la profesión.

Nuevamente, si recordamos las tres biografías destinadas a expulsar del campo a estos periodistas, la falta más grave que se les adjudica es carencia de valores morales, inexperiencia y, en el caso de la biografía sobre Verbitsky, la historia adolece justamente del motivo, del móvil que hubiera llevado a este periodista a ser un doble agente (de los Servicios de Inteligencia del Estado y de la Inteligencia Montonera).

Para los \emph{predicadores}, los periodistas agrupados en la categoría de \emph{defensores} eran críticos del gobierno por carecer de independencia de la línea ideológica editorial de sus mandantes. Sin embargo, en general, con la excepción de los que participaban del programa televisivo \emph{6 7 8}, este grupo de periodistas fue más respetuoso del \emph{habitus} correspondiente a no difamar a sus colegas.

En síntesis, la independencia en el campo periodístico es el documento de identidad de pertenencia al mismo. Es, por lo tanto, irrenunciable.

\chapter{Post scriptum}

Este trabajo concluye cuatro años después de haber sido imaginado y proyectado. Cuatro años en los que fuimos testigos de cambios significativos en la composición del campo periodístico, decenas de medios cerraron, miles de periodistas perdieron sus trabajos e incluso, agencias de noticias históricas hoy ya no existen.

Es probable que el cambio de signo político en el gobierno nacional permita la recuperación de algunos espacios. Sin embargo, difícilmente se regrese al inicio del ciclo.

El estilo \emph{6 7 8}, que signó una época periodística, probablemente no vuelva a encarnarse luego de haber perdido, provisoriamente, la batalla cultural que se había propuesto. Sin embargo, el campo periodístico parece salir debilitado de la confrontación. La crítica al interior del campo periodístico parece haber llegado, en la Argentina, para quedarse probablemente bajo otras formas.

La experiencia no es solo en Argentina. El debilitamiento del campo y su legitimidad se observa simultáneamente en los movimientos sociales de Chile, Colombia, Ecuador y Bolivia, donde las movilizaciones sociales, a diferencia de lo que ocurría hasta hace poco tiempo, ya no buscan a los medios dominantes para ser reconocidos, sino que desconfían de ellos y son repudiados.

Iniciamos este trabajo bajo la premisa que el conflicto generaba el riesgo de fractura del campo periodístico. Creemos que ese riesgo se ha consumado.

Septiembre de 2020

\chapter{Bibliografía}

%Alberdi, Juan B. (2015): Cartas sobre la prensa y la política militante de la República Argentina (cartas quillotanas). Biblioteca Saavedra Fajardo, Madrid. {[}1853{]}.
%
%Alonso, Paula (1997): ``\,``En la primavera de la historia'': el discurso político del roquismo de la década del ochenta a través de su prensa''. En: Boletín del Instituto de Historia Argentina y Americana ``Dr. Emilio Ravignani'', Tercera serie, N° 15, 1er. Semestre de 1997, Buenos Aires.
%
%Alonso, Paula (2004): ``La tribuna nacional y Sud-América: tensiones ideológicas en la construcción de la ``Argentina moderna'' en la década de 1880''. En: Construcciones impresas. Buenos Aires: Fondo de Cultura Económica.
%
%Altamirano, Carlos y Myers, Jorge (2008) Historia de los intelectuales en América Latina. Tomo I. Buenos Aires: Katz Editores.
%
%Altheide, D. y Rasmussen, P.(1976) Becoming News. A Study of Two Newsrooms, en Sociology of works an occupations, Vol. 3 N. 2, Pags. 223-226.
%
%Amado, Adriana (2016) El periodismo por los periodistas. Perfiles profesionales en las democracias de América Latina. Buenos Aires: Konrad-Adenauer-Stiftung e. v.
%
%Anguita Eduardo (2002) Grandes hermanos. Alianzas y negocios ocultos de los dueños de la información. Buenos Aires: Colihue.
%
%Arrueta, C. (2010). ¿Qué realidad construyen los diarios? Una mirada desde el periodismo en contextos de periferia. Buenos Aires: La Crujía Ediciones.
%
%Arrueta, Cesar (2013) Identidad(es) periodística(s) Una mirada desde los escenarios de tensión, en Revista:~Tram{[}p{]}as de la Comunicación y la Cultura; Núm 76, ISSN:~2314-274X pp. 81-87 Recuperado 05/03/2017: \url{http://sedici.unlp.edu.ar/handle/10915/37153}
%
%Auza, Néstor T. (1978) El periodismo de la Confederación. Buenos Aires: Eudeba.
%
%Baldoni (2012) Kirchnerismo, mediatización e identidades políticas: reflexiones en torno a la política, el periodismo y el discurso . 2003-2008 / Maestri Mariana ... {[}et.al.{]} ; coordinado por Irene Lis Gindin ; 1a ed. - Rosario: UNR Editora. Editorial de la Universidad Nacional de Rosario, 2014.
%
%Barbier, Frédéric y Bertho-Lavenir, Catherine (1999) Historia de los medios deDiderot a Internet, Buenos Aires: Colihue.
%
%Bargach, A. y Suárez, M. (2014). Télam, el hecho maldito del periodismo argentino. Una historia narrada por sus trabajadores. Buenos aires: Ediciones TXT.
%
%Barone, Orlando (2011) K Letra bárbara. Buenos Aires: Sudamericana.
%
%Barrera Carlos (2004) Historia del Periodismo Universal. Barcelona: Ariel.
%
%Beliz y Zuleta Puceiro (1998) La cultura profesional del periodismo argentino en Cuadernos Autrales de Comunicación 1. Buenos Aires: Universidad Austral
%
%Benhamou, F. (2015). El libro en la era digital: papel, pantallas y otras derivas. Ciudad Autónoma de Buenos Aires: Editorial Paidós.
%
%Bernays, Edward (2008) Propaganda. Madrid: Melusina
%
%Blaustein, Eduardo (2014) Las locuras del Rey Jorge. Buenos Aires: Ediciones B.
%
%Blaustein, Eduardo. (2013). Años de rabia. El periodismo, los medios y las batallas del kirchnerismo. Buenos Aires: Ediciones B.
%
%Borderia Ortiz, Enrique; Laguna Platero, Antonio; Martínez Gallego, Francesc (2015) Historia social de la comunicación: mediaciones y públicos. Madrid, Síntesis.
%
%Borges-Bioy Casares (1967). Crónicas de H. Bustos Domecq. Buenos Aires: Losada.
%
%Borrat, H. (2009). El periódico, actor político. Barcelona: Editorial Gustavo Gili.
%
%Bourdieu, P. (2008). Cuestiones de Sociología. Madrid: Ediciones Itsmo.
%
%Bourdieu, P. y Wacquant Loïc (2008) Una invitación a la Sociología reflexiva, Buenos Aires: Siglo XXI Editores.
%
%Bourdieu, Pierre (1997) Sobre la televisión. Barcelona: Anagrama.
%
%Bourdieu, Pierre (1999). La miseria del mundo, Buenos Aires: Fondo de Cultura Económica.
%
%Bourdieu, Pierre (2002). Campo de poder, campo intelectual, Barcelona: Montressor
%
%Brancatelli, Diego (2015). Todos contra BRANCA contra todos. Buenos Aires: Ediciones B
%
%Bücher, K. (1917) Die Anfänge des Zeitunswesens, en Die EnstehungderVolkswirtschaft, Vol I, Berlín.
%
%Capasso, J. (2013). Victor Hugo. Una historia de coherencia y convicción. Buenos Aires: Ediciones Al Arco.
%
%Castells, M. (1998). La era de la información. Economía, sociedad y cultura. Vol 2 El poder de la identidad. Madrid: Alianza Editorial.
%
%Cecchini, D. y Mancinelli, J. (2010). Silencio por sangre. La verdadera historia de Papel Prensa. Buenos Aires: Perito Mundo.
%
%Creswell, J. W. (1994). Research Design. Qualitative \& Quantitative Approaches. London: Sage.
%
%De Diego, J. (2012).~Concentración económica, nuevos editores, nuevos agentes. Primer Coloquio Argentino de Estudios sobre el Libro y la Edición, 31 de octubre, 1 y 2 de noviembre de 2012, La Plata, Argentina. EN: Actas. La Plata: UNLP-FAHCE.
%
%De Marco, Miguel Ángel (2006) Historia del periodismo argentino. Desde los orígenes hasta el Centenario de Mayo. Buenos Aires: EDUCA.
%
%Dessein, D y Roitberg G. (Comps.). (2014). Nuevos desafíos del periodismo. Buenos Aires: Ariel.
%
%Díaz Claudio (2009) Diario de guerra. Clarín el gran engaño argentino. Buenos Aires: Gárgola
%
%Díaz, César (2012). Comunicación y Revolución. 1759-1810. La Plata: Universidad Nacional de La Plata -- Ediciones EPC.
%
%Donsbach Wolfgang (2014). Cómo entender al periodismo : selección de la obra de Wolfgang Donsbach; compilado por Fernando J. Ruiz. - 1a ed. - Ciudad Autónoma de Buenos Aires : Konrad Adenauer Stiftung
%
%Fontevecchia, J. (2015). \href{https://www.boutiquedellibro.com.ar/9789504946175/Quienes+Fuimos+En+La+Era+K/}{Quiénes fuimos en la era k}. Buenos Aires. Planeta
%
%Fontevecchia, J. (2018) \url{https://www.perfil.com/noticias/medios/fontevecchia-quienes-critican-a-los-medios-estan-criticando-al-sitema-democratico.phtml}. Seminario ``Telecomunicaciones, Medios Privados y Públicos en la era de la Convergencia'', organizado por el Sistema Federal de Medios y Contenidos Públicos y la Embajada Británica en Argentina. 05/11/18
%
%Fontevecchia, J. (2018). Periodismo y Verdad. C.A.B.A: Editorial Paidós.
%
%Fraga Roserndo (comp.) (1997) Autopercepción del periodismo en la Argentina. Buenos Aires: Editorial de Belgrano.
%
%Gans (1979) Deciding What´s News, Newsweek an Time, Nueva York: PantheonBooks
%
%Gil Bellota, Carlos J. (2012) Periodismo, metaperiodismo y bienes públicos. \url{https://www.datanalytics.com/2012/09/07/periodismo-metaperiodismo-y-bienes-publicos/}
%
%Gindin Irene Lis (2014) Identidades fragmentadas: apuntes teóricos sobre las identidades políticas, en Kirchnerismo, mediatización e identidades políticas: reflexiones en torno a la política, el periodismo y el discurso .2003-2008. Rosario: UNR Editora.
%
%Glaser, B y Strauss, A (1967) The discovery of gronunded theory strategies for qualitative reserarch. New York: Aldine Publishing Company.
%
%Glaser, B. (1978),~ Theoretical Sensitivity. San Francisco, University of California, 1978.~Traducción: Ada Freytes Frey.
%
%Golding-Elliott (1979) Making the news, Londres: Longman
%
%González, Gustavo (2011) Noticias Bajo Fuego. Buenos Aires: Planeta
%
%Habermas, J. (2006). Historia y Crítica de la Opinión Pública. Barcelona: Editorial Gustavo Gili.
%
%Habermas, Jurguen (1982) Teoría de la Acción Comunicativa, Madrid, Ed.: Taurus
%
%Halperín, Jorge (2007), Noticias del poder, Buenas y malas artes del periodismo político. Buenos Aires: Aguilar.
%
%Humphreys, Peter (2008) Subvenciones a la prensa en Europa. Una visión histórica. Extraído de Revista \href{https://dialnet.unirioja.es/servlet/revista?codigo=6769}{Telos: Cuadernos de comunicación e innovación},~ISSN~0213-084X,~\href{https://dialnet.unirioja.es/ejemplar/195612}{Nº. 75, 2008},~págs.~71-84
%
%Jorge Asís (2011) El kirchnerismo póstumo. Buenos Aires: Ediciones B
%
%Jornet, C. y Dessein, D (Comps.). (2014). Tiempos Turbulentos. Medios y libertad de expresión en la Argentina de hoy. Buenos Aires: Ariel.
%
%Levinas, Gabriel (2013) El pequeño Timerman. Buenos Aires: Ediciones B
%
%Levinas, Gabriel (2015) Doble agente. La biografía inesperada de Horacio Verbitsky. Buenos Aires: Sudamericana
%
%Llonto, Pablo. (2007). La Noble Ernestina. Ciudad Autónoma de Buenos Aires: Editorial Punto de Encuentro
%
%Lopez Echagüe, Hernan (2015) El perro. Horacio Verbitsky, un animal político. Buenos Aires: Ediciones B
%
%Majul Luis (1999) Periodistas. Qué piensan y que hacen los que desiden en los medios. Buenos Aires: Editorial Sudamericana.
%
%Majul, L. (2012). Lanata. Buenos Aires: Margen Izquierdo
%
%Márquez, V. y Ces, A. (2010). Periodismo de infantería. Buenos Aires: Clarinete.
%
%Martini, Stella;~Luchessi, Lila (2004). Los que hacen la noticia: periodismo, información y poder. Buenos Aires: Biblos
%
%Mastrini y Becerra (2017). Medios en guerra. Balance, crítica y desguace de las políticas de comunicación 2003-2016. CABA: Biblos
%
%Mauss, M. (2009). Ensayo sobre el don. Forma y función del intercambio en las sociedades arcaicas. Buenos Aires: Katz Editores.
%
%Meneses Fernández, \href{https://www.researchgate.net/profile/Maria_Meneses_Fernandez}{M. Dolores (2010)} Aportaciones del metaperiodismo al estudio de la prensa y del periodismo en contextos históricos delimitados. La Transición democrática, en Zer, Vol. 15 - Núm. 29, ISSN: 1137-1102, pp. 175-191
%
%Meneses Fernández, María Dolores (2008). Noticias sobre la prensa. Imagen propia en la Transición democrática española. Madrid: Ediciones Fragua.
%
%Mercado, Silvia Diana (2012) El inventor del peronismo. Buenos Aires: Planeta
%
%Mochkofsky, G. (2011). Pecado original. Clarín, los Kirchner y la lucha por el poder. Buenos Aires: Planeta.
%
%Molina, Eugenia (2012) El poder de la opinión pública. Trayectos y avatares de una\\
%nueva cultura política en el Río de la Plata, 1800-1852. Santa Fe: UNL.
%
%Morales, Victor Hugo (2015) \href{http://www.cuspide.com/9789876842686/El+Rebenque+Del+Diablo/}{El rebenque del diablo}. Buenos Aires: Colihue
%
%Morales, Victor Hugo (2015) Mentime que me gusta. Buenos Aires: Aguilar.
%
%Moyano, Julio (1996) Prensa y Modernidad. Paraná: CEPCE -- EDUNER.
%
%Moyano, Julio (2008) Prensa, modernidad y transición. Buenos Aires, UBA, Facultad de Ciencias Sociales.
%
%Moyano, Julio (2013): ``Seis años decisivos: La Revolución de Mayo y la construcción de la prensa moderna en el Río de la Plata''. En: Pineda, Adriana, y Gantús, Fausta (Comp.): Miradas y acercamientos a la prensa decimonónica. México: Universidad Michoacana de San Nicolás de Hidalgo/Red de Historiadores de la Prensa y el Periodismo en Iberoamérica.
%
%Muchnik, Daniel (2012) Aquel periodismo: política, medios y periodistas en la Argentina, Buenos Aires: Edhasa
%
%Novaro, M. y Birmajer, M. (2015). Grandes y pequeñas mentiras que nos contaron. La guerra contra la prensa. Qué nos dejan doce años de acoso al periodismo. Buenos Aires: Planeta.
%
%Ojeda Alejandra y Moyano Julio (2019): ``En la forja de un diario moderno''. En: Ojeda, Alejandra, Moyano Julio y Levenberg Rubén: Prácticas de oficio e innovación tecnológica. Tensiones y estrategias en dos momentos clave del diario argentino La Nación. Buenos Aires: IEALC -- UBA.
%
%Ojeda, Alejandra (2010) ``De la Arenga Faccional al Reclame Publicitario''. En: revista Pensar la Publicidad, nº 2, Barcelona.
%
%Oliván, M. J. y Alabarces, P. (2010). 6 7 8. La creación de otra realidad. Buenos Aires: Paidós.
%
%Piccone, N. (2015). La inconclusa Ley de Medios. La historia menos contada. Buenos Aires: Ediciones Continente.
%
%Ramonet, I. (2011). La explosión del periodismo. Internet pone en jaque a los medios tradicionales. Buenos Aires: Capital Intelectual.
%
%Rivera, Jorge (1990) ``De la facción al folletín''. En: Diario Clarín, suplemento cultural Cultura y Nación, 23 de agosto.
%
%Rivera, Jorge (1998) El escritor y la industria cultural. Buenos Aires: Atuel.
%
%Rosso, Daniel (2012) Máquinas de captura. Los medios concentrados en tiempos del kirchnerismo. Buenos Aires: Colihue.
%
%Rotemberg, Abrasha (1999), Historia confidencia. La Opinión y otros olvidos. Buenos Aires: Editorial Sudamericana
%
%Ruiz, F. (2014). Guerras mediáticas. Las grandes batallas periodísticas desde la Revolución de Mayo hasta la actualidad. Ciudad Autónoma de Buenos Aires: Sudamericana.
%
%Saferstein, E. (2016). ``La década publicada'' Los bestsellers políticos y sus editores. Producción de libros, difusión de temas e intervención pública en el mercado editorial argentino (2003-2015), Tesis de doctorado, Universidad de Buenos Aires.
%
%Schlesinger, (1978) Putting ``reality'' together. BBC news, Londres: Constable.
%
%Shumway, Nicolás (2003) La invención de la Argentina. Buenos Aires: Emecé.
%
%Siebert F. y Peterson T. (1967) Tres teorías sobre la prensa en el mundo capitalista. Buenos Aires: De la Flor.
%
%Sirvén, Pablo (2011) Perón y los medios de comunicación. Buenos Aires: Sudamericana.
%
%Sirvén, Pablo (2013) Converso. Buenos Aires: Margen izquierdo.
%
%Sivak, M. (2013). Clarín, el gran diario argentino. Una historia. Buenos Aires: Planeta.
%
%Sivak, Martín (2015) Clarín, La era Magnetto. Buenos Aires: Planeta.
%
%Stefoni Jorge Andrés. (2013) Controversias contemporáneas en el periodismo argentino. Los nudos de la política y el debate sobre la condición profesional (2009-2011) Revista Astrolabio, Núm 10 Nueva Época ISSN 1668-7515 pp. 389 -- 419.
%
%Strauss, A. y Corbin, J. (1998). Basics of Qualitative Research: Techniques and Procedures for Developing Grounded Theory. London: Sage.
%
%Ure, Mariano y Schwarz Christian (2014) Las identidades del periodismo argentino : estudio cualitativo de la percepción de los propios periodistas, Ciudad Autónoma de Buenos Aires: Konrad Adenauer Stiftung.
%
%Vargas, Walter (2015) Periodistas depordivos. Buenos Aires: Ediciones al arco.
%
%Verón, E. (1987). Construir el acontecimiento. Buenos Aires: Gedisa.
%
%\href{http://www.cuspide.com/resultados.aspx?c=VILLARRUEL+DARIO\&por=AutorEstricto\&aut=291343\&orden=fecha}{Villarruel} Dario (2014) (In)justicia mediática. Buenos Aires: Sudamericana.
%
%Vommaro, G. y Baldoni, M. (2012). Bernardo y Mariano: las transformaciones del periodismo político en Argentina, de los años ochenta a los noventa, en Mediálogos, Revista de Comunicación Social, nº 2, Universidad Católica del Uruguay, Montevideo, pp. 59-82.
%
%Weber, Max (1992) Para una sociología de la prensa en \href{http://bddoc.csic.es:8080/ver/ISOC/revi/0034.html}{Revista Española de Investigaciones Sociológicas} nº 57 ISSN:~0210-5233. (pp. 251-259) Madrid.
%
%Wiñazki, Miguel (comp.) (2000) Puro periodismo. Buenos Aires: Editorial de Belgrano.
%
%Wolf, M. (2004). La investigación de la comunicación de masas: Crítica y perspectivas. Ciudad Autónoma de Buenos Aires: Paidós.
%
%Zukernik Eduardo (2006) , Hechos y noticias. Claroscuros de la prensa gráfica en la Argentina. Buenos Aires: La Crujía Ediciones - Konrad Adenauer Stiftung.
%
%Zunino, E. (2011) Patria o medios. Buenos Aires: Sudamericana.
%
%Zunino, E. (2014). Periodistas en el barro. Peleas, aprietes, traiciones y negocios. Miserias y razones de la guerra mediática en la Argentina reciente Buenos Aires: Sudamericana.


\end{document}
